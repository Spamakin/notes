\documentclass[letterpaper]{article}

\usepackage{style}  % If you feel like procrastinating, mess with this file
\usepackage{algo}   % Thank you Jeff, very cool!
\usepackage{quiver} % https://q.uiver.app/

\addbibresource{refs.bib}

% Required reading
% https://jmlr.csail.mit.edu/reviewing-papers/knuth_mathematical_writing.pdf
%   Along with required viewing:
%   https://www.youtube.com/watch?v=N6QEgbPWUrg&list=PLOdeqCXq1tXihn5KmyB2YTOqgxaUkcNYG
% https://faculty.math.illinois.edu/~west/grammar.html

% % % % % % % % % %
%     Cursor      %
%     Parking     %
%     Lot         %
% % % % % % % % % %

% Disable check for mismatched parens/brackets/braces
%     chktex-file 9
% Disable check for different counts of parents/brackets/braces
%     chktex-file 17

\title{Topology Notes and Exercises}
\author{Anakin Dey}

\begin{document}
\maketitle

\listoftheorems[ignoreall, show={defn}, title={List of Definitions}]
\clearpage

\listoftheorems[ignoreall, show={ex}, title={List of Examples and Counterexamples}]
\clearpage

\section{Metric Spaces}

The most familiar example of a metric space is $\R^{n}$ with $\dist(x, y) = \pqty{\sum_{i = 1}^{n} (x_{i} - y_{i})^{2}}^{1/2}$.
We also have a notion of continuity.
A function $f$ is \emph{continuous at a point $x$} if for all $\e > 0$, there exists $\delta > 0$ such that $\dist(x, y) < \delta \implies \dist(f(x), f(y)) < \e$.
We wish to generalize this notion.

\begin{defn}[Metric Spaces]
  A \emph{metric space} is a set $X$ equipped with a distance function $\dist\colon X \times X \to \R$ called a \emph{metric} satisfying
  \begin{enumerate}
    \item $\dist(x, y) \geq 0$ with equality if and only if $x = y$
    \item $\dist(x. y) = \dist(y, x)$
    \item $\dist(x, z) \leq \dist(x, y) + \dist(y, z)$.
  \end{enumerate}
\end{defn}

\begin{defn}[$\e$-ball]
  For a point $x$ in some metric space $X$, the \emph{$\e$-ball about x} is $B_{\e}(x) \defeq \set{y \in X | \dist(x, y) < \e}$.
\end{defn}

It is not hard to rephrase the definition of continuity in terms of $\e$-balls.

\begin{defn}[Open and Closed Sets in $\R$]
  A subset $U$ of some metric space $X$ is \emph{open} if for all $x \in U$ there exists $\e > 0$ such that $B_{\e}(x) \subseteq U$.
  A subset $C$ is \emph{closed} if its complement is open.
\end{defn}

We can actually even rephrase continuity in terms of open sets.

\begin{prop}[Continuity in terms of Open Sets]
  A function $f\colon X \to Y$ between metric spaces is continuous if and only if $f^{-1}(U)$ is open for every open set $U \subseteq Y$.
\end{prop}

\begin{pf}
  Suppose $f\colon X \to Y$ is continuous.
  Let $U \subset Y$ be some open set and let $f(x) \in U$.
  Then there exists $\e > 0$ such that $B_{\e}(f(x)) \subseteq U$.
  By continuity of $f$, there exists $\delta > 0$ such that $f(B_{\delta}(x)) \subseteq B_{e}(f(x))$.
  Thus $B_{\delta}(x) \subseteq f^{-1}(U)$ and $f^{-1}(U)$ is open.

  Conversely, suppose that $f^{-1}(U)$ is open for any open $U \subseteq Y$.
  Let $\e > 0$.
  Then $B_{\e}(f(x))$ is open and $f^{-1}(B_{\e}(f(x)))$ is open and contains $x$.
  Thus there exists $\delta > 0$ such that $B_{\delta}(x) \subseteq f^{-1}(B_{\e}(f(x)))$.
  Thus $f(B_{\delta}(x)) \subseteq B_{\e}(f(x))$ and $f$ is continuous.
\end{pf}

This can also all be rephrased in terms of closed sets, rather than open sets.

\begin{ex}[Other Metrics on $\R$]
  We can also equip other metrics on $\R^{n}$.
  \begin{align*}
    \dist_{2}(x, y) &= \sum_{i = 1}^{n} \abs{x_{i} - y_{i}} \\
    \dist_{3}(x, y) &= \max_{i = 1}^{n} (\abs{x_{i} - y_{i}})
  \end{align*}
\end{ex}

However it turns out that for $\R^{n}$, choice between these 3 metrics is irrelevant.

\clearpage

\begin{prop}[Equivalence of Open Sets]
  Suppose $\dist_{1}$ and $\dist_{2}$ are two metrics on the same set $X$ such that for any $x \in X$ and $\e > 0$, there exists $\delta > 0$ such that
  \[
    \dist_{1}(x, y) < \delta \implies \dist_{2}(x, y) < \e
  \]
  and
  \[
    \dist_{2}(x, y) < \delta \implies \dist_{1}(x, y) < \e.
  \]
  Then these metrics define the same open sets in $X$.
\end{prop}

\begin{pf}
  \quest{TODO: PROOF}
\end{pf}

\begin{cor}[Equivalence of Open Sets in $\R^{n}$]
  The following metrics define the same open sets in $\R^{n}$:
  \begin{align*}
    \dist(x, y) &= \pqty{\sum_{i = 1}^{n} (x_{i} - y_{i})^{2}}^{1/2} \\
    \dist_{2}(x, y) &= \sum_{i = 1}^{n} \abs{x_{i} - y_{i}} \\
    \dist_{3}(x, y) &= \max_{i = 1}^{n} (\abs{x_{i} - y_{i}})
  \end{align*}
\end{cor}

\clearpage

\begin{exercise}[Bredon 1.1.1]
  Consider the set $X$ of all continuous real valued functions on $[0, 1]$.
  Show that
  \[
    \dist(f, g) = \int_{0}^{1} \abs{f(x) - g(x)} \dd{x}
  \]
  defines a metric on $X$.
  Is this still the case if continuity is weakened to integrability?
\end{exercise}
\begin{pf}
  Positivity follows from the positivity of absolute value.
  If $f = g$ then clearly for all $x \in [0, 1]$ we have that $\abs{f(x) - g(x)} = 0$.
  Then if $f \neq g$ then there exists $x \in [0, 1]$ such that $\abs{f(x) - g(x)} \neq 0$.
  Since these functions are continuous, this yields that $\int_{0}^{1} \abs{f(x) - g(x)} \dd{x} > 0$.

  Symmetry follows from $\abs{f(x) - g(x)} = \abs{g(x) - f(x)}$.

  The triangle inequality follows from the triangle inequality for absolute value.

  Note that we require continuity.
  Let $f(x) \defeq 0$ and let $g(x) = 0$ for all $x > 0$ and $g(0) = 1$.
  Then $f$ and $g$ are integrable and $\dist(f, g) = 0$ but $f \neq g$.
\end{pf}

\begin{exercise}[Bredon 1.1.2]
  If $X$ is a metric space and $x_{0}$ is a given point in $X$, show that the function $f\colon X \to \R$ given by $f(x) = \dist(x, x_{0})$ is continuous.
\end{exercise}
\begin{pf}
  Let $\e > 0$.
  Suppose that $\dist(x, y) < \e$.
  Then we have that
  \[
    \abs{f(x), f(y)} = \abs{\dist(x, x_{0}) - \dist(y, x_{0})} \leq \abs{\dist(x, y)} = \dist(x, y) < \e.
  \]
\end{pf}

\begin{exercise}[Bredon 1.1.3]
  If $A$ is a subset of a metric space $X$ then define a real valued function $d$ on $X$ by $d(x) = \dist(x, A) \defeq \int\set{\dist(x, y) | y \in A}$.
  Show that $d$ is continuous.
\end{exercise}
\begin{pf}
  Note that for any $x, y \in X$ and $z \in A$ we have that
  \[
    \dist(x, z) \leq \dist(x, y) + \dist(y, z).
  \]
  Taking infimum yields that
  \[
    \dist(x, A) \leq \dist(x, y) + \dist(y, A).
  \]
  Thus $\dist(x, A) - \dist(y, A) \leq \dist(x, y)$ and similarly we have $\dist(y, A) - \dist(x, A) \leq \dist(x, y)$.
  This implies that $\abs{\dist(y, A) - \dist(x, A)} \leq \dist(x, y)$.

  Now let $\e > 0$.
  Suppose that $\dist(x, y) < \e$.
  Then we have that $\abs{d(x), d(y)} \leq \dist(x, y) < \e$.
\end{pf}


\clearpage

\section{Topological Spaces}

We usually only care about continuity, not the actual metrics.
Continuity can be formulated in terms of open sets.
It can be shown that a function is continuous if and only if $f^{-1}(U)$ is open for all open $U$ in the codomain.

\begin{defn}[Topological Spaces]
  A \emph{topological} space is a set $X$ with a collection of subsets of $X$ called ``open sets'' such that
  \begin{enumerate}
  \item the intersection of two open sets is open;
  \item the union of any collection of open sets is open; and
  \item $X, \emptyset$ are open.
  \end{enumerate}
  A subset $X \subseteq X$ is \emph{closed} if $X \setminus C$ is open.
\end{defn}

\begin{defn}[Continuous Functions]
  A function of topological spaces $f\colon X \to Y$ is \emph{continuous} if $f^{-1}(U)$ is open for all open $U \subseteq Y$.
  A \emph{map} is a continuous function.
\end{defn}

It isn't hard to see that a function $f\colon X \to Y$ is continuous if and only if $f^{-1}(C)$ is closed for all closed $C \subseteq Y$.

\begin{defn}[Neighborhood]
  If $X$ is a topological space, a set $N \subseteq X$ is a \emph{neighborhood} of $x \in X$ if it contains an open set $U \subseteq N$ such that $x \in U$.
\end{defn}

It is immediate that arbitrary unions and finite intersections of neighborhoods of a point $x \in X$ are still neighborhoods of $x$.

\begin{defn}[Neighborhood Basis]
  Let $X$ be a topological space and $x \in X$.
  A collection $\textbf{B}_{x}$ of subsets of $X$ containing $x$ is called a \emph{neighborhood basis} at $x$ in $X$ if each neighborhood of $x$ contains some element of $\textbf{B}_{x}$ and each element of $\textbf{B}_{x}$ is a neighborhood of $x$.
\end{defn}

Neighborhood bases let us define continuity at a single point.

\begin{defn}[Continuity at a Point]
  A function $f\colon X \to Y$ between topological spaces is said to be \emph{continuous at $x$}, $x \in X$, if, given any neighborhood $N$ of $f(x)$ in $Y$, there is a neighborhood $M$ of $x$ in $X$ such that $f(M) \subseteq N$.
\end{defn}

This is the same as saying $f^{-1}(N)$ is a neighborhood of $x$.

\begin{prop}
  A function $f\colon X \to Y$ between topological spaces is continuous if and only if it is continuous at each point $x \in X$.
\end{prop}
\begin{pf}
  Suppose $f$ is continuous and let $N$ be a neighborhood of $f(x)$ in $Y$ with open $U \subseteq N$ such that $f(x) \in U$.
  Then $x \in f^{-1}(U) \subseteq f^{-1}(N)$ where $f^{-1}(U)$ is open in $X$.
  Thus $f^{-1}(N)$ is a neighborhood of $x$ in $X$ and $f\pqty{f^{-1}(N)} = N \subseteq N$ and $f$ is continuous at $x \in X$.

  Conversely now suppose that $f$ is continuous at each point in $X$ and let $U \subseteq Y$ be open.
  Then for any $x \in f^{-1}(U),$ we have that $f^{-1}(U)$ is a neighborhood of $x$.
  Thus there exists open $V_{x} \subseteq f^{-1}(U)$ with $x \in V_{x}$.
  Thus $f^{-1}(U)$ is the union of open sets $V_{x}$ ranging over $x \in f^{-1}(U)$ and $f^{-1}(U)$ is open which yields that $f$ is continuous.
\end{pf}

\clearpage

\begin{defn}[Homeomorphic Functions]
  A function $f\colon X \to Y$ between topological spaces is called a \emph{homeomorphism} if $f^{-1}\colon Y \to X$ exists and both $f$ and $f^{-1}$ are continuous.
  We notate that $X \approx Y$ meaning that $X$ is \emph{homeomorphic} to $Y$.
\end{defn}

Topological spaces are homeomorphic then if there is a bijection between them as sets but also there is a correspondence between the open sets.
These sets can then essentially be regarded as the same sets.

Describing topological spaces can be described in a more simple manner than listing all the open sets using the concept of a basis.
\begin{defn}[Basis]
  Let $X$ be a topological space and $\textbf{B}$ a collection of subsets of $X$.
  We say that $\textbf{B}$ is a \emph{basis} for the topology of $X$ if the open sets of $X$ are precisely the unions of members of $\textbf{B}$.
  A collection $\textbf{S}$ of subsets of $X$ is called a subbasis for the topology of $X$ if the set $\textbf{B}$ of finite intersections of members of $\textbf{S}$ forms a basis of $X$.
\end{defn}

\emph{Any} collection $\textbf{S}$ of subsets of any set $X$ is a subbasis for some topology on $X$, namely the topology where the open sets are arbitrary unions of finite intersections of members of $\textbf{S}$.
The empty set is the union of an empty collection and $X$ is the intersection of an empty collection.
Thus to specify a topology, a subbasis suffices.
In a metric space, the collection of all $\e$-balls, for all $\e > 0$, forms a basis.
So is the collection of $\e$-balls for $\e = 1, \frac{1}{2}, \frac{1}{3}, \ldots$.

\begin{ex}[Examples of Topologies]
  We now give examples of topological spaces:
  \begin{enumerate}
  \item (Trivial Topology) Any set $X$ where the only open sets are $X$ and $\emptyset$.
  \item (Discrete Topology) Any set $X$ where every subset of $X$ is open.
  \item Any sets $X$ where the closed sets are finite sets and $X$ itself.
  \item $X = \N \cup \set{\N}$ with the open sets being all subsets of $\N$ together with complements of finite sets.
  \item Let $X$ be any poset.
        For $\alpha \in X$ consider the one-sided intervals $\set{\beta \in X | \alpha < \beta}$ and $\set{\beta \in X | \alpha > \beta}$.
        The ``order topology'' on $X$ is the topology generated by these intervals.
        The ``strong order topology'' is the topology generated by these intervals together with the complements of finite sets.
  \item Let $X = I^{2}$ where $I = [0, 1]$ the unit intervals.
        Give this the ``dictionary ordering'' where $(x, y) < (s, t)$ if and only if either $x < s$ or $(x = s \text{ and } y < t)$.
        Let $X$ have the order topology for this ordering.
  \item Let $X$ be the real line bu with the topology generated by the ``half open intervals'' $[x, y)$.
        This is called the ``half open interval topology'' or the ``lower limit topology.''
  \item Let $X = \Omega \cup \set{\Omega}$ be the set of ordinal numbers up to and including the least uncountable ordinal $\Omega$.
        Give this the order topology.
  \end{enumerate}
\end{ex}

\begin{defn}[Countability]
  A topological space is \emph{first countable} if each point has a countable neighborhood basis.
  A topological space is \emph{second countable} if its topology has a countable basis.
\end{defn}

Every metric space is first countable but some are not second countable.
Consider the space if any uncountable set with metric $\dist(x, y) = 1$ if $x \neq y$ and $0$ otherwise (which yields the discrete topology).
Euclidian spaces are second countable since the $\e$-balls, with $\e$ rational, about points with rational coordinates, is a countable basis.

\begin{defn}[Uniform Convergence of Functions]
  A sequence $f_{1}, f_{2}, \ldots$ of functions from a topological space $X$ to a metric space $Y$ is said to \emph{converge uniformly} to a function $f\colon X \to Y$ if, for all $\e > 0$, there is a number $n$ such that for all $i > n$, $\dist(f_{i}(x), f(x)) < \e$ for all $x \in X$.
\end{defn}

\begin{thrm}
  If a sequence $f_{1}, f_{2}, \ldots$ of continuous functions from a topological space $X$ to a metric space $Y$ converges uniformly to a function $f\colon X \to Y$, then $f$ is continuous.
\end{thrm}
\begin{pf}
  Given $\e > 0$, let $n_{0}$ be such that for all $x \in X$, $n \geq n_{0}$ implies that $\dist(f_{n}(x), f(x)) < \frac{\e}{3}$.
  Given $n_{0}$, continuity of $f_{n_{0}}$ implies that there is a neighborhood $N$ of $x_{0}$ such that $x \in N$ implies that $\dist(f_{n_{0}}(x), f_{n_{0}}(x_{0})) < \frac{\e}{3}$.
  Thus for any $x \in N$ we have that
  \begin{align*}
    \dist(f(x), f(x_{0})) &\leq \dist(f(x), f_{n_{0}}(x)) + \dist(f_{n_{0}}(x), f_{n_{0}}(x_{0})) + \dist(f_{n_{0}}(x), f(x_{0})) \\
                          &< \frac{\e}{3} + \frac{\e}{3} + \frac{\e}{3} = \e.
  \end{align*}
  Thus $f$ is continuous.
\end{pf}

\begin{defn}[Open and Closed Functions]
  A function $f\colon X \to Y$ between topological spaces is said to be \emph{open} if $f(U)$ is open in $Y$ for all open $U \subseteq X$.
  It is said to be \emph{closed} if $f(C)$ is closed in $Y$ for all closed $C \subseteq X$.
\end{defn}

\begin{defn}[Smallest and Largest Topologies]
  If $X$ is a set and some condition is given on subsets of $X$, which may or may not hold for any particular subset, then if there is a topology $T$ whose open sets satisfy the condition, and such that, for any topology $T'$ whose open sets satisfy the condition, then the $T$-open sets are also $T'$-open (i.e. $T \subseteq T'$), then $T$ is called the \emph{smallest} (or \emph{weakest} or \emph{coarsest}) topology satisfying the condition.
  If, instead, for any topology $T'$ whose open sets satisfy the condition, any $T'$-open sets are also $T$-open (i.e. $T' \subseteq T$), then $T$ is called the \emph{largest} (or \emph{strongest} or \emph{finest}) topology satisfying the condition.
\end{defn}

\begin{ex}[Largest and Smallest Topology for a Condition]
  If $f\colon X \to Y$ is a function and $X$ is a topological space, then there is a largest topology on $Y$ making $f$ continuous by having open sets $\set{V \subseteq Y | f^{-1}(V) \text{ is open in } X}$.
  The smallest such topology is the trivial topology.
\end{ex}

If a topology is the largest one satisfying some condition, then there is always some other condition where the given topology is the smallest one satisfying the new condition.
For example, the topology described in the prior example is the smallest topology satisfying the condition ``for all spaces $Z$ and functions $g\colon Y \to Z$, $g \circ f$ being continuous implies $g$ is continuous.''
Thus there is no way to argue that a topology is ``large'' or ``small'' without knowing the defining condition.

\clearpage

\begin{exercise}[Bredon 1.2.1]
  Show that in a topological space $X$:
  \begin{enumerate}
  \item[a.] the union of two closed sets is closed;
  \item[b.] the intersection of any collection of closed sets is closed; and
  \item[c.] the empty set and the whole space $X$ are closed.
  \end{enumerate}
\end{exercise}
\begin{pf}
  \begin{enumerate}
  \item[a.] Suppose that $C_{1}, C_{2}$ are closed in $X$.
        Then we have that $X \setminus (C_{1} \cup \C_{2}) = (X \setminus C_{1}) \cup (X \setminus C_{2})$ which is the union of two open sets.
        Thus $C_{1} \cup C_{2}$ is closed.
  \item[b.] \quest{Sim.}
  \item[c.] We have that $X \setminus \emptyset = X$ which is open so $\emptyset$ is closed.
        Similarly we have that $X \setminus X = \emptyset$ which is open so $X$ is closed.
  \end{enumerate}
\end{pf}

\begin{exercise}[Bredon 1.2.2]
  Consider the topology on $\R$ generated by half open intervals $[x, y)$ together with those of the form $(x, y]$.
  Show that this coincides with the discrete topology.
\end{exercise}
\begin{pf}
  Note that $(x - 1, x] \cup [x, x + 1) = \set{x}$.
  Thus every singleton, and therefore every subset of $\R$, is open and we recover the discrete topology.
\end{pf}

\begin{exercise}[Bredon 1.2.3]
  Show that the space $\Omega \cup \set{\Omega}$ in the order topology cannot be given a metric consistent with its topology.
\end{exercise}
\begin{pf}
  \quest{TODO: What is $\Omega$?}
\end{pf}

\begin{exercise}[Bredon 1.2.4]
  If $f\colon X \to Y$ is a function between topological space, and $f^{-1}(U)$ is open for each open $U$ in some subbasis for the topology of $Y$, show that $f$ is continuous.
\end{exercise}
\begin{pf}
  Let \textbf{S} be a subbasis for $Y$ and let $U$ be open in $Y$.
  We have that \textbf{S} generates some basis \textbf{B} for $Y$ and for some indexing set $I$ we have $U = \bigcup_{i \in I} B_{i}$ where $B_{i} \in \textbf{B}$.
  But then each $B_{i} = S_{i_{1}} \cap S_{i_{n}}$ for $S_{i_{j}} \in \textbf{S}$.
  We have that
  \begin{align*}
    f^{-1}(B_{i}) &= \bigcap_{j = 1}^{n} f^{-1}(S_{i_{j}}) \text{ where each } f^{-1}(S_{i_{j}}) \text{ is open so } f^{-1}(B_{i}) \text{ is open }; \text{ and} \\
    f^{-1}(U) &= \bigcup_{i \in I} f^{-1}(B_{i}) \text{ where each } f^{-1}(B_{i}) \text{ is open so } f^{-1}(U) \text{ is open }.
  \end{align*}
  Thus $f$ is continuous.
\end{pf}

\clearpage

\begin{exercise}[Bredon 1.2.5]
  Suppose that $S$ is a set and we are given, for each $x \in S$, a collection $\textbf{N}(x)$ of subsets of $S$ satisfying:
  \begin{enumerate}
  \item $N \in \textbf{N}(x) \implies x \in N$;
  \item $N, M \in \textbf{N}(x) \implies \exists P \in \textbf{N}(x) \text{ such that } P \subseteq N \cap M$; and
  \item $x \in S \implies \textbf{N}(x) \neq \emptyset$.
  \end{enumerate}
  Then show that there is a unique topology on $S$ such that $\textbf{N}(x)$ is a neighborhood basis at $x$, for each $x \in S$.
  Thus a topology can be defined by giving such a collection of neighborhoods at each point.
\end{exercise}
\begin{pf}
  Let $S$ be the topology with open sets $\set{U \subseteq S | \forall x \in U,~ \exists N \in \textbf{N}(x) \text{ such that } N \subseteq U}$.
  \begin{itemize}
  \item $\emptyset$ is open in $S$ vacuously.
        Now take $x \in S$.
        $\textbf{N}(x) \neq \emptyset$ and so there exists $N \subseteq S$ such that $x \in N$.
        Thus $S$ is also open.
  \item Let $U, V$ be open.
        Then let $x \in U \cap V$.
        $x \in U$ implies there exists $N \in \textbf{N}(x)$ such that $N \subseteq U$.
        Similarly there exists $M \in \textbf{N}(x)$ such that $M \subseteq V$.
        We have then there exists $P \in \textbf{N}(x)$ such that $P \subseteq N \cap M$.
        Thus $U \cap V$ is open.
  \item \quest{TODO: Union}
  \end{itemize}

  Suppose that $Y$ is some other topology such that for each $x \in S$. $\textbf{N}(x)$ is a neighborhood basis of $x$.
  Let $U$ be open in $Y$ and $x \in U$.
  Then there must be $N \in \textbf{N}(x)$ with $N \subseteq U$ because $U$ is an open set containing $x$ and $\textbf{N}(x)$ is a neighborhood basis.
  Thus $U$ is also open in $S$.

  Now suppose that $U$ is open in $S$.
  Then for $x \in  U$ there exists $N \in \textbf{N}(x)$ such that $N \subseteq U$.
  $N$ is a neighborhood of $x$ i $Y$ also.
  Thus there exists open $V_{x} \subseteq N$ such that $x \in V_{x}$.
  Thus since $V_{x} \subseteq N \subseteq U$ we have that $U = \bigcup_{x \in U} V_{x}$.
  Thus $U$ is the union of open sets in $Y$ and $U$ is open in $Y$.

  Overall $S = Y$ and we have that $S$ is the unique topology we want.
\end{pf}

\clearpage

\section{Subspaces}

\begin{defn}[Subspace]
  If $X$ is a topological space and $A \subseteq X$, then the \emph{relative topology} or \emph{subspace topology} on $A$ is the collection of intersections of $A$ with open sets of $X$.
  With this topology, $A$ is called a \emph{subspace} of $X$.
\end{defn}

We have a series of basic consequences of this definition of subspace.

\begin{prop}
  If $Y$ is a subspace of $A$, then $A \subseteq Y$ is closed if and only if $A = Y \cap B$ for some closed $B \subseteq X$
\end{prop}
\begin{pf}
  Suppose $A$ is closed in $Y$.
  Then $Y \setminus A$ is open in $Y$.
  Then $Y \setminus A = Y \cap U$ for some open $U \subseteq X$.
  $X \setminus U$ is closed and $(Y \setminus A) \sqcup A = Y = Y \cap U \sqcup (Y \cap (X \setminus U))$.
  This implies $A = Y \cap (X \setminus U)$.

  Now suppose that $A = Y \cap B$ for some closed $B \subseteq X$.
  Then $X \setminus B$ is open in $X$.
  We have that $Y \setminus A = Y \setminus (Y \cap B) = Y \cap (X \setminus B)$ and so $Y \cap A$ is open meaning that $A$ is closed.
\end{pf}

\begin{prop}
  If $X$ is a topological space and $A \subseteq X$, then there exists a largest open set $U$ with $U \subseteq A$.
\end{prop}
\begin{pf}
  Let $\set{U_{i}}_{i \in I}$ be an indexed set of all open sets $\subseteq A$.
  Then $U \defeq \bigcup_{i \in I} U_{i}$ is the largest open set in $A$.
  This is because the union of open sets is open and if $O \subseteq A$ is open then $O = U_{j}$ for some $j$ and thus $O \subseteq U$.
\end{pf}
\begin{defn}[Interior]
  let $X$ be a topological space and $A \subseteq X$.
  The largest open set contained in $A$ is called the \emph{interior} of $A$ in $X$ and is denoted by $\inte(A)$ or $A^{\circ}$.
\end{defn}

\begin{prop}
  If $X$ is a toplogical space and $A \subseteq X$, then there exists a smallest closed set $F$ such that $A \subseteq F \subseteq X$.
\end{prop}
\begin{pf}
  Let $\set{F_{i}}_{i \in I}$ be the indexed set of all closed sets $A \subseteq F_{i} \subseteq X$.
  Then $F \defeq \bigcup_{i \in I} F_{i}$ is the closure of $A$.
  $F$ is closed since the intersection of closed sets is closed and $A \subseteq F$ since $A \subseteq F_{i}$ for all $i \in I$.
  If $A \subseteq C \subseteq X$ is closed then $C = F_{j}$ for some $j$ and thus $F \subseteq C$.
\end{pf}
\begin{defn}[Closure]
  let $X$ be a topological space and $A \subseteq X$.
  The smallest closed set containing $A$ is called the \emph{closure} of $A$ in $X$ and is denoted by $\cl(A)$ or $\overline{A}$.
\end{defn}

We give more precise definitions of the interior and closure of a set.

\begin{prop}[Bredon 1.3.1.a]\label{prop: defn_of_int_and_cl}
  Let $X$ be a topological space and $A \subseteq X$.
  Then we have that
  \begin{align*}
    \inte(A) &= \set{a \in A | \exists U \text{ open such that } a \in U \subseteq X} \\
    \overline{A} &= \set{x \in X | \forall U \text{ open such that } x \in U,~ U \cap A \neq \emptyset}
  \end{align*}
\end{prop}
\begin{pf}
  \quest{Immediate from defn.}
\end{pf}

\begin{prop}[Bredon 1.3.1.b]
  Let $X$ be a topological space and $A \subseteq X$.
  Then $A$ is open if and only if $\inte(A) = A$ and $A$ is closed if and only if $\overline{A} = A$.
\end{prop}
\begin{pf}
  \quest{Immediate from \Cref{prop: defn_of_int_and_cl}}.
\end{pf}

\clearpage

\begin{prop}[Bredon 1.3.1.c]
  Let $X$ be a topological space and $A \subseteq X$.
  Then $X \setminus \inte(A) = \overline{X \setminus A}$ and $X \setminus \overline{A} = \inte(X \setminus A)$.
\end{prop}
\begin{pf}
  Let $x \in X \setminus \inte(A)$.
  Then for any open $U$ such that $x \in U$, we have that $U \not\subseteq A$ and so $U \cap (X \setminus A) \neq \emptyset$.
  Thus by \Cref{prop: defn_of_int_and_cl}, $x \in \overline{X \setminus A}$.
  Conversely let $x \in \overline{X \setminus A}$.
  Then for any open set $U$ with $x \in U$, we have $U \cap (X \setminus A) \neq \emptyset$.
  Thus $U \not\subseteq A$ and $x \in X \setminus \inte(A)$.

  Now let $x \in X \setminus \overline{A}$.
  Then by \Cref{prop: defn_of_int_and_cl} there exists an open set $U$ such that $x \in U$ and $U \cap A = \emptyset$.
  Thus $U \subseteq X \setminus A$ and $x \in \inte(X \setminus A)$.
  \quest{TODO: Reverse Inclusion}
\end{pf}

\begin{prop}[Bredon 1.3.1.d]
  Let $X$ be a topological space and $A, B \subseteq X$.
  Then $\inte(A \cap B) = \inte(A) \cap \inte(B)$ and $\overline{A \cup B} = \overline{A} \cup \overline{B}$.
\end{prop}
\begin{pf}
  Suppose that $x \in \inte(A \cap B)$.
  Then there exists open $U$ such that $x \in U \subseteq A \cap B$.
  Thus $U \subseteq A$ and $U \subseteq B$ and $x \inte(A) \cap \inte(B)$.
  Now suppose $x \in \inte(A) \cap \inte(B)$. Then there exists open sets $U_{1}$ and $U_{2}$ such that $x \in U_{1} \subseteq A$ and $x \in U_{B} \subseteq B$.
  Thus $x \in U_{1} \cap U_{2} \subseteq A \cap B$ which implies that $x \in \inte(A \cap B)$.

  Now suppose that $x \in \overline{A \cup B}$.
  Then for all open $U$ with $x \in U$, $U \cap (A \cup B) \neq \emptyset$.
  Thus for all open $U$ with $x \in U$, $U \cap A$ or $U \cap B$ is nonempty and $x \in \overline{A} \cup \overline{B}$.
  Now suppose that $x \in \overline{A} \cup \overline{B}$.
  Then for all open $U$ with $x \in U$, $U \cap A$ or $U \cap B$  is nonempty.
  Thus $U \cap (A \cup B) \neq \emptyset$ and $x \in \overline{A \cup B}$.
\end{pf}

\begin{prop}[Bredon 1.3.1.e]
  Let $X$ be a topological space and $\set{A_{\alpha}}$ an indexed collection of subsets of $X$.
  \begin{enumerate}
  \item $\Bigcap_{\alpha} \inte(A_{\alpha}) \containseq \inte\pqty{\Bigcap_{\alpha} A_{\alpha}} = \inte\pqty{\Bigcap_{\alpha} \inte(A_{\alpha})}$.
  \item $\Bigcup_{\alpha} \overline{A_{\alpha}} \subseteq \overline{\Bigcup_{\alpha} A_{\alpha}} = \overline{\Bigcap_{\alpha} \overline{A_{\alpha}}}$.
  \item $\Bigcup_{\alpha} \inte(A_{\alpha}) \subseteq \inte\pqty{\Bigcup_{\alpha} A_{\alpha}}$.
  \item $\Bigcap_{\alpha} \overline{A_{\alpha}} \containseq \overline{\Bigcap_{\alpha} A_{\alpha}}$.
  \end{enumerate}
\end{prop}
\begin{pf}\
  \begin{enumerate}
  \item In the finite case, by induction, we have that $\Bigcap_{\alpha} \inte(A_{\alpha}) \subseteq \inte\pqty{\Bigcap_{\alpha} A_{\alpha}}$.
        \quest{The proof does not invoke finiteness / infiniteness and so this holds in the general case.}
        We actually have equality in the case of finite intersection.

        Note that $\inte\pqty{\bigcap_{\alpha} A_{\alpha}}$ is open and $\inte\pqty{\bigcap_{\alpha} A_{\alpha}} \subseteq \bigcap_{\alpha} \inte(A_{\alpha})$ and so $\inte\pqty{\bigcap_{\alpha} A_{\alpha}} \subseteq \inte\pqty{\bigcap_{\alpha} \inte(A_{\alpha})}$.
        Now let $x \in \inte\pqty{\bigcap_{\alpha} \inte(A_{\alpha})}$.
        Then there is some open $U$ with $x \in U$ such that $U \subseteq \bigcap_{\alpha} \inte(A_{\alpha})$.
        Thus $U \subseteq \inte(A_{\alpha}) \subseteq A_{\alpha}$ for all $\alpha$ and so $U \subseteq \bigcap_{\alpha} A_{\alpha}$ and $x \in \inte\pqty{\bigcap_{\alpha} A_{\alpha}}$.
  \item Let $x \in \bigcup_{\alpha} \overline{A_{\alpha}}$.
        Then for some $\alpha$, $x \in \overline{A_{\alpha}}$.
        So for all open $U$ with $x \in U$, $U \cap A_{\alpha} \neq \emptyset$.
        Thus for all open $U$ with $x \in U$, $U \cap \pqty{\bigcup_{\alpha} A_{\alpha}} \neq \emptyset$ and $x \in \overline{\bigcup_{\alpha} A_{\alpha}}$.

        $A_{\alpha} \subseteq \overline{A_{\alpha}}$ for all $\alpha$.
        So $\overline{\bigcup_{\alpha} A_{\alpha}} \subseteq \overline{\bigcup_{\alpha} \overline{A_{\alpha}}}$.
        Now let $x \in \overline{\bigcup_{\alpha} \overline{A_{\alpha}}}$.
        Then for all open $U$ with $x \in U$, $U \cap \bigcup_{\alpha} A_{\alpha} \neq \emptyset$.
        Since $\bigcup_{\alpha} \overline{A_{\alpha}} \subseteq \overline{\bigcup_{\alpha} A_{\alpha}}$, we have that $U \cap \overline{\bigcup_{\alpha} A_{\alpha}} \neq \emptyset$.
        Thus $x \in \overline{\bigcup_{\alpha}}$.
  \item Suppose that $x \in \bigcup_{\alpha} \inte(A_{\alpha})$.
        Then there exists $\alpha$ such that $x \in \inte(A_{\alpha})$.
        Thus there exists open $U$ with $x \in U$ such that $U \subseteq A_{\alpha}$.
        So $U \subseteq \bigcup_{\alpha} A_{\alpha}$ and overall $x \in \inte\pqty{\bigcup_{\alpha} A_{\alpha}}$.
  \item Let $x \in \overline{\bigcap_{\alpha} A_{\alpha}}$.
        Then for all open $U$ with $x \in U$, $U \cap \bigcap_{\alpha} A_{\alpha} \neq \emptyset$.
        Thus for all $\alpha$, $U \cap A_{\alpha} \neq \emptyset$ and $x \in \overline{A_{\alpha}}$ for all $\alpha$.
        Thus $x \in \bigcap_{\alpha} \overline{A_{\alpha}}$.
  \end{enumerate}
\end{pf}

\begin{ex}[Interior of Intersection is not Equal to Intersection of Interiors]
  We have that $\bigcap_{\alpha} \inte(A_{\alpha}) \containseq \inte\pqty{\bigcap_{\alpha} A_{\alpha}}$.
  We also know that we have equality in the case of finite intersections.
  However in general we have that $\bigcap_{\alpha} \inte(A_{\alpha}) \not\subseteq \inte\pqty{ \bigcap_{\alpha} A_{\alpha}}$.
  Let $A_{n} \defeq \pqty{1 - \frac{1}{n}, 1 + \frac{1}{n}} \subseteq \R$ for all $n \geq 1$.
  Then
  \begin{align*}
    \Bigcap_{n \geq 1} \inte(A_{n}) &= \Bigcap_{n \geq 1} \pqty{1 - \frac{1}{n}, 1 + \frac{1}{n}} = \set{1}; \text{ and } \\
    \inte\pqty{\Bigcap_{n \geq 1} A_{n}} &= \inte \set{1} = \emptyset.
  \end{align*}
\end{ex}

\begin{ex}[Union of Closures is not Equal to Closure of Union]
  We have that $\bigcup_{\alpha} \overline{A_{\alpha}} \subseteq \overline{\bigcup_{\alpha} A_{\alpha}}$.
  However in general we have that  $\overline{\bigcup_{\alpha} A_{\alpha}} \not\subseteq \bigcup_{\alpha} \overline{A_{\alpha}}$.
  Let $A_{n} = \left( \frac{1}{n}, 1 \right]$ for $n \geq 1$.
  Then
  \begin{align*}
    \Bigcup_{n \geq 1} \overline{\left( \frac{1}{n}, 1 \right]} &= \Bigcup_{n \geq 1} \bqty{\frac{1}{n}, 1} =  (0, 1] \\
    \overline{\Bigcup_{n \geq 1} \left( \frac{1}{n}, 1 \right]} &= \overline{(0, 1]} = [0, 1]
  \end{align*}
\end{ex}

\begin{ex}[Union of Interiors is not Equal to Interior of Union]
  We have that $\bigcup_{\alpha} \inte(A_{\alpha}) \subseteq \inte\pqty{\bigcup_{\alpha} A_{\alpha}}$.
  However in general we have that $\inte\pqty{\bigcup_{\alpha} A_{\alpha}} \not\subseteq \bigcup_{\alpha} \inte(A_{\alpha})$.
  Let $A_{1} = \bqty{0, \frac{1}{2}}$ and let $A_{2} = \bqty{\frac{1}{2}, 1}$.
  Then $\inte(A_{1} \cup A_{2}) = \inte([0, 1]) = (0, 1)$.
  However $\inte(A_{1}) \cup \inte(A_{2}) = \pqty{0, \frac{1}{2}} \cup \pqty{\frac{1}{2}, 1}$.
\end{ex}

\begin{ex}[Intersection of Closures is not Equal to Closure of Intersection]
  We have that $\overline{\bigcap_{\alpha} A_{\alpha}} \subseteq \bigcap_{\alpha} \overline{A_{\alpha}}$.
  However in general we have that $\bigcap_{\alpha} \overline{A_{\alpha}} \not\subseteq \overline{\bigcap_{\alpha} A_{\alpha}}$.
  Let $A_{1} = \pqty{0, \frac{1}{2}}$ and $A_{2} = \pqty{\frac{1}{2}, 1}$.
  Then $\overline{A_{1} \cap A_{2}} = \overline{\emptyset} = \emptyset$.
  However $\overline{A_{1}} \cap \overline{A_{2}} = \bqty{0, \frac{1}{2}} \cap \bqty{\frac{1}{2}, 1} = \set{\frac{1}{2}}$.
\end{ex}

\begin{prop}[Bredon 1.3.1.f]
  Let $X$ be a topological space and $A, B$ subsets of $X$.
  If $A \subset B$, then we have that $\overline{A} \subseteq \overline{B}$ and $\inte(A) \subseteq \inte(B)$.
\end{prop}
\begin{pf}
  Suppose $A \subseteq B$ and $x \in \overline{A}$.
  Then for all open sets $U$ with $x \in U$ we have $U \cap A \neq \emptyset$.
  Thus $U \cap B \neq \emptyset$ and $x \in \overline{B}$.

  Now let $x \in \inte(A)$.
  Then there exists open $U$ such that $x \in U \subseteq A \subseteq B$.
  Thus $x \in \inte(B)$.
\end{pf}

We may specify the space in which a closure is taken with the notation $\overline{A}^{X}$.
However, we do not need this notation very often.

\begin{prop}
  If $A \subseteq Y \subseteq X$, then $\overline{A}^{Y} = \overline{A}^{X} \cap Y$.
  Thus if $Y$ is closed, then $\overline{A}^{Y} = \overline{A}^{X}$.
\end{prop}
\begin{pf}
  Suppose $a \in \overline{A}^{Y}$.
  Then $a \in X$ since $\overline{A} \subseteq Y \subseteq X$.
  Thus $A \subseteq \overline{A} \subseteq X$ and since $a \in Y$ also, we have that $a \in \overline{A}^{X} \cap Y$.
  The reverse inclusion is also a similar argument and thus $\overline{A}^{Y} = \overline{A}^{X} \cap Y$.
  If $Y$ is closed, then $\overline{A} \subseteq Y$ and so $\overline{A}^{X} \cap Y = \overline{A}^{X}$.
\end{pf}

\clearpage

\begin{prop}
  If $Y \subseteq X$ then the set of intersection of $Y$ with a basis of $X$ is a basis of the relative topology of $X$.
\end{prop}
\begin{pf}
  \quest{High lvl: Take open $\subseteq Y$. This is equal to $Y \cap \text{ open } \subseteq X$. Use basis of $X$, distribute intersection}.
\end{pf}

\begin{prop}
  If $X, Y, Z$ are topological spaces and $Z$ is a subspace of $Y$, and $Y$ is a subspace of $X$, then $Z$ is a subspace of $X$.
\end{prop}
\begin{pf}
  Let $U \subseteq Z$ be open.
  Then $U = Z \cap U'$ for some open $Z' \subseteq Y$.
  But $U' = Y \cap U''$ for some open $U'' \subseteq X$.
  Note that $Z \subseteq Y$ so $Z \cap Y = Z$.
  Thus $U = Z \cap U' = Z \cap Y \cap U'' = Z \cap U''$ and overall $Z$ is a subspace of $X$.
\end{pf}

\begin{prop}
  If $X$ is a metric space and $A \subseteq X$, then $\overline{A}$ coincides with the set of limits in $X$ of sequences of points in $A$.
\end{prop}
\begin{pf}
  If $x$ is the limit point of a sequence of points in $A$, then any open sets about $x$ contains a point of $A$.
  Thus $x \notin \inte(X \setminus A)$.
  We have that $X \setminus \inte(X \setminus A) = \overline{A}$ and so $x \in \overline{A}$.

  Now suppose $x \in \overline{A}$.
  $B_{1 / n}(x)$ must contain a point in $A$.
  If it didn't then $x \in \int(X \setminus A)$ which is disjoint from $\overline{A}$.
  Let $x_{n}$ be a point in $A$ which is also in $B_{a / n}(x)$.
  Thus $x$ is a limit of a sequence of points in $A$.
\end{pf}

\begin{defn}[Boundary]
  If $X$ is a topological space and $A \subseteq X$, then the \emph{boundary} or \emph{frontier} of $A$ is defined to be $\partial(A)$ or $\bdry(A) = \overline{A} \cap \overline{X \setminus A}$.
\end{defn}

\begin{defn}[Dense and Nowhere Dense Sets]
  A subset $A$ of a topological space $X$ is called \emph{dense} in $X$ if $\overline{A} = X$.
  A subset $A$ is said to be \emph{nowhere dense} in $X$ if $\inte(\overline{A}) = \emptyset$.
\end{defn}

\clearpage

\begin{exercise}[Bredon 1.3.2]
  For $A \subseteq X$, we have that $X = \inte(A) \sqcup \partial(A) \sqcup X \setminus \overline{A}$.
\end{exercise}
\begin{pf}
  Recall that $X \setminus \overline{A} = \inte(X \setminus A)$ and $\partial A \defeq \overline{A} \cap \overline{X \setminus A}$.
  Let $x \in \inte(A)$.
  Then there exists open $N \subseteq A$ such that $x \in N$.
  Thus $N \cap (X \setminus A) = \emptyset$ and $x \notin \overline{X \setminus A}$.
  So $x \notin \partial A$.
  Now suppose that $x \in \partial A = \overline{A} \cap \overline{X \setminus A}$.
  Then for every open set $U$ such that $x \in U$, we have $U \cap (X \setminus A) \neq \emptyset$ and so in particular $U \not\subseteq A$.
  Thus overall $\inte(A)$ and $\partial A$ are disjoint.
  We automatically have that $\partial A$ and $X \setminus \overline{A}$ are disjoint since $x \in \partial A \implies x \in \overline{A}$.

  Now let $x \in X$ such that $x \notin \inte(A) \sqcup X \setminus \overline{A}$.
  $x \notin \inte(A)$ so for all open $U$ such that $x \in U$ we have $U \not\subseteq A$.
  So $U \cap (X \setminus A) \neq \emptyset$ for all open $U$ containing $x$.
  Thus $x \in \overline{X \setminus A}$.
  Also $x \notin X \setminus \overline{A}$ so $x \in \overline{A}$.
  Thus $x \in \partial A$ and overall $X = \inte(A) \sqcup \partial(A) \sqcup X \setminus \overline{A}$.
\end{pf}

\begin{exercise}[Bredon 1.3.3]
  Show that a metric space $X$ is second countable if and only if it has a countable dense set.
\end{exercise}
\begin{pf}
\end{pf}


\begin{defn}[Separable Space]
  A topological space is \emph{separable} if it has a countable dense set.
\end{defn}

\clearpage

\section{Connectivity and Components}

Intuitively, a connected space is a space where you can move from one space to another with no jumps.
Another intuition is that the space doesn't have two or more separated pieces.

\begin{defn}[Connected Sets and Separation]
  A topological space $X$ is \emph{connected} if it is not the disjoint union of two nonempty open subsets.
  If $A, B$ are two disjoint nonempty open subsets of $X$ such that $A \sqcup B = X$ then we say that $A$ and $B$ form a \emph{separation} of $X$.
\end{defn}

\begin{defn}[Clopen Sets]
  A subset $A$ of a topological space $X$ is \emph{clopen} if it is both open and closed in $X$.
\end{defn}

\begin{prop}
  A topological space $X$ is connected if and only if the open clopen sets are $X$ and $\emptyset$.
\end{prop}
\begin{prop}
  Suppose $X$ is connected and suppose there exists clopen $\emptyset \subsetneq U \subsetneq X$.
  Then $X \setminus U$ is also clopen and $X = (X \setminus U) \sqcup U$ forms a separation of $X$, contradiction.
\end{prop}

\begin{defn}[Discrete Valued Maps (DVM)]
  A \emph{discrete valued map (DVM)} is a map from a topological space $X$ to a discrete space $D$.
\end{defn}

\begin{prop}
  A topological space $X$ is connected if and only if every discrete valued map on $X$ is constant.
\end{prop}
\begin{pf}
  If $X$ is connected and $d\colon X \to D$ is a DVM and $y$ is in the range of $d$, then $d^{-1}(y)$ is clopen in and non-empty.
  Thus $d^{-1}(y) = X$ and $d$ is constant.

  Now suppose that $X$ is not connected and $U \sqcup V$ form a separation of $X$.
  Then $d\colon X \to \set{0, 1}$ where $d(x) = 0$ if and only if $x \in U$ forms a nonconstant DVM.\@
\end{pf}

\begin{prop}
  If $f\colon X \to Y$ is continuous and $X$ is connected, then $f(X)$ is connected.
\end{prop}
\begin{pf}
  Let $d\colon f(X) \to D$ be a DVM.\@
  Then $d \circ f$ is a DVM on $X$ and thus constant.
  This implies that $d$ is constant and thus $f(X)$ is connected.
\end{pf}

\begin{prop}
  If $\set{Y_{i}}_{i \in I}$ is a collection of connected sets in a topological space $X$ such that no two $Y_{i}$ are disjoint, then $\bigcup_{i \in I} Y_{i}$ is connected.
\end{prop}
\begin{pf}
  Let $d\colon \bigcup_{i \in I} Y_{i} \to D$ be a DVM.\@
  Let $p, q \in \bigcup_{i \in I} Y_{i}$ such that $p \in Y_{i}$, $q \in Y_{j}$ and $v \in Y_{i} \cap Y_{j}$.
  Then $d(p) = d(v) = d(q)$ and $d$ is constant which yields that $\bigcup_{i \in I} Y_{i}$ is connected.
\end{pf}

\begin{cor}\label{cor: connected_subset_eq}
  The relation ``$p \sim q$ if and only if $p$ and $q$ belong to a connected subset of $X$'' is an equivalence relation.
\end{cor}
\begin{pf}
  Immediate.
\end{pf}

\begin{defn}[Components]
  The equivalence classes of the relation stated in \Cref{cor: connected_subset_eq} are called the \emph{components} of $X$.
  These are the ``maximal'' connected subsets of $X$.
\end{defn}


\begin{lem}\label{lem: closure_of_connected_is_connected}
  Let $X$ be a connected set.
  Then $\overline{X}$ is connected.
\end{lem}
\begin{pf}
  Let $d\colon \overline{X} \to \set{0, 1}$ be a DVM.\@
  $X$ is connected so $f(X) =$, without loss of generality, $\set{0}$.
  Then $\set{0}$ is closed so $d^{-1}(0)$ is closed and contains $X$.
  Thus $\overline{X} \subseteq d^{-1}(0) \subseteq \overline{X}$ which means that $f$ is constant.
\end{pf}

\begin{prop}
  Components of a topological space $X$ are connected and closed.
  Each connected subset of $X$ is contained in a component.
  Components are equal or disjoint and their union is $X$.
\end{prop}
\begin{pf}
  The last sentence is immediate based off of \Cref{cor: connected_subset_eq}.
  By definition, the component of $X$ containing a point $p$ is the union of all connected sets containing $p$, which itself is connected.
  This implies that connected sets lie in components.
  We have by \Cref{lem: closure_of_connected_is_connected} that since a component $C$ is connected $\overline{C}$ is connected and since $C \subseteq \overline{C}$ we overall have that $C = \overline{C}$ which means that $C$ is closed.
\end{pf}

\begin{prop}\label{prop: dvm_eq_rel}
  The relation ``$p \sim q$ if and only if $d(p) = d(q)$ for every discrete valued map $d$ on $X$'' is an equivalence relation on $X$.
\end{prop}
\begin{pf}
  Immediate.
\end{pf}

\begin{defn}[Quasi-components]
  The equivalence classes of the relation stated in \Cref{prop: dvm_eq_rel} are called the \emph{quasi-components} of $X$.
\end{defn}

\begin{prop}
  Quasi-components of a space $X$ are closed.
  Each connected set is contained in a quasi-component and in particular each component is contained in a quasi-component.
  Quasi-component are either equal or disjoint and their union is $X$.
\end{prop}
\begin{pf}
  The last statement is immediate based off of \Cref{prop: dvm_eq_rel}.
  If $p \in X$, then the quasi-component is $\set{q \in X | d(q) = d(p) \text{ for all DVM's } d \text{ on } X}$.
  But this is $\bigcap \Set{d^{-1}(d(p)) | d \text{DVM on } X}$.
  We have that $d^{-1}(d(p))$ is closed since $d$ is continuous.
  The intersection of closed sets is closed, so quasi-components are closed.
  Components are constant on every DVM, so they are contained in some quasi-component.
\end{pf}

\clearpage

\section{Separation Axioms}

\begin{defn}[Separation Axioms, Hausdorff, Regular, Normal Spaces]
  The \emph{Separation Axioms}:
  \begin{itemize}
  \item[(T$_{0}$)] A topological space $X$ is called a \emph{T$_{0}$-space} if for any two points $x \neq y$ there is an open set containing one of them but not the other.
                   This says that point can be distinguished by the open sets where they lie.
  \item[(T$_{1}$)] A topological space $X$ is called a \emph{T$_{1}$-space} if for any two points $x \neq y$ there is an open set containing $x$ but not $y$ and another open set containing $y$ but not $x$.
        This says that singletons, and thus finite sets, are closed.
        Let $x \in X$.
        For each point $y \neq x$ let $U_{y}$ be an open set containing $y$ but not $x$.
        Then $X \setminus \set{x} = \bigcup_{y} U_{y}$ which is open and thus $\set{x}$ is closed.
        Conversely if $\set{x}$ is closed, then $X \setminus \set{x}$ is open and contains the other point.
  \item[(T$_{2}$)] A topological space $X$ is called a \emph{T$_{2}$-space} or \emph{Hausdorff} if for any two points $x \neq y$ there are disjoint open sets $U$, $V$ with $x \in U$ and $y \in V$.
        This is the most useful type of space.
        It essential means that ``limits'' are unique.
  \item[(T$_{3}$)] A T$_{1}$-space $X$ is called a \emph{T$_{4}$-space} of \emph{regular} if for any point $x$ and closed set $F$ not containing $x$ there are disjoint open sets $U$, $V$ with $x \in U$ and $F \subseteq V$.
  \item[(T$_{4}$)] A T$_{1}$-space $X$ is called a \emph{T$_{5}$-space} of \emph{normal} if for any two disjoint closed sets $F$, $G$ there are disjoint open sets $U$, $V$ with $F \subseteq U$ and $G \subseteq V$.
  \end{itemize}
\end{defn}

\begin{prop}
  A Hausdorff space $X$ is regular if and only if the closed neighborhoods of any point form a neighborhood basis of the point.
\end{prop}
\begin{pf}
  Suppose $X$ is regular.
  Let $x \in V$ for $V$ open and let $C \defeq X \setminus V$.
  Then by regularity there exist open sets $U$ and $W$ with $x \in U$, $C \subseteq W$, and $U \cap W = \emptyset$.
  Then $X \setminus W$ is closed and $X \setminus W \subseteq X \setminus C = V$ and so any neighborhood of $V$ of $x$ contains a closed neighborhood of $x$.

  Now suppose that every point has a closed neighborhood basis.
  Let $x \in C$ with $C$ closed and $V = X \setminus C$.
  Then there exists open $U$ with $\overline{U} \subseteq V = X \setminus C$ and $x \in U$.
  Then $C \subseteq X \setminus \overline{U}$ and $U \cap (X \setminus \overline{U}) = \emptyset$.
  Thus $X$ is regular.
\end{pf}

\begin{cor}
  A subspace of a regular space $X$ is regular.
\end{cor}
\begin{pf}
  If $A \subseteq X$ is a subspace, just intersect a closed neighborhood basis in $X$ of some $a \in A$ with $A$ to obtain a closed neighborhood basis of $a$ in A.
\end{pf}

\clearpage

\section{Nets (Moore-Smith Convergence)}

Many results in metric spaces are stated in terms of sequences.
We discuss a generalization of sequences called \emph{nets}.

\begin{defn}[Directed Sets]
  A \emph{directed set} $D$ is a poset such that for all $\alpha, \beta \in D$, there exists $\tau \in D$ such that $\tau \geq \alpha$ and $\tau \geq \beta$.
\end{defn}

\begin{defn}[Net]
  A \emph{net} in a topological space $X$ is a directed set $D$ along with a function $\phi\colon D \to X$.
\end{defn}

\begin{ex}[Sequences are Nets]
  Note that $\N$ with the usual ordering is a directed set.
  Sequences are nets with $\N$ as the directed set.
\end{ex}

\begin{defn}[Frequently, Eventually]
  If $\phi\colon D \to X$ is a net in a topological space $X$ and $A \subseteq X$, we say that $\phi$ is \emph{frequently} in $A$ if for all $\alpha \in D$ there exists $\beta \geq \alpha$ such that $\phi(\beta) \in A$.
  We say that $\phi$ is \emph{eventually} in $A$ if there exists $\alpha \in D$ such that for all $\beta \geq \alpha$ we have that $\phi(\beta) \in A$.
\end{defn}

\begin{defn}[Convergence of a Net]
  A net $\phi\colon D \to X$ in a topological space $X$ is said to \emph{converge to $x \in X$} if for any neighborhood $U$ of $x$, $\phi$ is eventually in $U$.
\end{defn}

\begin{prop}
  A topological space $X$ is Hausdorff if and only if any two limits of a convergent net are equal.
  Thus it makes sense to speak of the limit of a convergent net.
\end{prop}
\begin{pf}
  Suppose that $X$ is Hausdorff.
  If a net $\phi$ is eventually in two sets $U$ and $V$, then it is eventually in $U \cap V$.
  Also this means that $U \cap V \neq \emptyset$.
  Thus the forward direction is immediate.

  Now suppose that $X$ is not Hausdorff and that $x \neq y \in X$ are two points which cannot be separated by open sets.
  Consider a directed whose elements are pairs of open sets $(U, V)$ with $x \in U$, $y \in V$.
  We give this directed set the ordering $(U, V) \geq (A, B)$ if and only if $(U \subseteq A) \text{ and } (V \subseteq B)$ (so smaller sets are greater).
  Let $\phi$ be a net on this directed set such that $\phi(U, V) = $ some point in $U \cap V$.

  We claim that this net converges to both $x$ and $y$.
  Let $W$ be any neighborhood of $x$.
  Take any open set $V$ containint $y$ and an open set $U$ with $x \in U \subseteq W$.
  For any $(A, B) \geq (W, V)$ we have that $\phi(A, B) \in A \cap B \subseteq U \subseteq W$.
  Thus $\phi$ is eventually in $W$ and $\phi$ converges to $x$.
  A similar argument yields that $\phi$ also converges to $y$.
\end{pf}

\begin{prop}\label{prop: continuous_iff_net_converges}
  $f\colon X \to Y$ between topological spaces is continuous if and only if for any net $\phi$ converging to $x \in X$, the net $f \circ \phi$ in $Y$ converges to $f(x)$.
\end{prop}
\begin{pf}
  Suppose that $f$ is continuous.
  Let $\phi$ be a net in $X$ converging to $x$.
  Let $V \subseteq Y$ be an open set containing $f(x)$.
  We have that $U = f^{-1}(V)$ is a neighborhood of $x$.
  $\phi$ is eventually in $U$ so $f \circ \phi$ is eventually in $V$ and thus converges to $f(x)$.

  Now suppose that $f$ is not continuous.
  Then there is some open $V \subseteq Y$ such that $K \defeq f^{-1}(V)$ is not open.
  Let $x \in K \setminus \inte(K)$.
  Consider the directed set of open neighborhoods of $x$ with the ordering $A \geq B$ if and only if $A \subseteq B$.
  Choose any neighborhood $A$ of $x$.
  Note that $A \not\subseteq K$ so let $\phi(A) = w_{A} \in A \setminus K$ be a net.
  If $N$ is a neighborhood of $x$ and $B \geq N$, so $B \subseteq N$, then $\phi(B) = w_{B} \in B \setminus K \subseteq N$ and so $\phi$ is eventually in $N$.
  Thus $\phi$ converges to $x$.
  However $(f \circ \phi)(A) \notin V$ for any $A$ and so $f \circ \phi$ is not eventually in $V$ and thus does not converge to $f(x)$.
\end{pf}

\clearpage

Given a net $\phi\colon D \to X$, let $x_{\alpha} \defeq \phi(\alpha)$.
It's common to notate this net as $\set{x_{\alpha}}_{\alpha \in D}$.
So the condition in \Cref{prop: continuous_iff_net_converges} can be stated as
\[
  f\colon X \to Y \text{ continuous } \iff f(\lim x_{\alpha}) = \lim f(x_{\alpha}).
\]

\begin{prop}
  if $A \subseteq X$ then $\overline{A}$ is the set of limits of nets in $A$ which converge to $X$.
\end{prop}
\begin{pf}
  If $x \in \overline{A}$ then any open neighborhood of $x$ intersects nontrivially.
  We can make a net of this set of neighborhoods ordered by inclusion and have $x_{U} \in U \cap A$.
  This clearly converges to $x$.

  Now suppose that we have a net $\set{x_{\alpha}}$ of points in $A$ which converges to a point $x \in X$.
  Then this net is eventually in any neighborhood of $x$.
  Thus any neighborhood of $x$ has nontrivial intersection with a point in $A$ and $x \in \overline{A}$.
\end{pf}

\begin{defn}[Final Functions]
  If $D$ and $D'$ are directed sets, a function $h\colon D' \to D$ is \emph{final} if for all $\delta \in D$, there exists $\delta' \in D'$ such that $\alpha' \geq \delta'$ implies $h(\alpha') \geq \delta$.
\end{defn}

\begin{defn}[Subnets]
  A \emph{subnet} of a net $\phi\colon D \to X$ is the composition of a final map $h\colon D' \to D$ to a net $\phi \circ h$.
\end{defn}

\begin{prop}
  A net $\set{x_{\alpha}}$ is frequently in each neighborhood of a given point $x \in X$ if and only if it has a subnet which converges to $x$.
\end{prop}
\begin{pf}
  Consider the set $D'$ be ordered pairs $(\alpha, U)$ where $\alpha \in D$, $U$ is a neighborhood of $x$, and $x_{\alpha} \in U$.
  Give $D'$ the ordering on $D$ and by inclusion.
  If $(\alpha, U)$ and $(\beta, V)$ are in $D'$, then since $\set{x_{\alpha}}$ is frequently in $U$ and $V$, it is frequently in $U \cap V$.
  Thus there is some $\tau \geq \alpha, \beta$ with $x_{\tau} \in U \cap V$.
  Thus $(\tau, U \cap V) \in D'$ and $(\tau, U \cap V) \geq (\alpha, U), (\beta, V)$.
  So $D'$ is directed.

  Map $D' \to D$ by $(\alpha, U) to \alpha$.
  For any $\delta \in D$, $(\delta, X) \in D'$, and $(\alpha, X) \geq (\delta, X)$ implies that $\alpha \geq \delta$.
  So this map is final and $\set{x_{\alpha, U}}$ is a subset of $\set{x_{\alpha}}$.

  Let $N$ be a neighborhood of $x$.
  Then by assumption there exists some $x_{\beta} \in N$.
  If $(\alpha, U) \geq (\beta, N)$, then $x_{\alpha, U} = x_{\alpha} \in U \subseteq N$.
  So $\set{x_{\alpha, N}}$ is eventually in $N$.

  The converse is immediate.
\end{pf}

\begin{defn}[Universal Nets]
  A net in a set $X$ is \emph{universal} if for any $A \subseteq X$, the net is either eventually in $A$ or $X \setminus A$.
\end{defn}

\begin{prop}
  The composition of a universal net in $X$ with a function $f\colon X \to Y$ is a universal net in $Y$.
\end{prop}
\begin{pf}
  If $A \subseteq Y$, then the net is eventually either in $f^{-1}(A)$ or $X \setminus f^{-1}(A)$.
  But $X \setminus f^{-1}(A) = f^{-1}(Y \setminus A)$ so the composition is either in $A$ or $Y \setminus A$.
\end{pf}

\clearpage

\begin{thrm}\label{thrm: net_has_universal_subnet}
  Every net has a universal subset
\end{thrm}
\begin{pf}
  Let $\set{x_{\alpha} | \alpha \in P}$ be a net in $X$.
  Consider all collections \textbf{C} of subsets of $X$ such that
  \begin{enumerate}
  \item $A \in \textbf{C} \implies \set{x_{\alpha}}$ is frequently in $A$; and
  \item $A, B \in \textbf{C} \implies A \cap B \in \textbf{C}$.
  \end{enumerate}
  Note that $\textbf{C} = \set{X}$ is such a collection.
  Other the family of all such collections by inclusion.
  The union of any \quest{simply ordered set} of collections satisfying conditions is another such collection.
  Thus by the \quest{maximality principle} there exists a maximal such collection $\textbf{C}_{0}$.

  Let $P_{0} = \set{(A, \alpha) \in \textbf{C}_{0} \times P | x_{\alpha} \in A}$ and order it by
  \[
    (B, \beta) \geq (A, \alpha) \iff B \subseteq A \text{ and } \beta \geq \alpha.
  \]
  This makes $P_{0}$ a directed set.
  Map $(A, \alpha) \to \alpha$ which is clearly final and thus defines a subset $\set{x_{A, \alpha}}$.

  We claim this subnet is universal.
  Suppose $S$ is any subset of $X$ such that $\set{x_{A, \alpha}}$ is frequently in $S$.
  Then for any $(A, \alpha) \in P_{0}$, there exists $(B, \beta) \geq (A, \alpha)$ in $P_{0}$ with $x_{\beta} = x_{B, \beta} \in S$.
  Then $B \subseteq A$, $\beta \geq \alpha$, and $x_{\beta} \in B$.
  Thus $x_{\beta} \in S \cap B \subseteq S \cap A$.
  This means that $\set{x_\alpha}$ is frequently in $S \cap A$ for any $A \in \textbf{C}_{0}$.
  But $S$ and $S \cap A$, for $A \in \textbf{C}_{0}$, can be added to $\textbf{C}_{0}$ and these conditions still hold.
  So by maximality, $S \in \textbf{C}_{0}$.

  If $\set{x_{A, \alpha}}$ was also frequently in $X \setminus S$, then $X \setminus S \in \textbf{C}_{0}$ be a similar argument.
  Thus $S \cap (X \setminus S) = \emptyset \in \textbf{C}_{0}$.
  This contradicts the first condition.
  Thus $\set{X_{A, \alpha}}$ is not frequently in $X \setminus S$ and so it is indeed eventually in $S$.

  Overall we have that if $\set{x_{(A, \alpha)}}$ is frequently in $S$, it is eventually in $S$.
  Thus $\set{x_{A, \alpha}}$ is a universal subset.
\end{pf}

Note that here we have used the Axiom of Choice in the form of the \quest{maximality principle}.
This past theorem is in fact equal to the Axiom of Choice.

\begin{prop}
  Subnets of universal nets are universal.
\end{prop}
\begin{pf}
  \quest{Immediate Proof}
\end{pf}

\clearpage

\section{Compactness}

\begin{defn}[Covering, Open Covering, Subcover]
  A \emph{covering} of a topological space $X$ is a collection of sets whose union is $X$.
  An \emph{open covering} is a covering where each set is open.
  A \emph{subcover} is a subset of a cover which is still a cover.
\end{defn}

\begin{defn}[Compact, Heine-Borel Property]
  A topological space $X$ is said to be \emph{compact} or have the \emph{Heine-Borel property} if every open covering of $X$ has a finite subcover.
\end{defn}

\begin{defn}[Finite Intersection Property]
  A collection of sets has the \emph{finite intersection property} if the intersection of any finite subcollection is empty.
\end{defn}

The following theorem is just a translation of compactness in terms of open sets to an equivalent statement of about the closed complements of those sets.

\begin{thrm}\label{thrm: compact_iff_nonempty_intersection}
  A topological space $X$ is compact if and only if for every collection of closed subsets of $X$ which has the finite intersection property, the intersection of the whole collection is nonempty.
\end{thrm}
\begin{pf}
  \quest{Trivial Proof}
\end{pf}

\begin{thrm}
  If $X$ is a Hausdorff space, then any compact subset of $X$ is closed.
\end{thrm}
\begin{pf}
  Let $A \subseteq X$ be compact and suppose that $x \in X \setminus A$.
  For $a \in A$ let $a \in U_{a}$ and $x \in V_{a}$ be disjoint open sets.
  Now $A = \bigcup_{a \in A} (U_{a} \cap A)$ and so we have a cover.
  Thus by compactness of $A$, we have $a_{1}, \ldots, a_{n} \in A$ such that $A \subseteq u_{a_{1}} \cup \cdots \cup U_{a_{n}} = U$.
  But then $x \in V_{a_{1}} \cap \cdots \cap V_{a_{n}} = V$ which is open, and $U \cap V = \emptyset$.
  Thus $x \in V \subseteq X \setminus U \subseteq X \setminus A$ and $V$ is open.
  Since this holds for any $x \in X \setminus A$, we have that $X \setminus A$ is open and thus $A$ is closed.
\end{pf}

\begin{thrm}
  If $X$ is compact and $f\colon X \to Y$ is continuous, then $f(X)$ is compact.
\end{thrm}
\begin{thrm}
  We may as well replace $Y$ by $f(X)$ and so assume that $f$ is onto.
  For any open cover of $f(X)$, look at inverse images of the sets and apply compactness.
\end{thrm}

\begin{thrm}\label{thrm: closed_subset_of_compact_is_compact}
  If $X$ is compact and $A \subseteq X$ is closed, then $A$ is compact.
\end{thrm}
\begin{pf}
  Cover $A$ with open sets in $X$, add the open set $X \setminus A$, and then apply compactness of $X$.
\end{pf}

\begin{thrm}
  Suppose that $X$ is compact, $Y$ is Hausdorff, $f\colon X \to Y$ is a continuous bijection, then $f$ is a homeomorphism.
\end{thrm}
\begin{pf}
  We need to show that $f^{-1}$ is continuous.
  This is equivalent to showing that $f$ maps closed sets to closed sets.
  But if $A \subseteq X$ is closed, then $A$ is compact by \Cref{thrm: closed_subset_of_compact_is_compact}.
  Thus $f(A)$ is compact and $f(A)$ must be closed since $A$ is Hausdorff.
\end{pf}

\clearpage

\begin{ex}[Closed Intervals are compact in $\R$]
  We have that $[0, 1]$ is compact in $\R$.
\end{ex}
\begin{pf}
  Let $\textbf{U}$ be an open covering of $[0, 1]$.
  Let $S = \set{s \in [0, 1] | [0, s] \text{ is covered by a finite subcollection of } \textbf{U}}$.
  Then Let $b = \sup(S)$.
  Note that $S$ takes the form $[0, b]$ or $[0, b)$.
  Suppose that $S$ takes for form $[0, b)$.
  Consider a set $U \in \textbf{U}$ containing $b$.
  $U$ must contain the interval $[a, b]$ for some $a < b$.
  But we can throw this in with the interval $[0, a]$ and obtain $[0, b]$.
  So $S$ must take the form $[0, b]$.
  Now suppose that $b < 1$.
  Then \quest{a similar argument} yields that there exists $c > b$ such that there is a finite cover of $[0, c]$ which contradicts the definition of $b$.
  Thus $b = 1$ and we have a finite cover of $[0, 1]$.
\end{pf}

Overall we have that closed intervals $[a, b]$ and their closed subsets are compact.
Also note that these closed subsets must be bounded.
So in $\R$ we have that a subset is compact if and only if it is closed and bounded.
This does not hold in arbitrary metric spaces.

\begin{thrm}
  A real-valued map on a compact space takes a maximum.
\end{thrm}
\begin{pf}
  If $f\colon X \to \R$ is a real-valued map on a compact space, then $f(X)$ is compact.
  Since $f(X)$ is compact, it is closed and bounded.
  This means that $\sup(f(X))$ exists, is finite, and belongs to $f(X)$.
\end{pf}

\begin{thrm}
  Compact Hausdorff spaces are normal
\end{thrm}
\begin{pf}
  Suppose that $X$ is a compact Hausdorff space.
  First we show that $X$ is regular.
  Let $C$ be a closed subset of $X$ and let $x \not in C$.
  $X$ is Hausdorff so for any $y \in C$ there exists open disjoint $U_{y}$, $V_{y}$ such that $x \in U_{y}$ and $y \in V_{y}$.
  $C$ is closed and so it is compact by \Cref{thrm: closed_subset_of_compact_is_compact}.
  $V_{y}$ is an open cover of $C$ and so there exist $y_{1}, \ldots, y_{n}$ such that $C \subseteq V \defeq V_{y_{1}} \cup \cdots \cup V_{y_{n}}$.
  Let $U \defeq U_{y_{1}} \cap \cdots \cap U_{y_{n}}$.
  Then we have that $x \in U$, $C \subseteq V$, and $U, V$ disjoint and open.
  Thus $X$ is regular.

  Now repeat this same proof letting a closed set $F$ play the role of $x$ and the other closed set $G$ playing the role of $C$.
  Thus $X$ is normal.
\end{pf}

\begin{defn}[Proper Maps]
  A map $f\colon X \to Y$ is \emph{proper} if $f^{-1}(C)$ is compact for each compact $C \subseteq Y$.
\end{defn}

\clearpage

\begin{thrm}
  If $f\colon X \to Y$ is closed and $f^{-1}(y)$ is compact for each $y \in Y$, then $f$ is proper.
\end{thrm}
\begin{pf}
  Let $C \subseteq Y$ be compact and let $\set{U_{\alpha} | \alpha \in A}$ be an indexed collection of open sets whose union contains $f^{-1}(C)$.
  For any $y \in C$, there exists a finite subset of $A_{y} \subseteq A$ such that $f^{-1}(y) = \bigcup_{\alpha \in A_{y}} U_{\alpha}$.
  Now let
  \begin{align*}
    W_{y} &= \bigcup_{\alpha \in A_{y}} U_{\alpha}, \\
    V_{y} &= Y \setminus f(X \setminus W_{y}).
  \end{align*}
  These are both open sets.
  $f^{-1}(V_{y}) \subseteq W_{y}$ and $y \in V_{y}$.
  Since $C$ is compact and covered by $V_{y}$, there exists $y_{1}, \ldots, y_{n}$ such that $C \subseteq V_{y_{1}} \cup \cdots \cup V_{y_{n}}$.
  Thus
  \begin{align*}
    f^{-1}(C) &\subseteq f^{-1}(V_{y_{1}}) \cup \cdots \cup f^{-1}(V_{y_{n}}) \\
              &\subseteq W_{y_{1}} \cup \cdots \cup W_{y_{n}} \\
              &= \bigcup_{\substack{\alpha \in A_{y_{i}}, 1 \leq i \leq n}} U_{\alpha}
  \end{align*}
  which is a finite open subcover of $f^{-1}(C)$.
\end{pf}

\begin{thrm}
  For a topological space $X$, the following are equivalent:
  \begin{enumerate}
  \item $X$ is compact.
  \item Every collection of closed subsets of $X$ with the finite intersection property has non-empty intersection.
  \item Every universal net in $X$ converges.
  \item Every net in $X$ has a convergent subset.
  \end{enumerate}
\end{thrm}
\begin{pf}
  We have already seen the equivalent of $1.$ and $2.$ and now we show the rest.
  \begin{enumerate}
  \item[$1. \iff 2.$] See \Cref{thrm: compact_iff_nonempty_intersection}.
  \item[$1. \implies 3.$] Suppose that $\set{x_{\alpha}}$ is a universal net that does not converge.
        Given $x \in X$, there is an open neighborhood of $U_{x}$ such that $\set{x_{\alpha}}$ is not eventually in  $U_{x}$ and so the net is eventually in $X \setminus U_{x}$.
        So there exists $\beta_{x}$ such that for all $\alpha \geq \beta_{x}$ we have that $x_{\alpha} \notin U_{x}$.
        Now cover $X$ by some finite cover $U_{x_{1}} \cup \cdots U_{x_{n}}$.
        Let $\alpha \geq \beta_{x_{i}}$ for all $i$, then $x_{\alpha} \notin U_{x_{i}}$ for any $i$.
        Thus $x_{\alpha} \notin X$ which is a contradiction.
  \item[$3. \implies 4.$] This is clear by \Cref{thrm: net_has_universal_subnet}.
  \item[$4. \implies 2.$] Let $\textbf{F} = \set{C}$ be a collection of closed sets with the finite intersection property.
    Without loss of generality, we can assume that \textbf{F} is closed under finite intersection.
    Order \textbf{F} by $C \geq C'$ if and only if $C \subseteq C'$, making \textbf{F} a directed set.
    Let $\set{x_{C}}_{C \in \textbf{F}}$ be a net.
  \end{enumerate}
\end{pf}


\clearpage
\nocite{book:Bredon}
\printbibliography
\end{document}
