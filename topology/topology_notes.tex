\documentclass[letterpaper, 11pt, oneside]{book}

\usepackage{style}  % If you feel like procrastinating, mess with this file
\usepackage{algo}   % Thank you Jeff, very cool!

\addbibresource{refs.bib}

% Required reading
% https://jmlr.csail.mit.edu/reviewing-papers/knuth_mathematical_writing.pdf
%   Along with required viewing:
%   https://www.youtube.com/watch?v=N6QEgbPWUrg&list=PLOdeqCXq1tXihn5KmyB2YTOqgxaUkcNYG
% https://faculty.math.illinois.edu/~west/grammar.html

% % % % % % % % % %
%     Cursor      %
%     Parking     %
%     Lot         %
% % % % % % % % % %

% Disable check for mismatched parens/brackets/braces
%   chktex-file 9
% Disable check for different counts of parents/brackets/braces
%   chktex-file 17
% Exclude these environments from syntax checking
%   VerbEnvir { tikzcd }

\regtotcounter{figure}

\title{\vspace{-100pt}\includegraphics[width=0.75\textwidth]{figs/title.png} \\ {\Huge Topology Notes and Exercises} \\ {\small With $\total{figure}$ Figures}}
\nocite{title_pic}
\author{\Large Anakin Dey}
\DTMsavenow{now}
\date{\small Last Edited on \today\ at \DTMfetchhour{now}:\DTMfetchminute{now}}

% Cover page number chicanery
\newcommand{\CoverName}{Cover}

\begin{document}
\frontmatter
\renewcommand{\thepage}{\CoverName}
\maketitle

\pagenumbering{roman}

\section*{TODOs}

\quest{Change the style of enumerates from ``1.'' to ``(1)''}

\quest{Proper Exercise Header}

\quest{Proper Chapter Header}

\tableofcontents
\clearpage


\listoftheorems[ignoreall, show={defn}, title={List of Definitions}]

\listoftheorems[ignoreall, show={ex}, title={List of Examples and Counterexamples}]

\mainmatter

\chapter{Metric Spaces}

The most familiar example of a metric space is $\R^{n}$ with $\dist(x, y) = \pqty{\sum_{i = 1}^{n} (x_{i} - y_{i})^{2}}^{1/2}$.
We also have a notion of continuity.
A function $f$ is \emph{continuous at a point $x$} if for all $\e > 0$, there exists $\delta > 0$ such that $\dist(x, y) < \delta \implies \dist(f(x), f(y)) < \e$.
We wish to generalize this notion.

\begin{defn}[Metric Spaces]
  A \emph{metric space} is a set $X$ equipped with a distance function $\dist\colon X \times X \to \R$ called a \emph{metric} satisfying
  \begin{enumerate}
    \item $\dist(x, y) \geq 0$ with equality if and only if $x = y$
    \item $\dist(x. y) = \dist(y, x)$
    \item $\dist(x, z) \leq \dist(x, y) + \dist(y, z)$.
  \end{enumerate}
\end{defn}

\begin{defn}[$\e$-ball]
  For a point $x$ in some metric space $X$, the \emph{$\e$-ball about x} is
  \[
    B_{\e}(x) \defeq \set{y \in X | \dist(x, y) < \e}.
  \]
\end{defn}

It is not hard to rephrase the definition of continuity in terms of $\e$-balls.

\begin{defn}[Open and Closed Sets in $\R$]
  A subset $U$ of some metric space $X$ is \emph{open} if for all $x \in U$ there exists $\e > 0$ such that $B_{\e}(x) \subseteq U$.
  A subset $C$ is \emph{closed} if its complement is open.
\end{defn}

We can actually even rephrase continuity in terms of open sets.

\clearpage

\begin{prop}
  A function $f\colon X \to Y$ between metric spaces is continuous if and only if $f^{-1}(U)$ is open for every open set $U \subseteq Y$.
\end{prop}
\begin{pf}
  Suppose $f\colon X \to Y$ is continuous.
  Let $U \subset Y$ be some open set and let $f(x) \in U$.
  Then there exists $\e > 0$ such that $B_{\e}(f(x)) \subseteq U$.
  By continuity of $f$, there exists $\delta > 0$ such that $f(B_{\delta}(x)) \subseteq B_{e}(f(x))$.
  Thus $B_{\delta}(x) \subseteq f^{-1}(U)$ and $f^{-1}(U)$ is open.

  Conversely, suppose that $f^{-1}(U)$ is open for any open $U \subseteq Y$.
  Let $\e > 0$.
  Then $B_{\e}(f(x))$ is open and $f^{-1}(B_{\e}(f(x)))$ is open and contains $x$.
  Thus there exists $\delta > 0$ such that $B_{\delta}(x) \subseteq f^{-1}(B_{\e}(f(x)))$.
  Thus $f(B_{\delta}(x)) \subseteq B_{\e}(f(x))$ and $f$ is continuous.
\end{pf}

This can also all be rephrased in terms of closed sets, rather than open sets.

\begin{ex}[Other Metrics on $\R$]
  We can also equip other metrics on $\R^{n}$.
  \begin{align*}
    \dist_{2}(x, y) &= \sum_{i = 1}^{n} \abs{x_{i} - y_{i}} \\
    \dist_{3}(x, y) &= \max_{i = 1}^{n} (\abs{x_{i} - y_{i}})
  \end{align*}
\end{ex}

However it turns out that for $\R^{n}$, choice between these 3 metrics is irrelevant.

\begin{prop}
  Suppose $\dist_{1}$ and $\dist_{2}$ are two metrics on the same set $X$ such that for any $x \in X$ and $\e > 0$, there exists $\delta > 0$ such that
  \[
    \dist_{1}(x, y) < \delta \implies \dist_{2}(x, y) < \e
  \]
  and
  \[
    \dist_{2}(x, y) < \delta \implies \dist_{1}(x, y) < \e.
  \]
  Then these metrics define the same open sets in $X$.
\end{prop}

\begin{pf}
  It suffices to show that every $\dist_{1}$ open set is also $\dist_{2}$ open since for the other direction we have the opposite hypothesis.
  Suppose $U$ is open with respect to $\dist_{1}$.
  Then for each $x \in V $we have some $\e$ such that $B_{\e}(x) \subseteq V$.
  By hypothesis, we can find $\delta$ such that $B_{\delta}(x) \subseteq V$ because we can find $\delta$ such that $\dist_{2}(x, y) < \delta \implies \dist_{1}(x, y) < \e$.
  Thus $V$ is open with respect to $\dist_{2}$.
\end{pf}

\begin{cor}
  The following metrics define the same open sets in $\R^{n}$:
  \begin{align*}
    \dist(x, y) &= \pqty{\sum_{i = 1}^{n} (x_{i} - y_{i})^{2}}^{1/2} \\
    \dist_{2}(x, y) &= \sum_{i = 1}^{n} \abs{x_{i} - y_{i}} \\
    \dist_{3}(x, y) &= \max_{i = 1}^{n} (\abs{x_{i} - y_{i}})
  \end{align*}
\end{cor}

\clearpage

\subsection*{Exercises}

\begin{exercise}[\tcite{book:Bredon} 1.1.1]
  Consider the set $X$ of all continuous real valued functions on $[0, 1]$.
  Show that
  \[
    \dist(f, g) = \int_{0}^{1} \abs{f(x) - g(x)} \dd{x}
  \]
  defines a metric on $X$.
  Is this still the case if continuity is weakened to integrability?
\end{exercise}
\begin{pf}
  Positivity follows from the positivity of absolute value.
  If $f = g$ then clearly for all $x \in [0, 1]$ we have that $\abs{f(x) - g(x)} = 0$.
  Then if $f \neq g$ then there exists $x \in [0, 1]$ such that $\abs{f(x) - g(x)} \neq 0$.
  Since these functions are continuous, this yields that $\int_{0}^{1} \abs{f(x) - g(x)} \dd{x} > 0$.

  Symmetry follows from $\abs{f(x) - g(x)} = \abs{g(x) - f(x)}$.

  The triangle inequality follows from the triangle inequality for absolute value.

  Note that we require continuity.
  Let $f(x) \defeq 0$ and let $g(x) = 0$ for all $x > 0$ and $g(0) = 1$.
  Then $f$ and $g$ are integrable and $\dist(f, g) = 0$ but $f \neq g$.
\end{pf}

\begin{exercise}[\tcite{book:Bredon} 1.1.2]
  If $X$ is a metric space and $x_{0}$ is a given point in $X$, show that the function $f\colon X \to \R$ given by $f(x) = \dist(x, x_{0})$ is continuous.
\end{exercise}
\begin{pf}
  Let $\e > 0$.
  Suppose that $\dist(x, y) < \e$.
  Then we have that
  \[
    \abs{f(x), f(y)} = \abs{\dist(x, x_{0}) - \dist(y, x_{0})} \leq \abs{\dist(x, y)} = \dist(x, y) < \e.
  \]
\end{pf}

\begin{exercise}[\tcite{book:Bredon} 1.1.3]
  If $A$ is a subset of a metric space $X$ then define a real valued function $d$ on $X$ by $d(x) = \dist(x, A) \defeq \int\set{\dist(x, y) | y \in A}$.
  Show that $d$ is continuous.
\end{exercise}
\begin{pf}
  Note that for any $x, y \in X$ and $z \in A$ we have that
  \[
    \dist(x, z) \leq \dist(x, y) + \dist(y, z).
  \]
  Taking infimum yields that
  \[
    \dist(x, A) \leq \dist(x, y) + \dist(y, A).
  \]
  Thus $\dist(x, A) - \dist(y, A) \leq \dist(x, y)$ and similarly we have $\dist(y, A) - \dist(x, A) \leq \dist(x, y)$.
  This implies that $\abs{\dist(y, A) - \dist(x, A)} \leq \dist(x, y)$.

  Now let $\e > 0$.
  Suppose that $\dist(x, y) < \e$.
  Then we have that $\abs{d(x), d(y)} \leq \dist(x, y) < \e$.
\end{pf}

\clearpage

\chapter{Topological Spaces}

We usually only care about continuity, not the actual metrics.
Continuity can be formulated in terms of open sets.
It can be shown that a function is continuous if and only if $f^{-1}(U)$ is open for all open $U$ in the codomain.

\begin{defn}[Topological Spaces]
  A \emph{topological} space is a set $X$ with a collection of subsets of $X$ called ``open sets'' such that
  \begin{enumerate}
  \item the intersection of two open sets is open;
  \item the union of any collection of open sets is open; and
  \item $X, \emptyset$ are open.
  \end{enumerate}
  A subset $X \subseteq X$ is \emph{closed} if $X \setminus C$ is open.
\end{defn}

\begin{defn}[Continuous Functions]
  A function of topological spaces $f\colon X \to Y$ is \emph{continuous} if $f^{-1}(U)$ is open for all open $U \subseteq Y$.
  A \emph{map} is a continuous function.
\end{defn}

It isn't hard to see that a function $f\colon X \to Y$ is continuous if and only if $f^{-1}(C)$ is closed for all closed $C \subseteq Y$.

\begin{defn}[Neighborhood]
  If $X$ is a topological space, a set $N \subseteq X$ is a \emph{neighborhood} of $x \in X$ if it contains an open set $U \subseteq N$ such that $x \in U$.
\end{defn}

It is immediate that arbitrary unions and finite intersections of neighborhoods of a point $x \in X$ are still neighborhoods of $x$.

\begin{defn}[Neighborhood Basis]
  Let $X$ be a topological space and $x \in X$.
  A collection $\textbf{B}_{x}$ of subsets of $X$ containing $x$ is called a \emph{neighborhood basis} at $x$ in $X$ if each neighborhood of $x$ contains some element of $\textbf{B}_{x}$ and each element of $\textbf{B}_{x}$ is a neighborhood of $x$.
\end{defn}

Neighborhood bases let us define continuity at a single point.

\begin{defn}[Continuity at a Point]
  A function $f\colon X \to Y$ between topological spaces is said to be \emph{continuous at $x$}, $x \in X$, if, given any neighborhood $N$ of $f(x)$ in $Y$, there is a neighborhood $M$ of $x$ in $X$ such that $f(M) \subseteq N$.
\end{defn}

This is the same as saying $f^{-1}(N)$ is a neighborhood of $x$.

\begin{prop}
  A function $f\colon X \to Y$ between topological spaces is continuous if and only if it is continuous at each point $x \in X$.
\end{prop}
\begin{pf}
  Suppose $f$ is continuous and let $N$ be a neighborhood of $f(x)$ in $Y$ with open $U \subseteq N$ such that $f(x) \in U$.
  Then $x \in f^{-1}(U) \subseteq f^{-1}(N)$ where $f^{-1}(U)$ is open in $X$.
  Thus $f^{-1}(N)$ is a neighborhood of $x$ in $X$ and $f\pqty{f^{-1}(N)} = N \subseteq N$ and $f$ is continuous at $x \in X$.

  Conversely now suppose that $f$ is continuous at each point in $X$ and let $U \subseteq Y$ be open.
  Then for any $x \in f^{-1}(U)$, we have that $f^{-1}(U)$ is a neighborhood of $x$.
  Thus there exists open $V_{x} \subseteq f^{-1}(U)$ with $x \in V_{x}$.
  Thus $f^{-1}(U)$ is the union of open sets $V_{x}$ ranging over $x \in f^{-1}(U)$ and $f^{-1}(U)$ is open which yields that $f$ is continuous.
\end{pf}

\begin{defn}[Homeomorphic Functions]
  A function $f\colon X \to Y$ between topological spaces is called a \emph{homeomorphism} if $f^{-1}\colon Y \to X$ exists and both $f$ and $f^{-1}$ are continuous.
  We notate that $X \approx Y$ meaning that $X$ is \emph{homeomorphic} to $Y$.
\end{defn}

Topological spaces are homeomorphic then if there is a bijection between them as sets but also there is a correspondence between the open sets.
These sets can then essentially be regarded as the same sets.

Describing topological spaces can be described in a more simple manner than listing all the open sets using the concept of a basis.
\begin{defn}[Basis / Analytic Basis]
  Let $X$ be a topological space and \textbf{B} a collection of subsets of $X$.
  We say that \textbf{B} is a \emph{basis} for the topology of $X$ if the open sets of $X$ are precisely the unions of members of \textbf{B}.
  A collection \textbf{S} of subsets of $X$ is called a subbasis for the topology of $X$ if the set \textbf{B} of finite intersections of members of \textbf{S} forms a basis of $X$.

  Some sources call this an \emph{analytic basis}.
\end{defn}

\emph{Any} collection \textbf{S} of subsets of any set $X$ is a subbasis for some topology on $X$, namely the topology where the open sets are arbitrary unions of finite intersections of members of \textbf{S}.
The empty set is the union of an empty collection and $X$ is the intersection of an empty collection.
Thus to specify a topology, a subbasis suffices.
In a metric space, the collection of all $\e$-balls, for all $\e > 0$, forms a basis.
So is the collection of $\e$-balls for $\e = 1, \frac{1}{2}, \frac{1}{3}, \ldots$.

\clearpage

We can actually phrase continuity in terms of bases.
\begin{thrm}\label{thrm: continuity_for_bases}
  Let $X, Y$ be topological spaces and $f\colon X \to Y$ be a function.
  Let \textbf{B} be a basis for $Y$.
  Then $f$ is continuous if and only if for all $B \in \textbf{B}$ we have that $f^{-1}(B)$ is open in $X$.
\end{thrm}
\begin{pf}
  The forward direction is immediate since basis members are open by definition.
  Let $U \subseteq Y$ be open.
  Then for some $\textbf{A} \subseteq \textbf{B}$ we have that $U = \bigcup_{B \in \textbf{A}} B$.
  Then we have that
  \begin{align*}
    f^{-1}(U) &= f^{-1}\pqty{\bigcup_{B \in \textbf{A}} B} \\
              &= \bigcup_{B \in \textbf{A}} f^{-1}(B)
  \end{align*}
  and since each $f^{-1}(B)$ is open, $f^{-1}(U)$ is open and $f$ is continuous.
\end{pf}

An analogous result holds for elements of a subbasis with a very similar, and thus omitted, proof.

\begin{ex}[Examples of Topologies]
  We now give examples of topological spaces:
  \begin{enumerate}
  \item (Trivial / Indiscrete Topology) Any set $X$ where the only open sets are $X$ and $\emptyset$.
  \item (Discrete Topology) Any set $X$ where every subset of $X$ is open.
  \item Any sets $X$ where the closed sets are finite sets and $X$ itself.
  \item $X = \N \cup \set{\N}$ with the open sets being all subsets of $\N$ together with complements of finite sets.
  \item Let $X$ be any poset.
        For $\alpha \in X$ consider the one-sided intervals $\set{\beta \in X | \alpha < \beta}$ and $\set{\beta \in X | \alpha > \beta}$.
        The ``order topology'' on $X$ is the topology generated by these intervals.
        The ``strong order topology'' is the topology generated by these intervals together with the complements of finite sets.
  \item Let $X = I^{2}$ where $I = [0, 1]$ the unit intervals.
        Give this the ``dictionary ordering'' where $(x, y) < (s, t)$ if and only if either $x < s$ or $(x = s \text{ and } y < t)$.
        Let $X$ have the order topology for this ordering.
  \item Let $X$ be the real line bu with the topology generated by the ``half open intervals'' $[x, y)$.
        This is called the ``half open interval topology'' or the ``lower limit topology.''
  \item Let $X = \Omega \cup \set{\Omega}$ be the set of ordinal numbers up to and including the least uncountable ordinal $\Omega$.
        Give this the order topology.
  \end{enumerate}
\end{ex}

\begin{defn}[Countability]
  A topological space is \emph{first countable} if each point has a countable neighborhood basis.
  A topological space is \emph{second countable} if its topology has a countable basis.
\end{defn}

\clearpage

\begin{ex}[Metric Spaces are First Countable but some are not Second Countable]
  Let $x \in X$ where $X$ is a metric space.
  Then the set of open balls around $x$ with rational radius forms a countable neighborhood basis.

  Consider the space if any uncountable set with metric $\dist(x, y) = 1$ if $x \neq y$ and $0$ otherwise (which yields the discrete topology).
  Euclidean spaces are second countable since the $\e$-balls, with $\e$ rational, about points with rational coordinates, is a countable basis.
\end{ex}

\begin{prop}
  A topological space $X$ is second countable if and only if every basis for $X$ has a countable subbasis.
\end{prop}
\begin{pf}
  Suppose $X$ is second countable.
  Let $B = \set{B_{\alpha}}_{\alpha \in A}$ be a basis for $X$ and $C = \set{C_{i}}_{i \in \N}$ be a countable basis for $X$.
  It suffices to show that each $C_{i}$ can be expressed as a countable union of some subset of the $B_{\alpha}$'s.
  Let $C_{k}$ be a member of the countable basis.
  $B$ is a basis so $C_{k} = \bigcup_{\alpha \in I \subseteq A}$ for some $I$.
  For each $x \in C_{k}$, let $B_{i_{x}}$ be a member of the basis such that $i_{x} \in I$ and $x \in B_{i_{x}}$.
  But then $C$ is also a basis, so we can find $C_{i_{x}}$ such that $x \in C_{i_{x}} \subseteq B_{i_{x}}$.
  We can see that $\set{B_{i_{x}} | x \in C_{k}}$ is our desired countable subset.
  Clearly for $x \in C_{k}$ we have that there is some $i_{x}$ such that $x \in B_{i_{x}}$ so $C_{k} \subseteq \bigcup_{x \in C_{k}} B_{i_{x}}$.
  We also see that we only need countably many $i_{x}$ since each of the $x \in C_{i}$ are in some $C_{i_{x}}$ and there are only countably many $C_{i_{x}}$.
\end{pf}

\begin{defn}[Uniform Convergence of Functions]
  A sequence $f_{1}, f_{2}, \ldots$ of functions from a topological space $X$ to a metric space $Y$ is said to \emph{converge uniformly} to a function $f\colon X \to Y$ if, for all $\e > 0$, there is a number $n$ such that for all $i > n$, $\dist(f_{i}(x), f(x)) < \e$ for all $x \in X$.
\end{defn}

\begin{thrm}
  If a sequence $f_{1}, f_{2}, \ldots$ of continuous functions from a topological space $X$ to a metric space $Y$ converges uniformly to a function $f\colon X \to Y$, then $f$ is continuous.
\end{thrm}
\begin{pf}
  Given $\e > 0$, let $n_{0}$ be such that for all $x \in X$, $n \geq n_{0}$ implies that $\dist(f_{n}(x), f(x)) < \frac{\e}{3}$.
  Given $n_{0}$, continuity of $f_{n_{0}}$ implies that there is a neighborhood $N$ of $x_{0}$ such that $x \in N$ implies that $\dist(f_{n_{0}}(x), f_{n_{0}}(x_{0})) < \frac{\e}{3}$.
  Thus for any $x \in N$ we have that
  \begin{align*}
    \dist(f(x), f(x_{0})) &\leq \dist(f(x), f_{n_{0}}(x)) + \dist(f_{n_{0}}(x), f_{n_{0}}(x_{0})) + \dist(f_{n_{0}}(x), f(x_{0})) \\
                          &< \frac{\e}{3} + \frac{\e}{3} + \frac{\e}{3} = \e.
  \end{align*}
  Thus $f$ is continuous.
\end{pf}

\begin{defn}[Open and Closed Functions]
  A function $f\colon X \to Y$ between topological spaces is said to be \emph{open} if $f(U)$ is open in $Y$ for all open $U \subseteq X$.
  It is said to be \emph{closed} if $f(C)$ is closed in $Y$ for all closed $C \subseteq X$.
\end{defn}

\clearpage

\begin{defn}[Smallest and Largest Topologies]
  If $X$ is a set and some condition is given on subsets of $X$, which may or may not hold for any particular subset, then if there is a topology $T$ whose open sets satisfy the condition, and such that, for any topology $T'$ whose open sets satisfy the condition, then the $T$-open sets are also $T'$-open (i.e. $T \subseteq T'$), then $T$ is called the \emph{smallest} (or \emph{weakest} or \emph{coarsest}) topology satisfying the condition.
  If, instead, for any topology $T'$ whose open sets satisfy the condition, any $T'$-open sets are also $T$-open (i.e. $T' \subseteq T$), then $T$ is called the \emph{largest} (or \emph{strongest} or \emph{finest}) topology satisfying the condition.
\end{defn}

\begin{ex}[Largest and Smallest Topology for a Condition]
  If $f\colon X \to Y$ is a function and $X$ is a topological space, then there is a largest topology on $Y$ making $f$ continuous having open sets $\set{V \subseteq Y | f^{-1}(V) \text{ is open in } X}$.
  The smallest such topology is the trivial topology.
\end{ex}

If a topology is the largest one satisfying some condition, then there is always some other condition where the given topology is the smallest one satisfying the new condition.
For example, the topology described in the prior example is the smallest topology satisfying the condition ``for all spaces $Z$ and functions $g\colon Y \to Z$, $g \circ f$ being continuous implies $g$ is continuous.''
Thus there is no way to argue that a topology is ``large'' or ``small'' without knowing the defining condition.

\clearpage

\subsection*{Exercises}

\begin{exercise}[\tcite{book:Bredon} 1.2.1]
  Show that in a topological space $X$:
  \begin{enumerate}
  \item[a.] the union of two closed sets is closed;
  \item[b.] the intersection of any collection of closed sets is closed; and
  \item[c.] the empty set and the whole space $X$ are closed.
  \end{enumerate}
\end{exercise}
\begin{pf}
  \begin{enumerate}
  \item[a.] Suppose that $C_{1}, C_{2}$ are closed in $X$.
        Then we have that $X \setminus (C_{1} \cup C_{2}) = (X \setminus C_{1}) \cup (X \setminus C_{2})$ which is the union of two open sets.
        Thus $C_{1} \cup C_{2}$ is closed.
  \item[b.] Suppose $\set{C_{i}}_{i \in I}$ is some collection of closed sets in $X$.
        Then $X \setminus C_{i}$ is open for each $i \in I$ and thus
        \[
          \bigcup_{i \in I} X \setminus C_{i} =  X \setminus \bigcap_{i \in I} C_{i}
        \]
        is open, which means that
        \[
          \bigcap_{i \in I} C_{i}
        \]
        is closed.
  \item[c.] We have that $X \setminus \emptyset = X$ which is open so $\emptyset$ is closed.
        Similarly we have that $X \setminus X = \emptyset$ which is open so $X$ is closed.
  \end{enumerate}
\end{pf}

\begin{exercise}[\tcite{book:Bredon} 1.2.2]
  Consider the topology on $\R$ generated by half open intervals $[x, y)$ together with those of the form $(x, y]$.
  Show that this coincides with the discrete topology.
\end{exercise}
\begin{pf}
  Note that $(x - 1, x] \cup [x, x + 1) = \set{x}$.
  Thus every singleton, and therefore every subset of $\R$, is open and we recover the discrete topology.
\end{pf}

\clearpage

\begin{exercise}[\tcite{book:Bredon} 1.2.4]
  If $f\colon X \to Y$ is a function between topological space, and $f^{-1}(U)$ is open for each open $U$ in some subbasis for the topology of $Y$, show that $f$ is continuous.
\end{exercise}
\begin{pf}
  Let \textbf{S} be a subbasis for $Y$ and let $U$ be open in $Y$.
  We have that \textbf{S} generates some basis \textbf{B} for $Y$ and for some indexing set $I$ we have $U = \bigcup_{i \in I} B_{i}$ where $B_{i} \in \textbf{B}$.
  But then each $B_{i} = S_{i_{1}} \cap S_{i_{n}}$ for $S_{i_{j}} \in \textbf{S}$.
  We have that
  \begin{align*}
    f^{-1}(B_{i}) &= \bigcap_{j = 1}^{n} f^{-1}(S_{i_{j}}) \text{ where each } f^{-1}(S_{i_{j}}) \text{ is open so } f^{-1}(B_{i}) \text{ is open }; \text{ and} \\
    f^{-1}(U) &= \bigcup_{i \in I} f^{-1}(B_{i}) \text{ where each } f^{-1}(B_{i}) \text{ is open so } f^{-1}(U) \text{ is open }.
  \end{align*}
  Thus $f$ is continuous.
\end{pf}

\clearpage

\begin{exercise}[\tcite{book:Bredon} 1.2.5]
  Suppose that $S$ is a set and we are given, for each $x \in S$, a collection $\textbf{N}(x)$ of subsets of $S$ satisfying:
  \begin{enumerate}
  \item $N \in \textbf{N}(x) \implies x \in N$;
  \item $N, M \in \textbf{N}(x) \implies \exists P \in \textbf{N}(x) \text{ such that } P \subseteq N \cap M$; and
  \item $x \in S \implies \textbf{N}(x) \neq \emptyset$.
  \end{enumerate}
  Then show that there is a unique topology on $S$ such that $\textbf{N}(x)$ is a neighborhood basis at $x$, for each $x \in S$.
  Thus a topology can be defined by giving such a collection of neighborhoods at each point.
\end{exercise}
\begin{pf}
  Let $S$ be the topology with open sets $\set{U \subseteq S | \forall x \in U,~ \exists N \in \textbf{N}(x) \text{ such that } N \subseteq U}$.
  \begin{itemize}
  \item $\emptyset$ is open in $S$ vacuously.
        Now take $x \in S$.
        $\textbf{N}(x) \neq \emptyset$ and so there exists $N \subseteq S$ such that $x \in N$.
        Thus $S$ is also open.
  \item Let $U, V$ be open.
        Then let $x \in U \cap V$.
        $x \in U$ implies there exists $N \in \textbf{N}(x)$ such that $N \subseteq U$.
        Similarly there exists $M \in \textbf{N}(x)$ such that $M \subseteq V$.
        We have then there exists $P \in \textbf{N}(x)$ such that $P \subseteq N \cap M$.
        Thus $U \cap V$ is open.
  \item Consider some collection $\set{U_{i}}_{i \in I}$.
        Then for all $x \in U_{i}$ there is $N_{i} \in \textbf{N}(x)$ such that $N_{i} \subseteq U_{i}$.
        Then clearly for all $x \in \bigcup_{i \in I} U_{i}$, since $x$ must be in some $U_{i}$, we can find some such $N_{i} \subseteq U_{i} \subseteq \bigcup_{i \in I} U_{i}$.
        Thus $\bigcup_{i \in I} U_{i}$ is also open.
  \end{itemize}

  Suppose that $Y$ is some other topology such that for each $x \in S$. $\textbf{N}(x)$ is a neighborhood basis of $x$.
  Let $U$ be open in $Y$ and $x \in U$.
  Then there must be $N \in \textbf{N}(x)$ with $N \subseteq U$ because $U$ is an open set containing $x$ and $\textbf{N}(x)$ is a neighborhood basis.
  Thus $U$ is also open in $S$.

  Now suppose that $U$ is open in $S$.
  Then for $x \in  U$ there exists $N \in \textbf{N}(x)$ such that $N \subseteq U$.
  $N$ is a neighborhood of $x$ i $Y$ also.
  Thus there exists open $V_{x} \subseteq N$ such that $x \in V_{x}$.
  Thus since $V_{x} \subseteq N \subseteq U$ we have that $U = \bigcup_{x \in U} V_{x}$.
  Thus $U$ is the union of open sets in $Y$ and $U$ is open in $Y$.

  Overall $S = Y$ and we have that $S$ is the unique topology we want.
\end{pf}

\clearpage

\chapter{Subspaces}

\begin{defn}[Subspace]
  If $X$ is a topological space and $A \subseteq X$, then the \emph{relative topology} or \emph{subspace topology} on $A$ is the collection of intersections of $A$ with open sets of $X$.
  With this topology, $A$ is called a \emph{subspace} of $X$.
\end{defn}

We have a series of basic consequences of this definition of subspace.

\begin{prop}
  If $Y$ is a subspace of $A$, then $A \subseteq Y$ is closed if and only if $A = Y \cap B$ for some closed $B \subseteq X$
\end{prop}
\begin{pf}
  Suppose $A$ is closed in $Y$.
  Then $Y \setminus A$ is open in $Y$.
  Then $Y \setminus A = Y \cap U$ for some open $U \subseteq X$.
  $X \setminus U$ is closed and $(Y \setminus A) \sqcup A = Y = Y \cap U \sqcup (Y \cap (X \setminus U))$.
  This implies $A = Y \cap (X \setminus U)$.

  Now suppose that $A = Y \cap B$ for some closed $B \subseteq X$.
  Then $X \setminus B$ is open in $X$.
  We have that $Y \setminus A = Y \setminus (Y \cap B) = Y \cap (X \setminus B)$ and so $Y \cap A$ is open meaning that $A$ is closed.
\end{pf}

\begin{prop}
  If $X$ is a topological space and $A \subseteq X$, then there exists a largest open set $U$ with $U \subseteq A$.
\end{prop}
\begin{pf}
  Let $\set{U_{i}}_{i \in I}$ be an indexed set of all open sets $\subseteq A$.
  Then $U \defeq \bigcup_{i \in I} U_{i}$ is the largest open set in $A$.
  This is because the union of open sets is open and if $O \subseteq A$ is open then $O = U_{j}$ for some $j$ and thus $O \subseteq U$.
\end{pf}
\begin{defn}[Interior]
  let $X$ be a topological space and $A \subseteq X$.
  The largest open set contained in $A$ is called the \emph{interior} of $A$ in $X$ and is denoted by $\inte(A)$ or $A^{\circ}$.
\end{defn}

\begin{prop}
  If $X$ is a toplogical space and $A \subseteq X$, then there exists a smallest closed set $F$ such that $A \subseteq F \subseteq X$.
\end{prop}
\begin{pf}
  Let $\set{F_{i}}_{i \in I}$ be the indexed set of all closed sets $A \subseteq F_{i} \subseteq X$.
  Then $F \defeq \bigcup_{i \in I} F_{i}$ is the closure of $A$.
  $F$ is closed since the intersection of closed sets is closed and $A \subseteq F$ since $A \subseteq F_{i}$ for all $i \in I$.
  If $A \subseteq C \subseteq X$ is closed then $C = F_{j}$ for some $j$ and thus $F \subseteq C$.
\end{pf}
\begin{defn}[Closure]
  let $X$ be a topological space and $A \subseteq X$.
  The smallest closed set containing $A$ is called the \emph{closure} of $A$ in $X$ and is denoted by $\cl(A)$ or $\overline{A}$.
\end{defn}

We give more precise definitions of the interior and closure of a set.

\begin{prop}\label{prop: defn_of_int_and_cl}
  Let $X$ be a topological space and $A \subseteq X$.
  Then we have that
  \begin{align*}
    \inte(A) &= \set{a \in A | \exists U \text{ open such that } a \in U \subseteq A} \\
    \overline{A} &= \set{x \in X | \forall U \text{ open such that } x \in U,~ U \cap A \neq \emptyset}
  \end{align*}
\end{prop}
\begin{pf}
  Clearly $\set{a \in A | \exists U \text{ open such that } a \in U \subseteq A} \subseteq \inte(A)$.
  Suppose that $x \in \inte(A)$.
  Then $\inte(A)$ is open, a subset of $A$, and contains $x$.
  Thus $x \in \set{a \in A | \exists U \text{ open such that } a \in U \subseteq A}$.

  Suppose that $x \in \set{x \in X | \forall U \text{ open such that } x \in U,~ U \cap A \neq \emptyset}$ and let $C$ be a closed set containing $A$.
  If $x \notin C$ then $X \setminus C$ is an open set which has empty intersection with $A$.
  This is impossible.
  Thus $x \in C$ which implies that $x \in \overline{A}$.
  Now suppose that $x \in \overline{A}$.
  Suppose that $U$ is an open set containing $x$.
  If $U \cap A$ is empty, then $A \subseteq X \setminus U$ which is closed.
  Thus $\overline{A} \subseteq X \setminus U$.
  But $x \in \overline{A}$ which yields a contradiction.
  Thus $U \cap A$ is nonempty.
\end{pf}

\begin{prop}
  Let $X$ be a topological space and $A \subseteq X$.
  Then $A$ is open if and only if $\inte(A) = A$ and $A$ is closed if and only if $\overline{A} = A$.
\end{prop}
\begin{pf}
  Clearly $\inte(A) \subseteq A$ no matter what.
  Then if $A$ is open, it is an open set contained in itself so $\inte(A) \containseq A$.
  Similarly, $\overline{A} \containseq A$ always.
  If $A$ is closed, then $\overline{A} \subseteq A$.
\end{pf}

\begin{prop}
  Let $X$ be a topological space and $A \subseteq X$.
  Then $X \setminus \inte(A) = \overline{X \setminus A}$ and $X \setminus \overline{A} = \inte(X \setminus A)$.
\end{prop}
\begin{pf}
  Let $x \in X \setminus \inte(A)$.
  Then for any open $U$ such that $x \in U$, we have that $U \not\subseteq A$ and so $U \cap (X \setminus A) \neq \emptyset$.
  Thus by \Cref{prop: defn_of_int_and_cl}, $x \in \overline{X \setminus A}$.
  Conversely let $x \in \overline{X \setminus A}$.
  Then for any open set $U$ with $x \in U$, we have $U \cap (X \setminus A) \neq \emptyset$.
  Thus $U \not\subseteq A$ and $x \in X \setminus \inte(A)$.

  Now let $x \in X \setminus \overline{A}$.
  Then by \Cref{prop: defn_of_int_and_cl} there exists an open set $U$ such that $x \in U$ and $U \cap A = \emptyset$.
  Thus $U \subseteq X \setminus A$ and $x \in \inte(X \setminus A)$.
  Suppose now that $x \in \inte(X \setminus A)$.
  Then there is some open set $U$ containing $x$ such that $U \subseteq X \setminus A$.
  Thus $U$ is an open set containing $x$ which has empty intersection with $A$.
  Thus by \Cref{prop: defn_of_int_and_cl} $x \notin \overline{A}$.
\end{pf}

\clearpage

\begin{prop}
  Let $X$ be a topological space and $A, B \subseteq X$.
  Then $\inte(A \cap B) = \inte(A) \cap \inte(B)$ and $\overline{A \cup B} = \overline{A} \cup \overline{B}$.
\end{prop}
\begin{pf}
  Suppose that $x \in \inte(A \cap B)$.
  Then there exists open $U$ such that $x \in U \subseteq A \cap B$.
  Thus $U \subseteq A$ and $U \subseteq B$ and $x \inte(A) \cap \inte(B)$.
  Now suppose $x \in \inte(A) \cap \inte(B)$. Then there exists open sets $U_{1}$ and $U_{2}$ such that $x \in U_{1} \subseteq A$ and $x \in U_{B} \subseteq B$.
  Thus $x \in U_{1} \cap U_{2} \subseteq A \cap B$ which implies that $x \in \inte(A \cap B)$.

  Now suppose that $x \in \overline{A \cup B}$.
  Then for all open $U$ with $x \in U$, $U \cap (A \cup B) \neq \emptyset$.
  Thus for all open $U$ with $x \in U$, $U \cap A$ or $U \cap B$ is nonempty and $x \in \overline{A} \cup \overline{B}$.
  Now suppose that $x \in \overline{A} \cup \overline{B}$.
  Then for all open $U$ with $x \in U$, $U \cap A$ or $U \cap B$  is nonempty.
  Thus $U \cap (A \cup B) \neq \emptyset$ and $x \in \overline{A \cup B}$.
\end{pf}

\begin{prop}\label{prop: continuous_iff_im_closure_subset_closure_im}
  A function between topological spaces $f\colon X \to Y$ is continuous if and only if for all $A \subseteq X$ we have that $f\pqty{\overline{A}} \subseteq \overline{f(A)}$.
\end{prop}
\begin{pf}
  Suppose that $f$ is continuous and let $A \subseteq X$.
  Note that $A \subseteq f^{-1}(f(A)) \subseteq f^{-1}\overline{f(A)}$ which is closed by continuity.
  Thus $\overline{A} \subseteq f^{-1}\overline{f(A)}$.
  This yields that $f\pqty{overline{A}} \subseteq \overline{f(A)}$.

  Now for the converse, consider some closed set $C \subseteq Y$.
  By assumption we have that
  \[
    f\pqty{\overline{f^{-1}(C)}} \subseteq \overline{f\pqty{f^{-1}(C)}} = \overline{C} = C.
  \]
  Thus $\overline{f^{-1}(C)} \subseteq f^{-1}(C)$.
  Since we know that $f^{-1}(C) \subseteq \overline{f^{-1}(C)}$ we have equality and thus $f^{-1}(C)$ is closed as desired.
\end{pf}

\begin{prop}
  Let $X$ be a topological space and $\set{A_{\alpha}}$ an indexed collection of subsets of $X$.
  \begin{enumerate}
  \item $\Bigcap_{\alpha} \inte(A_{\alpha}) \containseq \inte\pqty{\Bigcap_{\alpha} A_{\alpha}} = \inte\pqty{\Bigcap_{\alpha} \inte(A_{\alpha})}$.
  \item $\Bigcup_{\alpha} \overline{A_{\alpha}} \subseteq \overline{\Bigcup_{\alpha} A_{\alpha}} = \overline{\Bigcap_{\alpha} \overline{A_{\alpha}}}$.
  \item $\Bigcup_{\alpha} \inte(A_{\alpha}) \subseteq \inte\pqty{\Bigcup_{\alpha} A_{\alpha}}$.
  \item $\Bigcap_{\alpha} \overline{A_{\alpha}} \containseq \overline{\Bigcap_{\alpha} A_{\alpha}}$.
  \end{enumerate}
\end{prop}
\begin{pf}\
  \begin{enumerate}
  \item In the finite case, by induction, we have that $\Bigcap_{\alpha} \inte(A_{\alpha}) \subseteq \inte\pqty{\Bigcap_{\alpha} A_{\alpha}}$.
        We actually have equality in the case of finite intersection.

        Note that $\inte\pqty{\bigcap_{\alpha} A_{\alpha}}$ is open and $\inte\pqty{\bigcap_{\alpha} A_{\alpha}} \subseteq \bigcap_{\alpha} \inte(A_{\alpha})$ and so $\inte\pqty{\bigcap_{\alpha} A_{\alpha}} \subseteq \inte\pqty{\bigcap_{\alpha} \inte(A_{\alpha})}$.
        Now let $x \in \inte\pqty{\bigcap_{\alpha} \inte(A_{\alpha})}$.
        Then there is some open $U$ with $x \in U$ such that $U \subseteq \bigcap_{\alpha} \inte(A_{\alpha})$.
        Thus $U \subseteq \inte(A_{\alpha}) \subseteq A_{\alpha}$ for all $\alpha$ and so $U \subseteq \bigcap_{\alpha} A_{\alpha}$ and $x \in \inte\pqty{\bigcap_{\alpha} A_{\alpha}}$.
  \item Let $x \in \bigcup_{\alpha} \overline{A_{\alpha}}$.
        Then for some $\alpha$, $x \in \overline{A_{\alpha}}$.
        So for all open $U$ with $x \in U$, $U \cap A_{\alpha} \neq \emptyset$.
        Thus for all open $U$ with $x \in U$, $U \cap \pqty{\bigcup_{\alpha} A_{\alpha}} \neq \emptyset$ and $x \in \overline{\bigcup_{\alpha} A_{\alpha}}$.

        $A_{\alpha} \subseteq \overline{A_{\alpha}}$ for all $\alpha$.
        So $\overline{\bigcup_{\alpha} A_{\alpha}} \subseteq \overline{\bigcup_{\alpha} \overline{A_{\alpha}}}$.
        Now let $x \in \overline{\bigcup_{\alpha} \overline{A_{\alpha}}}$.
        Then for all open $U$ with $x \in U$, $U \cap \bigcup_{\alpha} A_{\alpha} \neq \emptyset$.
        Since $\bigcup_{\alpha} \overline{A_{\alpha}} \subseteq \overline{\bigcup_{\alpha} A_{\alpha}}$, we have that $U \cap \overline{\bigcup_{\alpha} A_{\alpha}} \neq \emptyset$.
        Thus $x \in \overline{\bigcup_{\alpha}}$.
  \item Suppose that $x \in \bigcup_{\alpha} \inte(A_{\alpha})$.
        Then there exists $\alpha$ such that $x \in \inte(A_{\alpha})$.
        Thus there exists open $U$ with $x \in U$ such that $U \subseteq A_{\alpha}$.
        So $U \subseteq \bigcup_{\alpha} A_{\alpha}$ and overall $x \in \inte\pqty{\bigcup_{\alpha} A_{\alpha}}$.
  \item Let $x \in \overline{\bigcap_{\alpha} A_{\alpha}}$.
        Then for all open $U$ with $x \in U$, $U \cap \bigcap_{\alpha} A_{\alpha} \neq \emptyset$.
        Thus for all $\alpha$, $U \cap A_{\alpha} \neq \emptyset$ and $x \in \overline{A_{\alpha}}$ for all $\alpha$.
        Thus $x \in \bigcap_{\alpha} \overline{A_{\alpha}}$.
  \end{enumerate}
\end{pf}

\begin{ex}[Interior of Intersection is not Equal to Intersection of Interiors]
  We have that $\bigcap_{\alpha} \inte(A_{\alpha}) \containseq \inte\pqty{\bigcap_{\alpha} A_{\alpha}}$.
  We also know that we have equality in the case of finite intersections.
  However in general we have that $\bigcap_{\alpha} \inte(A_{\alpha}) \not\subseteq \inte\pqty{ \bigcap_{\alpha} A_{\alpha}}$.
  Let $A_{n} \defeq \pqty{1 - \frac{1}{n}, 1 + \frac{1}{n}} \subseteq \R$ for all $n \geq 1$.
  Then
  \begin{align*}
    \Bigcap_{n \geq 1} \inte(A_{n}) &= \Bigcap_{n \geq 1} \pqty{1 - \frac{1}{n}, 1 + \frac{1}{n}} = \set{1}; \text{ and } \\
    \inte\pqty{\Bigcap_{n \geq 1} A_{n}} &= \inte \set{1} = \emptyset.
  \end{align*}
\end{ex}

\begin{ex}[Union of Closures is not Equal to Closure of Union]
  We have that $\bigcup_{\alpha} \overline{A_{\alpha}} \subseteq \overline{\bigcup_{\alpha} A_{\alpha}}$.
  However in general we have that  $\overline{\bigcup_{\alpha} A_{\alpha}} \not\subseteq \bigcup_{\alpha} \overline{A_{\alpha}}$.
  Let $A_{n} = \left( \frac{1}{n}, 1 \right]$ for $n \geq 1$.
  Then
  \begin{align*}
    \Bigcup_{n \geq 1} \overline{\left( \frac{1}{n}, 1 \right]} &= \Bigcup_{n \geq 1} \bqty{\frac{1}{n}, 1} =  (0, 1] \\
    \overline{\Bigcup_{n \geq 1} \left( \frac{1}{n}, 1 \right]} &= \overline{(0, 1]} = [0, 1]
  \end{align*}
\end{ex}

\begin{ex}[Union of Interiors is not Equal to Interior of Union]
  We have that $\bigcup_{\alpha} \inte(A_{\alpha}) \subseteq \inte\pqty{\bigcup_{\alpha} A_{\alpha}}$.
  However in general we have that $\inte\pqty{\bigcup_{\alpha} A_{\alpha}} \not\subseteq \bigcup_{\alpha} \inte(A_{\alpha})$.
  Let $A_{1} = \bqty{0, \frac{1}{2}}$ and let $A_{2} = \bqty{\frac{1}{2}, 1}$.
  Then $\inte(A_{1} \cup A_{2}) = \inte([0, 1]) = (0, 1)$.
  However $\inte(A_{1}) \cup \inte(A_{2}) = \pqty{0, \frac{1}{2}} \cup \pqty{\frac{1}{2}, 1}$.
\end{ex}

\begin{ex}[Intersection of Closures is not Equal to Closure of Intersection]
  We have that $\overline{\bigcap_{\alpha} A_{\alpha}} \subseteq \bigcap_{\alpha} \overline{A_{\alpha}}$.
  However in general we have that $\bigcap_{\alpha} \overline{A_{\alpha}} \not\subseteq \overline{\bigcap_{\alpha} A_{\alpha}}$.
  Let $A_{1} = \pqty{0, \frac{1}{2}}$ and $A_{2} = \pqty{\frac{1}{2}, 1}$.
  Then $\overline{A_{1} \cap A_{2}} = \overline{\emptyset} = \emptyset$.
  However $\overline{A_{1}} \cap \overline{A_{2}} = \bqty{0, \frac{1}{2}} \cap \bqty{\frac{1}{2}, 1} = \set{\frac{1}{2}}$.
\end{ex}

\clearpage

\begin{prop}
  Let $X$ be a topological space and $A, B$ subsets of $X$.
  If $A \subset B$, then we have that $\overline{A} \subseteq \overline{B}$ and $\inte(A) \subseteq \inte(B)$.
\end{prop}
\begin{pf}
  Suppose $A \subseteq B$ and $x \in \overline{A}$.
  Then for all open sets $U$ with $x \in U$ we have $U \cap A \neq \emptyset$.
  Thus $U \cap B \neq \emptyset$ and $x \in \overline{B}$.

  Now let $x \in \inte(A)$.
  Then there exists open $U$ such that $x \in U \subseteq A \subseteq B$.
  Thus $x \in \inte(B)$.
\end{pf}

We may specify the space in which a closure is taken with the notation $\overline{A}^{X}$.
However, we do not need this notation very often.

\begin{prop}
  If $A \subseteq Y \subseteq X$, then $\overline{A}^{Y} = \overline{A}^{X} \cap Y$.
  Thus if $Y$ is closed, then $\overline{A}^{Y} = \overline{A}^{X}$.
\end{prop}
\begin{pf}
  Suppose $a \in \overline{A}^{Y}$.
  Then $a \in X$ since $\overline{A} \subseteq Y \subseteq X$.
  Thus $A \subseteq \overline{A} \subseteq X$ and since $a \in Y$ also, we have that $a \in \overline{A}^{X} \cap Y$.
  The reverse inclusion is also a similar argument and thus $\overline{A}^{Y} = \overline{A}^{X} \cap Y$.
  If $Y$ is closed, then $\overline{A} \subseteq Y$ and so $\overline{A}^{X} \cap Y = \overline{A}^{X}$.
\end{pf}

\begin{prop}
  If $Y \subseteq X$ then the set of intersections of $Y$ with a basis of $X$ is a basis of the relative topology of $X$.
\end{prop}
\begin{pf}
  By definition, intersections of $Y$ with elements of a basis of $X$ are open in $Y$.
  Now suppose that $U$ is open in $Y$.
  Then $U = V \cap Y$ for some $V$ open in $X$.
  But $V$ is the union of basis members of $X$.
  Thus $U$ is the union of the intersection of $Y$ with basis members of $X$.
  Overall we must have that the set of intersections of $Y$ with a basis of $X$ is a basis of the relative topology of $X$.
\end{pf}

\begin{prop}
  If $X, Y, Z$ are topological spaces and $Z$ is a subspace of $Y$, and $Y$ is a subspace of $X$, then $Z$ is a subspace of $X$.
\end{prop}
\begin{pf}
  Let $U \subseteq Z$ be open.
  Then $U = Z \cap U'$ for some open $Z' \subseteq Y$.
  But $U' = Y \cap U''$ for some open $U'' \subseteq X$.
  Note that $Z \subseteq Y$ so $Z \cap Y = Z$.
  Thus $U = Z \cap U' = Z \cap Y \cap U'' = Z \cap U''$ and overall $Z$ is a subspace of $X$.
\end{pf}

\begin{prop}
  If $X$ is a metric space and $A \subseteq X$, then $\overline{A}$ coincides with the set of limits in $X$ of sequences of points in $A$.
\end{prop}
\begin{pf}
  If $x$ is the limit point of a sequence of points in $A$, then any open sets about $x$ contains a point of $A$.
  Thus $x \notin \inte(X \setminus A)$.
  We have that $X \setminus \inte(X \setminus A) = \overline{A}$ and so $x \in \overline{A}$.

  Now suppose $x \in \overline{A}$.
  $B_{1 / n}(x)$ must contain a point in $A$.
  If it didn't then $x \in \int(X \setminus A)$ which is disjoint from $\overline{A}$.
  Let $x_{n}$ be a point in $A$ which is also in $B_{a / n}(x)$.
  Thus $x$ is a limit of a sequence of points in $A$.
\end{pf}

\clearpage

\begin{prop}\label{prop: closed_dist_inf_0}
  Let $X$ be a metric space $d$.
  Recall that for $C \subseteq X$, $x \in X$, that $d(x, C) \defeq \inf\set{d(x, c) | x \in C}$.
  Let $C$ be closed.
  Then $d(x, C) = 0$ if and only if $x \in C$
\end{prop}
\begin{pf}
  If $x \in C$ then the claim is clear.
  Now suppose $x \in X$ such that $d(x, C) = 0$.
  Then for all $r > 0$ we have that $B_{r}(x) \cap C \neq \emptyset$.
  Thus $x \in \overline{C} = C$.
\end{pf}

\begin{defn}[Boundary]
  If $X$ is a topological space and $A \subseteq X$, then the \emph{boundary} or \emph{frontier} of $A$ is defined to be $\partial(A)$ or $\bdry(A) = \overline{A} \cap \overline{X \setminus A}$.
\end{defn}

\begin{defn}[Dense and Nowhere Dense Sets]
  A subset $A$ of a topological space $X$ is called \emph{dense} in $X$ if $\overline{A} = X$.
  A subset $A$ is said to be \emph{nowhere dense} in $X$ if $\inte(\overline{A}) = \emptyset$.
\end{defn}

\begin{defn}[Separable Space]
  A topological space is \emph{separable} if it has a countable dense set.
\end{defn}

\begin{ex}[Interior, Closure, and Density in Discrete Topology and Indiscrete Topology]
  Let $X$ be the discrete topology and $A \subseteq X$.
  We have that $\inte(A) = A$ since every singleton is open.
  For $x \in A$ we have that $\set{x} \cap A \neq \emptyset$ but for $x \notin A$ we have that $\set{x} \cap \overline{A} \emptyset$ which implies that $\overline{A} = A$.
  Thus $\partial(A) = \overline{A} \cap \overline{X \setminus A} = A \cap (X \setminus A) =
  \emptyset$.
  We can see that the only dense subset of $X$ is $X$ itself and the only nowhere dense subset is $\emptyset$.

  Now let $X$ be the indiscrete topology and $A \neq X$.
  Then we have that $A^{\circ} = \overline{A^{\circ}}^{\circ} = \emptyset$.
  Now suppose $A \neq \emptyset$.
  Then we have $\overline{A} = \overline{\inte{\overline{A}}} = X$.
  If we have that $A \neq X$ and $A \neq \emptyset$, then we have $\partial(A) = \overline{A} \cap \overline{X \setminus A} = X \cap X = X$ and $\partial \partial(A) = \partial(X) = \emptyset$.
  Every non-empty subset of $X$ is dense and so in particular, singletons are dense meaning that $X$ is separable.
  In fact, by \Cref{exercise: countable_iff_countable_dense} we have that $X$ is thus second countable.

\end{ex}

\clearpage

\subsection*{Exercises}

\begin{exercise}[\tcite{book:Bredon} 1.3.2]
  For $A \subseteq X$, we have that $X = \inte(A) \sqcup \partial(A) \sqcup X \setminus \overline{A}$.
\end{exercise}
\begin{pf}
  Recall that $X \setminus \overline{A} = \inte(X \setminus A)$ and $\partial A \defeq \overline{A} \cap \overline{X \setminus A}$.
  Let $x \in \inte(A)$.
  Then there exists open $N \subseteq A$ such that $x \in N$.
  Thus $N \cap (X \setminus A) = \emptyset$ and $x \notin \overline{X \setminus A}$.
  So $x \notin \partial A$.
  Now suppose that $x \in \partial A = \overline{A} \cap \overline{X \setminus A}$.
  Then for every open set $U$ such that $x \in U$, we have $U \cap (X \setminus A) \neq \emptyset$ and so in particular $U \not\subseteq A$.
  Thus overall $\inte(A)$ and $\partial A$ are disjoint.
  We automatically have that $\partial A$ and $X \setminus \overline{A}$ are disjoint since $x \in \partial A \implies x \in \overline{A}$.

  Now let $x \in X$ such that $x \notin \inte(A) \sqcup X \setminus \overline{A}$.
  $x \notin \inte(A)$ so for all open $U$ such that $x \in U$ we have $U \not\subseteq A$.
  So $U \cap (X \setminus A) \neq \emptyset$ for all open $U$ containing $x$.
  Thus $x \in \overline{X \setminus A}$.
  Also $x \notin X \setminus \overline{A}$ so $x \in \overline{A}$.
  Thus $x \in \partial A$ and overall $X = \inte(A) \sqcup \partial(A) \sqcup X \setminus \overline{A}$.
\end{pf}

\begin{exercise}[\tcite{book:Bredon} 1.3.3]\label{exercise: countable_iff_countable_dense}
  Show that a metric space $X$ is second countable if and only if it has a countable dense set.
\end{exercise}
\begin{pf}
  Suppose that $X$ is second countable with a countable basis $\set{B_{i}}_{i \in \N}$.
  Without loss of generality, suppose all the $B_{i}$ are non-empty.
  Let $x_{i} \in B_{i}$ and $D = \set{x_{i} \mid i \in \N}$.
  We claim that $D$ is dense in $X$.
  Clearly $\overline{D} \subseteq X$ so let $x \in X$.
  Then for all open $U$ with $x \in U$, we have $U \cap D \neq \emptyset$.
  This is because $U$ is a union of members of $\set{B_{i}}$ and so contains some $x_{i}$.

  Now let $D$ be a countable dense set of $X$.
  Let $B = \set{B_{q}(x) | x \in D,~ q \in \Q^{\times}}$.
  We claim that $B$ is a countable basis for $X$.
  Let $U$ be an open set in $X$.
  Then for all $x \in U$, there exists $\e > 0$ such that $B_{\e}(x) \subseteq U$.
  $D$ is dense in $X$ and so there exists $d_{x} \in D$ such that $d_{x} \in B_{\e /2}(x)$.
  By the density of $\Q$ in $\R$, there exists $q_{x} \in \Q$ such that $\dist(x, d_{x}) < q_{x} < \frac{\e}{2}$.
  Thus $B_{q}(d_{x})$ contains $x$ and $B_{d_{x}}(x) \subseteq B_{e}(x) \subseteq U$ which means that
  \[
    U = \bigcup_{x \in U} B_{q_{x}}(d_{x})
  \]
  and $B$ is a countable basis for $X$.
\end{pf}

\begin{exercise}[\tcite{book:Bredon} 1.3.4]
  The union of two nowhere dense sets is nowhere dense.
\end{exercise}
\begin{pf}
  Let $A$ and $B$ be two nowhere dense subsets of a topological space $X$.
  Suppose we have some non-empty $U \subseteq \overline{A \cup B} = \overline{A} \cup \overline{B}$.
  Let $V = U \cap (X \setminus \overline{A})$.
  Note that $V$ is open.
  Suppose that $V$ was empty, then $U \subseteq \overline{A}$ which is impossible.
  Thus $V$ is nonempty.
  $U \subseteq \overline{A} \cup \overline{B}$ and has non-empty intersection with $X \setminus \overline{A}$ and thus $V \subseteq \overline{B}$.
  This is also impossible meaning that $U$ must be empty.
\end{pf}

\clearpage

\begin{defn}[Irreducible Space]
  A topological space $X$ is said to be \emph{irreducible} if whenever $X = F \cup G$ with $F, G$ closed we have $X = F$ or $X = G$.
  A subspace is irreducible if it is irreducible in the subspace topology.
\end{defn}

\begin{defn}[Zariski Space]
  A \emph{Zariski} space is a topological space such that every descending chain $F_{1} \containsneq F_{2} \containsneq F_{3} \containsneq \cdots$ of closed sets is eventually constant.
\end{defn}

\begin{exercise}[\tcite{book:Bredon} 1.3.8]
  Let $X = A \cup B$ where $A, B$ are closed.
  Let $f\colon X \to Y$ be a function.
  If $f\mid_{A}$ and $f\mid_{B}$ are both continuous, then $f$ is continuous.
\end{exercise}
\begin{pf}
  Let $U \subseteq Y$ be open.
  Then $f\mid_{A}^{-1}(U)$ and $f\mid_{B}^{-1}(U)$ are both open.
  Thus their union is open.
  We have
  \begin{align*}
    f\mid_{A}^{-1}(U) \cup f\mid_{B}^{-1}(U) &= \set{x \in A | f(x) \in U} \cup \set{x \in B | f(x) \in U} \\
                                             &=  \set{x \in A \cup B | f(x) \in U} \\
    &= f^{-1}(U)
  \end{align*}
  and so $f$ is continuous.
\end{pf}

\clearpage

\chapter{Connectivity and Components}

Intuitively, a connected space is a space where you can move from one space to another with no jumps.
Another intuition is that the space doesn't have two or more separated pieces.

\begin{defn}[Connected Sets and Separation]
  A topological space $X$ is \emph{connected} if it is not the disjoint union of two nonempty open subsets.
  If $A, B$ are two disjoint nonempty open subsets of $X$ such that $A \sqcup B = X$ then we say that $A$ and $B$ form a \emph{separation} of $X$.
\end{defn}

\begin{defn}[Clopen Sets]
  A subset $A$ of a topological space $X$ is \emph{clopen} if it is both open and closed in $X$.
\end{defn}

\begin{prop}
  A topological space $X$ is connected if and only if the open clopen sets are $X$ and $\emptyset$.
\end{prop}
\begin{prop}
  Suppose $X$ is connected and suppose there exists clopen $\emptyset \subsetneq U \subsetneq X$.
  Then $X \setminus U$ is also clopen and $X = (X \setminus U) \sqcup U$ forms a separation of $X$, contradiction.
\end{prop}

\begin{defn}[Discrete Valued Maps (DVM)]
  A \emph{discrete valued map (DVM)} is a map from a topological space $X$ to a discrete space $D$.
\end{defn}

\begin{prop}
  A topological space $X$ is connected if and only if every discrete valued map on $X$ is constant.
\end{prop}
\begin{pf}
  If $X$ is connected and $d\colon X \to D$ is a DVM and $y$ is in the range of $d$, then $d^{-1}(y)$ is clopen in and non-empty.
  Thus $d^{-1}(y) = X$ and $d$ is constant.

  Now suppose that $X$ is not connected and $U \sqcup V$ form a separation of $X$.
  Then $d\colon X \to \set{0, 1}$ where $d(x) = 0$ if and only if $x \in U$ forms a nonconstant DVM.\@
\end{pf}

\begin{prop}\label{prop: image_of_connected_is_connected}
  If $f\colon X \to Y$ is continuous and $X$ is connected, then $f(X)$ is connected.
\end{prop}
\begin{pf}
  Let $d\colon f(X) \to D$ be a DVM.\@
  Then $d \circ f$ is a DVM on $X$ and thus constant.
  This implies that $d$ is constant and thus $f(X)$ is connected.
\end{pf}

\begin{prop}
  If $\set{Y_{i}}_{i \in I}$ is a collection of connected sets in a topological space $X$ such that no two $Y_{i}$ are disjoint, then $\bigcup_{i \in I} Y_{i}$ is connected.
\end{prop}
\begin{pf}
  Let $d\colon \bigcup_{i \in I} Y_{i} \to D$ be a DVM.\@
  Let $p, q \in \bigcup_{i \in I} Y_{i}$ such that $p \in Y_{i}$, $q \in Y_{j}$ and $v \in Y_{i} \cap Y_{j}$.
  Then $d(p) = d(v) = d(q)$ and $d$ is constant which yields that $\bigcup_{i \in I} Y_{i}$ is connected.
\end{pf}

\begin{cor}\label{cor: connected_subset_eq}
  The relation ``$p \sim q$ if and only if $p$ and $q$ belong to a connected subset of $X$'' is an equivalence relation.
\end{cor}
\begin{pf}
  Immediate.
\end{pf}

\begin{defn}[Components]
  The equivalence classes of the relation stated in \Cref{cor: connected_subset_eq} are called the \emph{components} of $X$.
  These are the ``maximal'' connected subsets of $X$.
\end{defn}


\begin{lem}\label{lem: closure_of_connected_is_connected}
  Let $X$ be a connected set.
  Then $\overline{X}$ is connected.
\end{lem}
\begin{pf}
  Let $d\colon \overline{X} \to \set{0, 1}$ be a DVM.\@
  $X$ is connected so $f(X) =$, without loss of generality, $\set{0}$.
  Then $\set{0}$ is closed so $d^{-1}(0)$ is closed and contains $X$.
  Thus $\overline{X} \subseteq d^{-1}(0) \subseteq \overline{X}$ which means that $f$ is constant.
\end{pf}

\begin{prop}
  Components of a topological space $X$ are connected and closed.
  Each connected subset of $X$ is contained in a component.
  Components are equal or disjoint and their union is $X$.
\end{prop}
\begin{pf}
  The last sentence is immediate based off of \Cref{cor: connected_subset_eq}.
  By definition, the component of $X$ containing a point $p$ is the union of all connected sets containing $p$, which itself is connected.
  This implies that connected sets lie in components.
  We have by \Cref{lem: closure_of_connected_is_connected} that since a component $C$ is connected $\overline{C}$ is connected and since $C \subseteq \overline{C}$ we overall have that $C = \overline{C}$ which means that $C$ is closed.
\end{pf}

\begin{prop}\label{prop: dvm_eq_rel}
  The relation ``$p \sim q$ if and only if $d(p) = d(q)$ for every discrete valued map $d$ on $X$'' is an equivalence relation on $X$.
\end{prop}
\begin{pf}
  Immediate.
\end{pf}

\begin{defn}[Quasi-components]
  The equivalence classes of the relation stated in \Cref{prop: dvm_eq_rel} are called the \emph{quasi-components} of $X$.
\end{defn}

\clearpage

\begin{prop}
  Quasi-components of a space $X$ are closed.
  Each connected set is contained in a quasi-component and in particular each component is contained in a quasi-component.
  Quasi-component are either equal or disjoint and their union is $X$.
\end{prop}
\begin{pf}
  The last statement is immediate based off of \Cref{prop: dvm_eq_rel}.
  If $p \in X$, then the quasi-component is $\set{q \in X | d(q) = d(p) \text{ for all DVM's } d \text{ on } X}$.
  But this is $\bigcap \Set{d^{-1}(d(p)) | d \text{DVM on } X}$.
  We have that $d^{-1}(d(p))$ is closed since $d$ is continuous.
  The intersection of closed sets is closed, so quasi-components are closed.
  Components are constant on every DVM, so they are contained in some quasi-component.
\end{pf}

There is another closely related notion of connectedness which can be easier to deal with.

\begin{defn}[Arcwise / Pathwise Connected Space]
  A topological space $X$ is \emph{arcwise connected} or \emph{pathwise connected} if for any two points $p,q \in X$ there exists a map $\lambda\colon [0, 1] \to X$ with $\lambda(0) = p$ and $\lambda(1) = q$.
  $\lambda$ is called a \emph{path}.
\end{defn}

\begin{defn}[Locally Arcwise Connected Space]
  A topological space $X$ is \emph{locally arcwise connected} if every neighborhood of any point contains an arcwise connected neighborhood.
\end{defn}

We now prove some immediate properties of arcwise connectivity relating it to the other definition of connectivity we have seen.

\begin{prop}
  An arcwise connected space is connected.
\end{prop}
\begin{pf}
  Suppose $X$ is arcwise connected but not connected with separation $X = U \sqcup V$.
  Let $p \in U$ and $q \in V$ with $\lambda$ the path connecting the two points.
  $\lambda$ is continuous and so $\lambda^{-1}(U), \lambda^{-1}(V)$ are open and disjoint and thus form a separation of $[0, 1]$.
  This implies $[0, 1]$ is not connected which is a contradiction of the fact that $[0, 1]$ is connected (\Cref{exercise:unit_interval_connected}).
\end{pf}

\begin{prop}\label{prop: arc_eq_rel}
  The relation ``$p \sim q$ if and only if there exists a map $\lambda\colon [0, 1] \to X$ with $\lambda(0) = p$ and $\lambda(1) = q$'' over a space $X$ is an equivalence relation.
\end{prop}
\begin{pf}
  Straightforward.
\end{pf}

\begin{defn}[Arc Component]
  The equivalence classes of \Cref{prop: arc_eq_rel} are called \emph{arc components} and they are the maximally arcwise connected subsets of a space $X$.
\end{defn}

\begin{prop}
  An arc component of a space is contained in some component.
\end{prop}
\begin{pf}
  Arc components are connected and so an arc component must be contained in some component.
\end{pf}

\clearpage

\begin{prop}
  Arc components of locally arcwise connected space $X$ are clopen and coincide with the components.
\end{prop}
\begin{pf}
  We already know that arc components are connected which means they are closed.
  Now let $A$ be an arc component in a locally arcwise connected space $X$ and $x \in A$.
  $X$ is a neighborhood of $x$ and so there exists an arcwise connected neighborhood of $x$, call it $V$.
  $V$ is arcwise connected and contains $x$ so $V \subseteq A$.
  $V$ is a neighborhood of $x$ and so it contains an open subset $U$ such that $x \in U \subseteq V \subseteq A$ and thus $A$ is also open.

  Now let $C$ be the component that $A$ is contained in and suppose $A \subsetneq C$.
  let $Q$ be the union of all path connected components $\neq A$ that intersect $C$.
  $Q \subseteq C$ and since arc components are disjoint we have $C = A \sqcup Q$.
  $A$ is open and  $Q$ is open meaning that they form a separation of $C$ which is a contradiction to the fact that $C$ is connected.
  Thus $A = C$.
\end{pf}

\begin{ex}[An Arcwise Connected Space with Two Points]
  Let the two points be $p$ and $q$ and suppose we have open sets $\emptyset, \set{p}$ and $\set{p, q}$.
  Clearly $p$ and $q$ are arcwise connected to themselves.
  $p$ is arcwise connected to $q$ with $\lambda\colon [0, 1] \to \set{p, q}$ such that $\lambda(1) = q$ and $\lambda(x) = p$ otherwise.
  Note that $[0, 1)$ is open in $[0, 1]$ (but not in $\R$).
  Thus we have that $\lambda^{-1}(p)$ is open and $\lambda^{-1}(q)$ is closed as desired and $\lambda$ is our desired path.
  Thus this space is arcwise connected.
\end{ex}

\clearpage

\begin{ex}[Topologist's Sine Curve, A Space that is Connected but not Arcwise Connected]
  The \emph{Topologist's Sine Curve} is the subset of the real plane
  \[
    S = \Set{\pqty{x, \sin\pqty{\frac{1}{x}}} | 0 < x \leq 1 } \subseteq \R^{2}.
  \]

  \begin{figure}[h]
    \centering
    \includegraphics[width=0.5\textwidth]{figs/Topologists_Sine_Curve}
    \caption{The Topologist's Sine Curve}\label{fig:topologists_sine_curve}
  \end{figure}

  Clearly $S$ is arcwise connected and thus connected.
  This is because $\sin\pqty{\frac{1}{x}}$ is continuous and so we can just parameterize along the path.

  Consider the set
  \[
    T = (\set{0} \times [-1, 1]) \sqcup \Set{\pqty{x, \sin\pqty{\frac{1}{x}}} | 0 < x \leq 1 } \subseteq \R^{2}.
  \]
  We will show this path is connected but not arcwise connected.
  The intuition here is that we cannot ``get to'' a portion of the $y$-axis in finite time starting from a point on the sine curve due to the rapid oscillation.

  We claim that $\overline{S} = T$.
  Note that the function $\sin\pqty{\frac{1}{x}}$ is continuous for all $x > 0$ and so every point on it is a limit point.
  Now consider a point $(0, y)$ where $y \in [-1, 1]$.
  Then due to the rapid oscillation of $\sin\pqty{\frac{1}{x}}$ as $x \to 0$ we have that every open ball around $(0, y)$ has non-trivial intersection with the curve infinitely often and so $(0, y)$ is a limit point of some sequence of points in $S$.
  Thus every point in $T$ is a limit point of some sequence of points in $S$ and $T \subseteq \overline{S}$.

  To show that $\overline{S} \subseteq T$ it suffices to show that $T$ is closed since $S \subseteq T$.
  Closed sets contain their limit points so let $\set{(x_{n}, y_{n})}$ be a sequence of points in $T$ with limit $(x, y)$.
  We know that $x = \lim x_{n}$ and $y = \lim y_{n}$ so $x \geq 0$ and $y \in [-1, 1]$.
  If $x = 0$ then we are done since $(0, y) \in T$ for $y \in [-1, 1]$.
  So now suppose $x > 0$ and we can assume that after dropping some number of starting terms $x_{n} > 0$ for all $n$.
  Thus $(x_{n}, y_{n})$ lies on the curve $\sin\pqty{\frac{1}{x}}$.
  Since this function is continuous we get that $(x, y)$ lies on the curve and thus $T$ is closed.
  Thus $\overline{S} \subseteq T$ and overall we have equality.

\clearpage

  Note that since $S$ is connected, since it is arcwise connected, we have that $T$ is also connected by \Cref{lem: closure_of_connected_is_connected}. We now show that $T$ is not path-connected.
  Suppose that we have some path $\lambda$ connecting a point $\lambda(0) = p$ on the sine curve to $\lambda(1) = (0, 1)$.
  Let $\e = \frac{1}{2}$.
  By continuity of $\lambda$ there is some $\delta > 0$ such that $\abs{\lambda(t) - (0, 1)} < \frac{1}{2}$ for $t \in [1 - \delta, 1]$.
  The sine curve keeps escaping the disc of radius $\frac{1}{2}$ around $(0, 1)$.
  In particular this means that we cannot have continuity since the curve will eventually be of distance $> \frac{1}{2}$ from $(0, 1)$.
  Thus $T$ is not path-connected.
\end{ex}

\clearpage

\subsection*{Exercises}

\begin{exercise}[\tcite{book:Bredon} 1.4.1]
  If $A$ is a connected subset of a topological space $X$ and $A \subseteq B \subseteq \overline{A}$ then $B$ is connected.
\end{exercise}
\begin{pf}
  Suppose $U, V$ are a separation of $B$.
  Then note that $A = (U \cap A) \sqcup (V \cap A)$.
  But $U \cap A$ and $V \cap A$ are open in $A$ and since $A$ is connected we have that one of these, say $U \cap A$, is empty.
  Thus $A \subseteq X \setminus U$ which is closed.
  This means that $\overline{A} \subseteq X \setminus U$.
  So overall now we have $U \subseteq B \subseteq \overline{A} \subseteq X \setminus U$.
  This can hold if and only if $U$ is empty.
  Thus $B$ is connected.
\end{pf}

\begin{defn}[Locally Connected Space]
  A topological space $X$ is \emph{locally connected} if for each $x \in X$ and each neighborhood $N$ of $x$, there is a connected neighborhood $V$ of $x$ such that $V \subseteq N$.
\end{defn}

\begin{exercise}[\tcite{book:Bredon} 1.4.2]
  If $X$ is locally connected, then its components are open and equal to its quasi-components.
\end{exercise}
\begin{pf}
  Let $C$ be a component of $X$ and $x \in C$.
  Then $X$ itself is a neighborhood of $x$ and so there is a connected neighborhood $V$ of $x$ such that $V \subseteq X$.
  $V$ is connected so $V \subseteq C$.
  Since $V$ is a neighborhood of $x$, there exists an open $U \subseteq V$ such that $x \in U \subseteq C$.
  Thus $C$ is open.

  We already know that components are contained in quasi-components.
  Let $C$ be a component and $Q$ the quasi-component containing $C$ and suppose $C \subsetneq Q$.
  $C$ is open and we know that components are also closed and so $C$ is clopen in $X$.
  Let $x \in C$ and $y \in Q \setminus C$.
  Then $X = C \sqcup (Q \setminus C)$ is a separation of $X$.
  Thus there exists a DVM $d$ such that $d(x) \neq d(y)$ meaning that $x$ and $y$ are in different quasi-components.
  But $x \in C \subseteq Q$ and $y \in Q \setminus C \subseteq Q$ which is a contradiction and so $C = Q$.
\end{pf}

\begin{exercise}[\tcite{book:Bredon} 1.4.3]\label{exercise:unit_interval_connected}
  The unit interval $[0, 1]$ is connected.
\end{exercise}
\begin{pf}
  Suppose that $U, V$ formed a separation of $x$ such that $1 \in V$.
  Let $x = \sup(U) \in [0, 1]$ since the unit interval is closed.
  Suppose $x \in U$, then since $U$ is open there exists $\e > 0$ such that $B_{\e}(x) \subseteq U$ meaning that $x + \frac{\e}{2} \in U$ which is a contradiction to our choice of $x$.
  Thus $x \in V$ which is also open meaning there exists $\e > 0$ such that $B_{\e}(x) \subseteq V$.
  Thus $x - \frac{\e}{2} \in V$.
  Suppose that $x - \frac{\e}{2}$ is not an upper bound for $U$.
  Then there is some $p \in U$ such that $p > x - \frac{\e}{2}$.
  Thus $p \in \pqty{\frac{x - \e}{2}, x} \subseteq B_{\e}(x) \subseteq V$ and so $p \in V$ contradicting the disjointness of $U$ and $V$.
  Thus $x - \frac{\e}{2}$ is a lower upper bound of $U$ than $x = \sup(U)$ which is a contradiction.
  Thus we have found $x \in [0, 1]$ such that $x \notin U$ and $x \notin V$ which is a contradiction as well and $[0, 1]$ is connected.
\end{pf}

\clearpage

\chapter{Separation Axioms}

\begin{defn}[Separation Axioms, Hausdorff, Regular, Normal Spaces]
  The \emph{Separation Axioms}:
  \begin{itemize}
  \item[(T$_{0}$)] A topological space $X$ is called a \emph{T$_{0}$-space} if for any two points $x \neq y$ there is an open set containing one of them but not the other.
                   This says that point can be distinguished by the open sets where they lie.
  \item[(T$_{1}$)] A topological space $X$ is called a \emph{T$_{1}$-space} if for any two points $x \neq y$ there is an open set containing $x$ but not $y$ and another open set containing $y$ but not $x$.
        This says that singletons, and thus finite sets, are closed.
        Let $x \in X$.
        For each point $y \neq x$ let $U_{y}$ be an open set containing $y$ but not $x$.
        Then $X \setminus \set{x} = \bigcup_{y} U_{y}$ which is open and thus $\set{x}$ is closed.
        Conversely if $\set{x}$ is closed, then $X \setminus \set{x}$ is open and contains the other point.
  \item[(T$_{2}$)] A topological space $X$ is called a \emph{T$_{2}$-space} or \emph{Hausdorff} if for any two points $x \neq y$ there are disjoint open sets $U$, $V$ with $x \in U$ and $y \in V$.
        This is the most useful type of space.
        It essential means that ``limits'' are unique.
  \item[(T$_{3}$)] A T$_{1}$-space $X$ is called a \emph{T$_{4}$-space} of \emph{regular} if for any point $x$ and closed set $F$ not containing $x$ there are disjoint open sets $U$, $V$ with $x \in U$ and $F \subseteq V$.
  \item[(T$_{4}$)] A T$_{1}$-space $X$ is called a \emph{T$_{5}$-space} of \emph{normal} if for any two disjoint closed sets $F$, $G$ there are disjoint open sets $U$, $V$ with $F \subseteq U$ and $G \subseteq V$.
  \end{itemize}
\end{defn}

\begin{ex}[A Space that is not  $T_0$]
  The topology on $\set{x, y}$ where the only open sets are $\emptyset$ and $\set{x, y}$ is not $T_{0}$.
  There is no open set containing $x$ and not $y$ nor vice versa.
\end{ex}

\begin{ex}[A Space that is $T_0$ but not $T_1$]
  The topology on $\set{x, y}$ with open sets $\emptyset, \set{x}$, and $\set{x, y}$ is $T_{0}$.
  This is because $\set{x}$ is an open set containing $x$ but not $y$.
  However, there is no open set containing $y$ but not $x$ so the space is not $T_{1}$.
\end{ex}

\clearpage

\begin{prop}\label{prop: regular_iff_closed_neighborhood_basis}
  A Hausdorff space $X$ is regular if and only if the closed neighborhoods of any point form a neighborhood basis of the point.
\end{prop}
\begin{pf}
  Suppose $X$ is regular.
  Let $x \in V$ for $V$ open and let $C \defeq X \setminus V$.
  Then by regularity there exist open sets $U$ and $W$ with $x \in U$, $C \subseteq W$, and $U \cap W = \emptyset$.
  Then $X \setminus W$ is closed and $X \setminus W \subseteq X \setminus C = V$ and so any neighborhood of $V$ of $x$ contains a closed neighborhood of $x$.

  Now suppose that every point has a closed neighborhood basis.
  Let $x \in C$ with $C$ closed and $V = X \setminus C$.
  Then there exists open $U$ with $\overline{U} \subseteq V = X \setminus C$ and $x \in U$.
  Then $C \subseteq X \setminus \overline{U}$ and $U \cap (X \setminus \overline{U}) = \emptyset$.
  Thus $X$ is regular.
\end{pf}

\begin{cor}
  A subspace of a regular space $X$ is regular.
\end{cor}
\begin{pf}
  If $A \subseteq X$ is a subspace, just intersect a closed neighborhood basis in $X$ of some $a \in A$ with $A$ to obtain a closed neighborhood basis of $a$ in A.
\end{pf}

\clearpage

\subsection*{Exercises}

\begin{exercise}[\tcite{book:Bredon} 1.5.2]
  A finite $T_{1}$-space is discrete.
\end{exercise}
\begin{pf}
  Let $X$ be a finite $T_{1}$ space and $x \in X$.
  For all $y \neq x$ we can find an open set $U_{y}$ such that $y \notin U_{y}$ but $x \in U_{y}$.
  Then $\set{x} = \bigcap_{y \neq x} U_{y}$ is open since it is a finite intersection.
  Thus since singletons are open, their unions are open.
  This means arbitrary subsets of $X$ are open and we have the discrete topology.
\end{pf}

\begin{exercise}[\tcite{book:Bredon} 1.5.5]\label{exercise: Bredon 1.5.5}
  Subspaces of Hausdorff Spaces are Hausdorff.
\end{exercise}
\begin{pf}
  Let $X$ be Hausdorff and $Y$ a subspace of $X$.
  Let $x \neq y$ be elements in $Y$.
  Then there exists disjoint $U, V$ open in $X$ such that $x \in U$ and $y \in V$.
  But $U \cap Y$ and $U \cap V$ are open in $Y$ and contain $x$ and $y$ respectively and are still disjoint.
  Thus $Y$ is Hausdorff.
\end{pf}

\begin{exercise}[\tcite{book:Bredon} 1.5.6]\label{exercise: bredon_1.5.6}
  A Hausdorff space $X$ is normal if and only if for all open sets $U$ and closed sets $C \subseteq U$ there is an open set $V$ with $C \subseteq V \subseteq \overline{V} \subseteq U$.
\end{exercise}
\begin{pf}
  Suppose that $X$ is normal and let $U$ open and $C$ closed in $X$ such that $C \subseteq U \subseteq X$.
  $U$ is open so $X \setminus U$ is closed and disjoint from $C$.
  Thus there exists disjoint open $V, W$ such that $C \subseteq V$ and $(X \setminus U) \subseteq W$.
  But $V \subseteq X \setminus W$ and since $X \setminus W$ is closed, it contains $\overline{V}$.
  Then we also have that $X \setminus W \subseteq U$ since $X \setminus U \subseteq W$.
  Thus overall we have $C \subseteq V \subseteq \overline{V} \subseteq X \setminus W \subseteq U$.

  Now suppose that for any $U$ open, $C$ closed such that $C \subseteq U$ we have $V$ open such that $C \subseteq V \subseteq \overline{V} \subseteq U$.
  Let $C_{1}, C_{2}$ be disjoint closed subsets of $X$.
  Since $C_{1}$ is closed, $X \setminus C_{1}$ is open and contains $C_{2}$.
  Thus there exists open $V$ such that $C_{2} \subseteq V \subseteq \overline{V} \subseteq X \setminus C_{1}$.
  $C_{2} \subseteq \overline{V}$ and $\overline{V}$ closed means that $C_{1} \subseteq X \setminus \overline{V}$ which is open.
  Thus we have open sets $V, X \setminus \overline{V}$ that are disjoint and contain $C_{1}$ and $C_{2}$ respectively meaning that $X$ is normal.
\end{pf}

\begin{exercise}[\tcite{book:Bredon} 1.5.9]
  Metric spaces are normal.
\end{exercise}
\begin{pf}
  Let $X$ be a metric space with metric $d$.
  Recall from \Cref{prop: closed_dist_inf_0} that for closed $C \subseteq X$, $x \in X$ that $d(x, C) = 0$ if and only if $x \in C$.
  Let $C_{1}, C_{2}$ be disjoint closed sets in $X$.
  Define open sets $U_{1}$, $U_{2}$ such that
  \[
    U_{1} \defeq \bigcup_{x \in C_{1}} D_{\frac{d(x, C_{2})}{3}}(x),~ U_{2} \defeq \bigcup_{x \in C_{2}} D_{\frac{d(x, C_{1})}{3}}(x).
  \]
  We have that $C_{1} \subseteq U_{1}$ and $C_{2} \subseteq U_{2}$ and $U_{1}, U_{2}$ are open and disjoint.
  Thus $X$ is normal.
\end{pf}

\clearpage

\chapter{Nets (Moore-Smith Convergence)}

Many results in metric spaces are stated in terms of sequences.
We discuss a generalization of sequences called \emph{nets}.

\begin{defn}[Directed Sets]
  A \emph{directed set} $D$ is a poset such that for all $\alpha, \beta \in D$, there exists $\tau \in D$ such that $\tau \geq \alpha$ and $\tau \geq \beta$.
\end{defn}

\begin{defn}[Net]
  A \emph{net} in a topological space $X$ is a directed set $D$ along with a function $\phi\colon D \to X$.
\end{defn}

\begin{ex}[Sequences are Nets]
  Note that $\N$ with the usual ordering is a directed set.
  Sequences are nets with $\N$ as the directed set.
\end{ex}

\begin{defn}[Frequently, Eventually]
  If $\phi\colon D \to X$ is a net in a topological space $X$ and $A \subseteq X$, we say that $\phi$ is \emph{frequently} in $A$ if for all $\alpha \in D$ there exists $\beta \geq \alpha$ such that $\phi(\beta) \in A$.
  We say that $\phi$ is \emph{eventually} in $A$ if there exists $\alpha \in D$ such that for all $\beta \geq \alpha$ we have that $\phi(\beta) \in A$.
\end{defn}

\begin{defn}[Convergence of a Net]
  A net $\phi\colon D \to X$ in a topological space $X$ is said to \emph{converge to $x \in X$} if for any neighborhood $U$ of $x$, $\phi$ is eventually in $U$.
\end{defn}

\clearpage

\begin{prop}
  A topological space $X$ is Hausdorff if and only if any two limits of a convergent net are equal.
  Thus it makes sense to speak of the limit of a convergent net.
\end{prop}
\begin{pf}
  Suppose that $X$ is Hausdorff.
  If a net $\phi$ is eventually in two sets $U$ and $V$, then it is eventually in $U \cap V$.
  Also this means that $U \cap V \neq \emptyset$.
  Thus the forward direction is immediate.

  Now suppose that $X$ is not Hausdorff and that $x \neq y \in X$ are two points which cannot be separated by open sets.
  Consider a directed whose elements are pairs of open sets $(U, V)$ with $x \in U$, $y \in V$.
  We give this directed set the ordering $(U, V) \geq (A, B)$ if and only if $(U \subseteq A) \text{ and } (V \subseteq B)$ (so smaller sets are greater).
  Let $\phi$ be a net on this directed set such that $\phi(U, V) = $ some point in $U \cap V$.

  We claim that this net converges to both $x$ and $y$.
  Let $W$ be any neighborhood of $x$.
  Take any open set $V$ containing $y$ and an open set $U$ with $x \in U \subseteq W$.
  For any $(A, B) \geq (W, V)$ we have that $\phi(A, B) \in A \cap B \subseteq U \subseteq W$.
  Thus $\phi$ is eventually in $W$ and $\phi$ converges to $x$.
  A similar argument yields that $\phi$ also converges to $y$.
\end{pf}

\begin{prop}\label{prop: continuous_iff_net_converges}
  $f\colon X \to Y$ between topological spaces is continuous if and only if for any net $\phi$ converging to $x \in X$, the net $f \circ \phi$ in $Y$ converges to $f(x)$.
\end{prop}
\begin{pf}
  Suppose that $f$ is continuous.
  Let $\phi$ be a net in $X$ converging to $x$.
  Let $V \subseteq Y$ be an open set containing $f(x)$.
  We have that $U = f^{-1}(V)$ is a neighborhood of $x$.
  $\phi$ is eventually in $U$ so $f \circ \phi$ is eventually in $V$ and thus converges to $f(x)$.

  Now suppose that $f$ is not continuous.
  Then there is some open $V \subseteq Y$ such that $K \defeq f^{-1}(V)$ is not open.
  Let $x \in K \setminus \inte(K)$.
  Consider the directed set of open neighborhoods of $x$ with the ordering $A \geq B$ if and only if $A \subseteq B$.
  Choose any neighborhood $A$ of $x$.
  Note that $A \not\subseteq K$ so let $\phi(A) = w_{A} \in A \setminus K$ be a net.
  If $N$ is a neighborhood of $x$ and $B \geq N$, so $B \subseteq N$, then $\phi(B) = w_{B} \in B \setminus K \subseteq N$ and so $\phi$ is eventually in $N$.
  Thus $\phi$ converges to $x$.
  However $(f \circ \phi)(A) \notin V$ for any $A$ and so $f \circ \phi$ is not eventually in $V$ and thus does not converge to $f(x)$.
\end{pf}

Given a net $\phi\colon D \to X$, let $x_{\alpha} \defeq \phi(\alpha)$.
It's common to notate this net as $\set{x_{\alpha}}_{\alpha \in D}$.
So the condition in \Cref{prop: continuous_iff_net_converges} can be stated as
\[
  f\colon X \to Y \text{ continuous } \iff f(\lim x_{\alpha}) = \lim f(x_{\alpha}).
\]

\begin{prop}
  if $A \subseteq X$ then $\overline{A}$ is the set of limits of nets in $A$ which converge to $X$.
\end{prop}
\begin{pf}
  If $x \in \overline{A}$ then any open neighborhood of $x$ intersects nontrivially.
  We can make a net of this set of neighborhoods ordered by inclusion and have $x_{U} \in U \cap A$.
  This clearly converges to $x$.

  Now suppose that we have a net $\set{x_{\alpha}}$ of points in $A$ which converges to a point $x \in X$.
  Then this net is eventually in any neighborhood of $x$.
  Thus any neighborhood of $x$ has nontrivial intersection with a point in $A$ and $x \in \overline{A}$.
\end{pf}

\begin{defn}[Final Functions]
  If $D$ and $D'$ are directed sets, a function $h\colon D' \to D$ is \emph{final} if for all $\delta \in D$, there exists $\delta' \in D'$ such that $\alpha' \geq \delta'$ implies $h(\alpha') \geq \delta$.
\end{defn}

\begin{defn}[Subnets]
  A \emph{subnet} of a net $\phi\colon D \to X$ is the composition of a final map $h\colon D' \to D$ to a net $\phi \circ h$.
\end{defn}

\begin{prop}
  A net $\set{x_{\alpha}}$ is frequently in each neighborhood of a given point $x \in X$ if and only if it has a subnet which converges to $x$.
\end{prop}
\begin{pf}
  Consider the set $D'$ be ordered pairs $(\alpha, U)$ where $\alpha \in D$, $U$ is a neighborhood of $x$, and $x_{\alpha} \in U$.
  Give $D'$ the ordering on $D$ and by inclusion.
  If $(\alpha, U)$ and $(\beta, V)$ are in $D'$, then since $\set{x_{\alpha}}$ is frequently in $U$ and $V$, it is frequently in $U \cap V$.
  Thus there is some $\tau \geq \alpha, \beta$ with $x_{\tau} \in U \cap V$.
  Thus $(\tau, U \cap V) \in D'$ and $(\tau, U \cap V) \geq (\alpha, U), (\beta, V)$.
  So $D'$ is directed.

  Map $D' \to D$ by $(\alpha, U) to \alpha$.
  For any $\delta \in D$, $(\delta, X) \in D'$, and $(\alpha, X) \geq (\delta, X)$ implies that $\alpha \geq \delta$.
  So this map is final and $\set{x_{\alpha, U}}$ is a subset of $\set{x_{\alpha}}$.

  Let $N$ be a neighborhood of $x$.
  Then by assumption there exists some $x_{\beta} \in N$.
  If $(\alpha, U) \geq (\beta, N)$, then $x_{\alpha, U} = x_{\alpha} \in U \subseteq N$.
  So $\set{x_{\alpha, N}}$ is eventually in $N$.

  The converse is immediate.
\end{pf}

\begin{defn}[Universal Nets]
  A net in a set $X$ is \emph{universal} if for any $A \subseteq X$, the net is either eventually in $A$ or $X \setminus A$.
\end{defn}

\begin{prop}\label{prop: composition_with_universal_is_universal}
  The composition of a universal net in $X$ with a function $f\colon X \to Y$ is a universal net in $Y$.
\end{prop}
\begin{pf}
  If $A \subseteq Y$, then the net is eventually either in $f^{-1}(A)$ or $X \setminus f^{-1}(A)$.
  But $X \setminus f^{-1}(A) = f^{-1}(Y \setminus A)$ so the composition is either in $A$ or $Y \setminus A$.
\end{pf}

\clearpage

\begin{thrm}\label{thrm: net_has_universal_subnet}
  Every net has a universal subset
\end{thrm}
\begin{pf}
  Let $\set{x_{\alpha} | \alpha \in P}$ be a net in $X$.
  Consider all collections \textbf{C} of subsets of $X$ such that
  \begin{enumerate}
  \item $A \in \textbf{C} \implies \set{x_{\alpha}}$ is frequently in $A$; and
  \item $A, B \in \textbf{C} \implies A \cap B \in \textbf{C}$.
  \end{enumerate}
  Note that $\textbf{C} = \set{X}$ is such a collection.
  Other the family of all such collections by inclusion.
  The union of any \quest{simply ordered set} of collections satisfying conditions is another such collection.
  Thus by the \quest{maximality principle} there exists a maximal such collection $\textbf{C}_{0}$.

  Let $P_{0} = \set{(A, \alpha) \in \textbf{C}_{0} \times P | x_{\alpha} \in A}$ and order it by
  \[
    (B, \beta) \geq (A, \alpha) \iff B \subseteq A \text{ and } \beta \geq \alpha.
  \]
  This makes $P_{0}$ a directed set.
  Map $(A, \alpha) \to \alpha$ which is clearly final and thus defines a subset $\set{x_{A, \alpha}}$.

  We claim this subnet is universal.
  Suppose $S$ is any subset of $X$ such that $\set{x_{A, \alpha}}$ is frequently in $S$.
  Then for any $(A, \alpha) \in P_{0}$, there exists $(B, \beta) \geq (A, \alpha)$ in $P_{0}$ with $x_{\beta} = x_{B, \beta} \in S$.
  Then $B \subseteq A$, $\beta \geq \alpha$, and $x_{\beta} \in B$.
  Thus $x_{\beta} \in S \cap B \subseteq S \cap A$.
  This means that $\set{x_\alpha}$ is frequently in $S \cap A$ for any $A \in \textbf{C}_{0}$.
  But $S$ and $S \cap A$, for $A \in \textbf{C}_{0}$, can be added to $\textbf{C}_{0}$ and these conditions still hold.
  So by maximality, $S \in \textbf{C}_{0}$.

  If $\set{x_{A, \alpha}}$ was also frequently in $X \setminus S$, then $X \setminus S \in \textbf{C}_{0}$ be a similar argument.
  Thus $S \cap (X \setminus S) = \emptyset \in \textbf{C}_{0}$.
  This contradicts the first condition.
  Thus $\set{X_{A, \alpha}}$ is not frequently in $X \setminus S$ and so it is indeed eventually in $S$.

  Overall we have that if $\set{x_{(A, \alpha)}}$ is frequently in $S$, it is eventually in $S$.
  Thus $\set{x_{A, \alpha}}$ is a universal subset.
\end{pf}

Note that here we have used the Axiom of Choice in the form of the \quest{maximality principle}.
This past theorem is in fact equal to the Axiom of Choice.

\begin{prop}
  Subnets of universal nets are universal.
\end{prop}
\begin{pf}
  Let $\set{x_{\alpha}}$ be a universal net $\phi\colon D \to X$ and $\set{x_{h(\alpha)}}$ a subnet with final $h\colon D' \to D$.
  Let $A \subseteq X$ and without loss of generality suppose that $\set{x_{\alpha}}$ is eventually in $A$.
  Then there exists $\alpha \in D$ such that for all $\beta \geq \alpha$, $x_{\beta} \in A$.
  We have that $h$ is final and so there exists $\alpha'$ such that $\beta' \geq \alpha'$ implies that $h(\beta') \geq \beta$.
  Thus the subnet is eventually in $A$.
\end{pf}

\clearpage

\subsection*{Exercises}

\begin{exercise}[\tcite{book:Bredon} 1.6.1]
  A sequence is universal if and only if it is eventually constant.
\end{exercise}
\begin{pf}
  Let $x = \set{x_{i}}_{i \in \N}$ be a sequence in a metric space with metric $d$.
  Suppose the sequence was not eventually constant.
  Then for any $i \in \N$ we can find $\e > 0$ and $j > i$ such that $x_{j} \notin B_{\e}(x_{i})$.
  Thus the sequence is not universal.

  The converse is clear.
\end{pf}

\clearpage

\chapter{Compactness}

\begin{defn}[Covering, Open Covering, Subcover]
  A \emph{covering} of a topological space $X$ is a collection of sets whose union is $X$.
  An \emph{open covering} is a covering where each set is open.
  A \emph{subcover} is a subset of a cover which is still a cover.
\end{defn}

\begin{defn}[Compact, Heine-Borel Property]
  A topological space $X$ is said to be \emph{compact} or have the \emph{Heine-Borel property} if every open covering of $X$ has a finite subcover.
\end{defn}

\begin{defn}[Finite Intersection Property]
  A collection of sets has the \emph{finite intersection property} if the intersection of any finite subcollection is empty.
\end{defn}

The following theorem is just a translation of compactness in terms of open sets to an equivalent statement of about the closed complements of those sets.

\begin{thrm}\label{thrm: compact_iff_nonempty_intersection}
  A topological space $X$ is compact if and only if for every collection of closed subsets of $X$ which has the finite intersection property, the intersection of the whole collection is nonempty.
\end{thrm}
\begin{pf}
  \quest{Trivial Proof}
\end{pf}

\begin{thrm}\label{thrm: compact_subset_Hausdorff_closed}
  If $X$ is a Hausdorff space, then any compact subset of $X$ is closed.
\end{thrm}
\begin{pf}
  Let $A \subseteq X$ be compact and suppose that $x \in X \setminus A$.
  For $a \in A$ let $a \in U_{a}$ and $x \in V_{a}$ be disjoint open sets.
  Now $A = \bigcup_{a \in A} (U_{a} \cap A)$ and so we have a cover.
  Thus by compactness of $A$, we have $a_{1}, \ldots, a_{n} \in A$ such that $A \subseteq u_{a_{1}} \cup \cdots \cup U_{a_{n}} = U$.
  But then $x \in V_{a_{1}} \cap \cdots \cap V_{a_{n}} = V$ which is open, and $U \cap V = \emptyset$.
  Thus $x \in V \subseteq X \setminus U \subseteq X \setminus A$ and $V$ is open.
  Since this holds for any $x \in X \setminus A$, we have that $X \setminus A$ is open and thus $A$ is closed.
\end{pf}

\begin{thrm}\label{thrm: image_of_compact_is_compact}
  If $X$ is compact and $f\colon X \to Y$ is continuous, then $f(X)$ is compact.
\end{thrm}
\begin{pf}
  We may as well replace $Y$ by $f(X)$ and so assume that $f$ is onto.
  For any open cover of $f(X)$, look at inverse images of the sets and apply compactness.
\end{pf}

\begin{thrm}\label{thrm: closed_subset_of_compact_is_compact}
  If $X$ is compact and $A \subseteq X$ is closed, then $A$ is compact.
\end{thrm}
\begin{pf}
  Cover $A$ with open sets in $X$, add the open set $X \setminus A$, and then apply compactness of $X$.
\end{pf}

\begin{thrm}\label{thrm: bij_comp_to_Haus_is_homeo}
  Suppose that $X$ is compact, $Y$ is Hausdorff, $f\colon X \to Y$ is a continuous bijection, then $f$ is a homeomorphism.
\end{thrm}
\begin{pf}
  We need to show that $f^{-1}$ is continuous.
  This is equivalent to showing that $f$ maps closed sets to closed sets.
  But if $A \subseteq X$ is closed, then $A$ is compact by \Cref{thrm: closed_subset_of_compact_is_compact}.
  Thus $f(A)$ is compact and $f(A)$ must be closed since $A$ is Hausdorff.
\end{pf}

\begin{ex}[Closed Intervals are compact in $\R$]\label{ex: 01_compact}
  We have that $[0, 1]$ is compact in $\R$.
\end{ex}
\begin{pf}
  Let $\textbf{U}$ be an open covering of $[0, 1]$.
  Let $S = \set{s \in [0, 1] | [0, s] \text{ is covered by a finite subcollection of } \textbf{U}}$.
  Then Let $b = \sup(S)$.
  Note that $S$ takes the form $[0, b]$ or $[0, b)$.
  Suppose that $S$ takes for form $[0, b)$.
  Consider a set $U \in \textbf{U}$ containing $b$.
  $U$ must contain the interval $[a, b]$ for some $a < b$.
  But we can throw this in with the interval $[0, a]$ and obtain $[0, b]$.
  So $S$ must take the form $[0, b]$.
  Now suppose that $b < 1$.
  Then \quest{a similar argument} yields that there exists $c > b$ such that there is a finite cover of $[0, c]$ which contradicts the definition of $b$.
  Thus $b = 1$ and we have a finite cover of $[0, 1]$.
\end{pf}

Overall we have that closed intervals $[a, b]$ and their closed subsets are compact.
Also note that these closed subsets must be bounded.
So in $\R$ we have that a subset is compact if and only if it is closed and bounded.
This does not hold in arbitrary metric spaces.

\begin{thrm}
  A real-valued map on a compact space takes a maximum.
\end{thrm}
\begin{pf}
  If $f\colon X \to \R$ is a real-valued map on a compact space, then $f(X)$ is compact.
  Since $f(X)$ is compact, it is closed and bounded.
  This means that $\sup(f(X))$ exists, is finite, and belongs to $f(X)$.
\end{pf}

\begin{thrm}\label{thrm: compact_haus_is_norm}
  Compact Hausdorff spaces are normal
\end{thrm}
\begin{pf}
  Suppose that $X$ is a compact Hausdorff space.
  First we show that $X$ is regular.
  Let $C$ be a closed subset of $X$ and let $x \notin C$.
  $X$ is Hausdorff so for any $y \in C$ there exists open disjoint $U_{y}$, $V_{y}$ such that $x \in U_{y}$ and $y \in V_{y}$.
  $C$ is closed and so it is compact by \Cref{thrm: closed_subset_of_compact_is_compact}.
  $V_{y}$ is an open cover of $C$ and so there exist $y_{1}, \ldots, y_{n}$ such that $C \subseteq V \defeq V_{y_{1}} \cup \cdots \cup V_{y_{n}}$.
  Let $U \defeq U_{y_{1}} \cap \cdots \cap U_{y_{n}}$.
  Then we have that $x \in U$, $C \subseteq V$, and $U, V$ disjoint and open.
  Thus $X$ is regular.

  Now repeat this same proof letting a closed set $F$ play the role of $x$ and the other closed set $G$ playing the role of $C$.
  Thus $X$ is normal.
\end{pf}

\begin{defn}[Proper Maps]
  A map $f\colon X \to Y$ is \emph{proper} if $f^{-1}(C)$ is compact for each compact $C \subseteq Y$.
\end{defn}

\clearpage

\begin{thrm}\label{thrm: closed_single_point_preimage_compact}
  If $f\colon X \to Y$ is closed and $f^{-1}(y)$ is compact for each $y \in Y$, then $f$ is proper.
\end{thrm}
\begin{pf}
  Let $C \subseteq Y$ be compact and let $\set{U_{\alpha} | \alpha \in A}$ be an indexed collection of open sets whose union contains $f^{-1}(C)$.
  For any $y \in C$, there exists a finite subset of $A_{y} \subseteq A$ such that $f^{-1}(y) = \bigcup_{\alpha \in A_{y}} U_{\alpha}$.
  Now let
  \begin{align*}
    W_{y} &= \bigcup_{\alpha \in A_{y}} U_{\alpha}, \\
    V_{y} &= Y \setminus f(X \setminus W_{y}).
  \end{align*}
  These are both open sets.
  $f^{-1}(V_{y}) \subseteq W_{y}$ and $y \in V_{y}$.
  Since $C$ is compact and covered by $V_{y}$, there exists $y_{1}, \ldots, y_{n}$ such that $C \subseteq V_{y_{1}} \cup \cdots \cup V_{y_{n}}$.
  Thus
  \begin{align*}
    f^{-1}(C) &\subseteq f^{-1}(V_{y_{1}}) \cup \cdots \cup f^{-1}(V_{y_{n}}) \\
              &\subseteq W_{y_{1}} \cup \cdots \cup W_{y_{n}} \\
              &= \bigcup_{\substack{\alpha \in A_{y_{i}} \\ 1 \leq i \leq n}} U_{\alpha}
  \end{align*}
  which is a finite open subcover of $f^{-1}(C)$.
\end{pf}


\begin{thrm}\label{thrm: compact_iff_FIP_iff_universal_etc}
  For a topological space $X$, the following are equivalent:
  \begin{enumerate}
  \item $X$ is compact.
  \item Every collection of closed subsets of $X$ with the finite intersection property has non-empty intersection.
  \item Every universal net in $X$ converges.
  \item Every net in $X$ has a convergent subset.
  \end{enumerate}
\end{thrm}
\begin{pf}
  We have already seen the equivalence of $1$ and $2$ and now we show the rest.
  \begin{enumerate}
  \item[] $1 \iff 2$ See \Cref{thrm: compact_iff_nonempty_intersection}.
  \item[] $1 \implies 3$ Suppose that $\set{x_{\alpha}}$ is a universal net that does not converge.
        Given $x \in X$, there is an open neighborhood of $U_{x}$ such that $\set{x_{\alpha}}$ is not eventually in  $U_{x}$ and so the net is eventually in $X \setminus U_{x}$.
        So there exists $\beta_{x}$ such that for all $\alpha \geq \beta_{x}$ we have that $x_{\alpha} \notin U_{x}$.
        Now cover $X$ by some finite cover $U_{x_{1}} \cup \cdots U_{x_{n}}$.
        Let $\alpha \geq \beta_{x_{i}}$ for all $i$, then $x_{\alpha} \notin U_{x_{i}}$ for any $i$.
        Thus $x_{\alpha} \notin X$ which is a contradiction.
  \item[] $3 \implies 4$ This is clear by \Cref{thrm: net_has_universal_subnet}.

\clearpage

  \item[] $4 \implies 2$ Let $\textbf{F} = \set{C}$ be a collection of closed sets with the finite intersection property.
        Without loss of generality, we can assume that \textbf{F} is closed under finite intersection.
        Order \textbf{F} by $C \geq C'$ if and only if $C \subseteq C'$, making \textbf{F} a directed set.
        Let $\set{x_{C}}_{C \in \textbf{F}}$ be a net.
        By assumption, there is a convergent subnet given by a final map $f\colon D \to \textbf{F}$.
        Thus for $\alpha \in D$, $f(\alpha) \in \textbf{F}$ and $x_{f(\alpha)} \in f(\alpha)$.
        Suppose that $x_{f(\alpha)}$ converges to $x$.
        Let $C \in \textbf{F}$.
        Then there is $\beta \in D$ such that for all $\alpha \geq \beta$ we have $f(\alpha) \subseteq C$.
        Thus $x_{f(\alpha)} \in f(\alpha) \subseteq C$.
        Since $C$ is closed, it contains its limit points and $x \in C$ meaning that $x \in \bigcap_{C \in \textbf{F}} C$ and the total intersection is nonempty.
  \end{enumerate}
\end{pf}

\begin{prop}\label{prop: closed_cap_compact_is_compact}
  The intersection of a closed set and a compact set is compact.
\end{prop}
\begin{pf}
  Let $L$ be closed and $K$ compact.
  Suppose that $\set{x_{i}}$ is a net in $L \cap K$.
  Since $K$ is compact, by \Cref{thrm: compact_iff_FIP_iff_universal_etc} there must be some subnet converging to $x \in K$.
  But $L$ is closed, so it contains its limit points, and thus $x \in L$.
  Since every net in $K \cap L$ has a convergent subnet, $K \cap L$ must be compact.
\end{pf}

\clearpage

\subsection*{Exercises}

\begin{exercise}[\tcite{book:Bredon} 1.7.1]
  Show that if $X$ is compact, then every net in $X$ has a convergent subnet without using universal nets.
\end{exercise}
\begin{pf}
  Recall that for any collection of closed subsets of $X$ with the finite intersection property, the intersection of the whole collection has nonempty intersection.
  Let $\set{x_{\alpha}}_{\alpha \in A}$ be a net in $X$ and for each $\alpha$ let $E_{\alpha} \defeq \set{x_{\beta} | \beta \geq \alpha}$.
  Then note that the collection $\set{\overline{E_{\alpha}}}$ has the finite intersection property.
  Consider a finite intersection $\bigcap_{\alpha \in I} \overline{E_{\alpha}}$.
  Let $\alpha^{*} \geq \alpha$ for all $\alpha \in I$, which exists since $A$ is a directed set.
  Then $\alpha^{*} \in \overline{E_{\alpha}}$ for all $\alpha \in I$ and so in particular the finite intersection is non-empty.
  Thus we have that the intersection the whole collection $\bigcap_{\alpha \in A} \overline{E_{\alpha}}$ is non-empty containing some element $x \in X$.

  Consider the set $B = \set{\alpha, U_{\alpha} | \alpha \in A,~ U_{\alpha} \text{ is a neighborhood of } x_{\alpha}}$.
  This can be made directed by $(\alpha, U_{\alpha}) \geq (\beta, U_{\beta})$ if and only if $\alpha \geq \beta$ and $U_{\alpha} \subseteq U_{\beta}$.
  Consider the map $h\colon B \to A$ such that $h(\alpha, U_{\alpha}) \to \alpha$.
  This is clearly final.

  This subnet $\set{x_{(\alpha, U_{\alpha})}}$ converges to $x$.
  Let $U$ be a neighborhood of $x$.
  By construction, there exists some $\alpha$ such that $x_{\alpha} \in U$ since each of the $\overline{E_{\alpha}}$ are closed sets.
  So consider $(\alpha, U) \in B$.
  Then for all $(\beta, U_{\beta}) \geq (\alpha, U)$ we have that $h(\beta, U_{\beta}) \geq \alpha$ and thus $x_{\beta} \in U_{\beta} \subseteq U$, a convergent subnet.
\end{pf}

\begin{exercise}[\tcite{book:Bredon} 1.7.2]
  Let $X$ be a compact space and $\set{C_{\alpha} | \alpha \in A}$ a collection of closed sets which is also closed with respect to finite intersection.
  Let $C \defeq \bigcap C_{\alpha}$ and suppose that $C \subseteq U$ for some $U$ open.
  Then for some $\alpha$, we have that $C_{\alpha} \subseteq U$.
\end{exercise}
\begin{pf}
  Note that $V \defeq X \setminus U$ and $V \cap C_{\alpha}$ are both closed and thus compact.
  Also note that
  \[
    \bigcap_{\alpha \in A} V \cap C_{\alpha} = (X \setminus U) \cap \bigcap_{\alpha \in A} C_{\alpha} = (X \setminus U) \cap C = \emptyset.
  \]
  Let $F_{\alpha} \defeq X \setminus (V \cap C_{\alpha})$ which is open.
  Fix some $\beta \in A$.
  We have that $\bigcup_{\alpha \in A} F_{\alpha}$ is an open cover for $V \cap C_{\beta}$.
  This is because if $x \in V \cap C_{\beta}$, then since $\bigcap_{\alpha \in A} V \cap C_{\alpha} = \emptyset$ there is some $\alpha$ such that $x \notin V \cap C_{\alpha}$ meaning that $x \in F_{\alpha}$.
  But $V \cap C_{\beta}$ is compact, so there is some finite subcover $A' \subseteq A$ such that
  \[
    \bigcup_{\alpha \in A'} F_{\alpha} = \bigcup_{\alpha \in A'} X \setminus (V \cap C_{\alpha}) \containseq V \cap C_{\beta}.
  \]
  This implies that
  \[
    (V \cap C_{\beta}) \cap \pqty{\bigcap_{\alpha \in A'} V \cap C_{\alpha}} = V \cap \pqty{C_{\beta \cap \bigcap_{\alpha \in A'} C_{\alpha}}} = \emptyset.
  \]
  Thus we have found a finite intersection of closed sets from the collection, which itself is a member of the collection, disjoint from $X \setminus U$ meaning that is contained in $U$ as desired.
\end{pf}

\begin{exercise}[\tcite{book:Bredon} 1.7.3]
  Show that the hypothesis in \Cref{thrm: closed_single_point_preimage_compact} that $f$ is closed is necessary.
\end{exercise}
\begin{pf}
  Recall that subsets of $\R$ are compact if and only if they are closed and bounded.
  Note that $(0, 1]$ is bounded but not closed.
  This is because $\R \setminus (0, 1]$ is open, and in particular there is no open ball around $0$ completely contained in $R \setminus (0, 1]$.

  Now consider the inclusion $f\colon (0, 1] \into [0, 1]$.
  This satisfies the hypothesis that $f^{-1}(y)$ is compact for all $y \in [0, 1]$.
  Clearly $f^{-1}(0) = \emptyset$ is compact.
  We also have for $y > 0$ that because $[y, 1]$ is compact and $\set{y}$ is closed, $f^{-1}(y) = \set{y}$ is compact.
  However $f$ is not proper since $[0, 1]$ is compact but $f^{-1}([0, 1]) = (0, 1]$ is not compact.
\end{pf}

\clearpage

\chapter{Products}

\begin{defn}[Product Topology, Tychonoff Topology]
  Let $X$ and $Y$ be topological spaces.
  Then the \emph{product topology} on $X \times Y$ is the topology with the subbasis $\set{U \times V | U, V \text{ open },~ U \subseteq X,~ V \subseteq Y}$.
  Similarly, one can define a product topology on a finite product $X_{1} \times \cdots \times X_{n}$.

  For an infinite product $\Bigtimes_{\alpha \in A} X_{\alpha}$, we define the product topology as the topology with basis
  \[
    \bigtimes_{\substack{\alpha \in A \\ U_{\alpha} \subseteq X_{\alpha} \text{ open }}} U_{\alpha}
  \]
  such that $U_{\alpha} = X_{\alpha}$ for all but a finite number of $\alpha$ (``almost all $\alpha$'').
  This topology has subbasis $U_{\alpha} \times \Bigtimes_{\beta \neq \alpha} X_{\beta}$ for $U_{\alpha}$ open in $X_{\alpha}$.
  This is also known as the \emph{Tychonoff Topology}.
\end{defn}

\begin{prop}
  The projections $\pi_{X}\colon X \times Y \to X$ and $\pi_{Y}\colon X \times Y \to Y$ are continuous and the product topology is the smallest topology for which this is true.
  This also holds for infinite products.
\end{prop}
\begin{pf}
  The subbasis described is exactly the subbasis needed for continuity to hold.
\end{pf}

\begin{prop}
  The projection map $\pi_{\beta}\colon \Bigtimes_{\alpha} X_{\alpha} \to X_{\beta}$ is an open map.
\end{prop}
\begin{pf}
  It suffices to show for basis elements of $\Bigtimes_{\alpha} X_{\alpha}$.
  Let $\Bigtimes_{\alpha} U_{\alpha}$ be a basis element of the product topology.
  Then $\pi_{\beta}\pqty{\Bigtimes_{\alpha} U_{\alpha}} = U_{\beta}$ is open in $X_{\beta}$ which shows that $\pi_{\beta}$ is an open maps.
\end{pf}

\clearpage

\begin{prop}\label{prop: compact_proj_closed}
  If $X$ is compact, then the projection $\pi_{Y}\colon X \times Y \to Y$ is closed.
\end{prop}
\begin{pf}
  Let $C \subseteq X \times Y$ be closed.
  We want to show that $Y \setminus \pi_{Y}(C)$ is open.
  Let $y \notin \pi_{Y}(C)$, meaning that $(x, y) \notin C$ for all $x \in X$.
  Then for any $x \in X$, we have open sets $U_{x} \subseteq X$ and $V_{x} \subseteq Y$ such that $x \in U_{x}, y \in V_{x}$ and $(U_{x} \times V_{x}) \cap C = \emptyset$.

  Since $X$ is compact, we can find $x_{1}, ldots, x_{n} \in X$ such that $X = U_{x_{1}} \cup \cdots \cup U_{x_{n}}$.
  Let $V = V_{x_{1}} \cap \cdots \cap V_{x_{n}}$.
  Then we have
  \[
    (X \times V) \cap C = ( U_{x_{1}} \cup \cdots \cup U_{x_{n}} ) \times ( V_{x_{1}} \cap \cdots \cap V_{x_{n}} ) \cap C = \emptyset.
  \]
  Thus $y \in V \subseteq Y \setminus \pi_{Y}(C)$ where $V$ is open meaning that $\pi_{Y}(C)$ is closed.
\end{pf}

\begin{cor}\label{cor: compact_projection_proper}
  If $X$ is compact, then the projection $\pi_{Y}\colon X \times Y \to Y$ is proper.
\end{cor}
\begin{pf}
  Immediate from \Cref{thrm: closed_single_point_preimage_compact} and \Cref{prop: compact_proj_closed}.
\end{pf}

\begin{cor}\label{cor: prod_of_compact_is_compact}
  If $X$ and $Y$ are both compact, then $X \times Y$ is compact.
\end{cor}
\begin{pf}
  The projections are proper by \Cref{cor: compact_projection_proper} so we can recover finite subcovers.
\end{pf}

\begin{cor}[Tychonoff Theorem for Finite Products]\label{cor: tychonoff_finite}
  If each of the $X_{i}$ are compact, then $X_{1} \times \cdots \times X_{n}$ is compact.
\end{cor}
\begin{pf}
  Simple inductive argument from \Cref{cor: prod_of_compact_is_compact}.
\end{pf}

\begin{cor}\label{cor: I^n_compact}
  Let $I = [0, 1]$.
  Then $I^{n} \subseteq \R^{n}$ is compact.
\end{cor}
\begin{pf}
  Recall that $I$ is compact by \Cref{ex: 01_compact} and apply \Cref{cor: tychonoff_finite}.
\end{pf}

\begin{cor}
  A subspace $X$ of $\R^{n}$ is compact if and only if it is closed and bounded.
\end{cor}
\begin{pf}
  Suppose that $X$ is compact and thus closed since $\R^{n}$ is Hausdorff \Cref{thrm: compact_subset_Hausdorff_closed}.
  Cover $X$ by the open balls of radius $k$ about the origin, $k = 1, 2, \ldots$.
  Then $X$ has a finite subcover which implies $X$ is bounded.

  Now suppose $X$ is closed and bounded,
  Then it is contained in some ball of radius $k$ about the origin.
  This ball is contained in $[-k, k]^{n} \subseteq \R^{n}$, which is compact by \Cref{cor: tychonoff_finite}.
  Thus $X$ is a closed subset of a compact set which is known to be compact \Cref{thrm: closed_subset_of_compact_is_compact}.
\end{pf}

\clearpage

\begin{prop}\label{prop: net_of_product_converges}
  A net in a product space $X = \Bigtimes_{\alpha} X_{\alpha}$ converges to a point $(\ldots, x_{\alpha}, \ldots)$ if and only if its composition with each projection $\pi_{\alpha}\colon X \to X_{\alpha}$ converges to $x_{\alpha}$.
\end{prop}
\begin{pf}
  Let $h\colon D \to X$ be the net in question.
  Then $\pi_{\alpha} \circ h\colon D \to X_{\alpha}$ is a net as well.
  Suppose $h$ converges to $x = (\ldots, x_{\alpha}, \ldots)$.
  Let $U_{\alpha}$ be a neighborhood of $x_{\alpha}$.
  Then $U_{\alpha} \times \Bigtimes_{\beta \neq \alpha} X_{\beta}$ is a neighborhood of $x$.
  Thus the net $h$ is eventually in this neighborhood.
  But then $\pi_{\alpha} \circ h$ is eventually in $U_{\alpha}$ and thus $\pi_{\alpha} \circ h$ is a net converging to $x_{\alpha}$.

  \quest{Converse is clear}
\end{pf}

\begin{thrm}[Tychonoff Theorem]\label{thrm: tychonoff}
  The product of an arbitrary collection of compact spaces is compact.
\end{thrm}
\begin{pf}
  Let $X = \Bigtimes_{\alpha} X_{\alpha}$ where each of the $X_{\alpha}$ are compact.
  Let $f\colon D \to X$ be a universal net in $X$.
  Then the composition $\pi_{\alpha} \circ h$ is also universal by \Cref{prop: composition_with_universal_is_universal}.
  Thus the composition converges to some $x_{\alpha}$ by \Cref{thrm: compact_iff_FIP_iff_universal_etc}.
  Thus means that the original net $f$ converges to a point $(\ldots, x_{\alpha}, \ldots)$ by \Cref{prop: net_of_product_converges}.
  Thus since every universal net in $X$ converges, by \Cref{thrm: compact_iff_FIP_iff_universal_etc} $X$ is also compact.
\end{pf}

The ease of the proof of this result is entirely due to the use of universal nets.
This usage depends on the Axiom of Choice.
In fact, \Cref{thrm: tychonoff} is equivalent to the Axiom of Choice.
However the finite case \Cref{cor: tychonoff_finite} does not depend on the Axiom of Choice.

If $X$ is a space and $A$ is a set, the product of $A$ copies of $X$ is denoted by $X^{A}$.
It can be though of as the space of functions $f\colon A \to X$.
Using this, \Cref{prop: net_of_product_converges} can be restated as follows:
\begin{prop}
  A net $\set{f_{\alpha}}$ in $X^{A}$ converges to $f \in X^{A}$ if and only if for all $x \in X$ we have that $f_{\alpha}(x) \to f(x)$.
  In particular we have that $\lim(f_{\alpha}(x)) = (\lim f_{\alpha})(x)$.
\end{prop}

When $A$ itself is also a topological space, we use $X^{A}$ to denote the set of all continuous functions $f\colon A \to X$.

\begin{prop}
  Let $X = \Bigtimes_{\alpha} X_{\alpha}$ be a product of topological spaces.
  If $X$ is normal, then each of the $X_{\alpha}$ are normal.

\end{prop}
\begin{pf}
  To see this, let $F_{\beta}, G_{\beta}$ be disjoint closed sets in $X_{\beta}$.
  Then since $\pi_{\beta}$ is continuous, we have that $\pi_{\beta}^{-1}(F_{\beta})$ and $\pi_{\beta}^{-1}(G_{\beta})$ closed.
  Thus by the normality of $X$, there exist disjoint open sets $U, V \subseteq X$ such that $\pi_{\beta}^{-1}(F_{\beta}) \subseteq U$ and $\pi_{\beta}^{-1}(G_{\beta}) \subseteq V$.
  Then since $\pi_{\beta}$ is open, we have that $\pi_{\beta}(U)$ and $\pi_{\beta}(V)$ are disjoint and open and contain $F_{\beta}$ and $G_{\beta}$ respectively.
  Thus $X_{\alpha}$ is normal.
\end{pf}

\clearpage

\begin{ex}[Sorgenfrey's Half-Open Square Topology, Product of Normal may not be Normal]\

  Let $S$ denote the real line $\R$ with the half-open topology with half-open intervals $[a, b)$ as basis elements.
  We claim that $S$ is normal.
  % TODO: redo with tikz?
  \begin{figure}[h]
    \centering
    \includegraphics[width=0.25\textwidth]{figs/Sorgengrey_Half_Open.png}
    \caption{$S(p, \e)$}\label{fig:sorgenfrey_half_open}
  \end{figure}
  Let $F, G$ be disjoint closed sets in $S$.
  For each $x \in F$, choose some $x'$ such that $[x, x') \cap G$ is disjoint.
  Let $U = \bigcup_{x \in F}[x, x')$.
  We can find such $x'$ because $G$ is closed and $F$ is disjoint from $G$ and so there exists an open set around $x$ such that it is disjoint from $G$ and we can choose $x'$ from that set.
  Similarly, for each $y \in G$, choose some $y'$ such that $[y, y') \cap F$ is disjoint and let $V = \bigcup_{y \in G}[y, y')$.
  We have that $F \subseteq U$ and $G \subseteq V$.
  We also have that $U$ and $V$ are disjoint.
  To see this, it suffices to show that for all $x \in F$, $y \in G$ that $[x, x')$ and $[y, y')$ are disjoint.
  Let $x \in F$ and $y \in G$ and without loss of generality suppose $x < y$.
  Note that $x'$ was chosen such that $[x, x') \cap G = \emptyset$ and so in particular we have that $[x, x') \cap [y, y') = \emptyset$.
  Thus $x' \leq y$ which shows that $U$ and $V$ are disjoint and $S$ is normal.

  Let $X = S^{2}$ with the half-open square topology.
  Let $S(p, \e)$ denote the basis element the half-open square with sidelengths $\e$ and $p \in X$ as the lower left corner as in \Cref{fig:sorgenfrey_half_open}.
  Define $L = \set{(x, -x) | x \in \R}$.
  We have that $K = \set{(x, -x) | x \text{ is irrational }}$ is closed in $X$.
  This is because for any $(x, y) \in X \setminus K$, we have that $S((x, y), \e)$ is disjoint from $K$ for some $\e$ based on if it is above or below the line.
  If $(x, y)$ lies on the line, then since the square $S((x, y), \e)$ opens up and to the right, it is also disjoint from $K$.
  Similarly, $L \setminus K$ is closed.

  We now show that $X$ is not normal.
  \quest{Something something Baire Category Theorem, move when covered later}
\end{ex}

\clearpage

\begin{defn}[Topological Sum, Disjoint Union]
  If $X, Y$ are topological space, then their \emph{topological sum} or \emph{disjoint union} $X + Y$ is the set $(X \times \set{0}) \cup (Y \times \set{1})$ with the topology making $X \times \set{0}$ and $Y \times \set{1}$ clopen and their inclusions $X \to X + Y$ such that $x \mapsto (x, 0)$ and $Y \to X + Y$ such that $y \mapsto (y, 1)$ homeomorphisms to their image.

  More generally, if $\set{X_{\alpha} | \alpha \in A}$ is an indexed family of spaces, then their \emph{topological sum} $\bigplus_{\alpha} X_{\alpha}$ is $\bigcup_{\alpha} X_{\alpha} \times \set{\alpha}$ given the topology making each $X_{\alpha} \times \set{\alpha}$ clopen and each inclusion $X_{\beta} \to \bigplus_{\alpha} X_{\alpha}$ such that $x \to (x, \beta)$ is a homeomorphism to its image $X_{\beta} \times \set{\beta}$.
\end{defn}

In ordinary usage, if $X$ and $Y$ are disjoint space, we can regard $X + Y$ as $X \cup Y$ with the topology making $X$ and $Y$ open subspaces.

\clearpage

\subsection*{Exercises}

\begin{exercise}[\tcite{book:Bredon} 1.8.3]\label{exercise: Bredon 1.8.3}
  Let $\set{Y_{\alpha}}$ be a collection of spaces and $Y = \Bigtimes_{\alpha} Y_{\alpha}$.
  Prove that the function $f\colon X \to Y$ is continuous if and only if each composition $\pi_{\alpha} \circ f$ is continuous.
\end{exercise}
\begin{pf}
  Suppose that $f$ is continuous.
  We know the composition of two continuous functions is continuous, so $\pi_{\alpha} \circ f$ is continuous for all $\alpha$.

  Now suppose that each composition $\pi_{\alpha} \circ f$ is continuous.
  It suffices by \Cref{thrm: continuity_for_bases} to show that for each element $S$ of some subbasis \textbf{S} of $Y$, $f^{-1}(B)$ is open.
  We use the standard subbasis for the infinite product which is $U_{\alpha} \times \bigtimes_{\beta \neq \alpha} X_{\beta}$.
  We have that for any $U_{\alpha}$ that $\pqty{\pi_{\alpha} \circ f}^{-1}(U_{\alpha})$ is open by assumption.
  Thus preimages of basis elements are open and $f$ is continuous.
\end{pf}

\begin{exercise}[\tcite{book:Bredon} 1.8.4]
  An arbitrary product of Hausdorff spaces is Hausdorff and an arbitrary product of regular spaces is regular.
\end{exercise}
\begin{pf}
  Let $H = \Bigtimes_{\alpha} H_{\alpha}$ be a product of Hausdorff spaces.
  Let $x \neq y \in H$.
  Since $x \neq y$ there exists $\beta$ such that $x_{\beta} \neq y_{\beta} \in H_{\beta}$.
  Then since $H_{\beta}$ is Hausdorff let $U_{\beta}$ and $V_{\beta}$ be disjoint open sets such that $x_{\beta} \in U_{\beta}$ and $y_{\beta} \in V_{\beta}$.
  For $\alpha \neq \beta$ let $U_{\alpha} = V_{\alpha} = H_{\alpha}$.
  Let $U = \bigtimes_{\alpha} U_{\alpha}$ and $V = \bigtimes_{\alpha} V_{\alpha}$.
  We have constructed disjoint open sets $U$ and $V$ such that $x \in U$, $y \in V$ proving that $H$ is Hausdorff.

  Now let $R = \Bigtimes_{\alpha} R_{\alpha}$ be a product of regular space.
  By \Cref{prop: regular_iff_closed_neighborhood_basis}, it suffices to show that for each point $x$ in $R$, the closed neighborhoods of $x$ form a neighborhood basis of $x$.
  Let $x \in R$ and let $V$ be a neighborhood of $x$.
  Then each $V_{\alpha}$ is a neighborhood of $x_{\alpha}$ in $R_{\alpha}$ and by \Cref{prop: regular_iff_closed_neighborhood_basis} we can find $C_{\alpha}$ closed such that $x \in C_{\alpha} \subseteq R_{\alpha}$.
  Thus $C = \bigtimes_{\alpha} C_{\alpha}$ is a closed set such that $x \in C \subseteq V$ and closed neighborhoods of $x$ form a closed neighborhood basis of $x$.
  This proves that $R$ is regular.
\end{pf}

\begin{exercise}[\tcite{book:Bredon} 1.8.5]
  If $X$ is a topological space, the \emph{diagonal} of $X \times X$ is the subspace $\Delta = \set{\pqty{x, x} | x \in X}$.
  Show that $X$ is Hausdorff if and only if $\Delta$ is closed in $X \times X$.
\end{exercise}
\begin{pf}
  Suppose that $X$ is Hausdorff and let $(x, y) \in (X \times X) \setminus \Delta$.
  Then we have that there exist disjoint open sets $U_{x}, U_{y}$ such that $x \in U_{x}$ and $y \in U_{y}$.
  We have that $U_{x} \times U_{y}$ is open in $X \times X$ and contains $(x, y)$ and is contained in $(X \times X) \setminus \Delta$.
  Thus $(X \times X) \setminus \Delta$ is open meaning that $\Delta$ is closed.

  Now suppose that $\Delta$ is closed, so $(X \times X) \setminus \Delta$ is open.
  Let $x \neq y \in X$.
  Then $(x, y) \in (X \times X) \setminus \Delta$ meaning that there exists a basis element $U_{x} \times U_{y} \subseteq (X \times X) \setminus \Delta$ containing $(x, y)$.
  This implies that $U_{x} \cap U_{y} = \emptyset$.
  Thus we have found disjoint open sets $U_{x}, U_{y}$ contained in $X$ such that $x \in U_{x}$ and $y \in U_{y}$ meaning that $X$ is Hausdorff.
\end{pf}

\clearpage

\begin{exercise}[\tcite{book:Bredon} 1.8.6]
  Let $f, g\colon X \to Y$ be two maps.
  If $Y$ is Hausdorff then show that the subspace $A = \set{x \in X \mid f(x) = g(x)}$ is closed in $X$.
\end{exercise}
\begin{pf}
  Since $f$ and $g$ are continuous, we have that $h(x) = (f(x), g(x)) \subseteq Y \times Y$ is also continuous.
  Let $\Delta \subseteq Y \times Y$ be as defined in the previous problem.
  Then note that $A = h^{-1}(\Delta)$.
  Since $Y$ is Hausdorff, we have that $\Delta$ is closed which means by continuity of $h$ that $A$ is closed.
\end{pf}

\clearpage

\chapter{Metric Spaces Again}

In this section we will learn about two new concepts surrounding metric spaces.
The first will be the concept of \emph{completeness} which is a condition on sequence convergence.
The second will be about when a topological space can be equipped with a metric yielding the same open sets.

\begin{defn}[Cauchy Sequence]
  A \emph{Cauchy Sequence} in a metric space $X$ is a sequence $\set{x_{i}}_{i \in \N}$ such that for all $\e > 0$ there exists $N > 0$ such that if $n, m > N$ we have that $\dist(x_{n}, x_{m}) < \e$.
\end{defn}

\begin{defn}[Completeness]
  A metric space $X$ is \emph{complete} if every Cauchy sequence in $X$ converges in $X$.
\end{defn}

\begin{defn}[Totally Bounded]
  A metric space $X$ is \emph{totally bounded} if, for each $\e > 0$, $X$ can be covered by a finite number of $\e$ balls.
\end{defn}

Before we continue, we state an equivalent criterion for compactness in metric spaces, although it can be generalized to general topological spaces using nets.

\begin{defn}[Sequential Compactness]
  A metric space $X$ is \emph{sequentially compact} if every sequence in $X$ has a convergent subsequence.
\end{defn}

\clearpage

\begin{prop}
  A metric space $X$ is compact if and only if it is sequentially compact.
\end{prop}
\begin{pf}
  Suppose that $X$ is compact and that $x \in X$ is not the limit of a subsequence of $\set{x_{n}}$.
  Then there is an open neighborhood $U_{x}$ of $x$ containing $x_{n}$ for a finite number of $n$.
  But we can find a finite number of these $U_{x}$ that cover $X$.
  Their union would cover $X$ but only contain a finite number of the $x_{n}$ which is impossible.

  Now suppose $X$ is sequentially compact.
  First we show that $X$ is totally bounded and complete.
  Suppose that $X$ is not totally bounded.
  Then there exists $\e > 0$ such that no finite collection of $\e$-balls covers $X$.
  Let $x_{1} \in X$ and note that $B_{\e}(x_{1}) \subsetneq X$.
  Thus there exists $x_{2} \in X \setminus B_{\e}(x_{1})$ and similarly there exists $x_{e} \in X \setminus (B_{\e}(x_{1}) \cup B_{\e}(x_{2}))$.
  We can continue to iterate this construction to find a sequence of points $\set{x_{n}}_{n \in N}$.
  Then we note that for $i \neq j$ we have $\dist(x_{i}, x_{j}) > \e$ and thus we have a sequence with no convergent subsequence.
  Thus sequential compactness of $X$ implies that the set $X$ is totally bounded.
  For completeness, note that by sequential compactness any Cauchy sequence in $X$ has a convergent subsequence with limit $x \in X$.
  However using a triangle inequality argument, we can show that the whole Cauchy sequence converges to $x \in X$.
  Thus sequential compactness of $X$ implies $X$ is totally bounded and complete.

  Now suppose that $X$ is totally bounded and complete.
  Let $\set{U_{i}}_{i \in I}$ be an open cover for $X$ and for the sake of contradiction suppose it has no finite subcover.
  Then for each $N \geq 1$ we have that $X$ is covered by finitely many open balls $B_{1 / N}(x)$ for some finite set of $\set{x_{N_{i}}}$.
  We have that the $\set{U_{i}}$ has no finite subcover but finitely many open balls $B_{1 / N}(x)$ for $x \in \set{x_{N_{i}}}$ do cover $X$.
  Thus there exists some $x_{0} \in \set{x_{0_{i}}}$ such that $B_{1}(x_{0})$ is not covered by any finite subcollection of the $\set{U_{i}}$.
  If no such $x_{0}$ exists, we could find a finite subcover which would be a contradiction.
  Now we can cover $B_{1}(x_{0})$ by balls of radius $\frac{1}{2}$ such that their centers are of distance less than $1 + \frac{1}{2}$ from $x_{0}$.
  Similarly, some $x_{1}$ exists such that $B_{1/2}(x_{1})$ is not covered by any finite subcollection of the $\set{U_{i}}$.
  This continues on and on and we get a sequence $\set{x_{i}}_{i \in \N}$ which is Cauchy and by completeness converges to some point $x \in X$.
  We have that for some $i$ that $x \in U_{i}$.
  However we must also have by convergence of the sequence that there exists $n > 0$ such that $B_{2^{-n}}(x) \subseteq U_{i}$ which is a contradiction.
  Thus $X$ must be compact.
\end{pf}

We have actually proven in metric spaces an equivalence of 3 conditions.

\begin{cor}
  In a metric space $X$ the following are equivalent:
  \begin{enumerate}
  \item $X$ is compact;
  \item Every sequence in $X$ has a convergent subsequence; and
  \item $X$ is totally bounded and complete.
  \end{enumerate}
\end{cor}

\begin{defn}[Metrizable]
  A topological space $X$ is called \emph{metrizable} if it can be equipped with a metric such that the epsilon balls defined by that metric form a basis of $X$.
\end{defn}

\clearpage

\begin{defn}[Completely Regular, $T_{3 \frac{1}{2}}$, Tychonoff Space]\label{defn: completely_regular}
  A Hausdorff space $X$ is \emph{completely regular} or $T_{3 \frac{1}{2}}$ if, for each point $x$ and closed set $C \subset X$ with $x \notin C$, there is a map $f\colon X \to [0, 1]$ such that $f(x) = 0$ and $f(C) = \set{1}$.
\end{defn}

We can follow such a function with a map $[0, 1] \to [0, 1]$ which is $0$ on $\bqty{0, \frac{1}{2}}$ and stretches $\bqty{\frac{1}{2}, 1}$ onto $[0, 1]$.
This means the function $f$ in \Cref{defn: completely_regular} can be taken such that $f$ is $0$ on a neighborhood of $x$.
In fact, we can take $f^{-1}\pqty{\left[ 0, \frac{1}{2} \right)}$ and $f^{-1}\pqty{\left( \frac{1}{2}, 1 \right]}$ as disjoint open neighborhoods of of $x$ and $C$ respectively.

\begin{prop}\label{prop: bounded_metric_leq_1}
  Suppose $X$ is a metric space. Define:
  \[
  \dist'(x, y) = \begin{cases}
                   1           & \text{if } \dist(x, y) > 1, \\
                   \dist(x, y) & \text{if } \dist(x, y) \leq 1.
                 \end{cases}
  \]
  Then $\dist$ and $\dist'$ have the same topology on $X$.
\end{prop}
\begin{pf}
  The basis for the topology on $X$ only depends on the open $\e$-balls for small $\e$ and so we immediately have the same basis.
\end{pf}

\begin{prop}\label{prop: bounded_metrics_prod_top}
  Let $\set{X_{i}}_{i \in \N}$ be collection of metric spaces with metrics bounded by $1$ (\Cref{prop: bounded_metric_leq_1}).
  Define a metric on $\bigtimes_{i \in \N} X_{i}$ by $\dist(x, y) = \Sum_{i \in \N} = \frac{\dist(x_{i}, y_{i})}{2^{i}}$.
  Then this metric gives rise to the product topology.
\end{prop}
\begin{pf}
  Let $X$ denote the product set with the product topology and $X'$ denote the same set with the metric topology.
  To see that $X \to X'$ is continuous, by \Cref{exercise: Bredon 1.8.3} we just need to show that it's composition to each $X_{i}$ is continuous.
  However, all this projection does is decrease distance and multiply the distance by $2^{i}$ which clearly implies continuity.
  For the reverse, it suffice to show that for any point $x \in X$ the $\e$-ball around $x$ contains a neighborhood of $x$ in $X$.
  We have that
  \[
    B_{\e}(x) = \Set{y \in X' | \sum_{i} \frac{\dist(x_{i}, y_{i})}{2^{i}} < \e}.
  \]

  \clearpage

  Take $n$ such that $2^{-n} < \frac{\e}{4}$ and let $y_{i} \in X_{i}$ be such that $\dist(x_{i}, y_{i}) < \frac{\e}{2}$ for $i = 1, \ldots, n - 1$ and $\dist(x_{i}, y_{i})$ is arbitrary for $i \geq n$.
  Then we have that
  \begin{align*}
    \dist(x, y) &= \sum_{i = 1}^{n - 1} \frac{\dist(x_{i}, y_{i})}{2^{i}} + \sum_{i = n}^{\infty} \frac{\dist(x_{i}, y_{i})}{2^{i}} \\
                &< \sum_{i = 1}^{n - 1} \frac{\e}{2^{i + 1}} + \frac{\e}{4} \sum_{i = 0}^{\infty} \frac{1}{2^{i}} \\
                &< \frac{\e}{2} + \frac{\e}{2} = \e.
  \end{align*}
  Thus we have that
  \[
    x \in B_{\e/2}(x_{1}) \times \cdots \times B_{\e/2}(x_{n - 1}) \times X_{n} \times \cdot \subseteq B_{\e}(x)
  \]
  and note that the middle term is a basis element of the product topology as desired.
\end{pf}

\begin{lem}\label{lem: Hausdorff_and_comp_reg-ish_embedding}
  Suppose that $X$ is Hausdorff and that $f_{i}\colon X \to [0, 1]$ with $i = 1, 2, \ldots$ such that for any $x \in X$ and closed $C \subseteq X$ with $x \notin C$, there is an index $i$ such that $f_{i}(x) = 0$ and $f_{i}(C) = \set{1}$.
  Then define $f\colon X \to \Bigtimes_{i = 1, 2, \ldots} [0, 1]$ by $f(x) = (f_{1}(x), f_{2}(x), \ldots)$.
  Then $f$ is an embedding, \ie, a homeomorphism onto its image.
\end{lem}
\begin{pf}
  We have that $f$ is continuous by \Cref{exercise: Bredon 1.8.3}.
  We also have that $f$ is an injection but not a surjection because if $x$ and $y$ are distinct points, just take $C = \set{y}$ and apply the statement.
  Then note to show that $f^{-1}$ is continuous it suffices to show that $f$ is closed because $f^{-1}$ is continuous if and only if for all closed $C \subseteq X$ we have that $\pqty{f^{-1}}^{-1}(C) = f(C)$ is closed.
  Let $C \subseteq X$ be closed and suppose we have some sequence $\set{c_{i}} \subseteq C$ such that $f_{c_{i}} \to f(x)$.
  Then we just need to show the limit point $x$ is in $C$.
  If $x$ is not in $C$ then there exists an index $i$ such that $f_{i}(x) = 0$ but $f_{i}(C) = \set{1}$.
  This means that $1 = f_{i}(c_{n}) \to f_{i}(x) = 0$ which is impossible.
  Thus $x \in C$ and $f^{-1}$ is continuous and we have that $f$ is a homeomorphism onto its image.
\end{pf}

\begin{lem}\label{lem: second_count_comp_reg_family_F}
  Suppose that $X$ is second countable and is completely regular and let \textbf{S} be a countable basis for the open sets.
  For each pair of $U, V \in \textbf{S}$ with $\overline{U} \subseteq V$, take some $f\colon X \to [0, 1]$ which is $0$ on $U$ and $1$ on $X \setminus V$, provided such a function exists.
  Call \textbf{F} this (possibly empty) collection of maps.
  Note that \textbf{F} is also countable.
  Then for each $x \in X$ and each closed $C \subseteq X$ such that $x \notin C$, there is an $f \in \textbf{F}$ such that $f = 0$ on a neighborhood of $x$ and $f = 1$ on $C$.
\end{lem}
\begin{pf}
  Let $x$ and $C$ be as stated in the lemma.
  Since $C$ is closed, we have by definition of basis some $V \in \textbf{S}$ such that $x \in V \subseteq X \setminus C$.
  By $X$ being completely regular, we have some $g\colon X \to [0, 1]$ such that $g(x) = 0$ and $1$ on $X \setminus V$.
  Without loss of generality, as stated before, we can take $g = 0$ on some neighborhood of $x$.
  This neighborhood contains some $U \in \textbf{S}$ such that $\overline{U} \subseteq V$ \quest{why}, $f(U) = \set{0}$ and $f(X \setminus V) = \set{1}$.
  Thus by assumption, this $g$ can be replaced by some $f \in \textbf{F}$ satisfying the same properties as $g$ and satisfying the final requirements of the lemma.
\end{pf}

\begin{thrm}[Urysohn Metrization Theorem]\label{thrm: urysohn_metrization}
  If a space $X$ is second countable and completely regular, then it is metrizable.
\end{thrm}
\begin{pf}
  Construct a countable family of functions \textbf{F} satisfying \Cref{lem: second_count_comp_reg_family_F}.
  Then apply \Cref{lem: Hausdorff_and_comp_reg-ish_embedding} to get an embedding of $X$ into a countable product of unit intervals $[0, 1]$.
  Then by \Cref{prop: bounded_metrics_prod_top}, this countable product of intervals is metrizable.
  Thus $X$ is metrizable by the embedding.
\end{pf}

\begin{defn}[Diameter in A Metric Space]
  Let $X$ be a metric space and $A \subseteq X$. Then the \emph{diameter of $A$}, denoted $\diam(A)$, is $\sup\set{\dist(p, q) | p, q \in A}$.
\end{defn}

\begin{lem}[Lebesgue Lemma]\label{lem: lebesgue_lemma}
  Let $X$ be a compact metric space and let $\set{U_{\alpha}}$ be an open covering of $X$.
  Then there is $\delta > 0$ such that for all $A \subseteq X$ such that $\diam(A) < \delta$, there exists $\alpha$ such that $A \subseteq U_{\alpha}$.
\end{lem}
\begin{pf}
  For each $x \in X$, there exists some $\e(x) > 0$ such that $B_{2\e(x)}(x) \subseteq U_{\alpha}$ for some $\alpha$.
  Then $X$ is covered by a finite number of the balls $B_{\e(x)}$ for say $x = x_{1}, \ldots, x_{n}$.
  Let $\delta = \min_{1 \leq i \leq n} \e(x_{i})$.
  Suppose that $\diam(A) < \delta$ and pick some $a_{0} \in A$.
  There is some index $1 \leq i \leq n$ such that $\dist(a_{0}, x_{i}) < \e(x_{i})$.
  If $a \in A$ then we have that $\dist(a, a_{0}) < \delta \leq \e(x_{i})$.
  We have that $\dist(a, x_{i}) \leq \dist(a, a_{0}) + \dist(a_{0}, x_{i}) < \e(x_{i}) + \e(x_{i}) = 2\e(x_{i})$.
  Thus $A \subseteq B_{2\e(x_{i})}(x_{i}) \subseteq U_{\alpha}$ for some $\alpha$.
\end{pf}

\begin{defn}[Lebesgue Number]
  The $\delta$ in \Cref{lem: lebesgue_lemma} is called a \emph{Lebesgue Number}.
\end{defn}

\clearpage

\subsection*{Exercises}

\begin{exercise}[\tcite{book:Bredon} 1.9.2]
  A subspace of a completely regular space is completely regular
\end{exercise}
\begin{pf}
  Let $X$ be a completely regular space and $S \subseteq X$ a subspace.
  Note that by \Cref{exercise: Bredon 1.5.5} we have that $S$ is Hausdorff.
  Let $x \in S$ and $C \subseteq S$ be closed such that $x \notin C$.
  Since $C$ is closed in $S$, we have that $C = S \cap D$ for some $D$ closed in $X$.
  Also note that we must have that $x \notin D$ since $C = S \cap D$ and $x \notin C$ but $x \in S$.
  Thus by the fact that $X$ is completely regular, we have that there exists some map $f\colon X \to [0, 1]$ such that $f(x) = 0$ and $f(D) = \set{1}$.
  We can restrict $f$ to $S$ and get a map $f\mid_{S}\colon X \to [0, 1]$ such that $f\mid_{S}(0) = x$ and $f\mid_{S}(C) = f\mid_{S}(D \cap S) = f(D) = \set{1}$ as desired.
  Thus $S$ is completely regular.
\end{pf}

\clearpage

\chapter{Existence of Real Valued Functions}

In \Cref{thrm: urysohn_metrization} we relied on the fact that some set of continuous real-valued functions of sufficiently large size existed.
We now address purely topological assumptions that guarantee such functions.

\begin{lem}\label{lem: dyadic_lem_for_urysohn}
  Suppose that on a topological space $X$ we are given, for each dyadic rational number $r = \frac{m}{2^n}$ where $0 \leq m \leq 2^{m}$, an open set $U_{r}$ such that $r < s \implies \overline{U_{r}} \subseteq U_{s}$.
  Then the function
  \begin{align*}
    f\colon X &\to \R \\
      x &\mapsto
          \begin{cases}
            \inf\set{r | x \in U_{r}} & x \in U_{1} \\
            1                         & x \notin U_{1}
          \end{cases}
  \end{align*}
  is continuous.
\end{lem}
\begin{pf}
  Note that for $r$ dyadic we have
  \begin{align*}
    f(x) <    r \implies    x \in U_{r}, && \text{ hence } &&  &f(x) \geq r \impliedby x \notin U_{r}, \\
    f(x) \leq r \impliedby  x \in U_{r}, && \text{ hence } &&  &f(x) > r \implies      x \notin U_{r} \implies x \in X \setminus U_{r}.
  \end{align*}
  Thus for $\alpha$ real we have
  \[
    f^{-1}(-\infty, \alpha) = \set{x | f(x) < \alpha} = \bigcup_{r < \alpha} U_{r}
  \]
  which is open, and for $\beta$ real we have
  \[
    f^{-1}(\beta, \infty) = \set{x | f(x) > \beta} = \bigcup_{r > \beta} X \setminus U_{r} = \bigcup_{s > \beta} X \setminus \overline{U_{s}}
  \]
  which is also open.
  These half open intervals form a subbasis for $\R$ and thus $f$ is continuous.
\end{pf}

\begin{lem}[Urysohn's Lemma]\label{lem: urysohn_lemma}
  If $X$ is normal and $F \subseteq U$ where $F$ is closed and $U$ is open, then there is a map $f\colon X \to [0, 1]$ which is $0$ on $F$ and $1$ on $X \setminus U$.
\end{lem}
\begin{pf}
  Let $U_{1} = U$ and use normality of $X$ to find $U_{0}$ such that $F \subseteq U_{0} \subseteq \overline{U_{0}} \subseteq U_{1}$.
  We can do this by \Cref{exercise: bredon_1.5.6}.
  We can iterate this to find $U_{1/2}$ such that $\overline{U_{0}} \subseteq U_{1/2} \subseteq \overline{U_{1/2}} \subseteq U_{1}$.
  We also find $U_{1/4}$ such that $\overline{U_{0}} \subseteq U_{1/4} \subseteq \overline{U_{1/4}} \subseteq U_{1/2}$ and $U_{3/4}$ such that $\overline{U_{1/2}} \subseteq U_{3/4} \subseteq \overline{U_{3/4}} \subseteq U_{1}$.
  Iterate this construction repeatedly and apply \Cref{lem: dyadic_lem_for_urysohn}.
\end{pf}

\begin{cor}\label{cor: normal_implies_comp_reg}
  Normality implies Complete Regularity. $\qed$
\end{cor}

\begin{thrm}[Tietze Extension Theorem]
  Let $X$ be normal and $F \subseteq X$ closed and let $f\colon F \to \R$ be continuous.
  Then there is a map $g\colon X \to \R$ such that $g(x) = f(x)$ for all $x \in F$.
  Moreover, it can be arranged that
  \[
    \sup_{x \in F} f(x) = \sup_{x \in X} g(x) \text{ and } \inf_{x \in F} f(x) = \inf_{x \in X} g(x).
  \]
\end{thrm}
\begin{pf}
  First we consider the case that $f$ is bounded.
  Without loss of generality, suppose that $0 \leq f(x) \leq 1$ with infimum $0$ and supremum $1$.
  By \Cref{lem: urysohn_lemma} and the fact that $[0, 1]$ is homeomorphic to $\bqty{0, \frac{1}{3}}$, there exists a function $g_{1}\colon X \to \bqty{0, \frac{1}{3}}$ such that
  \[
    g_{1}(x) = \begin{cases}
                 0           & \text{ if } x \in F \text{ and } f(x) \leq \frac{1}{3}, \\
                 \frac{1}{3} & \text{ if } x \in F \text{ and } f(x) \geq \frac{2}{3}.
               \end{cases}
  \]
  This can be done by taking the closed set as $F \cap f^{-1}\pqty{\left(-\infty, \frac{1}{3}\right]}$ and the open set as $f^{-1}\pqty{\pqty{-\infty, \frac{2}{3}}}$.
  Let $f_{1} = f - g_{1}$.
  Note that $0 \leq f_{1}(x) \leq \frac{2}{3}$ for all $x \in F$.
  Then we can repeat this construction and find $g_{3}\colon X \to \bqty{0, \frac{1}{3} \cdot \frac{2}{3}}$ such that
  \[
    g_{2}(x) = \begin{cases}
                 0           & \text{ if } x \in F \text{ and } f(x) \leq \frac{1}{3} \cdot \frac{2}{3}, \\
                 \frac{1}{3} \cdot \frac{2}{3} & \text{ if } x \in F \text{ and } f(x) \geq \frac{2}{3} \cdot \frac{2}{3}.
               \end{cases}
  \]
  Let $f_{2} = f_{1} = g_{2}$ and note that $0 \leq f_{2}(x) \leq \pqty{\frac{2}{3}}^{2}$ for all $x \in F$.
  Inductively we can find $f_{n}$ with $0 \leq f_{n}(x) \leq \pqty{\frac{2}{3}}^{n}$ for $x \in F$.
  Then find $g_{n + 1}\colon X \to \bqty{0, \pqty{\frac{1}{3}} \cdot \pqty{\frac{2}{3}}^{n}}$ such that
  \[
    g_{n + 1}(x) = \begin{cases}
                 0           & \text{ if } x \in F \text{ and } f(x) \leq \frac{1}{3} \cdot \pqty{\frac{2}{3}}^{n}, \\
                 \frac{1}{3} \cdot \pqty{\frac{2}{3}}^{n} & \text{ if } x \in F \text{ and } f(x) \geq \frac{2}{3} \cdot \pqty{\frac{2}{3}}^{n}.
               \end{cases}
  \]
  Let $f_{n + 1} = f_{n} - g_{n + 1}$.

  \clearpage

  Now let $g(x) = \sum_{n = 1}^{\infty} g_{n}(x)$.
  This converges uniformly since $0 \leq g_{n}(x) \leq \pqty{\frac{1}{3}} \cdot\pqty{\frac{2}{3}}^{n - 1}$ which then implies that $g$ is continuous.
  For $x \in F$ we have that
  \begin{align*}
    f - g_{1}     &= f_{1}, \\
    f_{1} - g_{2} &= f_{2}, \\
                  &\vdots
  \end{align*}
  and by adding and cancelling we get
  \[
    f - (g_{1} + g_{2} + \cdots + g_{n}) = f_{n} \text{ and } 0 \leq f_{n} \leq \pqty{\frac{2}{3}}^{n}.
  \]
  Taking the limit yields that $g(x) = f(x)$ on $F$.
  Clearly the bounds are also correct.
  The infimum of $f$ is $0$ and the infimum of each of the $g_{n}$ is also $0$.
  The supremum of $f$ is $1$ and we have that
  \[
    \sup_{x \in X} g(x) = \sum_{n = 1}^{\infty} g_{n}(x) = \frac{1}{3} \sum_{n = 1}^{\infty} \pqty{\frac{2}{3}}^{n} = \frac{1}{3} * 3 = 1.
  \]

  We now consider the unbounded cases
  \begin{align*}
   \text{Case I:\@}   &f \text{ is unbounded in both directions}, \\
   \text{Case II:\@}  &f \text{ is bounded below by } a, \\
   \text{Case III:\@} &f \text{ is bounded above by } b.
  \end{align*}

  Let $h$ be a homeomorphism:
  \begin{align*}
    (-\infty, \infty) &\to (0, 1) &\text{ in Case I, }  \\
    [a, \infty)       &\to [0, 1) &\text{ in Case II, } \\
    (-\infty, b]      &\to (0, 1] &\text{ in Case III.\@}
  \end{align*}
  Then we have that $h \circ f$ is bounded by $0, 1$ and we can extend it to $g_{1}$.
  If we can arrange that $g_{1}(x)$ is never $0$ (respectively $1$) if $h \circ f$ is never $0$ (respectively $1$) then $g = h^{-1} \circ g_{1}$ would be defined and would extend $f$.
  Thus put
  \begin{align*}
    C &= \set{x | g_{1}(x) = 0 \text{ or 1 }} &\text{ in Case I, }  \\
    C &= \set{x | g_{1}(x) = 1 } &\text{ in Case II, } \\
    C &= \set{x | g_{1}(x) = 0 } &\text{ in Case III.\@}
  \end{align*}
  Then $C$ is closed and $C \cap F = \emptyset$.
  So there exists a function $k\colon X \to [0, 1]$ such that $k = 0$ on $C$ and $k = 1$ on $F$.
  Let $g_{2} = k \cdot g_{1} + (1 - k) \cdot \frac{1}{2}$.
  Then $g_{2}$ is always between $g_{1}$ and $\frac{1}{2}$ with $g_{2} \neq g_{1}$ on $C$.
  Also $g_{2} = g_{1} = h \circ f$ on $F$.
  \quest{Thus $g = h^{-1} \circ g_{2}$ extends $f$ in the desired manner}.
\end{pf}

\clearpage

\subsection*{Exercises}

\begin{exercise}[\tcite{book:Bredon} 10.2]
  If $F$ is a closed subspace of the normal space $X$ then any map $F \to \R^{n}$ can be extended to $X$.
\end{exercise}
\begin{pf}
  Tietze Extension but on each projection.
\end{pf}

\clearpage

\chapter{Locally Compact Spaces}

Many spaces, such as Euclidean spaces are not compact themselves but contain ``enough'' compact subspaces that are important to the properties of the spaces.
Locally Compact spaces are one such class of spaces.

\begin{defn}[Locally Compact Space]
  A topological space is \emph{locally compact} if every point has a compact neighborhood.
\end{defn}

\begin{prop}\label{prop: closed_subset_local_compact}
  Closed subsets of locally compact spaces are locally compact.
\end{prop}
\begin{pf}
  Let $X$ be locally compact and $F \subseteq X$ closed.
  Let $x \in F$.
  We have that $x \in X$ so it has a compact neighborhood $K$.
  Clearly $K \cap F$ is also a neighborhood of $x$.
  Let $\set{U_{\alpha}}$ be an open cover of $K \cap F$.
  $F$ is closed so $X \setminus F$ is open.
  We have that $\set{U_{\alpha}} \cup \set{X \setminus F}$ is an open cover for $K$ and thus has a finite subcover, set $\set{V_{\alpha}}$.
  Without loss of generality, we can discard $X \setminus F$ from $\set{V_{\alpha}}$ if it is included and thus we are left with a finite subcover of $K \cap F$.
  Thus $F$ is locally compact.
\end{pf}

However, the similar statement for open subsets is much more complicated.

\begin{lem}\label{lem: separate_point_compact_Hausdorff}
  Suppose that $X$ is Hausdorff, $K \subseteq X$ compact, and $p \in X \setminus K$.
  Then there are disjoint open sets $U, W$ such that $p \in U$ and $K \subseteq W$.
\end{lem}
\begin{pf}
  If $q \in K$, then by Hausdorffness we can find disjoint open sets $U_{q}$ and $V_{q}$ such that $p \in U_{q}$ and $q \in V_{q}$.
  Since $K$ is compact, we can find $q_{1}, \ldots, q_{n}$ such that $K \subseteq V_{q_{1}} \cup \cdots \cup V_{q_{n}}$.
  Then the sets
  \[
    U = U_{q_{1}} \cap \cdots \cap U_{q_{n}} \text{ and } W = V_{q_{1}} \cup \cdots \cup V_{q_{n}}
  \]
  satisfy the claim.
\end{pf}

\clearpage

\begin{lem}\label{lem: finite_subcollection_empty}
  If $\set{K_{\alpha}}$ is some collection of compact subsets of a Hausdorff space $X$ and $\bigcap_{\alpha} K_{\alpha} = \emptyset$ then some finite subcollection of the $\set{K_{\alpha}}$ also has empty intersection.
\end{lem}
\begin{pf}
  Let $V_{\alpha} = X \setminus K_{\alpha}$.
  Let $K_{1} \in \set{K_{\alpha}}$.
  Since no point of $K_{1}$ belongs to every $K_{\alpha}$, we have that $\set{V_{\alpha}}$ is an open cover of $K_{1}$.
  Thus $K_{1} \subseteq V_{\alpha_{1}} \cup \cdots \cup V_{\alpha_{n}}$ which implies that $K_{1} \cap K_{\alpha_{1}} \cap \cdots \cap K_{\alpha_{n}} = \emptyset$.
\end{pf}

\begin{prop}\label{prop: open_between_compact_and_open}
  If $X$ is a locally compact Hausdorff space and $K \subseteq U \subseteq X$ where $K$ is compact and $U$ is open, then there exists an open subset $V \subseteq X$ such that $K \subseteq V \subseteq \overline{V} \subseteq U$ and $\overline{V}$ is compact.
\end{prop}
\begin{pf}
  Since every point in $K$ has a neighborhood with compact closure due to the fact that closed subsets of compact sets are compact, and since $K$ is covered by the union of finitely many of these neighborhoods, we have that $K$ lies in an open set $G$ with compact closure.
  If $U = X$, then we are done by taking $V = G$.
  Otherwise, $U \subsetneq X$.
  Let $C = X \setminus U$.
  By \Cref{lem: separate_point_compact_Hausdorff}, we can separate $K$ and $p \in C$ by an open set $W_{p}$ such that $K \subseteq W_{p}$ and $p \notin \overline{W_{p}}$.
  Thus $\set{C \cap \overline{G} \cap \overline{W_{p}}}_{p \in C}$ is a collection of compact sets (by \Cref{prop: closed_cap_compact_is_compact}) with empty intersection.
  By \Cref{lem: finite_subcollection_empty}, we can find $p_{1}, \ldots, p_{n} \in C$ such that
  \[
    C \cap \overline{G} \cap \overline{W_{p_{1}}} \cap \cdots \cap \overline{W_{p_{n}}} = \emptyset.
  \]
  Now let
  \[
    V = G \cap W_{p_{1}} \cap W_{p_{n}}.
  \]
  We have that $K \subseteq V$ since $K \subseteq G$ and $K \subseteq$ each of the $W_{p_{i}}$.
  Then note that we have
  \[
    \overline{V} \subseteq \overline{G} \cap \overline{W_{p_{1}}} \cap \cdots \cap \overline{W_{p_{n}}}.
  \]
  since $C = X \setminus U$.
  We also have that $\overline{V}$ is compact.
  This is because $\overline{V}$ is contained in the intersection of a compact set $\overline{G}$ with a bunch of closed sets.
  Since this intersection is compact, $\overline{V}$ is a closed subset of a compact set which itself is compact.
\end{pf}

\begin{prop}\label{prop: open_subset_local_compact}
  Open subsets of locally compact Hausdorff spaces are locally compact.
\end{prop}
\begin{pf}
  Suppose that $X$ is locally compact and $U \subseteq X$ is open.
  Let $x \in U$.
  We have that $\set{x}$ is clearly compact.
  Then by \Cref{prop: open_between_compact_and_open} we can find open $V$ such that $\set{x} \subseteq V \subseteq \overline{V} \subseteq U$ with $\overline{V}$ compact.
  Thus $x$ has a compact neighborhood $\overline{V} \subseteq U$.
\end{pf}

\clearpage

\begin{thrm}
  If $X$ is a locally compact Hausdorff space then each neighborhood of a point $x \in X$ contains a compact neighborhood of $x$.
  That is, the compact neighborhoods of $x$ form a neighborhood basis at $x$.
  Moreover, $X$ is completely regular.
\end{thrm}
\begin{pf}
  Let $C$ be a compact neighborhood of $x$ and $U$ an arbitrary neighborhood of $x$.
  Let $V \subseteq C \cap U$ be open with $x \in V$.
  Then $\overline{V} \subseteq C$ is compact Hausdorff and therefore regular (\Cref{thrm: compact_haus_is_norm}).
  Thus there is a neighborhood $N \subseteq V$ of $x$ in $C$ which is closed in $\overline{V}$ and thus closed in $C$ since regular spaces have a closed neighborhood basis (\Cref{prop: regular_iff_closed_neighborhood_basis}).
  Since $N$ is closed in $C$, it is compact by \Cref{thrm: closed_subset_of_compact_is_compact}.
  Then since $N$ is a neighborhood of $x$ in $\overline{V}$ and $N = N \cap V$, $N$ is a neighborhood of $x$ in the open set $V$ and thus in $X$.
\end{pf}

\begin{thrm}\label{thrm: one_point_compactification}
  Let $X$ be a locally compact Hausdorff space.
  Define $X^{+} \defeq X \cup \set{\infty}$ where $\infty$ is just some point not in $X$.
  Define the open sets of $X^{+}$ to be open in $X$ or open in $X \setminus C$ where $C \subseteq X$ is compact.
  This defines a topology on $X^{+}$ which makes $X^{+}$ into a compact Hausdorff space.
  Moreover, this topology on $X^{+}$i s the only topology making $X^{+}$ a compact Hausdorff space with $X$ as a subspace.
\end{thrm}
\begin{pf}
  The whole space $X^{+}$ and $\emptyset$ are clearly open.
  Let $U = X^{+} \setminus C$ for $C \subseteq X$ compact.
  Then for $V$ open we have $U \cap V = V \setminus C$ which is open in $X$ since compact subsets of Hausdorff spaces are closed (\Cref{thrm: compact_subset_Hausdorff_closed}).
  The other cases of intersection of two open sets are trivial.

  Let $\set{U_{\alpha}}$ be a collection of open sets and $U = \bigcup_{\alpha} U_{\alpha}$.
  If all the $U_{\alpha}$ are open subsets of $X$ the result is immediate.
  Suppose some $U_{\beta} = X^{+} \setminus C$ where $C \subseteq X$ is compact.
  Then
  \[
    X^{+} \setminus U = \bigcap_{\alpha} X^{+} \setminus U_{\alpha} = C \cap \pqty{\bigcap_{\alpha \neq \beta} X \setminus U_{\alpha}}
  \]
  which is closed in $C$ and thus compact.
  Therefore $X^{+} \setminus U$ is closed and $U$ is open and we have a well defined topology.

  Suppose that $\set{U_{\alpha}}$ is an open cover of $X^{+}$.
  One of these sets, say $U_{\beta}$, contains $\infty$.
  $X \setminus U_{\beta}$ is closed and thus compact and has a finite subcover.
  Adding $U_{\beta}$ to this finite subcover yields a finite subcover of $X^{+}$.

  To show that $X^{+}$ is Hausdorff we just need to separate $\infty$ from any $x \in X$.
  Let $V$ be an open neighborhood of $x$ in $X$ such that $\overline{V} \subseteq X$ is compact.
  Then $x \in V$ and $\infty \in X^{+} \setminus \overline{V}$ is the required separation.

  For uniqueness, let $U \subseteq X^{+}$ be an open set in some such topology.
  Then for $C = X^{+} \setminus U$ is closed and therefore compact.
  If $C \subseteq X$ then $U$ is open in the topology described in the statement.
  If $C \not\subseteq X$ then $U \subseteq X$ must be open in $X$ since $X$ is a subspace and again $U$ is open in the topology described in the open sets.
  We now show that the open sets in the topology described in the statement are open in the given topology.
  If $U$ is open in $X$, then $U = U' \cap X$ for some $U'$ open in $X^{+}$ since $X$ is a subspace.
  But $X$ is open in $X^{+}$ since singletons are closed in Hausdorff spaces.
  Thus $U = U' \cap X$ is open in $X^{+}$.
  Next if $C$ is compact in $X$ then it must be compact in $X^{+}$ and thus $C$ is closed in $X^{+}$.
  It follows that $X^{+} \setminus X$ is open in the given topology.
\end{pf}

\begin{defn}[One-Point Compactification]
  The set $X^{+}$ in \Cref{thrm: one_point_compactification} is called the \emph{one-point compactification} of $X$.
\end{defn}

Note that if $X$ itself is compact, then $\infty$ is an isolated point (clopen) in $X^{+}$ and thus $X$ is clopen in $X^{+}$.

\begin{thrm}
  Suppose that $X$ and $Y$ are locally compact Hausdorff spaces and that $f\colon X \to Y$ is continuous.
  Then $f$ is proper if and only if $f$ extends to a continuous function $f^{+}\colon X^{+} \to Y^{+}$ with $f^{+}(\infty) = \infty$.
\end{thrm}
\begin{pf}
  $f^{+}$ exists as a function so we just need to show continuity.
  Suppose that $U \subseteq Y^{+}$ is open.
  If $U \subseteq Y$ then the result is immediate.
  Otherwise, $U = Y^{+} \setminus C$ with $C \subseteq Y$ compact.
  Thus $\pqty{f^{+}}^{-1}(U) = X^{+} \setminus f^{-1}(C)$.
  Since $f$ is proper, $f^{-1}(C)$ is compact and therefore closed.
  The result follows.

  Now suppose that we have $f^{+}$ as described.
  Then we have that $\pqty{f^{+}}^{-1}(\infty) = \infty$ and thus $\pqty{f^{+}}^{-y}(Y) = X$.
  If $C \subseteq Y$ is compact then it is closed and so $f^{-1}(C)$ is closed in $X^{+}$ and hence compact and contained in $X$.
  Thus $f$ is proper.
\end{pf}

\begin{prop}
  If $f\colon X \to Y$ is a proper map between locally compact Hausdorff spaces, then $f$ is closed.
\end{prop}
\begin{pf}
  We have an extension $f^{+}\colon X^{+} \to Y^{+}$.
  If $F \subseteq X$ is closed in $X$ then $F \cup \set{\infty}$ is closed in $X^{+}$ and hence compact.
  Thus$ f^{+}(F \cup \set{\infty})$ is compact by \Cref{thrm: image_of_compact_is_compact} and hence closed by \Cref{thrm: compact_subset_Hausdorff_closed}.
  But then $f(F) = f^{+}(F \cup \set{\infty}) \cap Y$ is closed in $Y$.
\end{pf}

\begin{defn}[Locally Closed Space]
  A subspace $A$ of a topological space is said to be \emph{locally closed} if each point $a \in A$ has an open neighborhood $U_{a}$ such that $U_{a} \cap A$ is closed in $U_{a}$.
\end{defn}

\begin{prop}\label{prop: subspace_locally_closed_iff}
  A subspace $A \subseteq X$ is locally closed if and only if it has the form $A = C \cap U$ where $U$ is open in $X$ and $C$ is closed in $X$.
\end{prop}
\begin{pf}
  Let $U = \bigcup_{a \in A} U_{a}$ which is open and $C = \overline{A}$ which is closed.
  Then we have that
  \[
    C \cap U = \overline{A} \cap \bigcup_{a \in A} U_{a} = \bigcup_{a} \overline{A} \cap U_{a} = A \cap U = A.
  \]
\end{pf}

\clearpage

\begin{thrm}
  For a Hausdorff space $X$, the following are equivalent:
  \begin{enumerate}
  \item $X$ is locally compact;
  \item $X$ is a locally closed subspace of a compact Hausdorff space; and
  \item $X$ is a locally closed subspace of a locally compact Hausdorff space.
  \end{enumerate}
\end{thrm}
\begin{pf}
  If $X$ is locally compact, then it is an open subspace of its one-point compactification $X^{+}$ which is a compact Hausdorff space by \Cref{thrm: one_point_compactification}.
  Clearly $X = X^{+} \cap X$ and so by \Cref{prop: subspace_locally_closed_iff} $X$ is a locally closed subspace.
  Thus $1$ implies $2$.
  Compact implies locally compact so $2$ implies $3$ immediately.
  If $Y \containseq X$ is locally compact and $X = C \cap U$ for $C \subseteq Y$ closed and $U \subseteq Y$ open, then $C$ is locally compact by \Cref{prop: closed_subset_local_compact} and $X = C \cap U$ is open in $C$ and hence also locally compact.
  Thus $3$ implies $1$.
\end{pf}

The prior result begs the question of when a topological space $X$ can be embedded into some compact, Hausdorff space $Y$ as a subspace.
Since $Y$ is normal (\Cref{thrm: compact_haus_is_norm}), we have that $Y$ is completely regular.
Since subspaces of completely regular sets are completely regular, we must have that $X$ is completely regular.

\begin{defn}[Stone-\Cech\ compactification]
  Let $X$ be a completely regular space and \textbf{F} the set of all maps $f\colon X \to [0, 1]$.
  Then let
  \begin{align*}
    \Phi\colon X &\to [0, 1]^{\textbf{F}} = \Bigtimes_{f \in \textbf{F}} [0, 1] \\
            \Phi(x)(f) &= f(x).
  \end{align*}
  Here we are regarding an element of $[0, 1]^{\textbf{F}}$ as a function $\textbf{F} \to [0, 1]$.
  The closure of $\Phi(X)$ is the \emph{Stone-\Cech\ compactification} of $X$ and is denoted $\beta(X)$.
\end{defn}

\clearpage

\begin{thrm}
  If $X$ is a completely regular space, then $\beta(X)$ is compact Hausdorff and $\Phi(X) \to \beta(X)$ is an embedding.
\end{thrm}
\begin{pf}
  The function $\Phi$ is one-to-one since if $\Phi(x) = \Phi(y)$ then $f(x) = f(y)$ for all $f\colon: X \to [0, 1]$ which by complete regularity implies that $x = y$.

  To prove continuity, let $\set{x_{\alpha}}$ be a net in $X$ converging to $x$.
  Then we have that
  \[
    \lim(\Phi(x_{\alpha})(f)) = \lim(f(x_{\alpha})) = f(x) = \Phi(x)(f)
  \]
  for all $f\colon X \to [0, 1]$.
  This implies that $\lim(\Phi(x)) = \Phi(x)$ by \Cref{prop: net_of_product_converges}.
  We get continuity thus by \Cref{prop: continuous_iff_net_converges}.

  For continuity of the inverse, take some net $\set{x_{\alpha}}$ in $X$ such that $\Phi(x_{\alpha})$ converges to $\Phi(x)$.
  Then for all $f\colon X \to [0, 1]$ we have
  \[
    \lim(f(x_{\alpha})) = \lim(\Phi(x_{\alpha})(f)) = \Phi(x)(f) = f(x).
  \]
  Suppose that $\set{x_{\alpha}}$ does not converge to $x$.
  Then we have some neighborhood $U$ such that $x_{\alpha}$ is frequently in $X \setminus U$.
  But some map $f\colon X \to [0, 1]$ must exist such that $f$ is $0$ at $x$ and $1$ on $X \setminus U$.
  Thus $f(x_{\alpha})$ is frequently $1$ but $f(x) = 0$ which is a contradiction to the convergence of $f(x_{\alpha})$.
\end{pf}

\begin{thrm}
  If $X$ is completely regular and $f\colon X \to\ R$ is a bounded real valued map, then $f$ can be extended uniquely to a map $\beta(X) \to \R$.
\end{thrm}
\begin{pf}
  Without loss of generality, suppose the image of $f$ is bounded between $0$ and $1$.
  Recall that $[0, 1]^{\textbf{F}}$ is the set of all functions $\textbf{F} \to [0, 1]$.
  Consider the function
  \begin{align*}
    \overline{f}\colon [0, 1]^{\textbf{F}} &\to \R \\
                \mu &\mapsto \mu(f)
  \end{align*}
  Suppose that $\set{\mu_{\alpha}}$ is a net in $[0, 1]^{\textbf{F}}$ converging to $\mu$.
  Then we have that
  \[
    \lim(\overline{f}(\mu_{\alpha})) = \lim(\mu_{\alpha}(f)) = \mu(f) = \overline{f}(\mu),
  \]
  which shows that $\overline{f}$ is continuous.
  If $x \in X$ then we have that $\overline{f}(\Phi(x)) = \Phi(x)(f) = f(x)$ which shows that $\overline{f}$ does extend $f$.
\end{pf}

\clearpage

\begin{lem}
  Let $A \subseteq X$ and $f\colon A \into Z$ be a map into a Hausdorff space $Z$.
  Then there is at most one extension of $f$ into a map $\overline{f}\colon \overline{A} \to Z$.
\end{lem}
\begin{pf}
  Suppose $\overline{f}, \overline{f}'\colon \overline{A} \to Z$ are two different such extensions of $f$.
  Then choose $x$ such that $\overline{f}(x) \neq \overline{f}'(x)$.
  Then by Hausdorffness of $Z$ there exists disjoint open neighborhoods $U$ and $U'$ such that $\overline{f}(x) \in U$ and $\overline{f}' \in U'$.
  Choose a neighborhood $V$ of $x$ such that $\overline{f}(V) \subseteq U, \overline{f}'(V) \subseteq U'$.
  By characterization of closure, $V$ intersects $A$ and this intersection contains some $y \in A \cap V$.
  Thus $\overline{f}(y) \in U, \overline{f}'(y) \in U'$.
  However by extension we have that $\overline{f}(y) = f(y) = \overline{f}'(y)$ contradicting disjointness of $U$ and $U'$.
\end{pf}

\begin{prop}
  Let $X$ be a completely regular space and $C$ a compact Hausdorff space.
  The Stone-\Cech\ compactification $\beta(X)$ satisfies the following universal property.
  If $f\colon X \to C$ is any compactification then there exists a unique map $g\colon\beta(X) \to C$ such that the following diagram commutes:
  \begin{figure}[h]
    \centering
    % https://q.uiver.app/#q=WzAsMyxbMCwwLCJYIl0sWzIsMCwiXFxiZXRhKFgpIl0sWzIsMiwiQyJdLFswLDFdLFsxLDIsIlxcZXhpc3RzIX5nIiwwLHsic3R5bGUiOnsiYm9keSI6eyJuYW1lIjoiZGFzaGVkIn19fV0sWzAsMiwiZiIsMl1d
    \[
      \begin{tikzcd}[every label/.append style={font=\normalsize}]
        X && {\beta(X)} \\
        \\
        && C
        \arrow[from=1-1, to=1-3]
        \arrow["{\exists!~g}", dashed, from=1-3, to=3-3]
        \arrow["f"', from=1-1, to=3-3]
      \end{tikzcd}
    \]
    \caption{The universal property of the Stone\Cech\ Compatification}\label{fig:stone-cech-univ}
  \end{figure}
\end{prop}
\begin{pf}
  Note that compact spaces are immediately locally compact \quest{and locally compact spaces are completely regular} so $C$ is completely regular.
  By \Cref{lem: Hausdorff_and_comp_reg-ish_embedding} we have that $C$ can be embedded into $[0, 1]^{J}$ for some $J$ so without loss of generality suppose that $C \subseteq [0, 1]^{J}$.
  Then each component function $f_{\alpha}$ of $f$ is a bounded, continuous, real-valued function of $X$.
  Thus $f_{\alpha}$ can be uniquely extended to a map $\overline{f_{\alpha}}\colon \beta(X) \to \R$.
  Define $\overline{f}\colon \beta(X) \to \R^{J}$ by $\overline{f}(x) = (\overline{f_{\alpha}}(x))_{\alpha \in J}$.
  Then $\overline{f}$ is continuous by product topology.
  In fact, we now have that $\overline{f}$ maps $\beta(X)$ into $C \subseteq \R^{J}$.
  Now let $x \in X$ and recall that $\Phi\colon X \to \beta(X)$ is the embedding of $X$ into its Stone-\Cech\ compactification.
  We have that
  \[
    \overline{f}(\Phi(x)) = (\overline{f_{\alpha}}(\Phi(x)))_{\alpha \in J} = (f_{\alpha}(x))_{\alpha \in J} = f(x)
  \]
  which shows that the diagram \Cref{fig:stone-cech-univ} commutes.
  Uniqueness is immediate by the prior lemma.
\end{pf}

\clearpage

\subsection*{Exercises}

\begin{exercise}[\tcite{book:Bredon} 1.11.1]
  Show that the Stone-\Cech\ compactification $\beta(\cdot)$ is a functor on completely regular spaces by showing that a map $f\colon X \to Y$ induces a unique commutative diagram
  \begin{figure}[h]
    \centering
    % https://q.uiver.app/#q=WzAsNCxbMCwwLCJYIl0sWzIsMCwiWSJdLFsyLDIsIlxcYmV0YShZKSJdLFswLDIsIlxcYmV0YShYKSJdLFswLDEsImYiXSxbMSwyLCJpX1kiXSxbMywyLCJcXGJldGEoZikiLDJdLFswLDMsImlfWCIsMl1d
    \[
      \begin{tikzcd}[every label/.append style={font=\normalsize}]
        X && Y \\
        \\
        {\beta(X)} && {\beta(Y)}
        \arrow["f", from=1-1, to=1-3]
        \arrow["{i_Y}", from=1-3, to=3-3]
        \arrow["{\beta(f)}"', from=3-1, to=3-3]
        \arrow["{i_X}"', from=1-1, to=3-1]
      \end{tikzcd}
    \]
    \addtocounter{figure}{1}
  \end{figure}
  such that $\beta(f \circ g) = \beta(f) \circ \beta(g)$ and $\beta(1_{X}) = 1_{\beta(X)}$.
\end{exercise}
\begin{pf}
  Suppose that $X, Y$ are completely regular spaces, $1_{X}\colon X \to \beta(X)$ and $1_{Y}\colon Y \to \beta(Y)$ their respective compactifications, and $f$ a map $X \to Y$.
  \begin{figure}[h]
    \centering
    % https://q.uiver.app/#q=WzAsMyxbMCwwLCJYIl0sWzIsMCwiXFxiZXRhKFgpIl0sWzIsMiwiXFxiZXRhKFkpIl0sWzAsMSwiaV9YIl0sWzEsMiwiXFxleGlzdHMhflxcYmV0YShmKSIsMCx7ImNvbG91ciI6WzI0MCw2MCw2MF0sInN0eWxlIjp7ImJvZHkiOnsibmFtZSI6ImRhc2hlZCJ9fX0sWzI0MCw2MCw2MCwxXV0sWzAsMiwiaV95IFxcY2lyYyBmIiwyXV0=
    \[
      \begin{tikzcd}[every label/.append style={font=\normalsize}]
        X && {\beta(X)} \\
        \\
        && {\beta(Y)}
        \arrow["{i_X}", from=1-1, to=1-3]
        \arrow["{\exists!~\beta(f)}", color={rgb,255:red,92;green,214;blue,214}, dashed, from=1-3, to=3-3]
        \arrow["{i_y \circ f}"', from=1-1, to=3-3]
      \end{tikzcd}
    \]
    \addtocounter{figure}{1}
  \end{figure}

  Since $\beta(Y)$ is compact Hausdorff, by the universal property of $\beta(X)$ we have that there exists a unique map $\beta(f)\colon \beta(X) \to \beta(Y)$ such that $\beta(f) \circ i_{X} = i_{Y} \circ f$.
  Letting $Y = X$ and $f$ be $1_{X}$ and using the fact that $1_{\beta(X)}$ is a map $\beta(X) \to \beta(X)$ yields that $\beta(1_{X}) = 1_{\beta(X)}$.

  Now let $Z$ be another completely regular space and $g$ a map $Y \to Z$.
  \begin{figure}[h]
    \centering
    % https://q.uiver.app/#q=WzAsNixbMCwwLCJYIl0sWzIsMCwiWSJdLFs0LDAsIloiXSxbMCwyLCJcXGJldGEoWCkiXSxbMiwyLCJcXGJldGEoWSkiXSxbNCwyLCJcXGJldGEoWikiXSxbMCwxLCJmIl0sWzEsMiwiZyJdLFswLDMsImlfWCIsMl0sWzEsNCwiaV9ZIiwyXSxbMiw1LCJpX1oiXSxbMyw0LCJcXGJldGEoZikiXSxbNCw1LCJcXGJldGEoZykiXSxbMyw1LCJcXGJldGEoZyBcXGNpcmMgZikiLDIseyJjdXJ2ZSI6Mn1dLFswLDIsImcgXFxjaXJjIGYiLDAseyJjdXJ2ZSI6LTJ9XV0=
    \[
      \begin{tikzcd}[every label/.append style={font=\normalsize}]
    	  X && Y && Z \\
    	  \\
    	  {\beta(X)} && {\beta(Y)} && {\beta(Z)}
    	  \arrow["f", from=1-1, to=1-3]
    	  \arrow["g", from=1-3, to=1-5]
    	  \arrow["{i_X}"', from=1-1, to=3-1]
    	  \arrow["{i_Y}"', from=1-3, to=3-3]
    	  \arrow["{i_Z}", from=1-5, to=3-5]
    	  \arrow["{\beta(f)}", from=3-1, to=3-3]
    	  \arrow["{\beta(g)}", from=3-3, to=3-5]
    	  \arrow["{\beta(g \circ f)}"', curve={height=24pt}, from=3-1, to=3-5]
    	  \arrow["{g \circ f}", curve={height=-24pt}, from=1-1, to=1-5]
      \end{tikzcd}
    \]
    \addtocounter{figure}{1}
  \end{figure}

  Then by the uniqueness of the universal property, we must have that $\beta(g) \circ \beta(f) = \beta(g \circ f)$.
\end{pf}

\clearpage


\chapter{Paracompact Spaces}

\begin{defn}[Refinement]
  If \textbf{U} and \textbf{V} are open coverings of a space then \textbf{U} is a \emph{refinement} of \textbf{V} if each element of \textbf{U} is a subset of some element of \textbf{V}.
\end{defn}

\begin{defn}[Locally Finite]
  A collection \textbf{U} of open subsets of a space $X$ is said to be \emph{locally finite} if each point $x \in X$ has a neighborhood $N$ which meets, nontrivially, only a finite number of the members of \textbf{U}.
\end{defn}

\begin{defn}[Paracompactness]
  A Hausdorff space $X$ is \emph{paracompact} if every open covering of $X$ has an open, locally finite refinement.
\end{defn}

\begin{prop}
  A closed subspace of a paracompact space is paracompact.
\end{prop}
\begin{pf}
  Let $A$ be a closed subspace of a paracompact space $X$.
  Cover $A$ with sets open in $X$ and add $X \setminus A$.
  This is an open covering of $X$ so can find a locally finite refinement of this open covering and intersect it with $A$.
  This gives a locally finite refinement of the originally covering of $A$.
  Indeed this is a refinement and it is open.
  For $a \in A$, we have that $a$ has a neighborhood $N$ which meets nontrivially with only a finite number of the members of the locally finite refinement of $X$.
  Thus $N$ must meet nontrivially with only a finite number of the members of the locally finite refinement of the covering of $A$.
\end{pf}

\clearpage

\begin{thrm}\label{thrm: paracompact_is_normal}
  A paracompact space is normal.
\end{thrm}
\begin{pf}
  We will first show that a paracompact space $X$ is regular.
  Suppose that $x \in X$ and $C \subseteq X$ is closed with $x \notin C$.
  For each point $y \in C$ there are disjoint open sets $U_{y}, V_{y}$ such that $x \in U_{y}$ and $y \in V_{y}$.
  Cover $X$ by $X \setminus C$ along with the $V_{y}$.
  Then we can find a open locally finite refinement $\set{U_{\alpha}}$.
  Let
  \[
    U = \bigcup_{\substack{\alpha \\ U_{\alpha} \subseteq \text{ some } V_{y}}} U_{\alpha}.
  \]
  Then we have that $C \subseteq U$.
  Since this is a locally finite collection, \quest{its closure $\overline{U}$ is equal to the union of the closures of the same $U_{\alpha}$'s.}
  However, $x$ is not in any of the $\overline{U_{\alpha}}$ and so $x \notin \overline{U}$.
  Thus $U$ and $X \setminus \overline{U}$ separates $x$ and $C$.

  The same argument with $C$ playing the role of $x$ and the other closed set playing the role of $C$ shows that $X$ is normal.
\end{pf}

Recall by \Cref{cor: normal_implies_comp_reg} that normal spaces are completely regular.
Thus paracompact spaces just need second countability to be metrizable.
It is known that metric spaces are paracompact, however the proof is beyond our scope for now.
\quest{Find examples.}
We can find paracompact spaces that are not metrizable, as well as paracompact spaces with subspaces that are not paracompact.

Recall that normality implies that there are many real valued maps.
\begin{defn}[Support]
  If $f$ is a real valued map, then the \emph{support of $f$} is
  \[
    \supp(f) = \overline{\set{x | f(x) \neq 0}}.
  \]
\end{defn}

\begin{defn}[Partition of Unity]
  Let $\set{U_{\alpha}}_{\alpha \in A}$ be an open covering of a space $X$.
  Then a \emph{partition of unity} subordinate to this covering is a collection of maps
  \[
    \set{f_{\beta}\colon X \to [0, 1]}_{\beta \in B}
  \]
  such that
  \begin{enumerate}
  \item There is a locally finite open refinement $\set{V_{\beta}}_{\beta \in B}$ such that $\supp(f_{\beta}) \subseteq V_{\beta}$ for each $\beta \in B$; and
  \item $\Sum_{\beta \in B} f_{\beta}(x) = 1$ for each $x \in X$.
  \end{enumerate}
\end{defn}

\clearpage

\begin{lem}\label{lem: paracompact_refinement_closed_subset}
  Suppose $X$ is a paracompact space.
  If $\textbf{U} = \set{U_{\alpha}}_{\alpha \in A}$ is an open cover of $X$, then there exists a open locally finite refinement $\textbf{V} = \set{V_{\alpha}}_{\alpha \in A}$ such that $\overline{V_{\alpha}} \subseteq U_{\alpha}$ for all $\alpha \in A$.
\end{lem}
\begin{pf}
  Recall by \Cref{exercise: bredon_1.5.6} that each $x \in X$ has a neighborhood $Y_{x}$ such that $\overline{Y_{x}} \subseteq U_{\alpha}$ for some $\alpha$.
  The cover $\set{Y_{x}}_{x \in X}$ has an open locally finite refinement $\textbf{Z} = \set{Z_{\beta}}_{\beta \in B}$ by paracompactness of $X$.
  Thus for each $\beta \in B$ there is some $x$ such that $Z_{\beta} \subseteq Y_{x}$ and thus there is some $\alpha$ such that $\overline{Z_{\beta}} \subseteq \overline{Y_{x}} \subseteq U_{\alpha}$.
  This defines a function $a\colon B \to A$ by choosing some such $\alpha$ for each $\beta$ and setting $a(\beta) = \alpha$.
  Then for each $\alpha \in A$, we have an open subset
  \[
    V_{\alpha} \defeq \bigcup_{\substack{\beta \in B \\ a(\beta) = \alpha}} Z_{\beta} \subseteq X.
  \]
  If no such $\beta$ exist then $V_{\alpha} = \emptyset$.
  \quest{Since \textbf{Z} is locally finite, we have that}
  \[
    \overline{V_{\alpha}} \defeq \bigcup_{\substack{\beta \in B \\ a(\beta) = \alpha}} \overline{Z_{\beta}}.
  \]
  This is contained in $U_{\alpha}$ as required.
\end{pf}

\begin{thrm}
  If $X$ is paracompact and \textbf{U} is an open covering of $X$ then there exists a partition of unity subordinate to \textbf{U}.
\end{thrm}
\begin{pf}
  Let $\textbf{U} = \set{U_{\alpha}}_{\alpha \in A}$ be an open cover of $X$.
  Apply \Cref{lem: paracompact_refinement_closed_subset} twice to get open locally finite covers $\textbf{V} = \set{V_{\alpha}}_{\alpha \in A}$ and $\textbf{W} = \set{W_{\alpha}}_{\alpha \in A}$ such that $\overline{W_{\alpha}} \subseteq V_{\alpha}$ and $\overline{V_{\alpha}} \subseteq U_{\alpha}$.
  By \Cref{lem: urysohn_lemma} for each $\alpha \in A$ we can find maps $f_{\alpha}\colon X \to [0, 1]$ such that $f_{\alpha}$ is $1$ on $\overline{W_{\alpha}}$ and $\supp(f_{\alpha}) \subseteq V_{\alpha}$.
  Let $f\colon X \to \R$ such that $f(p) = \Sum_{\alpha}f_{\alpha}(p)$.
  Since $\supp(f_{\alpha}) \subseteq V_{\alpha}$ we must have that $\set{\supp(f_{\alpha})}_{\alpha \in A}$ is locally finite, and thus only finitely many terms of the sum are non-zero and $f$ is continuous.
  $\set{W_{\alpha}}$ covers $X$ and so for each $p \in X$ we can find $\alpha$ such that $p \in W_{\alpha}$ and thus $f_{\alpha}(p) = 1$.
  Let $g_{\alpha} = f_{\alpha} / f$.
  We see that $\set{g_{\alpha}}_{\alpha \in A}$ is the desired partition of unity.
\end{pf}

The definition of paracompact is hard to verify.
The following criterion will apply in most cases we care about.
\begin{defn}[$\sigma$-compactness]
  A space is \emph{$\sigma$-compact} if it is the union of countably many compact subspaces.
\end{defn}

\clearpage

\begin{thrm}\label{thrm: paracompact_iff_union_sigma_compact}
  A locally compact, Hausdorff space $X$ is paracompact if and only if it is the disjoint union of open $\sigma$-compact subsets.
\end{thrm}
\begin{pf}
  $\implies\colon$ Using local compactness, we may cover $X$ by open sets $U_{\alpha}$ such that $\overline{U_{\alpha}}$ is compact.
  Using paracompactness, we can replace this covering by one which is also locally finite.
  We now inductively construct open sets whose closures are compact.
  First let $V_{1} = U_{\beta}$ for some given $\beta$.
  Now suppose $V_{n}$ has already been defined and consider all $U_{\alpha}$ which intersect $\overline{V_{n}}$.
  By compactness of $\overline{V_{n}}$ and local finiteness there are finitely many such $U_{\alpha}$.
  Let $V_{n + 1}$ be the union of these $U_{\alpha}$.
  Then $\overline{V_{n + 1}}$ is the union of the closures of these $U_{\alpha}$ and thus compact.
  Note that we have that $\overline{V_{n}} \subseteq V_{n + 1}$.
  Let $V = \bigcup_{n > 0} V_{n}$.
  Then $V$ is the union of countably many of these $U_{\alpha}$.
  By local finiteness, $\overline{V}$ is the union of the closures of the $U_{\alpha}$'s but each of these closures is contained in some $\overline{V} \subseteq V_{n + 1} \subseteq V$.
  Thus $\overline{V} = V$ is clopen by construction and is also $\sigma$-compact.
  \quest{The rest of this proof is a Maximality Principle argument which is left to the reader}.

  $\impliedby\colon$ Now suppose that $X$ is the disjoint union of open $\sigma$-compact spaces.
  It is clear that the disjoint union of $\sigma$-compact spaces is $\sigma$-compact and so we may assume that $X$ itself is $\sigma$-compact so $X = C_{1} \cup C_{2} \cup \cdots$ where each of the $C_{i}$ are compact.
  In order, alter each of the $C_{i}$ by adding to each of the $C_{j}$ that come after a finite union of compact sets, found by local compactness, whose interiors cover $C_{i}$.
  Thus we have that $C_{i} \subseteq \inte(C_{i + 1})$ for all $i$.
  Now define compact sets $A_{1} = C_{1}$ and $A_{i} = C_{i} \setminus \inte(C_{i - 1})$ for all $i > 1$.
  We can think of the $C_{i}$ as concentric dists and $A_{i}$ the rings between them \quest{fig}.
  Note that $A_{i}$ intersects nontrivially at most with $A_{i - 1}$ and $A_{i + 1}$.
  Each of the $A_{i}$ are compact and thus normal (\Cref{thrm: compact_haus_is_norm}).
  Thus we can enlarge each of the $A_{i}$ to find compact subsets $B_{i}$ such that $\inte(B_{i}) \containseq A_{i}$ and such that $B_{i}$ only nontrivially intersects $B_{i - 1}$ and $B_{i + 1}$.
  Now, given an open covering of $X$, we consider the induced covering of each of the $A_{i}$.
  We can \quest{select a finite refinement still covering $A_{i}$} and with none of the covering sets \quest{overflowing} from $B_{i}$.
  These finite coverings provide an open locally finite refinement of the original covering showing that $X$ is paracompact.
\end{pf}

\begin{thrm}
  If $X$ is locally compact, Hausdorff, and second countable then its one-point compactification $X^{+}$ is metrizable and $X$ is $\sigma$-compact and paracompact.
\end{thrm}
\begin{pf}
  Let \textbf{B} be a countable basis for $X$ and $C \subseteq X$ compact.
  We first show that $X^{+}$ is second countable by forming a countable basis for it.
  If we have an open set in $X$, we are done.
  We consider the other case.
  If $x \in C$ then $x$ has a compact neighborhood $N$ and there is some $U_{x} \in \textbf{B}$ such that $x \in U_{x} \subseteq N$.
  By compactness, $C$ is covered by a finite union $U_{x_{1}} \cup \cdots \cup U_{x_{n}}$.
  Let $V = X \setminus \bigcup_{i = 1}^{n} \overline{U_{i}}$.
  Then $V \cup \set{\infty}$ is a neighborhood of $\infty$ in $X^{+}$ contained in $X^{+} \setminus C$.
  These $V$ are indexed by a finite number of elements of \textbf{B} and thus there are countably many such $V$.
  Thus $X^{+}$ is second countable since $X$ is second countable by assumption.
  Since $X^{+}$ is second countable and locally compact, it is $\sigma$-compact and thus paracompact by \Cref{thrm: paracompact_iff_union_sigma_compact} $X^{+}$.
  Since $X^{+}$ is compact and Hausdorff, by \Cref{thrm: compact_haus_is_norm} it is normal.
  By \Cref{cor: normal_implies_comp_reg}, we have that $X^{+}$ is completely regular since it is normal.
  Then by the Urysohn Metrization Theorem (\Cref{thrm: urysohn_metrization}) since $X^{+}$ is completely regular and second countable, it is metrizable.
\end{pf}

\clearpage

\subsection*{Exercises}

\begin{exercise}[\tcite{book:Bredon} 1.12.1]
  Without using \Cref{thrm: paracompact_iff_union_sigma_compact} or the fact that metric spaces are paracompact, show that any open subspace of euclidean space is $\sigma$-compact.
  By \Cref{thrm: paracompact_iff_union_sigma_compact}, the space is also paracompact.
\end{exercise}
\begin{pf}
  Let $X \subseteq \R^{n}$ be an open subspace.
  Recall that $I^{n}$ is compact for $I = [0, 1]$ (\Cref{cor: I^n_compact}).
  This is of course translation invariant and there are countably many such translations since $\Z^{n}$ is countable.
  Every point in $X$ is contained in some translation $I^{n}$ and this forms a compact cover of $X$.
  Intersect each translation with $X$ to yield that $X$ is the union of countably many compact subspaces.
\end{pf}

\clearpage

\chapter{Quotient Spaces}

Quotient spaces, or identification spaces, are a central idea in topology.
It gives a foundation for the idea of ``pasting / gluing'' spaces together.
It also lets us construct many new spaces out of existing ones.

\begin{defn}[Quotient Topology, Topology Induced by a Function]
  Let $X$ be a topological space, $Y$ a set, and $f\colon X \onto Y$ a surjective function.
  Then the \emph{topology induced by $f$} or the \emph{quotient topology} is the topology where $V \subseteq Y$ is open if and only if $f^{-1}(V)$ is open in $X$.
  This is the largest topology on $Y$ making $f$ continuous.
\end{defn}

\begin{defn}[Quotient Space]
  Let $X$ be a topological space and $\sim$ an equivalence relation on $X$.
  Let $Y = X / \sim$ be the equivalence classes and $\pi\colon X \to Y$ the canonical projection.
  Then $Y$, with the topology induced by $\pi$, is a \emph{quotient space} of $X$.
\end{defn}

\begin{prop}
  A quotient space of a quotient space is a quotient space. $\qed$
\end{prop}

\begin{defn}[Identification Map]
  A map $f\colon X \to Y$ is called an \emph{identification map} if it is surjective and $Y$ has the quotient topology induced by $f$.
\end{defn}

\begin{prop}\label{prop: id_map_iff_all_g_cont}
  A surjective function $f\colon X \to Y$ is an identification map if and only if for all functions $g\colon Y \to Z$, we have that $g \circ f$ is continuous if and only if $g$ is continuous.
\end{prop}
\begin{pf}
  $\implies\colon$ Suppose that $f\colon X \to Y$ is an identification map and $g\colon Y \to Z$ some function between topological spaces.
  Suppose that $g \circ f$ is continuous and let $V \subseteq Z$ be open.
  Then we have that $f^{-1}\pqty{g^{-1}\pqty{V}}$ is open.
  But by definition of quotient topology on $Y$, we have that $f^{-1}\pqty{g^{-1}\pqty{V}}$ is open if and only if $g^{-1}(V)$ is open.
  Thus $g$ is continuous.
  Obviously the composition of continuous functions is continuous.

  $\impliedby\colon$ For the reverse direction, we consider the case  $Z = Y$ as sets with the identification topology on $Z$ and $g\colon Y \to Z$ the identity function $1_{Y}$.
  Then $1_{Y} \circ f$ is continuous meaning that $1_{Y}$ is continuous by assumption.
  But we also have that $1_{Y}^{-1}$ is continuous because $1_{Y}^{-1} \circ 1_{Y} \circ f = f$ is continuous.
  Since $Z$ has the quotient topology, $g$ is a homeomorphism.
  Thus, $Y = Z$ as spaces and $Y$ has the quotient topology making $f$ an identification map.
\end{pf}

\begin{prop}
  Let $X$ be a topological space and $\sim$ an equivalence relation on $X$ and $\pi\colon X \to X / \sim$ the canonical projection.
  Let $f\colon X \to Y$ be any map such that $x \sim y \implies f(x) = f(y)$ for all $x, y \in X$.
  Then there exists a unique map $\overline{f}\colon X / \sim \to Y$ such that the following diagram commutes:
  \begin{figure}[h]
    \centering
    % https://q.uiver.app/#q=WzAsMyxbMCwwLCJYIl0sWzIsMCwiWSJdLFswLDIsIlggLyBcXHNpbSJdLFswLDEsImYiXSxbMCwyLCJcXHBpIiwyXSxbMiwxLCJcXGV4aXN0cyF+XFxvdmVybGluZXtmfSIsMix7InN0eWxlIjp7ImJvZHkiOnsibmFtZSI6ImRhc2hlZCJ9fX1dXQ==
    \[
      \begin{tikzcd}[every label/.append style={font=\normalsize}]
        X && Y \\
        \\
        {X / \sim}
        \arrow["f", from=1-1, to=1-3]
        \arrow["\pi"', from=1-1, to=3-1]
        \arrow["{\exists!~\overline{f}}"', dashed, from=3-1, to=1-3]
      \end{tikzcd}
    \]
    \addtocounter{figure}{1}
  \end{figure}
\end{prop}
\begin{pf}
  For existence, let $[x] \in X / \sim$ and let $x' \in [x]$.
  Then set $\overline{f}([x]) = f(x')$.
  The assumption on $f$ yields that this function is well defined and we have that $\overline{f}(\pi(x)) = \overline{f}([x]) = f(x') = f(x)$ as desired.
  For uniqueness, note that any function $\overline{f}$ must satisfy $f(x) = \overline{f}(\pi(x)) = \overline{f}([x])$ for all $x \in X$.
  Thus we have uniqueness.

  For continuity of $\overline{f}$, let $U$ be open in $Y$.
  We know that $f$ is continuous and thus $f^{-1}(U) \subseteq X$ is open.
  Then since $f(x) = f(x')$ for all $x \sim x'$ in $X$ we have that $f^{-1}(U) = \bigcup_{x \in f^{-1}(U)} [x]$, a union of equivalence classes.
  Note that we have
  \[
    f^{-1}(U) = \bigcup_{x \in f^{-1}(U)} [x] = \bigcup_{[x] \in \overline{f}^{-1}(U)} [x]
  \]
  which shows that $\pqty{\overline{f}}^{-1}(U)$ is open in $X / \sim$.
\end{pf}

\clearpage

\begin{ex}[Equivalent Descriptions of the Projective Plane]\
  \begin{figure}[h]
    \centering
    \includegraphics[width=\textwidth]{figs/Projective_Plane_Sphere_Disk.png}
    \caption{Representation of the Projective Plane as both $\textbf{S}^2$ and $\textbf{D}^{2}$~\cite{proj_plane_pic}}\label{fig:proj_plane}
    \nocite{proj_plane_pic}
  \end{figure}

  The projective place is defined as the sphere $\textbf{S}^{2}$ with antipodal points identified.
  This means that we give $\textbf{S}^{2}$ the quotient topology with the relation that antipodal points are equivalent.
  We can give a second description of the unit $\textbf{D}^{2}$ with antipodal points on the boundary identified.
  Regard $D^{2}$ as homeomorphic to the upper hemisphere and consider the following diagram:

  \begin{figure}[h]
    \centering
    % https://q.uiver.app/#q=WzAsNCxbMCwwLCJcXHRleHRiZntEfV4yIl0sWzIsMCwiXFx0ZXh0YmZ7U31eMiJdLFswLDIsIlxcdGV4dGJme0R9XjIvIFxcc2ltIl0sWzIsMiwiXFx0ZXh0YmZ7U31eMiAvIFxcc2ltIl0sWzAsMSwiaSJdLFsxLDMsImciXSxbMiwzLCJrIiwyXSxbMCwyLCJmIiwyXV0=
    \[
      \begin{tikzcd}[every label/.append style={font=\normalsize}]
        {\textbf{D}^2} && {\textbf{S}^2} \\
        \\
        {\textbf{D}^2/ \sim} && {\textbf{S}^2 / \sim}
        \arrow["i", from=1-1, to=1-3]
        \arrow["g", from=1-3, to=3-3]
        \arrow["k"', from=3-1, to=3-3]
        \arrow["f"', from=1-1, to=3-1]
      \end{tikzcd}
    \]
    \addtocounter{figure}{1}
  \end{figure}

  The maps $f, g$ are the identifications, $i$ is inclusion, and $k$ is induced as the only function making the diagram commute by the universal property of the quotient.
  If $U \subseteq \textbf{S}^{2} / \sim$ is open, then $g^{-1}(U)$ is open and thus $(g \circ i)^{-1}(U)$ is open.
  But this is equivalent to $(k \circ f)^{-1}(U)$ which means that $k^{-1}(U)$ is open by definition of quotient topology.
  Thus $k$ is continuous.
  We have that $k([x]) = [i(x)]$ which is clearly surjective and injective.
  Moreover, $\textbf{D}^{2} / \sim$ is compact by \Cref{thrm: image_of_compact_is_compact} since $\textbf{D}^{2}$ is compact.
  Also $\textbf{S}^{2} / \sim$ is \quest{easily seen to be Hausdorff (basically take small open disks on surface of $\textbf{S}^{2}$)}.
  Thus by \Cref{thrm: bij_comp_to_Haus_is_homeo}, we have that $k$ is a homeomorphism.
  Thus these two definitions of the projective plane are topologically equivalent.
\end{ex}

A special case of quotient spaces is the idea of ``collapsing'' a subspace:
\begin{defn}[$X / A$]
  If $X$ is a space and $A \subseteq X$, then $X / A$ is the quotient obtained by the equivalence relation whose classes are $A$ and singletons $\set{x}$ for $x \in X \setminus A$.
\end{defn}

\clearpage

\begin{prop}
  If $X$ is regular and $A$ is closed, then $X / A$ is Hausdorff.
  If $X$ is normal and $A$ is closed, then $X / A$ is normal.
\end{prop}
\begin{pf}
  Denote the projection as $\pi\colon X \to X / A$.
  Suppose that $X$ is regular and $A$ is closed.
  Let $x, y \in X / A$ be distinct elements.
  If one of $x$ or $y$ are equal to $A$, say $x$, then since $X$ is regular we can separate $\pi^{-1}(A) = A$ and $\pi^{-1}(y) = \set{y}$ by regularity of $X$.
  If neither of $x$ or $y$ are equal to $A$, we can do something similar with $X$ being Hausdorff.

  The proof for the second fact is similar.
\end{pf}

\begin{ex}[Disk as Quotient of Cylinder]\label{ex: disk_quotient_cylinder}\
  \begin{figure}[h]
    \centering
    \includegraphics[width=\textwidth]{figs/Disk_Quotient_Cylinder.png}
    \caption{$\textbf{D}^2$ as a Quotient of $\textbf{S}^1 \times I$}\label{fig:disk_quotient_cylinder}
  \end{figure}

  Consider the cylinder $\textbf{S}^{n} \times I$.
  Let $f\colon \textbf{S}^{n} \times I \to \textbf{D}^{n + 1}$ by mapping $(x, t) \to tx$.
  This maps $S^{n} \times \set{0}$ to the origin and thus $f$ factors through $\textbf{S}^{n} \times I / \textbf{S}^{n} \times \set{0}$.
  The resulting map $g\colon \textbf{S}^{n} \times I / \textbf{S}^{n} \times \set{0} \to \textbf{D}^{n + 1}$ is clearly one-to-one and onto (see \Cref{fig:disk_quotient_cylinder}).
  Thus it is a homeomorphism by \Cref{thrm: bij_comp_to_Haus_is_homeo}.
\end{ex}

\clearpage

\begin{ex}[Sphere as Quotient of Disk]\
  \begin{figure}[h]
    \centering
    \includegraphics[width=0.33\textwidth]{figs/Sphere_Quotient_Disk.png}
    \caption{$\textbf{S}^2$ as a Quotient of $\textbf{D}^2$~\cite{book:Bredon}}\label{fig:sphere_quotient_disk}
  \end{figure}

  Consider the $n$-dimensional disk $\textbf{D}^{n}$.
  This is homeomorphic to the lower $n$-hemisphere of radius $2$ centered at $1$ on the ``vertical axis.''
  We can map this onto the $n$-sphere $S^{n}$ of radius $1$ centered at the origin by projection towards this vertical axis.
  This maps the boundary of the disk to the north pole of the sphere.
  This function is distance decreasing and thus continuous.
  Consider the quotient space $D^{n} / S^{n - 1}$.
  We can factor the projection of the disk to the sphere through this space.
  By an argument similar to \Cref{ex: disk_quotient_cylinder} one can show that the map $D^{n} / S^{n - 1} \to S^{n}$ is a homeomorphism.
\end{ex}

If we cannot use the argument from the previous two examples, we may be able to use the following criterion.

\begin{defn}[Saturation]
  If $A \subseteq X$ and if $\sim$ is an equivalence relation on $X$, then the \emph{saturation} of $A$ is $\set{x \in X | x \sim a \text{ for some } a \in A}$.
  We call a set \emph{saturated} if it is equal to its own saturation.
\end{defn}

\begin{prop}
  If $A \subseteq X$ and $\sim$ an equivalence relation of $X$ such that every equivalence class intersects $A$ nontrivially, then the induced map $k\colon A / \sim \to X / \sim$ is a homeomorphism if the saturation of every open (resp.\ closed) is open (resp.\ closed) in $X$.
\end{prop}
\begin{pf}
  Let $f\colon A \to A / \sim$ and $g\colon X \to X / \sim$ be the canonical maps and $U$ open in $A / \sim$.
  Note that $g^{-1}(V)$ is saturated for any $V$.
  This is because the saturation of $g^{-1}(V)$ is the set of elements in $x \in X$ such that $x$ equivalent to some element in $g^{-1}(V)$ which by definition of $g$ yields the claim.
  Then we have that $g^{-1}(k(U))$ is the saturation of $f^{-1}(U)$ by commutativity \quest{make diagram}.
  By definition, we have that $g^{-1}(k(U))$ is open if and only if $k(U)$ is open in $X / \sim$.
  Thus $k^{-1}$ is continuous.
  We also have that $k$ is injective by disjointness of equivalence classes.
  By the given assumption, $k$ is surjective.
  Then $k$ is continuous.
  We have that $f\colon A \to A / \sim$ is an identification map.
  Thus $k$ is continuous if and only if $k \circ f$ is continuous.
  However, $k \circ f = g \circ i$ which is clearly continuous.
  Thus $k$ is continuous and overall $k$ is a homeomorphism.
\end{pf}

\clearpage

We can also consider quotients built by ``attachments.''

\begin{defn}[$X \cup_{f} Y$]
  Let $X, Y$ be spaces and $A \subseteq Y$ be closed.
  Let $f\colon A \to X$ be a map.
  We have a quotient space on the disjoint union $X + Y$ by the relation generated $a \sim f(a)$ for all $a \in A$.
  In this relation, we have that $u \sim v$ if one of the following holds:
  \begin{enumerate}
  \item $u = v$;
  \item $u, v \in A$ and $f(u) = f(v)$; or
  \item $u \in A$ and $v = f(u) \in X$.
  \end{enumerate}
  If $X$ is a one-point space then this quotient space is just $Y / A$.
  This quotient space $(X + Y) / \sim$ is denoted by $X \cup_{f} Y$.
\end{defn}

\begin{prop}\label{prop: attachment_mappings}
  The canonical map $X \to X \cup_{f} Y$ is an embedding onto a closed subspace.
  The canonical map $Y \setminus A \to X \cup_{f} Y$ is an embedding onto an open subspace.
\end{prop}
\begin{pf}

  \quest{TODO:\@ Show that every map that is injective, continuous and either open or closed is an embedding}

  Let $q\colon X + Y \to X \cup_{f} Y$ be the quotient map.
  Then $q\mid_{X}\colon X \to X \cup_{f} Y$ is our desired canonical map.
  We first show that $q\mid_{X}$ is a closed map, so let $B \subseteq X$ be closed.
  Then we have that $q\mid_{X}(B) = q(B) \subseteq X \cup_{f} Y$ is closed if and only if $q^{-1}(q(B)) \subseteq X + Y$ is closed, since the quotient map is continuous.
  However, this is true if and only if $q^{-1}(q(B)) \cap X$ and $q^{-1}(q(B)) \cap Y$ are both closed.
  We have that
  \[
    q^{-1}(q(B)) \cap X = B, q^{-1}(q(B)) \cap Y = f^{-1}(B).
  \]
  The first equality is immediate.
  So suppose that $y \in q^{-1}(q(B)) \cap Y$.
  Note that $y \in q^{-1}(q(B))$ if and only for some $x \in q(B)$ we have that $y \sim x$.
  By the equivalence relation, we must have that $y \in f^{-1}(B)$.
  Now suppose that $y \in f^{-1}(B) \subseteq A \subseteq Y$.
  Then $f(y) \in B$ and by the equivalence relation we again have $f(y) \in q^{-1}(q(B))$.
  Since both $B$ and $f^{-1}(B)$ are closed by assumption that $B$ is closed and $f$ is a map, we have that $q\mid_{X}$ is a closed map.

  We now must show that $q\mid_{X}$ is injective.
  However, for any $u, v \in X$, since $A$ is disjoint from $X$ we just have that $q(u) = q(v) \implies q(u) \sim q(v)$ which holds if and only if $y = v$.
  Thus $q\mid_{X}$ is injective and overall $q\mid_{X}$ is an embedding onto a closed subspace since $X$ is closed in $X + Y$.

  Let $U \subseteq Y \setminus A$ be open.
  Then $q\mid_Y(U) = q(U) \subseteq X \cup_f Y$ is open if and only if $q^{-1}(q(U))$ is open in $X \sqcup U$ which is true if and only if $q^{-1}(q(U)) \cap X$ and $q^{-1}(q(U)) \cap Y$ are both open.
  Note that we immediately have that $q^{-1}(q(U)) \cap X$ is empty.
  This is because for $y \in Y \setminus A$, we must have that $[y] \in X \cup_{f} Y = \set{y}$ as the only possibility.
  Using this, we also immediately have that $q^{-1}(q(U)) \cap Y = U \cap Y = U$ which is open by assumption.
  Thus $q$ is an open map.
  We also have injectivity by similar logic and thus $q\mid_{Y}$ is an embedding on an open subspace since $Y \setminus A$ is open in $X + Y$.
\end{pf}

\clearpage

\begin{defn}[Retraction]
  If $A$ is a subspace of a space $X$, then a map $f\colon X \to A$ such that $f(a) = a$ for all $a \in A$ is called a \emph{retraction} and $A$ is a \emph{retract} of $X$.
\end{defn}

\begin{defn}[Mapping Cylinder $M_{f}$]\

  \begin{figure}[h]
    \centering
    \includegraphics[width=0.5\textwidth]{figs/Mapping_Cylinder.png}
    \caption{The Mapping Cylinder~\cite{book:hatcher_AT}}\label{fig:mapping_cylinder}
  \end{figure}

  If $f\colon X \to Y$ is a map then the \emph{mapping cylinder} of $f$ is the space $M_{f} = Y \cup_{f_{0}} X \times I$ where $f_{0}\colon X \times \set{0} \to Y$ is the map sending $(x, 0) \to f(x)$.
\end{defn}

Note that $X \approx X \times \set{1}$ is embedded as a closed subset of $M_{f}$.
We abuse notation and regard this as an inclusion.
Also note that there is a factorization of $f$ by $X \subseteq M_{f} \xrightarrow{r} Y$ where $r$ is the retraction of $M_{f}$ onto $Y$ induced by the projection $X \times I \to X \times \set{0}$.

\begin{defn}[Mapping Cone $C_{f}$]
  If $f\colon X \to Y$ is a map, then the \emph{mapping cone} of $f$ is the space $C_{f} = M_{f} / (X \times \set{1})$.
\end{defn}

We are often interested to know when a function taking a mapping cylinder to another space is continuous.
The following is an easy criterion for this.

\begin{prop}\label{prop: mapping_continuity_criterion}
  A function $M_{f} \to Z$ is continuous if and only if the induced functions $X \times I \to Z$ and $Y \to Z$ are both continuous.
\end{prop}
\begin{pf}
  \quest{TODO:\@ Forward is immediate by \Cref{prop: attachment_mappings}? Reverse by universal property?}
\end{pf}

\begin{prop}\label{prop: prod_of_id_and_local_compact}
  If $f\colon X \to Y$ is an identification map and $K$ is a locally connected Hausdorff space, then $f \times 1\colon X \times K \to Y \times K$ is an identification map.
\end{prop}
\begin{pf}
  let $g\colon Y \times K \to W$ and let $h = g \circ (f \times 1)$.
  Then by \Cref{prop: id_map_iff_all_g_cont} we just have to show that $h$ continuous implies $g$ continuous.
  Let $U \subseteq W$ be open and suppose that $g(y_{0}, k_{0}) \in U$.
  Let $f(x_{0}) = y_{0}$.
  Then $h(x_{0}, k_{0}) = g(y_{0}, k_{0}) \in U$.
  Therefore there is a compact neighborhood $N$ of $k_{0}$ such that $h(\set{x_{0}} \times N) \subseteq U$.
  Let $A = \set{y \in Y | g(\set{y} \times N) \subseteq U}$.
  Then $y_{0} \in A$ and we just need to show that $A$ is open \quest{why}.
  Since $f^{-1}$ is continuous we just need to show that $f^{-1}(A)$ is open.
  We have that
  \[
    f^{-1}(A) = \set{x \in X | h(\set{x} \times N) = g(\set{f(x)} \times N) \subseteq U}
  \]
  and thus $X \setminus f^{-1}(A) = \pi_{X}\pqty{h^{-1}(W \setminus U) \cap (X \times N)}$ is closed by \Cref{prop: compact_proj_closed}.
\end{pf}

\clearpage

\subsection*{Exercises}
\begin{exercise}[\tcite{book:Bredon} 1.13.1]
  If $f\colon X \to A$ and $g\colon Y \to B$ are open identification maps, then $f \times g\colon X \times Y \to A \times B$ is also an open identification map.
\end{exercise}
\begin{pf}
  It is immediate that $f\ times g$ is continuous and surjective.
  Also since $f$ and $g$ are each open, then $f \times g$ must be open since the products of open sets in $X$ and $Y$ forms the basis of $A \times B$ by definition of product topology.
  Let $V \subseteq Y$.
  If $(f \times g)^{-1}(V)$ is open we have that $(f \times g)\pqty{(f \times g)^{-1}(V)} = V$ is open since $f \times g$ is open.
  Now suppose that $V$ is open.
  Then immediately we have that $(f \times g)^{-1}(V)$ is open.
  Thus $f \times g$ is an identification map.
\end{pf}

\clearpage

\chapter{Homotopy}

\begin{defn}[Homotopy]
  If $X$ and $Y$ are spaces, then a \emph{homotopy} of maps from $X$ to $Y$ is a map $F\colon X \times I \to Y$ where $I = [0, 1]$.
  Two maps $f_{0}, f_{1}\colon X \to Y$ are \emph{homotopic} if there exists a homotopy $F\colon X \times I \to Y$ such that $F(x, 0) = f_{0}(x)$ and $F(x, 1) = f_{1}(x)$ for all $x \in X$.
\end{defn}

The relation ``$f$ is homotopic to $g$'' is an equivalence relation on the set of all maps $X \to Y$ as we will see later.
This is denoted by $f \simeq g$.

\begin{prop}
  If we have maps $f, g\colon X \to Y$ and maps $h\colon X' \to X$ and $k\colon Y \to Y'$ then if $f \simeq g$ we must have that $f \circ h \simeq g \circ h$ and $k \circ f \simeq k \circ g$. $\qed$
\end{prop}

\begin{defn}[Homotopy Equivalence, Homotopy Inverse, Homotopy Type]
  A map $f\colon X \to Y$ is said to be a \emph{homotopy equivalence} with \emph{homotopy inverse} $g$ if there exists a map $g\colon Y \to X$ such that $g \circ f \simeq 1_{X}$ and $f \circ g \simeq 1_{Y}$.
  This relationship is denoted by $X \simeq Y$.
  We also say that that $X$ and $Y$ have the same \emph{homotopy type}.
\end{defn}

This is an equivalence relation between spaces.
If $h\colon Y \to Z$ is another homotopy equivalence with homotopy inverse $k\colon Z \to Y$ then we have
\begin{align*}
  (g \circ k) \circ (h \circ f) &= g \circ (k \circ h) \circ f = g \circ 1_{Y} \circ f = g \circ f = 1_{X}; \text{ and } \\
  (h \circ f) \circ (g \circ k) &= h \circ (f \circ g) \circ k = h \circ 1_{Y} \circ k = g \circ k = 1_{Z}.
\end{align*}

\begin{defn}[Contractible]
  A space is \emph{contractible} if it is homotopy equivalent to the one-point space.
\end{defn}

\clearpage

\begin{prop}\label{prop: contractible_iff_single_point}
  A space $X$ is contractible if and only if the identity map $1_{X}$ is homotopic to a map $r\colon X \to X$ whose image is a single point.
\end{prop}
\begin{pf}
  Suppose that $X$ is contractible, and so homotopy equivalent to a one point space $\set{x_{0}}$.
  This means there exists maps $f\colon X \to \set{x_{0}}$ and $g\colon \set{x_{0}} \to X$ such that $g \circ f \simeq 1_{X}$ and $f \circ g \simeq 1_{x_{0}}$.
  But saying that $g \circ f \simeq 1_{X}$ is exactly what we need since $\im(g \circ f)$ is a single point.

  Now suppose that $1_{X}$ is homotopic to a map $r\colon X \to X$ where $\im(r) = \set{x_{0}}$.
  We have an inclusion map $i\colon \set{x_{0}} \to X$ and the only possible map $r\colon X \to Y$.
  Then we have that $r \circ i = 1_{x_{0}}$ and equivalent maps are clearly homotopic.
  We also have that $i \circ r \simeq 1_{X}$ by assumption since $i \circ r$ is a map $X \to X$ with image a single point.
\end{pf}

\begin{ex}[Constant Map $\simeq 1_{X}$]
  Let $X = \R^{n}$.
  We have a homotopy $F\colon X \times I \to X$ given by $F(x, t) = tx$.
  This is a homotopy between $F(x, 1) = 1_{X}$ and $F(x, 0)$ which is the constant map sending everything to $0$.
  By \Cref{prop: contractible_iff_single_point} this means that $\R^{n}$ is contractible.
  Note that for $t > 0$, $F(x, t)$ is surjective.
  However, $F(x, 0)$ is not surjective.
\end{ex}

\begin{ex}[Unit Sphere $\simeq$ Punctured Euclidean Space]
  Consider the unit sphere $S^{n - 1}$ in $\R^{n}$ and the punctured euclidean space $\R^{n} \setminus \set{0}$.
  Let $i\colon S^{n - 1} \into \R^{n} \setminus \set{0}$ be the inclusion map and let $r\colon \R^{n} \setminus \set{0} \to S^{n - 1}$ be the projection $r(x) = \frac{x}{\norm{x}}$.
  Then we have that
  \[
    r \circ i = 1_{S^{n - 1}} \text{ and } i \circ r \simeq 1_{\R^{n} \setminus \set{0}}.
  \]
  The latter homotopy is given by $F\colon \pqty{\R^{n} \setminus \set{0}} \times I \to \R^{n} \setminus \set{0}$ such that $F(x, t) = tx + (1 - t)\frac{x}{\norm{x}}$.
  Thus $S^{n - 1} \simeq \R^{n} \setminus \set{0}$.
\end{ex}

These two examples suggest the following definition:

\begin{defn}[(Strong) Deformation Retract]
  A subspace $A \subseteq X$ is called a \emph{strong deformation retract} of $X$ if there is a homotopy $F\colon X \times I \to X$, called a \emph{deformation}, such that
  \begin{enumerate}
  \item $F(x, 0) = x$
  \item $F(x, 1) \in A$
  \item $F(a, t) = a$ for all $a \in A$ and $t \in I$.
  \end{enumerate}
  If we only require that $(3)$ holds for $t = 1$ then this is just known as a \emph{deformation retract}.
  A deformation retract $A$ of a space $X$ is homotopically equivalent to $X$.
\end{defn}

\begin{ex}[Mapping Cylinder is Strong Deformation Retraction]
  If $f\colon X \to Y$ is a map, then the canonical map $r\colon M_{f} \to Y$ is a strong deformation retraction.
  This will be seen later.
  Hence $M_{f} \simeq Y$.
  Thus, we can use the mapping cylinder to replace ``up to homotopy'' the map $f$ by the inclusion $X \into M_{f}$.
\end{ex}

\clearpage

\begin{defn}[Homotopy Relative to Subspace, Constant Homotopy]
  If $A \subseteq X$, then a homotopy $F\colon X \times I \to Y$ is \emph{relative to $A$}, denoted as $\rel\ A$, if $F(a, t)$ is independent of $t$ for all $a \in A$.
  A homotopy this is $\rel\ X$ is said to be a \emph{constant homotopy}.
\end{defn}

\begin{defn}[Concatenation]
  If $F\colon X \times I \to Y$ and $G\colon X \times I \to Y$ are two homotopies such that $F(x, 1) = G(x, 0)$ for all $x \in X$, then we define a homotopy $F * G\colon X \times I \to Y$ called the \emph{concatenation of $F$ and $G$} by
  \[
    (F * G)(x, t) =
    \begin{cases}
      F(x, 2t)      & \text{ if } t \leq \frac{1}{2}, \\
      G(x, 2t - 1)  & \text{ if } t \geq \frac{1}{2}.
    \end{cases}
  \]
  Note that we do not have to combine these homotopies at $t = \frac{1}{2}$, this generalizes to any $t$.
  \begin{figure}[ht]
    \centering
    \includegraphics[width=0.5\textwidth]{figs/concatenation.png}
    \caption{Concatenation of Homotopies~\cite{book:Bredon}}\label{fig:concatenation}
  \end{figure}
\end{defn}

\begin{lem}[Reparametrization Lemma]\label{lem: reparametrization}
  Note that $\partial I = \set{0, 1}$.
  Let $\phi_{1}$ and $\phi_{2}$ be two maps $(I, \partial I) \to (I, \partial I)$ which are equivalent on $\partial I$.
  (Note the case where one of these maps is the identity.)
  Let $F\colon X \times I \to Y$ be a homotopy and $G_{i}(x, t) = F(x, \phi_{i}(t))$ for $i \in \set{1, 2}$.
  Then $G_{1} \simeq G_{2} \rel\ X \times \partial I$.
\end{lem}
\begin{pf}
  Define $H\colon X \times I \times I \to Y$ by $H(x, t, 0) = F(x, s \cdot \phi_{2}(t) + (1 - s) \cdot \phi_{1}(t))$.
  Then we have that
  \begin{align*}
    H(x, t, 0) &= F(x, \phi_{1}(t)) = G_{1}(x, t), \\
    H(x, t, 1) &= F(x, \phi_{2}(t)) = G_{2}(x, t), \\
    H(x, 0, s) &= F(x, \phi_{1}(0)) = G_{1}(x, 0), \\
    H(x, 1, s) &= F(x, \phi_{2}(1)) = G_{2}(x, 1).
  \end{align*}
\end{pf}

\clearpage

\Cref{lem: reparametrization} has some easy and immediate consequences.
Let $C$ denote a constant topology.
This will depend on context.
For example, $F * C$ is concatenation with the constant homotopy where $C(x, t) = F(x, 1)$ but $C * F$ is concatenation with the constant homotopy where $C(x, t) = F(x, 0)$.

\begin{prop}
  We have that $F * C \simeq F \rel\ X \times \partial I$ and $C * F \simeq F \rel\ X \times \partial I$.
\end{prop}
\begin{pf}
  Let $\phi_{1}(t) = 2t$ for $t \leq \frac{1}{2}$ and $\phi_{1}(t) = 1$ for $t \geq \frac{1}{2}$ and letting $\phi_{2}(t) = t$.
  Then the first case immediately follows by applying \Cref{lem: reparametrization}.
  Similarly, the second case follows from letting $\phi_{1}(t) = 0$ for $t \leq \frac{1}{2}$ and $\phi_{1}(t) = 2t - 1$ for $t \geq \frac{1}{2}$ and $\phi_{2}(t) = t$.
\end{pf}

\begin{defn}[Inverse of a Homotopy]
  If $F\colon X \times I \to Y$ is a homotopy, let $F^{-1}\colon X \times I \to Y$ be the homotopy running the reverse direction defined by $F^{-1}(x, t) = F(x, 1 - t)$.
\end{defn}

\begin{prop}
  For a homotopy $F$ we have that $F * F^{-1} \simeq C \rel\ X \times \partial I$ where $C(x, t) = F(x, 0)$ for all $x \in X$ and $t \in I$.
\end{prop}
\begin{pf}
  This follows from \Cref{lem: reparametrization} by letting $\phi_{1}(t) = 2t$ for $t \leq \frac{1}{2}$, $\phi_{1}(t) = 2 - 2t$ for $t \geq \frac{1}{2}$ and $\phi_{2}(t) = 0$ for all $t$.
\end{pf}

\begin{prop}
  For any homotopies where concatenations $F * G$ and $G * H$ are defined, we have that $(F * G) * H \simeq F * (G * H) \rel\ X \times \partial I$.
\end{prop}
\begin{pf}
  \quest{TODO:\@ Easy application of \Cref{lem: reparametrization}??}.
\end{pf}
If follows that homotopies between maps of pairs $(X, A) \to (Y, B)$ are an equivalence relation.
The set of homotopy classes of these maps is denoted $[X, A ; Y, B]$ or simply $[X, Y]$ if $A = \emptyset$.

\clearpage

We now show that the homotopy type of a mapping cylinder or cone only depends on the homotopy class of the map.

\begin{thrm}\label{thrm: homotopy_on_map_cyl_class_of_map}
  If $f_{0} \simeq f_{1}\colon X \to Y$ then $M_{f_{0}} \simeq M_{f_{1}} \rel\ X + Y$ and $C_{f_{0}} \simeq C_{f_{1}} \rel Y + \emph{vertex}$.
\end{thrm}
\begin{pf}
  The proof for the cone follows from the mapping cylinder.
  Let $F\colon X \times I \to Y$ be the homotopy between $f_{0}$ and$ f_{1}$.
  Let $h\colon M_{f_{0}} \to M_{f_{1}}$ be the map such that $h(y) = y$ for $y \in Y$ and we have that
  \[
    h(x, t) = \begin{cases}
                F(x, 2t) & t \leq \frac{1}{2}, \\
                F(x, 2t - 1) & t \geq \frac{1}{2}.
              \end{cases}
  \]
  Note that $h\pqty{x, \frac{1}{2}} = F(x, 1) = f_{1}(x) = (x, 0)$ \quest{??}.
  To prove continuity, we need that the compositions $Y \to M_{f_{1}}$ and $X \times I \to M_{f_{1}}$ are continuous by \Cref{prop: mapping_continuity_criterion}.
  However, this is \quest{immediate}.

  Let $k\colon M_{f_{1}} \to M_{f_{0}}$ be defined analogously.
  Consider the composition $k \circ h\colon M_{f_{0}} \to M_{f_{0}}$.
  This is clearly the identity on $Y$.
  On the cylinder portion we have that this composition is equivalent to $F * \pqty{F^{-1} * E}$ where $E\colon X \times I \to M_{f_{0}}$ is induced by the identity on $X \times I \to X \times I$ (see \Cref{fig:mapping_cylinder_deformation}).
  This is homotopic to the identity $\rel\ X \times \set{1} + Y$ \quest{??}.
  Similar statements hold for $h \circ k$.
  For continuity of the homotopy $M_{f_{0}} \times I \to M_{f_{0}}$ it suffices to see that $M_{f_{0}} \times I \approx M_{f_{0} \times I}$ because then we just need continuity of the composition
  \[
    (X \times I + Y) \times I \to M_{f_{0}} \times I \to M_{f_{0}}
  \]
  which is \quest{trivial}.
  This is because on $Y \times I$ we have the constant homotopy and on $X \times I \times I$ we have that $F * \pqty{F^{-1} * E} \simeq E \rel\ X \times \partial I$.
  That is, it suffices to show that $M_{f_{0}} \times I$ has the identification topology from the map $f_{0} \times I$.
  This is a consequence of \Cref{prop: prod_of_id_and_local_compact}.

  \quest{This is all gibberish}

  \begin{figure}[ht]
    \centering
    \includegraphics[width=0.5\textwidth]{figs/Mapping_Cylinder_Deformation.png}
    \caption{Deformation of a Mapping Cylinder~\cite{book:Bredon}}\label{fig:mapping_cylinder_deformation}
  \end{figure}
\end{pf}

\clearpage

We conclude this section by studdying the effect of mapping cones when we change the target space via a homotopy equivalence.
\begin{thrm}
  Let $f\colon X \to Y$.
  If $\phi\colon Y \to Y'$ is a map, then we have an induced map $F\colon M_{f} \to M_{\phi \circ f}$ induced from both $\phi$ on $Y$ and the identity on $X \times I$.
  If $\phi$ is a homotopy equivalence, then so if $F\colon(M_{f}, X) \to (M_{\phi \circ f}, X)$ and hence so is $F\colon C_{f} \to C_{\phi \circ f}$.
\end{thrm}
\begin{pf}
  Let $\psi\colon Y' \to Y$ be the homotopy inverse of $\phi$.
  Let $G\colon M_{\phi \circ f} \to M_{\psi \circ \phi \circ f}$ be the map induced by $\psi$ on $Y'$ and the identity on $X \times I$.
  We have that $G \circ F\colon M_{f} \to M_{\psi \circ \phi \circ f}$ is induced by $\psi \circ \phi\colon Y \to Y$ and the identity on $X \times I$.
  Let $H\colon Y \times I \to I$ be a homotopy from $1$ to $\psi \circ \phi$; so we have that $H(y, 0) = y$ and $H(y, 1) = (\psi \circ \phi)(y)$.
  By the proof of \Cref{thrm: homotopy_on_map_cyl_class_of_map} there is a homotopy equivalence $h\colon M_{f} \to M_{\psi \circ \phi \circ f} \rel\ X$ which is given by $h(y) = y$ on $Y$ and
  \[
    h(x, t) =
    \begin{cases}
      H(f(x), 2t) & t \leq \frac{1}{2}, \\
      (x, 2t - 1) & t \geq \frac{1}{2}.
    \end{cases}
  \]
  We have that $h \simeq G \circ F \rel\ X$.
  We have that the homotopy $H$ can be extended to $M_{f} \times I \to M_{\psi \circ \phi \circ f}$ by setting
  \[
     H((x, s), t) =
     \begin{cases}
       H(f(x), 2s + t) & 2s + t \leq \frac{1}{2}, \\
       (x, 2s + t - 1) & 2s + t \geq \frac{1}{2}.
     \end{cases}
   \]
   Note that we have that $H((x, s), 0) = h(x, s)$, $H(y, 0) = y$, and $H((x, s), 1) = (\psi \circ \phi)(y)$.
   Thus $H(\cdot, 0) = h$ and $H(\cdot, 1) = G \circ F$.

   This means that $G \circ F$ is a homotopy equivalence, since $h$ is one.
   A similar proof yields that $F' \circ G$ is a homotopy equivalence where $F'\colon M_{\psi \circ \phi \circ f} \to  M_{\phi \circ \psi \circ \phi \circ f}$ is defined in a manner similar to $F$.

   If $k$ is a homotopy inverse of $G \circ F$ then $G\ circ F \circ k \simeq 1$.
   If $k'$ is a homotopy inverse of $F' \circ G$ then $k' \circ F' \circ G \simeq 1$.
   Thus $G$ has a homotopy right inverse $R = F \circ k$ and a homotopy left inverse $L = k' \circ F'$.
   That is $L \circ G \simeq 1 \simeq G \circ R$.
   Then
   \[
     R = 1 \circ R \simeq (L \circ G) \circ R = L \circ (G \circ R) \simeq L \circ 1 = L,
   \]
   and so $R \simeq L$ is a homotopy inverse of $G$.
   Therefore, $G$ is a homotopy equivalence.
   Since $G$ and $G \circ F$ are homotopy equivalences, so is $F$.
   To make this explicit, if $l$ is a homotopy inverse of $G$ then we have
   \[
     F \circ k \circ G \simeq (l \circ G) \circ F \circ k \circ G = l \circ (G \circ F \circ k) \circ G \simeq l \circ G \simeq 1
   \]
   and so $k \circ G$ is a homotopy inverse of $F$, since $k \circ G \circ F \simeq 1$ by the definition of $k$.
\end{pf}

\clearpage

\section*{Exercises}

\begin{exercise}[\tcite{book:Bredon} 1.14.1]
  Let $\pqty{\textbf{S}^{2}, x_{0}}$ and $\pqty{\textbf{S}^{1}, y_{0}}$ be ``pointed spaces'' meaning spaces with a distinguished ``base'' point.
  Then their ``one-point union'' $\textbf{S}^{2} \lor \textbf{S}^{1}$ is the quotient space of $\textbf{S}^{2} + \textbf{S}^{1}$ with the equivalence relation identifying $x_{0}$ with $y_{0}$.
  Show that this space is homotopically equivalent to the space $\textbf{S}^{2} \cup A$ where $A$ is the line segment joining the north and south poles.
\end{exercise}

\clearpage

\chapter{Topological Groups}

\begin{defn}[Topological Group]
  A \emph{topological group} is a Hausdorff topological space $G$ such that group multiplication $G \times G \to G$ and inversion $G \to G$ are both continuous.
\end{defn}

\begin{defn}[Subgroup]
  A \emph{subgroup} $H$ of a topological group $G$ is a subspace which is also a subgroup in the algebraic sense.
\end{defn}

\begin{defn}[Homomorphism]
  If $G$ and $G'$ are both topological groups, then a \emph{homomorphism} $f\colon G \to G'$ is a group homomorphism which is also continuous.
\end{defn}

\begin{defn}[Translation]
  For a topological group $G$ and $g \in G$ define \emph{left translation by $g$} and \emph{right translation by $g$} respectively by the maps

  \noindent\begin{minipage}{.5\linewidth}
    \begin{align*}
      L_{g}\colon G &\to G \\
      h &\mapsto gh
    \end{align*}
  \end{minipage}%
  \begin{minipage}{.5\linewidth}
    \begin{align*}
      R_{g}\colon G &\to G \\
      h &\mapsto hg^{-1}
    \end{align*}
  \end{minipage}
\end{defn}

\begin{prop}
  If a topological group $G$ we have that $L_{g} \circ L_{h} = L_{gh}$ and $R_{g} \circ R_{h} = R_{hg}$.
  Moreover, both $L_{g}$ and $R_{g}$ are homeomorphisms as are congugation by $g$ mapping $h \mapsto ghg^{-1}$ and inversion $h \mapsto h^{-1}$.
  $\hfill\qed$
\end{prop}

\begin{defn}[Symmetric]
  A subset $A$ of a topological group is \emph{symmetric} if $A = A^{-1}$.
\end{defn}

\begin{prop}
  In a topological group $G$ with identity $e$, the symmetric neighborhoods of $e$ form a neighborhood basis at $e$.
\end{prop}
\begin{pf}
  Let $U$ be a neighborhood of $e^{-1}$.
  We must have that $U^{-1}$ is also a neighborhood of $e$ since inversion is a homeomorphism.
  Thus $U \cap U^{-1}$ is a neighborhood of $e$ and we see that $U \cap U^{-1}$ is symmetric.
\end{pf}

\clearpage

\begin{prop}\label{prop: symmetric_VgV-1}
  If $G$ is a topological group, $g \in G$, and $U$ is a neighborhood of $g$, then there is a symmetric neighborhood $V$ of $e$ such that $VgV^{-1} \subseteq U$.
\end{prop}
\begin{pf}
  Consider $f_{g}\colon G \times G \to G$ mapping $(x, y) \mapsto xgy^{-1}$.
  Since multiplication is continuous, $f_{g}$ is continuous.
  Thus, there exists some open set $W$ of $G \times G$ containing $(e, e)$ such that $f(W) \subseteq U$.
  By the product topology, $W = V_1 \times V_{2}$ for some open $V_{1}, V_{2} \subseteq G$ where $e \in V_{1}, V_{2}$.
  Let $V = V_{1} \cap V_{1}^{-1} \cap V_{2} \cap V_{2}^{-1}$.
  We have that $V$ is a neighborhood of $e$, $V$ is symmetric, and $VgV^{-1} \subseteq U$ because $VgV^{-1} = f\pqty{V, V} \subseteq f(W) \subseteq U$.
\end{pf}

\begin{prop}
  If $G$ is a topological group with identity $e$, $U$ any neighborhood of $e$, and $n$ any positive integer, then there exists a symmetric neighborhood $V$ of $e$ such that $V^{n} \subseteq U$.
\end{prop}
\begin{pf}
  \quest{Induction, can do later.}
\end{pf}

\begin{prop}
  If $H$ is any subgroup of a topological group then $\overline{H}$ is also a subgroup of $G$.
  If $H$ is a normal subgroup then so is $\overline{H}$.
\end{prop}
\begin{pf}
  Recall by \Cref{prop: continuous_iff_im_closure_subset_closure_im} that a function is continuous if and only if the image of the closure is a subset of the closure of the image.
  Continuity of inversion and multiplication implies that $\overline{H}^{-1} \subseteq \overline{H}$ and that $\overline{H} \overline{H} \subseteq \overline{H}$.
  Thus $\overline{H}$ is a subgroup.
  Suppose now that $\overline{H}$ is normal.
  Let $g \in G$.
  Then since conjugation is continuous we again have that $g\overline{H}g^{-1} \subseteq \overline{gHg^{-1}} = \overline{H}$.
  We get the reverse inclusion by considering the same thing with $g^{-1}$.
  Thus $\overline{H}$ is also normal.
\end{pf}

\begin{prop}\label{prop: quotient_hausdorff}
  If $G$ is a topological group and $H$ is a closed subgroup then $G / H$, with the topology induced by the canonical map $\pi\colon G \to G / H$ of left cosets, is a Hausdorff space.
  Moreover, $\pi$ is open and continuous.
\end{prop}
\begin{pf}
  If $U \subseteq G$ is open then we have that
  \[
    \pi^{-1}(\pi(U)) = UH = \bigcup_{h \in H} Uh
  \]
  is the union of open sets in $G / H$.
  Thus $\pi(U)$ is open.

  Consider two points $g_{1}H, g_{2}H$ in $G / H$.
  Since $g_{1}H \neq g_{2}H$ we have that $g_{1}^{-1}g_{2} \notin H$.
  Since $G \setminus H$ is open and contains $g_{1}^{-1} g_{2}$, by \Cref{prop: symmetric_VgV-1} there is a symmetric open neighborhood $V$ of $e$ such that $V g_{1}^{-1} g_{2} V^{-1}$ is disjoint from $H$.
  Since $V = V^{-1}$ we have that
  \begin{align*}
    \pqty{V g_{1}^{-1} g_{2} V^{-1}} \cap H = \emptyset &\implies g_{1}^{-1} g_{2} V \cap VH = \emptyset \\
                                                        &\implies g_{2} V \cap g_{1} VH = \emptyset \\
                                                        &\implies g_{2} V H \cap g_{1} V H = \emptyset.
  \end{align*}
  Thus $\pi\pqty{g_{i}} \in g_{i} VH$ which are disjoint open sets in $G / H$ and thus $H$ is Hausdorff.
\end{pf}

\begin{prop}
  If $H$ is a closed normal subgroup of $G$ then $G / H$ equipped with the quotient topology is a topological group.
\end{prop}
\begin{pf}
  $G / H$ is Hausdorff by \Cref{prop: quotient_hausdorff}.
  We now show that the group operations on $G / H$ are continuous.
  Consider the following diagram, where the horizontal arrows are group multiplication.
  \quest{TODO:\@ diagram}
  A corollary of the fact that $\pi$ is open is that $\pi \times \pi$ is an identification map.
  Taking an open set in $G / H$, we want to show that it's inverse image in $G / H \times G / H$ is open.
  However, this is equivalent to showing that it's inverse image in $G \times G$ is open.
  But this is open since the maps along the right and top sides of the diagram are continuous.
  Thus multiplication in $G / H$ is continuous.
  A similar argument with a similar diagram yields that inversion in $G / H$ is open.
\end{pf}

The most important topological groups are Lie groups which have a differentiable structure.

\begin{ex}[Classical Lie Groups]
  The set of $n \times n$ matrices $M_{n}$ with real entries is a euclidean space of dimension $n^{2}$.
  We can view a matrix as an element of $\R^{n^{2}}$ and then use any norm since they all define the same open sets.
  The determinant function $\det\colon M_{n} \to \R$ is continuous.
  This is because it is a polynomial in the matrix coefficients.
  Thus the inverse of $\set{0}$ is closed.
  It's complement, the set of nonsingular matrices, is thus open.
  This is a multiplicative group known as the general linear group $\GL(n, \R)$.
  Matrix multiplication is also a polynomial in the coefficients and thus is continuous.
  Matrix inversion is a rational function by Cramer's rule, and thus is continuous.
  Thus $\GL(n, \R)$ is a topological group.
  Similarly, $\GL(n, \C)$ is a topological group.

  The special linear group $\SL(n, \R)$ is the subgroup of $\GL(n, \R)$ consisting of matrices with determinant $1$.
  We also can consider $\SL(n, \C)$.

  The general linear group $\GL(n, \mathbb{H})$ over the quaternions is also a topological space.
  We note that there is no determinant function, so the argument is not as clean.

  The set $\text{O}(n)$ is the set of orthogonal real matrices and is a subgroup of $\GL(n, \R)$ and also is a closed subset since it is defined by \quest{continuous relations} ($A A^{T} = I$).
  Since the coefficients of an orthogonal matrix are bounded by $1$, we have that $\text{O}(n)$ is a bounded closed subset of euclidean $n^{2}-$space and thus is compact since closed and bounded euclidean spaces are compact.
  Similarly, the set $\text{U}(n)$ of unitary matrices ($A A^{*} = I$) forms a compact subgroup of $\GL(n, \C)$.

  The quaternionic analouge of the orthogonal and unitary groups is the sympletic group $\Sp(n)$.
  Its elements are quaternionic matricies $A$ such that $A A^{*} = I$ where $A^{*}$ is the quaternionic conjugate transpose of $A$.
  Quaternionic conjugation means reversal of the three imaginary components.
  This forms a compact subgroup of $\GL(n, \mathbb{H})$.
\end{ex}

Note that the map $\GL(n, \R) \times \R^{n} \to \R^{n}$ is given by polynomials in the coefficients of the matrix and the vector, and thus is continuous.

\clearpage

An orthogonal matrix $A \in \text{O}(n)$, as a transformation of euclidean $n$-space, preserves lengths of vectors.
This makes it a map of the sphere $\textbf{S}^{n - 1}$ to itself.
Regard $\text{O}(n - 1)$ as the subgroup of $\text{O}(n)$ fixing the last coordinate.
Consider the $n$-dimensional point $(0, \ldots, 0, 1)$, which is fixed by $\text{O}(n - 1)$.
Define a map $f$ taking a matrix in $O\text{O}(n)$ to where it moves the point $(0, \ldots, 0, 1)$ on the sphere $\textbf{S}^{n - 1}$:
\begin{align*}
  f\colon \text{O}(n) &\to \textbf{S}^{n - 1} \\
             A &\mapsto A (0, \ldots, 0, 1)^{T}.
\end{align*}
If $B \in \text{O}(n - 1)$ then $f(AB) = f(A)$ meaning that $f$ factors through the coset space $\text{O}(n) / \text{O}(n - 1)$.
It can be shown that the induced function $\text{O}(n) / \text{O}(n - 1) \to \textbf{S}^{n - 1}$ is injective, surjective, and continuous.
Since this is a injective mapping of a compact space onto a Hausdorff space, it is a homeomorphism by \Cref{thrm: bij_comp_to_Haus_is_homeo}.
We can abstract this.

\begin{defn}[Group Action, Orbit, Isotropy / Stability, Transitive, Effective / Faithful]
  If $G$ is a topological group and $X$ a space, then an \emph{action} $G \acts X$ is a map $G \times X \to X$ where the image of $(g, x)$ is $g \cdot x$.
  The map is such that
  \begin{enumerate}
  \item $(gh) \cdot x = g \cdot (h \cdot x)$; and
  \item $e \cdot x = x$.
  \end{enumerate}
  For a point $x \in X$ we have that $G \cdot x \defeq \set{g \cdot x | g \in G}$ is the \emph{orbit} of $x$.
  The subgroup $G_{x} \defeq \set{g \in G | g \cdot x = x}$ is the \emph{stabilizer} or \emph{isotropy} group of $x$.
  An action is \emph{transitive} if it only has one orbit, the whole space $X$.
  The action is \emph{effective} if $g \cdot x = x$ for all $x$ implies that $g = e$.
\end{defn}

\begin{prop}
  If $G$ is a compact topological group acting on a Hausdorff space $X$ and $G_{x}$ is the isotropy group at $x$, then the map $\phi\colon G / G_{x} \to G \cdot x$ given by $g G_{x} \mapsto g \cdot x$ is a homeomorphism.
\end{prop}
\begin{pf}
  We immediately have that $\phi$ is surjective.
  If $g_{1} \cdot x = g_{2} \cdot x$, then $g_{1}^{-1} g_{2} \in G_{x}$.
  Thus $g_{1} G_{x} = g_{2} G_{x}$, meaning that $\phi$ is injective.
  It is continuous by the definition of the quotient topology.
  The result then follows by \Cref{thrm: bij_comp_to_Haus_is_homeo}.
\end{pf}

We also have that $\text{U}(n)$ acts on $\textbf{S}^{2n - 1}$ \quest{how?} and it is transitive because we can find a unitary matrix moving any vector from length $1$ to any other.
The isotropy group at $(0, \ldots, 0, 1)$ is $\text{U}(n - 1)$ and we have that $\text{U}(n) / \text{U}(n - 1) \approx \textbf{S}^{n - 1}$.
Similarly, $\Sp(n) / \Sp(n - 1) \approx \textbf{S}^{4n - 1}$.

More generally, let $V_{n, k}$ denote the ``Stiefel Manifold'' of $k$-frames in $n$-space.
A $k$-frame is an orthonormal set of $k$ vectors in $n$ space.
We have that $\text{O}(n)$ will act transitively on $V_{n, k}$ with isotropy group $\text{O}(n - k)$ and so $\text{O}(n) / \text{O}(n - k) \approx V_{n, k}$.
Analogous observations for unitary and symplectic cases can be made.

We can also form special groups of other matrices:
\begin{align*}
  \SO(n) &= \set{A \in \text{O}(n) | \det(A) = 1} = \text{the special orthogonal group; and} \\
  \SU(n) &= \set{A \in \text{U}(n) | \det(A) = 1} = \text{the special unitary group.}
\end{align*}
There is no analogue in the symplectic case.

With some restriction, these groups can also act transitively on spheres and we get
\begin{align*}
  \SO(n) / \SO(n - 1) &\approx \textbf{S}^{n - 1} & \text{for } n \geq 2; \text{ and } \\
  \SU(n) / \SU(n - 1) &\approx \textbf{S}^{2n - 1} & \text{for } n \geq 2.
\end{align*}
Similar results can be obtained for Stiefel manifolds.

\clearpage

\subsection*{Exercises}

\begin{exercise}[\tcite{book:Bredon} 15.5]
  If $G$ is a topological group and $H$ a closed subgroup, show that if $H$ and $G / H$ are both connected then so is $G$.
\end{exercise}
\begin{pf}
  Suppose that $G = A \sqcup B$ where $A, B$ are nonempty open subsets of $G$.
  Since $H$ is connected, its translations $L_{g}(H)$ are also connected by \Cref{prop: image_of_connected_is_connected}.
  Since $A$ and $B$ are disjoint, each left coset must be completely contained in one or the other.
  Thus, $A$ and $B$ are disjoint unions of left cosets of $H$.
  Let $\pi\colon G \to G/H$ be the quotient map.
  Then $\pi(A)$ and $\pi(B)$ are nonempty and disjoint.
  But then since $\pi$ is open by \cref{prop: quotient_hausdorff}, $\pi(A)$ and $\pi(B)$ is a disconnection of $G / H$.
  This is a contradiction.
  Thus $G$ must be connected.
\end{pf}

\begin{exercise}[\tcite{book:Bredon} 15.7]
  Show that $\SO(2) \approx \textbf{S}^{1}$, $\SU(2) \approx \textbf{S}^{3}$, and $\Sp(1) \approx \textbf{S}^{3}$ as spaces.
\end{exercise}
\begin{pf}
  Suppose that $A = \begin{pmatrix} a & b \\ c & d \end{pmatrix} \in \SO(2)$.
  Then we must have that $a^{2} + c^{2} = 1 = b^{2} + d^{2}$.
  Thus $a = \cos(\theta)$, $b = \sin(\theta)$, $c = \cos(\alpha)$, and $d = \sin(\alpha)$ for some $\theta, \alpha \in [0, 2\pi)$.
  However, we also have that $ab + cd = 0$ by orthogonality and that $ad - bc = 1$.
  Thus $\cos(\alpha - \theta) = 0$ and $\sin(\alpha - \theta) = 1$.
  Using this we get that $\alpha = \theta + \frac{\pi}{2}$ which implies that $A = \begin{pmatrix} \cos(\theta) & \sin(\theta) \\ -\sin(\theta) & \cos(\theta) \end{pmatrix}$.
  This corresponds to $e^{i \theta} \in \textbf{S}^{1}$.

  \quest{TODO:\@ the other ones later}
\end{pf}

\begin{exercise}[\tcite{book:Bredon} 15.8]
  Show that $\SO(n)$ is connected.
\end{exercise}
\begin{pf}
  Note that $\SO(n) / \SO(n - 1) \approx \textbf{S}^{n - 1}$ for $n \geq 2$.
  Also $\textbf{S}^{n}$ is connected for $n \geq 1$.
  Clearly $\SO(1)$ is connected.
  \quest{TODO:\@ Something something induction.}
  It is clear that the identity is in $\SO(n)$.
\end{pf}

\begin{exercise}[\tcite{book:Bredon} 15.9]
  Show that $\text{U}(n)$ and $\SU(n)$ are both connected and that $\text{U}(n)/\SU(n) \approx \textbf{S}^{1}$.
\end{exercise}
\begin{pf}
  To show that $\text{U}(n)$ is connected, suffices to describe a path from the identity to any other matrix $A \in \text{U}(n)$.
  Note that $A$ can be diagonalized by another unitary matrix $S$ meaning that $A = S \diag\pqty{e^{i \theta_{1}}, \ldots, e^{i \theta_{n}}} S^{-1}$.
  Thus we can take our path to be the map $t \mapsto S \diag\pqty{e^{ti \theta_{1}}, \ldots, e^{ti \theta_{n}}} S^{-1}$.
  This is a path from the identity to $A$.
  Similar logic yields that $\SU(n)$ is also connected by making the right choice of diagonal matrix.

  Note that $\text{U}(n)$ is compact and the quotient is continuous and thus $\text{U}(n) / \SU(n)$ is compact by \Cref{thrm: image_of_compact_is_compact}.
  Then $\textbf{S}^{1}$ is Hausdorff.
  Thus by \Cref{thrm: bij_comp_to_Haus_is_homeo} we have that $\text{U}(n)/\SU(n) \approx \textbf{S}^{1}$.
\end{pf}


\begin{exercise}[\tcite{book:Bredon} 15.12]
  Show that $\GL(n, \mathbb{H})$ is open in $M_{n}(\mathbb{H})$.
\end{exercise}
\begin{pf}
  \quest{TODO:\@ We can do it \href{https://math.stackexchange.com/questions/451402/quaternionic-general-linear-group-is-open}{this way} but I have no intuition for why you would come up with that}.
\end{pf}

\clearpage

\chapter{Convex Bodies}

We often want to know whether certain objects are homeomorphic.
A disk in euclidean space is homeomorphic to a cube, to a cylinder, and to a simplex, which is the analogue of a tetrahedron, as well as others.

\begin{defn}[Convex Body]
  A \emph{convex body} in $\R^{n}$ is a closed set $C \subseteq \R^{n}$ with the property that whenever $p, q \in C$ the whole line segment between $p$ and $q$ is contained in $C$.
\end{defn}

\begin{prop}\label{prop: convex_ray_intersect_boundary}
  If $C \subseteq \R^{n}$ is a convex body and the origin $0 \in \inte(C)$ then any ray from the origin intersects $\partial C = C \setminus \inte(C)$ in at most one point.
\end{prop}
\begin{pf}
  Suppose $R$ is a ray from the origin and $p, q \in R \cap C$, with neither $p$ or $q$ being the origin.
  Suppose also that $q$ is further from the origin than $p$.
  Since the origin is assumed to lie in $\inte(C)$ there is a ball $B$ centered at $0$ completely contained in $C$.
  Consider the union of all line segments from points in $B$ to $q$, which forms the cone on $B$ subtended from $q$.
  Clearly, $p$ is in the interior of the cone and thus is completely contained in $C$ since $C$ is convex.
  Thus $p \in \inte(C)$.
\end{pf}

\begin{prop}\label{prop: homeo_boundry_to_sphere}
  Let $C \subseteq \R^{n}$ be a compact convex body with $0 \in \inte(C)$.
  Then the function $f\colon \partial C \to \textbf{S}^{n - 1}$ given by $x \mapsto \frac{x}{\norm{x}}$ is a homeomorphism.
\end{prop}
\begin{pf}
  Since $f$ is the composition of the inclusion $\partial C \into \R^{n} \setminus \set{0}$ with the radial retraction $r\colon \R^{n} \setminus \set{0} \to \textbf{S}^{n - 1}$, it is continuous.
  We have that \Cref{prop: convex_ray_intersect_boundary} implies that $f$ is injective, and clearly it is surjective.
  Thus $f$ is a homeomorphism by \Cref{thrm: bij_comp_to_Haus_is_homeo}.
\end{pf}

\clearpage

\begin{thrm}
  A compact convex body $C$ in $\R^{n}$ with nonempty interior is homeomorphic to the closed $n$-ball and $\partial C \approx \textbf{S}^{n - 1}$.
\end{thrm}
\begin{pf}
  By translation, assume that the origin is in the interior of $C$.
  Let $\textbf{D}^{n}$ denote the unit disk in $\R^{n}$ and let $f$ be as in \Cref{prop: homeo_boundry_to_sphere}.
  Then let $k\colon \textbf{D}^{n} \to C$ given by $x \mapsto \norm{x} f^{-1}\pqty{\frac{x}{\norm{x}}}$ for $x \neq 0$ and $0 \mapsto 0$.
  We have that $k$ maps $\textbf{D}^{n}$ onto $C$ and is continuous everywhere except possibly at the origin.
  However, $C$ is compact and so there is a bound $M$ for $\set{\norm{x} | x \in C}$.
  Thus $\norm{k(x)} \leq M \cdot \norm{x}$ which implies continuity at the origin.
  It is clear that $k$ is also injective and thus by \Cref{thrm: bij_comp_to_Haus_is_homeo} we have that $k$ is a homeomorphism.
\end{pf}

\clearpage

\chapter{Baire Category Theorem}

We may be interested in a condition on points in a space that is satisfied by an open dense set.
For example, if $p(\overline{x})$ is a function on $\R^{n}$ then the condition $p(x) \neq 0$ has this property.
A special case of this is the determinant of matrices.
If we consider sets satisfying two such conditions, then the set of points satisfying both is still open and dense.
This holds for any finite number of conditions.
A natural question is when this may hold for a countably infinite number of such conditions?
We cannot expect that openness still holds, but it turns out that density survives for a wide class of spaces.

\begin{thrm}[Baire Category Theorem]\label{thrm: baire_category}
  Let $X$ be either a complete metric space or a locally compact Hausdorff space.
  Then the union of countably many nowhere dense subsets of $X$ has empty interior.
\end{thrm}
\begin{pf}
  Let $U$ be an open subset of $X$ and suppose for $i = 0, 1, \ldots$ that $A_{i} \subseteq X$ is nowhere dense.
  Construct a sequence of open sets $V_{1}, V_{2}, \ldots$ such that $\overline{V_{i + 1}} \subseteq V_{i} \setminus \overline{A_{i}}$ where $V_{0} = U$.
  In the complete metric case, this may be done by taking $V_{i + 1} = B_{\e}(x)$ for some $x \in V_{i} \setminus \overline{A_{i}}$ such that $B_{2\e}(x) \subseteq V_{i} \setminus \overline{A_{i}}$.
  \quest{How do we do this for locally compact Hausdorff space?}

  If $X$ is locally compact, then also construct the $V_{i}$ such that $\overline{V_{1}}$, and hence all of the $\overline{V_{i}}$, are compact.
  Then each of the $\overline{V_{i}}$ satisfy the finite intersection property and $\emptyset \neq \bigcap_{i} \overline{V_{i}} \subseteq U \setminus \bigcup_{i} \overline{A_{i}}$.

  If $X$ is a complete metric space, then construct the $V_{i}$ so that $\diam(V_{i}) < 2^{-i}$.
  Then a sequence of points $x_{i} \in V_{i}$ is Cauchy.
  Thus $x_{n} \in \overline{V_{i}}$ for all $n \geq i$ and so $x = \lim(x_{n}) \in \overline{V_{i}}$ for all $i$.
  Thus $x \in \bigcap_{i} \overline{V_{i}} \subseteq U \setminus \bigcup_{i} \overline{A_{i}}$.

  In both of these cases, we have that $U \not\subseteq \bigcup_{i} \overline{A_{i}}$.
  Since $U$ was an arbitrary open set, we have that $\inte\pqty{\overline{A_{i}}} = \emptyset$.
\end{pf}

\begin{defn}[First and Second Category, Residual]
  A subset $S$ of a space $X$ is said to be of the \emph{first category} if it is the countable union of nowhere dense subsets.
  Otherwise, it is of the \emph{second category}.
  A set of the second category is \emph{residual} if its complement is of the first category.
\end{defn}

\clearpage

We can rephrase \Cref{thrm: baire_category} as follows: ``An open subset of a complete metric space or locally compact Hausdorff space is of the second category in itself.''
The contrapositive of \Cref{thrm: baire_category} is also quite useful.

\begin{cor}
  Let $X$ be either a complete metric space or a locally compact Hausdorff space.
  Then the intersection of any countable family of dense open sets in $X$ (\ie\ a residual set) is dense.
\end{cor}

\begin{cor}
  If $\set{f_{n}}$ is a sequence of continuous functions $f_{n}\colon X \to Y$ from a complete metric space $X$ to a complete metric space $Y$ and if $f(x) = \lim f_{n}(x)$ exists for each $x$ then the set of points of continuity of $f$ is residual and hence dense.
\end{cor}
\begin{pf}
  For positive integers $m, k$ let
  \[
    U_{m, k} = \bigcup_{n \geq m} \Set{x | \dist(f_{n}(x), f_{m}(x)) \geq \frac{1}{k}}
  \]
  which is open.
  Since the set
  \[
    \bigcap_{m \geq 1} U_{m, k}
  \]
  consists of points where $f_{n}(x)$ does not converge, it is empty.
  It follows that
  \[
    \bigcap_{m \geq 1} \overline{U_{m, k}} \subseteq \bigcup_{m \geq 1} \pqty{\overline{U_{m, k} \setminus U_{m, k}}}
  \]
  which is a countable union of nowhere dense sets.
  Thus its complement
  \[
    C = \bigcap_{k \geq 1} \bigcup_{m \geq 1} \inte\pqty{x | \dist(f_{n}(x), f_{m}(x)) \leq \frac{1}{k}}
  \]
  is residual.
  Suppose that $y \in C$.
  Then for all $k \geq 1$, there exists $m \geq 1$ such that
  \[
    \exists~\delta > 0 \text{ s.t. } \dist(x, y) < \delta \implies \forall n \geq m,~ \dist(f_{n}(x), f_{m}(x)) \geq \frac{1}{k}.
  \]
  Hence, for such $k, m, \delta$ and $\dist(x, y) < \delta$, we have that $\dist(f(x), f_{m}(x)) \leq \frac{1}{k}$ and also that $\dist(f(y), f_{m}(y)) \leq \frac{1}{k}$.
  By taking $\delta$ smaller as necessary, we can guarantee that $\dist(f_{m}(x), f_{m}(y)) \leq \frac{1}{k}$ by the continuity of $f_{m}$.
  Thus, $\dist(f(x), f(y)) \leq \frac{3}{k}$ for these choices, which implies continuity of $f$ at $y$.
\end{pf}

\clearpage

\begin{cor}
  In the space $\R^{I}$ of continuous functions $I \to \R$ in the uniform metric, the set of functions which are nowhere differentiable is dense.
  Indeed, it is residual in $\R^{I}$.
\end{cor}
\begin{pf}
  For a positive integer $n$, define the set
  \[
    U_{n} = \Set{f \in \R^{I} | \forall~t \in I, \exists~ s \neq t \text{ such that } \abs{\frac{f(t) - f(s)}{t - s}} > n}.
  \]
  We claim that $U_{n}$ is open.
  To see this, note that for a given $f \in U_{n}$ and $t \in I$, there exists $\e > 0$ and $s \neq t$ such that
  \[
    \abs{\frac{f(t) - f(s)}{t - s}} > n + \e.
  \]
  Then, for some such $s = s(t)$ and $\e = \e(t)$, there is an open neighborhood $V_{t}$ of $t$ such that $s(t) \notin \overline{V_{t}}$ and such that
  \[
    \abs{\frac{f(t') - f(s)}{t - s}} > n + \e
  \]
  for all $t' \in V_{t}$.
  The $V_{t}$ cover $I$ so that some finite union $V_{t_{1}} \cup \cdots \cup V_{t_{k}} \containseq I$.
  Let $\e = \min_{i} \e(t_{i})$ and $\delta =  \min_{i} \dist\pqty{s(t_{i}), \overline{V_{t_{i}}}}$.
  Suppose that $\norm{f - g} < \frac{\e\delta}{2}$.
  Then for any $t \in I$ we have that $t \in V_{t_{i}}$ for some $i$ and for $s - s(t)i$ we have that
  \[
    n + \e < \abs{\frac{f(t) - f(s)}{t - s}} \leq \abs{\frac{f(t) - g(t)}{t - s}} + \abs{\frac{g(t) - g(s)}{t - s}} + \abs{\frac{g(s) - f(s)}{t - s}}.
  \]
  Since $\abs{t - s} \geq \delta$, the first and third terms on the right hand side are at most $\frac{\e \delta}{2} \cdot \frac{1}{\delta} = \frac{\e}{2}$.
  We also have that
  \[
    \abs{\frac{g(t) - g(s)}{t - s}} > n + \e - \e = n,
  \]
  and hence $g \in U_{n}$.
  Therefore, each $U_{n}$ is open as claimed.

  Next, we claim that each $U_{n}$ is dense.
  Let $f \in \R^{I}$ and $\e > 0$ be given.
  Let $m$ be such that $\frac{2}{m} < \e$.
  By uniform continuity of $f$, there is some $k$ so large that
  \[
    \abs{x - y} \leq \frac{1}{k} \implies \abs{f(x) - f(y)} \leq \frac{1}{m}.
  \]
  Also, take $k$ so large that $k > nm$.
  Let $a_{i} = \frac{i}{k}, b_{i} = a_{i} + \frac{1}{3k}, c_{i} = a_{i} + \frac{2}{3k}$, and $y_{i} = f(a_{i})$.
  Consider the interval $[a_{i}, a_{i + 1}]$.
  Define a function $g$ on this interval whose graph consists of the three line segments $(a_{i}, y_{i})$ to $(b_{i}, y_{i} - (1 / m))$ to $(c_{i}, y_{i} + (1 / m))$ to $(a_{i + 1}, y_{i + 1})$.
  \quest{See figure}.
  These fit together to define $g$ on all $I$.
  By construction, $\norm{f - g} \leq \frac{2}{m} < \e$.
  Let $t \in [a_{i}, a_{i + 1}]$.
  if $g(t) > y_{i}$, then let $s = b_{i}$.
  Otherwise, let $s = c_{i}$.
  Then we have that
  \[
    \abs{\frac{g(t) - g(s)}{t - s}} \geq \frac{1 / m}{1 / k} = \frac{k}{m} > \frac{nm}{m} = n.
  \]
  Hence $g \in U_{n}$ and $\norm{f - g} < \e$ which proves that $U_{n}$ is dense.

  Since $\R^{I}$ is a complete metric space \quest{why?} we conclude that $A = \bigcap_{n} U_{n}$ is residual.
  We claim that any function $f \in A$ is nowhere differentiable.
  Suppose otherwise that $f$ is differentiable at some $t \in I$.
  Then $\abs{\frac{f(s) - f(t)}{s - t}}$ has a limit as $s \to t$ and so it is bounded for all $s \in I$, $s \neq t$.
  If $n$ is larger than such a bound, then $f \notin U_{n}$ which is a contradiction.
\end{pf}

\clearpage

\subsection*{Exercises}

\clearpage

\chapter{The Inverse Function Theorem}

We will discuss three equivalent theorems from real analysis: the inverse function theorem, the implicit function theorem, and the constant rank theorem.
These discuss the local behavior of a $C^{\infty}$ map from $\R^{n} \to \R^{n}$.

\begin{defn}[Locally Invertible / Local Diffeomorphism]
  A $C^{\infty}$ map $f\colon U \to \R^{n}$ defined on an open $U \subseteq \R^{n}$ is \emph{locally invertible} or a \emph{local diffeomorphism} at a point $p \in U$ if $f$ has a $C^{\infty}$ inverse in some neighborhood of $p$.
\end{defn}

\begin{defn}[Jacobian]
  Let $f\colon U \to \R^{n}$ be a $C^{\infty}$ map defined on an open $U \subseteq \R^{n}$.
  The \emph{Jacobian matrix of $f$} is the matrix $Jf$ where $(Jf)_{i, j} = \frac{\partial f_{i}}{\partial x_{j}}$.
  We call the determinant of $Jf$ the \emph{Jacobian determinant of $f$}.
\end{defn}

\begin{thrm}[Inverse Function Theorem]\label{thrm: inverse_func}
  Let $f\colon U \to \R^{n}$ be a $C^{\infty}$ map defined on an open $U \subseteq \R^{n}$.
  At any point $p \in U$, the map $f$ is invertible in some neighborhood of $p$ if and only if the Jacobian determinant of $f$ evaluated at $p$ is non-zero.
\end{thrm}
Note that although \Cref{thrm: inverse_func} reduces invertibility on an open set to evaluation at a point, the nonvanishing of the Jacobian determinant at a point is equivalent to the nonvanishing in a neighborhod since the Jacobian determinant is continuous.
Since the linear map represented by the Jacobian matrix $Jf(p)$ is the best linear approximation to $f$ at $p$, we can surmise that $f$ is invertible in a neighborhood of $p$ if and only if $Jf(p)$ also is, \ie\ $\det(Jf(p)) \neq 0$.

In an equation such as $f(x, y) = 0$, it is often impossible to explicitly solve for one of the variables in terms of the other.
If we show the existence of some function $y = h(x)$, which may or may not be explicit, such that $f(x, h(x)) = 0$, then we say that $f(x, y) = 0$ can be solved \emph{implicitly} for $y$ in terms of $x$.
The implicit function theorem provides a sufficient condition on a system of equations $f_{i}(x_{1}, \ldots, x_{n}) = 0$, $i = 1, \ldots, m$ on which locally a set of variables can be solved implicitly as $C^{\infty}$ functions of the other variables.

\clearpage

\begin{ex}[Solving for Variables on the Circle]
  Consider the equation $f(x, y) = x^{2} + y^{2} - 1 = 0$ with solution set the unit circle centered at the origin.
  We can see that at any point other than $(\pm 1, 0)$, $y$ is a function of $x$ with $y = \pm \sqrt{1 - x^{2}}$.
  Both of these functions are $C^{\infty}$ as long as $x \neq \pm 1$.
  At these points, there is no neighborhood where $y$ is a function of $x$.
  In general, for a smooth curve $f(x, y) = 0$ in $\R^{2}$ we have that
  \begin{align*}
         & \text{ $y$ can be expressed as a function of $x$ in a neighborhood of a point $(a, b)$} \\
    \iff & \text{ the tangent line to } f(x, y) = 0 \text{ at } (a, b) \text{ is not vertical } \\
    \iff & \text{ the normal vector } \grad\ f \defeq \gen{f_{x}, f_{y}} \text{ to } f(x, y) = 0 \text{ at } (a, b) \text{ is not horizontal } \\
    \iff & f_{y}(a, b) \neq 0.
  \end{align*}
  The implicit function theorem, \Cref{thrm: implicit_func}, generalizes this to higher dimensions.
\end{ex}

\begin{thrm}[Implicit Function Theorem]\label{thrm: implicit_func}
  Let $U$ be an open subset of $\R^{n} \times \R^{m}$ and $f\colon U \to \R^{m}$ a $C^{\infty}$ map.
  Write $(x, y) = (x_{1}, \ldots, x_{n}, y_{1}, \ldots, y_{m})$ for a point in $U$.
  At a point $(a, b) \in U$ such that $f(a, b) = 0$ and the determinant $\det\bqty{\partial f_{i} / \partial y_{i}(a, b)}$ is nonzero, there exists a neighborhood $A \times B$ of $(a, b)$ in $U$ and a unique function $h\colon A \to B$ such that in $A \times B \subseteq U \subseteq \R^{n} \times \R^{m}$ we have that
  \[
    f(x, y) = 0 \iff y = h(x).
  \]
  Moreover, we have that $h$ is $C^{\infty}$.
\end{thrm}
\begin{pf}
  To solve for $f(x, y) = 0$ using the inverse function theorem, we first consider an inverse problem.
  For this, we need a map between two open sets of the same dimension.
  Since $f(x, y)$ is a map from an open set $U$ in $\R^{n + m}$ to $\R^{m}$, we can easily extend it to a map $F\colon  U \to \R^{n + m}$ by adjoining $x$ to the first $n$ components so that we have
  \[
    F(x, y) = (u, v) = (x, f(x, y)).
  \]

  For here on, assume that $n = m = 1$.
  Then we have Jacobian
  \[
    JF = \begin{matrix} 1 & 0 \\ \partial f / \partial x & \partial f / \partial y \end{matrix}.
  \]
  At the point $(a, b)$ we have that
  \[
    \det JF(a, b) = \frac{\partial f}{\partial y}(a, b) \neq 0.
  \]
  Since the Jacobian determinant at $(a, b)$ is nonzero, by the inverse function theorem (\Cref{thrm: inverse_func}) we have that there are neighborhoods $U_{1}$ of $(a, b)$ and $V_{1}$ of $F(a, b) = (a, 0)$ in $\R^{2}$ such that $F\colon U_{1} \to V_{1}$ is a diffeomorphism with $C^{\infty}$ inverse $F^{-1}$.
  See Figure \quest{TODO}.
  Since $F\colon U_{1} \to V_{1}$ is defined by $F(x, y) = (u, v) = (x, f(x, y))$ the inverse map $F^{-1}\colon V_{1} \to U_{1}$ must be of the form $F^{-1}(u, v) = (u, g(u, v))$ for some $C^{\infty}$ function $g\colon V_{1} \to \R$.

  The compositions of $F$ and $F^{-1}$ yield
  \begin{align*}
    (x, y) &= \pqty{F^{-1} \circ F}(x, y) = F^{-1}(x, f(x, y)) = (x, g(x, f(x, y))); \text{ and } \\
    (u, v) &= \pqty{F \circ F^{-1}}(u, v) = F(u, g(u, v)) = (u, f(u, g(u, v))).
  \end{align*}
  Hence we must have that $y = g(x, f(x, y))$ for all $(x, y) \in U_{1}$ and that $v = f(u, g(u, v))$ for all $(u, v) \in V_{1}$.
  If $f(x, y) = 0$ then $y = g(x, 0)$.
  This implies that we should define $h(x) = g(x, 0)$ for all $x \in \R$ where $(x, 0) \in V_{1}$.
  The set of all such $x$ is homeomorphic to $V_{1} \cap \pqty{\R \times \set{0}}$ and is an open subset of $\R$.
  Since $g$ is $C^{\infty}$ by the inverse function theorem, we must have that $h$ is also $C^{\infty}$.

  This also proves that $h$ is unique.
  We claim that for all $(x, y) \in U_{1}$ such that $(x, 0) \in V_{1}$ we must have that
  \[
    f(x, y) = 0 \iff y = h(x).
  \]
  Since $y = g(x, f(x, y))$ for all $(x, y) \in U_{1}$ we have that $y = g(x, f(x, y)) = g(x, 0) = h(x)$.
  Conversely, if $y = h(x)$ take $v = f(u, g(u, v))$ and set $(u, v) = (x, 0)$.
  Then we have that $0 = f(x, g(x, 0)) = f(x, h(x)) = f(x, y)$.

  By this claim, in some neighborhood of $(a, b)$ in $U_{1}$ we have that the zero set of $f(x, y)$ is the graph of $h$.
  We now need to find a product neighborhood of $(a, b)$ as in the statement.
  Let $A_{1} \times V$ be some neighborhood of $(a, b)$ contained in $U_{1}$.
  Let $A = h^{-1}(B) \cap A_{1}$.
  Since $h$ is continuous, $A$ is open in the domain of $H$ and hence in $\R^{1}$.
  Then $h(A) \subseteq B$ and we have
  \[
    A \times B \subseteq A_{1} \times B \subseteq U_{1}, \text{ and } A \times \set{0} \subseteq V_{1}.
  \]
  By the claim, for $(x, y) \in A \times B$ we have $f(x, y) = 0 \iff y = h(x)$.

  Replacing the partial derivative $\frac{\partial f}{\partial y}$ with a Jacobian matrix $\bqty{\partial f_{i} / \partial y_{j}}$, we recover the general form the of theorem.
  We also do not require that $y_{1}, \ldots, y_{m}$ be the last $m$ coordinates in $\R^{n + m}$.
  They can be any set of $m$ coordinates.
\end{pf}

\clearpage

\begin{thrm}
  The implicit function theorem is equivalent to the inverse function theorem.
\end{thrm}
\begin{pf}
  We saw in the proof of \Cref{thrm: implicit_func} that the inverse function theorem implies the implicit function theorem, at least for one typical case.

  Suppose the implicit function theorem holds and let $f\colon U \to \R^{n}$ be a $\C^{\infty}$ function defined on an open subset $U$ of $\R^{n}$ such that at some point $p \in U$ the Jacobian determinant $\det\bqty{\partial f_{i} / \partial x_{j}(p)}$ is nonzero.
  Finding a local inverse for $y = f(x)$ near $p$ amounts to solving
  \[
    g(x, y) = f(x) - y = 0
  \]
  for $x$ in terms of $y$ near $(p, f(p))$.
  Note that $\frac{\partial g_{i}}{\partial x_{j}} = \frac{\partial f_{i}}{\partial x_{j}}$.
  Hence we have that
  \[
    \det\bqty{\frac{\partial g_{i}}{\partial x_{j}} (p, f(p))} = \det\bqty{\frac{\partial f_{i}}{\partial x_{j}}(p, f(p))} \neq 0.
  \]
  Thus $y = f(h(y))$.
  Since $y = f(x)$ we have that $x = h(y) = h(f(x))$.
  Therefore, $f$ and $h$ are inverses defined near $p$ and $f(p)$ respectively.
\end{pf}

\begin{defn}[Rank of a $C^\infty$ Map]
  For a given $C^{\infty}$ map $f\colon U \to \R^{n}$ defined on an open $U \subseteq \R^{n}$, the \emph{rank of $p \in U$} is the rank of the Jacobian matrix $\bqty{\partial f_{i} / \partial x_{j}(p)}$.
\end{defn}

\begin{thrm}[Constant Rank Theorem]\label{thrm: constant_rank}
  If $f\colon \R^{n} \containseq U \to \R^{m}$ has constant rank $k$ in a neighborhood of a point $p \in U$, then after a suitable change of coordinates near $p \in U$ and $f(p) \in \R^{m}$, the map assumes the form
  \[
    \pqty{x_{1}, \ldots, x_{n}} \mapsto \pqty{x_{1}, \ldots, x_{k}, 0, \ldots, 0}.
  \]
  More precisely, there are diffeomorphisms $G$ from a neighborhood $p$ in $U$ sending $p$ to the origin and a $F$ of a neighborhood of $f(p)$ in $\R^{m}$ sending $f(p)$ to the origin such that
  \[
    \pqty{F \circ f \circ G^{-1}}\pqty{x_{1}, \ldots, x_{n}} = \pqty{x_{1}, \ldots, x_{k}, 0, \ldots, 0}.
  \]
\end{thrm}
\begin{pf}
  We consider the case where $n = m = 2$ and $k = 1$.
  Suppose $f = (f_{1}, f_{2}\colon \R^{2} \containseq U \to \R^{2})$ has constant rank $1$ in a neighborhood $p \in U$.
  By reordering the functions $f_{1}, f_{2}$ or the variables $x, y$ we may assume that $\frac{\partial f_{1}}{\partial x}(p) \neq 0$ by the fact that $f$ has rank $\geq 1$ at $p$.
  Define $G\colon U \to \R^{2}$ by $G(x, y) = (u, v) = \pqty{f_{1}(x, y), y}$.
  The Jacobian matrix of $G$ is
  \[
    JG = \begin{bmatrix} \partial f_{1} / \partial x & \partial f_{1} / \partial y \\ 0 & 1 \end{bmatrix}.
  \]

  \clearpage

  Since $\det JG(p) = \partial f_{1} / \partial x (p) \neq 0$, by the inverse function theorem there are neighborhoods $U_{1}$ of $p \in \R^{2}$ and $V_{1}$ of $G(p) \in \R^{2}$ such that $G\colon U_{1} \to V_{1}$ is a diffeomorphism.
  By making $U_{1}$ a sufficiently small neighborhood of $p$, we may assume that $f$ has constant rank $1$ on $U_{1}$.

  On $V_{1}$ we have that
  \[
    (u, v) = \pqty{G \circ G^{-1}} = \pqty{f_{1} \circ G^{-1}, y \circ G^{-1}}(u, v).
  \]
  Thus $u = \pqty{f_{1} \circ G^{-1}}(u, v)$.
  Thus we have that
  \begin{align*}
    \pqty{f \circ G^{-1}}(u, v) &= \pqty{f_{1} \circ G^{-1}, f_{2} \circ G^{-1}}(u, v) \\
                                &= \pqty{u, f_{2} \circ G^{-1}(u, v)} \\
                                &= (u, h(u, v))
  \end{align*}
  here $h \defeq f_{2} \circ G^{-1}$.

  Since $G^{-1}\colon V_{1} \to U_{1}$ is a diffeomorphism and $f$ has constant rank $1$ on $U_{1}$, the composite $f \circ G_{1}$ has constant rank $1$ on $V_{1}$.
  It's Jacobian matrix is
  \[
    J\pqty{f \circ G^{-1}} = \begin{bmatrix} 1 & 0 \\ \partial h / \partial u & \partial h / \partial v \end{bmatrix}.
  \]
  For this matrix to have constant rank $1$, $\partial h / \partial v$ must be $0$ on $V_{1}$ since $f$ has rank $\leq 1$ in a a neighborhood of $p$.
  Thus $h$ is a function of $u$ alone and we have $\pqty{f \circ G^{-1}}(u, v) = (u, h(u))$.
  Let $F\colon \R^{2} \to \R^{2}$ be the the change of coordinates $F(x, y) = (x, y - h(x))$.
  Then we have as desired
  \[
    \pqty{F \circ f \circ G^{-1}}(u, v) = F(u, h(u)) = (u, h(u) - h(u)) = (u, 0).
  \]
\end{pf}

If a $C^{\infty}$ map $f\colon \R^{n} \containseq U \to \R^{n}$ defined on an open subset $U$ of $\R^{n}$ has nonzero Jacobian determinant $\det(Jf(p))$ at a point $p \in U$, then by continuity it has nonzero Jacobian determinant in a neighborhood of $p$.
Therefore, it has constant rank $n$ in a neighborhood of $p$.

\clearpage
\nocite{book:TuAITM}
\nocite{note:conrad_topologist_sine}
\nocite{note:compactness_review}
\nocite{book:Rudin_RCA}
\nocite{book:Steen_Seebach}
\nocite{book:Munkres}
\nocite{book:Lee_ITM}
\nocite{book:hatcher_AT}
\printbibliography
\end{document}
