\documentclass[letterpaper, 11pt, oneside]{book}

\usepackage{style}  % If you feel like procrastinating, mess with this file
\usepackage{algo}   % Thank you Jeff, very cool!

\addbibresource{refs.bib}

% Required reading
% https://jmlr.csail.mit.edu/reviewing-papers/knuth_mathematical_writing.pdf
%   Along with required viewing:
%   https://www.youtube.com/watch?v=N6QEgbPWUrg&list=PLOdeqCXq1tXihn5KmyB2YTOqgxaUkcNYG
% https://faculty.math.illinois.edu/~west/grammar.html

% % % % % % % % % %
%     Cursor      %
%     Parking     %
%     Lot         %
% % % % % % % % % %

% Disable check for mismatched parens/brackets/braces
%   chktex-file 9
% Disable check for different counts of parents/brackets/braces
%   chktex-file 17
% Exclude these environments from syntax checking
%   VerbEnvir { tikzcd }

\regtotcounter{figure}

\title{\vspace{-100pt} {\Huge Algorithms in Invariant Theory Exercises} \\ {\small With $\total{figure}$ Figures}}
\author{\Large Anakin Dey}
\DTMsavenow{now}
\date{\small Last Edited on \today\ at \DTMfetchhour{now}:\DTMfetchminute{now}}

% Cover page number chicanery
\newcommand{\CoverName}{Cover}

\begin{document}
\frontmatter
\renewcommand{\thepage}{\CoverName}
\maketitle

\pagenumbering{roman}

\tableofcontents
\clearpage


% \listoftheorems[ignoreall, show={defn}, title={List of Definitions}]
%
% \listoftheorems[ignoreall, show={ex}, title={List of Examples and Counterexamples}]

\chapter*{Preface}

A core interest of mine (at the time of writing this) is algorithms in the context of commutative algebra and algebraic geometry.
As such, Bernd Sturmfel's text \emph{Algorithms in Invariant Theory}~\cite{book:AlgosInInvTheory} is a good resource for first learning this stuff.
This solely just contains exercises for now.
Perhaps in the future I'll include notes and some source code

\mainmatter

\chapter{Introduction}

\section*{Symmetric Polynomials}

\begin{exercise}[\tcite{book:AlgosInInvTheory} 1.1.5]
  Prove the following explicit formula for elementary symmetric polynomials in terms of the power sums~\cite[Page 29]{book:MacdonaldSymmetricHall}.
  \[
    \sigma_{k} = \frac{1}{k!}
    \det
    \begin{pmatrix}
      p_{1} & 1 & 0 & \cdots & 0 \\
      p_{2} & p_{1} & 2 & \cdots & 0 \\
      \vdots & \vdots & \ddots & \ddots & \vdots \\
      p_{k - 1} & p_{k - 2} & \cdots & p_{1} & k - 1 \\
      p_{k} & p_{k - 1} & \cdots & \cdots & p_{1}
    \end{pmatrix}
  \]
\end{exercise}
\begin{pf}
  From~[Page 23, 2.11']\cite{book:MacdonaldSymmetricHall}, we have the following identities
  \begin{align*}
    1 \cdot \sigma_{1} &= p_{1}, \\
    2 \cdot \sigma_{2} &= p_{1} \sigma_{1} - p_{2}, \\
    3 \cdot \sigma_{3} &= p_{1} \sigma_{2} - p_{2} \sigma_{1} + p_{3}, \\
                       &\vdots \\
    k \cdot \sigma_{k} &= \sum_{r = 1}^{k} (-1)^{r - 1} p_{r} \sigma_{k - r}.
  \end{align*}
Treating the $\sigma_{i}$ as indeterminants, we can re-express the above system of equations:
\begin{align*}
  p_{1} &= 1 \cdot \sigma_{1}, \\
  p_{2} &= p_{1} \sigma_{1} - 2 \cdot \sigma_{2}, \\
  p_{3} &= p_{2} \sigma_{1} - p_{1} \sigma_{2} + 3 \cdot \sigma_{3}, \\
        &\vdots \\
  p_{k} &= \pqty{\sum_{r = 1}^{k - 1} (-1)^{r - 1} p_{k - r} \sigma_{r}} + (-1)^{k - 1} \cdot k \sigma_{k}.
\end{align*}
Consider $\sigma_{1}, -\sigma_{2}, \sigma_{3}, \ldots, (-1)^{n} \sigma_{n - 1}, \sigma_{n}$ as indeterminants.
Thus, we obtain the following matrix equation:
\[
  \begin{pmatrix}
    1 & 0 & 0 & \cdots & 0 \\
    p_{1} & 2 & 0 & \cdots & 0 \\
    p_{2} & p_{1} & 3 & \cdots & 0 \\
    \vdots & \vdots & \ddots & \ddots & \vdots \\
    p_{k - 1} & p_{k - 2} & \cdots & \cdots & (-1)^{k} \cdot k
  \end{pmatrix}
  \begin{pmatrix}
    \sigma_{1} \\
    -\sigma_{2} \\
    \vdots \\
    (-1)^{k} \sigma_{k - 1} \\
    \sigma_{k}
  \end{pmatrix}
  =
  \begin{pmatrix}
    p_{1} \\
    p_{2} \\
    \vdots \\
    p_{k - 1} \\
    p_{k}
  \end{pmatrix}.
\]
Then, Cramer's rule yields that
\begin{align*}
  \sigma_{k} &=
  \det
  \begin{pmatrix}
    1 & 0 & 0 & \cdots & p_{1} \\
    p_{1} & 2 & 0 & \cdots & p_{2} \\
    p_{2} & p_{1} & 3 & \cdots & p_{3} \\
    \vdots & \vdots & \ddots & \ddots & \vdots \\
    p_{k - 1} & p_{k - 2} & \cdots & \cdots & p_{k}
  \end{pmatrix}
  \det
  \begin{pmatrix}
    1 & 0 & 0 & \cdots & 0 \\
    p_{1} & 2 & 0 & \cdots & 0 \\
    p_{2} & p_{1} & 3 & \cdots & 0 \\
    \vdots & \vdots & \ddots & \ddots & \vdots \\
    p_{k - 1} & p_{k - 2} & \cdots & \cdots & (-1)^{k} \cdot k
  \end{pmatrix}^{-1} \\
    &= \frac{(-1)^{k}}{k!}
  \det
  \begin{pmatrix}
    1 & 0 & 0 & \cdots & p_{1} \\
    p_{1} & 2 & 0 & \cdots & p_{2} \\
    p_{2} & p_{1} & 3 & \cdots & p_{3} \\
    \vdots & \vdots & \ddots & \ddots & \vdots \\
    p_{k - 1} & p_{k - 2} & \cdots & \cdots & p_{k}
  \end{pmatrix} \\
    &= \frac{(-1)^{k}}{k!} \cdot (-1)^{k}
  \det
  \begin{pmatrix}
    p_{1} & 1 & 0 & \cdots & 0 \\
    p_{2} & p_{1} & 2 & \cdots & 0 \\
    \vdots & \vdots & \ddots & \ddots & \vdots \\
    p_{k - 1} & p_{k - 2} & \cdots & p_{1} & k - 1 \\
    p_{k} & p_{k - 1} & \cdots & \cdots & p_{1}
  \end{pmatrix}
     = \frac{1}{k!}
  \det
  \begin{pmatrix}
    p_{1} & 1 & 0 & \cdots & 0 \\
    p_{2} & p_{1} & 2 & \cdots & 0 \\
    \vdots & \vdots & \ddots & \ddots & \vdots \\
    p_{k - 1} & p_{k - 2} & \cdots & p_{1} & k - 1 \\
    p_{k} & p_{k - 1} & \cdots & \cdots & p_{1}
  \end{pmatrix}.
\end{align*}
\end{pf}

\section*{\Grobner\ Bases}

\begin{exercise}[\tcite{book:AlgosInInvTheory} 1.2.1]
  Let $\prec$ be an monomial order and let $I$ be any ideal in $\C[x_{1}, \ldots, x_{n}]$.
  A monomial $m$ is called \emph{minimally nonstandard} if $m$ is nonstandard and all proper divisors of $m$ are standard.
  Show that the set of minimally nonstandard monomials is finite.
\end{exercise}
\begin{pf}
  Let $M$ be a set of monomial generators for $\init(I)$ and let $m$ be minimally nonstandard.
  Since $m$ is a monomial and in $\init(I)$, we have that $m' \mid m$ for some monomial $m' \in M$.
  However, note that $m' \in \init(I)$ and furthermore by the fact that $m$ is minimally nonstandard, we cannot have that $m'$ strictly divides $m$.
  Thus, $m' = m$ and $m \in M$.
  Then by Dickson's Lemma, we have that $M$ is a finite set and thus there are finitely many minimally nonstandard monomials.
\end{pf}

\begin{exercise}[\tcite{book:AlgosInInvTheory} 1.2.2]
  Prove that the reduced \Grobner\ basis $\mathcal{G}_{\text{red}}$ of $I$ with respect to $\prec$ is unique (up to multiplicative constants from $\C$).
  Given an algorithm which transforms an arbitrary \Grobner\ basis into $\mathcal{G}_{\text{red}}$.
\end{exercise}
\begin{pf}
  This is~\cite[Chapter 2, \S 7, Theorem 5]{book:IVA}.
\end{pf}

\begin{exercise}[\tcite{book:AlgosInInvTheory} 1.2.3]
  Let $I \subseteq \C[x_{1}, \ldots, x_{n}]$ be an ideal, given by a finite set of generators.
  Using \Grobner\ bases, describe an algorithm for computing the \emph{elimination ideals} $I \cap \C[x_{1}, \ldots, x_{i}]$ for $i = 1, \ldots, n - 1$, and prove its correctness.
\end{exercise}
\begin{pf}
  This is~\cite[Chapter 3, \S 1, Theorem 2]{book:IVA}.
\end{pf}

\begin{exercise}[\tcite{book:AlgosInInvTheory} 1.2.4]
  Find a characterization of all monomial orders on the polynomial ring $\C[x_{1}, x_{2}]$.
  Hint: each variable receives a certain ``weight'' which behaves additively under multiplication of variables.
  Generalize your result to $n$ variables.
\end{exercise}
\begin{pf}
  \quest{Look at~\cite[Chapter 1, \S 2, Exercise 6]{book:UsingAG}.}
\end{pf}

\begin{exercise}[\tcite{book:AlgosInInvTheory} 1.2.6]
  Let $\mathcal{F}$ be a set of polynomials whose initial monomials are pairwise relatively prime.
  Show that $\mathcal{F}$ is a \Grobner\ basis for its ideal.
\end{exercise}
\begin{pf}
  First, we recall some definitions.
  The $S$-polynomial of $f$ and $g$ is
  \[
    S(f, g) \defeq \frac{\lcm(\lm(f), \lm(g))}{\lt(f)} f - \frac{\lcm(\lm(f), \lm(g))}{\lt(g)} g.
  \]
  For a set of polynomials $\mathcal{F} = \set{f_{1}, \ldots, f_{t}} \subseteq k[x_{1}, \ldots, x_{n}]$, we write $f \to_{\mathcal{F}} 0$ if there exists $a_{1}, \ldots, a_{t} \in k[x_{1}, \ldots, x_{n}]$ such that $a_{1} f_{1} + \cdots + a_{t} f_{t} = 0$.
  Then~\cite[Chapter 2, \S 9, Theorem 3]{book:IVA} says that a basis $\mathcal{F} = \set{f_{1}, \ldots f_{t}}$ is a \Grobner\ basis for $G$ if and only if $S(f_{i}, f_{j}) \to_{\mathcal{F}} 0$ for all $i \neq j$.
  But~\cite[Chapter 2, \S 9, Proposition 4]{book:IVA} says that for $f, g \in \mathcal{F}$ with relatively prime initial monomials, we have that $S(f, g) \to_{\mathcal{F}} 0$.
  This proves the claim.
\end{pf}

\printbibliography
\end{document}
