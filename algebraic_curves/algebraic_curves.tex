\documentclass[letterpaper, 11pt, oneside]{book}

\usepackage{style}  % If you feel like procrastinating, mess with this file
\usepackage{algo}   % Thank you Jeff, very cool!

\addbibresource{refs.bib}

% Required reading
% https://jmlr.csail.mit.edu/reviewing-papers/knuth_mathematical_writing.pdf
%   Along with required viewing:
%   https://www.youtube.com/watch?v=N6QEgbPWUrg&list=PLOdeqCXq1tXihn5KmyB2YTOqgxaUkcNYG
% https://faculty.math.illinois.edu/~west/grammar.html

% % % % % % % % % %
%     Cursor      %
%     Parking     %
%     Lot         %
% % % % % % % % % %

% Disable check for mismatched parens/brackets/braces
%   chktex-file 9
% Disable check for different counts of parents/brackets/braces
%   chktex-file 17
% Exclude these environments from syntax checking
%   VerbEnvir { tikzcd }

\regtotcounter{figure}

\title{\vspace{-100pt} {\Huge Algebraic Curves} \\ {\small With $\total{figure}$ Figures}}
\author{\Large Anakin Dey}
\DTMsavenow{now}
\date{\small Last Edited on \today\ at \DTMfetchhour{now}:\DTMfetchminute{now}}

% Cover page number chicanery
\newcommand{\CoverName}{Cover}

\begin{document}
\frontmatter
\renewcommand{\thepage}{\CoverName}
\maketitle

\pagenumbering{roman}

\tableofcontents
\clearpage


% \listoftheorems[ignoreall, show={defn}, title={List of Definitions}]
%
% \listoftheorems[ignoreall, show={ex}, title={List of Examples and Counterexamples}]

\chapter*{Preface}

At the time of writing this, I am starting my PhD at The Ohio State University.
Currently a large part of my interests in algebra are about algorithms as they relate to polynomials and algebraic geometry.
As such, I've been primarily looking at \emph{Ideals, Varieties, and Algorithms}~\cite{book:IVA},
I've been doing a bunch of problems from \emph{Ideals, Varieties, and Algorithms}~\cite{book:IVA}, \emph{Using Algebraic Geometry}~\cite{book:UAG} and stealing glances at Sturmfel's \emph{Algorithms in Invariant Theory}~\cite{book:AlgosInInvTheory}.
However, this book is such a classic and I'm of the opinion that you can never know too much about curves and varieties.
To put my money where my mouth is, I've started to \TeX\ up some exercises from Fulton's \emph{Algebraic Curves}~\cite{book:Curves}.

\mainmatter

\chapter{Affine Algebraic Sets}

\section{Algebraic Preliminaries}

\begin{sol}[\cite{book:Curves} Ex. 1.1]\label{ex:Curves_1.1}
  As normal, we denote a monomial $x_{1}^{i_{1}} \cdots x_{n}^{i_{n}}$ as $\overline{x}^{I}$ where $\abs{I} = i_{1} + \cdots + i_{n} = d \in \N$ and $I \in \N^{n}$.
  Thus, if $f(x_{1}, \ldots, x_{n}), g(x_{1}, \ldots, x_{n}) \in R[x_{1}, \ldots, x_{n}]$ are forms of degree $r$ and $s$ respectively, we have that
  \begin{align*}
    f(x_{1}, \ldots, x_{n}) &= \sum_{I \in \N ^{n}, \abs{I} = r} f_{I} \overline{x}^{I}, \\
    g(x_{1}, \ldots, x_{n}) &= \sum_{J \in \N ^{n}, \abs{J} = s} g_{J} \overline{x}^{J}.
  \end{align*}
  Then, we have by distribution that
  \[
    f(x_{1}, \ldots, x_{n}) \cdot g(x_{1}, \ldots, x_{n}) = \sum_{\substack{I \in \N ^{n}, \abs{I} = r \\ J \in \N ^{n}, \abs{J} = s}} f_{I} \overline{x}^{I} g_{J} \overline{x}^{J}.
  \]
  For each $I, J$ we have that if $f_{I}, g_{J} \neq 0$ then $f_{I} g_{J} \overline{x}^{I} \overline{x}^{J} = (fg)_{I, J} \overline{x}^{I + J}$ where $(fg)_{I, J} = (fg)_{I, J} = f_{I} g_{J}$
  Notice that $\abs{I + J} = r + s$ by coordinatewise addition and so each term is either $0$ or a degree $r + s$ homogeneous form.
  Thus after combining like terms, we have that $f(x_{1}, \ldots, x_{N}) \cdot g(x_{1}, \ldots, x_{n})$ is either degree $0$ or a degree $r + s$ homogeneous form.
  As $0$ has any degree we choose, we have that $f \cdot g$ is a degree $r + s$ form.

  If we wanted to show that the product of two nonzero polynomials is still nonzero, it can be shown that $R$ a domain implies $R[x_{1}, \ldots, x_{n}]$ is a domain.
  This can be seen by looking at leading terms and inducting on the number of variables.
\end{sol}

\clearpage

\begin{sol}[\cite{book:Curves} Ex. 1.2]\label{ex:Curves_1.2}
  The expression of $z \in K = \mathrm{Frac}(R)$ as $z = \frac{a}{b}$ for some $a \in R, b \in R$ such that $a, b$ have no common factors, is obvious as we only consider finite sums so we can always take common denominators.
  Suppose that $z = \frac{a}{b} = \frac{a'}{b'}$ two different representatives.
  Since $R$ is a UFD, say that $a = u_{a}a_{1} \cdots a_{n_{a}}$ and $a' = u_{a'}a'_{1} \cdots a'_{n_{a'}}$ are factorizations of $a$ and $a'$.
  We have similar factorizations for $b$ and $b'$.
  Then we have that $ab' = a'b \in R$.
  However, as $R$ is a unique factorization domain, we must have
  \[
    u_{a}a_{1} \cdots a_{n_{a}} u_{b'}b'_{1} \cdots b'_{n_{b'}} = u_{a'}a'_{1} \cdots a'_{n_{a'}} u_{b}b_{1} \cdots b_{n_{b}}
  \]
  are two equivalent factorizations up to unit.
  This implies that $\frac{a}{b}$ and $\frac{a'}{b'}$ differ only up to unit.
\end{sol}

\begin{sol}[\cite{book:Curves} Ex. 1.3]\label{ex:Curves_1.3}
  \begin{enumerate}[label= (\alph*)]
    \item We know that $P = (p)$ for some nonzero, nonunital $p \in R$.
          Suppose that $p$ was reducible, so $p = ab$ for some nonzero nonunital $a, b \in R$.
          Then as $p \nmid a$ and $p \nmid b$, since such divisibility would imply that $a$ or $b$ were equal to $p$ up to unit, we have that $a, b \notin (p)$.
          But then we have elements $a, b \in R$ such that $a \notin (p)$, $b \notin (p)$ but $ab = p \in (p)$ which contradicts the primality of $P = (p)$.
          Then $p$ is irreducible.
    \item We show that $P = (p)$ is maximal.
          There exists a maximal ideal $M = (m)$ such that $(p) \subseteq (m)$.
          Thus, $p \mid m$.
          Notice that by \textbf{(a)} that both $p$ and $m$ are irreducible elements of $R$.
          As such, if $p \mid m$ we must have that $m \mid p$ as if not, $m$ would be reducible.
          Thus, $p = um$ for some unit $u \in R^{\times}$.
          Thus, $(p) = (m)$ and $P = (p)$ is maximal.
  \end{enumerate}
\end{sol}


\printbibliography
\end{document}
