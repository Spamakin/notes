\documentclass[letterpaper, 11pt, oneside]{book}

\usepackage{style}  % If you feel like procrastinating, mess with this file
\usepackage{algo}   % Thank you Jeff, very cool!

\addbibresource{refs.bib}

% Required reading
% https://jmlr.csail.mit.edu/reviewing-papers/knuth_mathematical_writing.pdf
%   Along with required viewing:
%   https://www.youtube.com/watch?v=N6QEgbPWUrg&list=PLOdeqCXq1tXihn5KmyB2YTOqgxaUkcNYG
% https://faculty.math.illinois.edu/~west/grammar.html

% % % % % % % % % %
%     Cursor      %
%     Parking     %
%     Lot         %
% % % % % % % % % %

% Disable check for mismatched parens/brackets/braces
%   chktex-file 9
% Disable check for different counts of parents/brackets/braces
%   chktex-file 17
% Exclude these environments from syntax checking
%   VerbEnvir { tikzcd }

\regtotcounter{figure}

\title{\vspace{-100pt} {\Huge Using Algebraic Geometry} \\ {\small With $\total{figure}$ Figures}}
\author{\Large Anakin Dey}
\DTMsavenow{now}
\date{\small Last Edited on \today\ at \DTMfetchhour{now}:\DTMfetchminute{now}}

% Cover page number chicanery
\newcommand{\CoverName}{Cover}

\begin{document}
\frontmatter
\renewcommand{\thepage}{\CoverName}
\maketitle

\pagenumbering{roman}

\tableofcontents
\clearpage


% \listoftheorems[ignoreall, show={defn}, title={List of Definitions}]
%
% \listoftheorems[ignoreall, show={ex}, title={List of Examples and Counterexamples}]

\chapter*{Preface}

At the time of writing this, I am starting my PhD at The Ohio State University.
Currently a large part of my interests in algebra are about algorithms as they relate to polynomials and algebraic geometry.
I've been doing a bunch of problems from \emph{Ideals, Varieties, and Algorithms}~\cite{book:IVA}.
However,  it seems that \emph{Using Algebraic Geometry}~\cite{book:UAG} moves through the material faster as it assumes you know more algebra.
So I've moved onto working through this book as well as trying to comprehend Sturmfel's \emph{Algorithms in Invariant Theory}~\cite{book:AlgosInInvTheory}.

\mainmatter

\chapter{Introduction}

\section{Polynomials and Ideals}

\begin{exercise}[\tcite{book:UAG} 1.1.1]\label{ex:UAG_1.1.1}
  \begin{enumerate}[label= (\alph*)]
    \item Show that $x^{2} \in \ideal{x - y^{2}, xy}$ in $k[x, y]$.
    \item Show that $\ideal{x - y^{2}, xy, y^{2}} = \ideal{x, y^{2}}$.
    \item Is $\ideal{x - y^{2}, xy} = \ideal{x^{2}, xy}$? Why or why not?
  \end{enumerate}
\end{exercise}

\begin{pf}
  \begin{enumerate}[label= (\alph*)]
    \item We have that $x(x - y^{2}) + y(xy) = x^{2} - xy^{2} + xy^{2} = x^{2}$.
    \item It suffices to check for generators.
          We have that $x + (-1)(y^{2}) = x - y^{2}, y(x) = xy$, and $y^{2} = y^{2}$ showing that $\ideal{x - y^{2}, xy, y^{2}} \subseteq \ideal{x, y^{2}}$.
          Then $x - y^{2} + y^{2} = x$ and $y^{2} = y^{2}$ shows the reverse containment and overall the ideals are equal.
    \item We already know from 1.\ that $x^{2}$ lives in $\ideal{x - y^{2}, xy}$.
          Since $xy = xy$, we overall have that $\ideal{x^{2}, xy} \subseteq \ideal{x - y^{2}, xy}$.
          It remains to check if $x - y^{2} \in \ideal{x^{2}, xy}$.
          However, notice that every element of $\ideal{x^{2}, xy}$ is divisible by $x$ while $x - y^{2}$ is clearly not divisible by $x$.
          Thus $x - y^{2} \notin \ideal{x^{2}, xy}$ and the two ideals are not equal.
  \end{enumerate}
\end{pf}

\clearpage

\begin{exercise}[\tcite{book:UAG} 1.1.2]\label{ex:UAG_1.1.2}
  Show that $\ideal{f_{1}, \ldots, f_{s}}$ is closed under sums in $k[x_{1}, \ldots, x_{n}]$.
  Also show that if $f \in \ideal{f_{1}, \ldots, f_{s}}$ and $p \in k[x_{1}, \ldots, x_{n}]$ then $p \cdot f \in \ideal{f_{1}, \ldots, f_{s}}$.
\end{exercise}
\begin{pf}
  Let $f, g \in \ideal{f_{1}, \ldots, f_{s}}$.
  Then $\exists p_{1}, \ldots, p_{s}, q_{1}, \ldots, q_{s}$ such that $f = \sum_{i = 1}^{s} p_{i} \cdot f_{i}$ and $g = \sum_{i = 1}^{s} q_{i} \cdot f_{i}$.
  Thus $f + g = \sum_{i = 1}^{s} (p_{i} + q_{i}) \cdot f_{i}$ which shows that $f + g \in \ideal{f_{1}, \ldots, f_{s}}$.
  Then let $p \in k[x_{1}, \ldots, x_{n}]$.
  We have that $p \cdot f = p\sum_{i = 1}^{s} p_{i} f_{i} = \sum_{i = 1}^{s} (p \cdot p_{i}) \cdot f_{i}$ which shows that $\ideal{f_{1}, \ldots, f_{s}}$ is an ideal.
\end{pf}

\begin{exercise}[\tcite{book:UAG} 1.1.3]\label{ex:UAG_1.1.3}
  Show that $\ideal{f_{1}, \ldots, f_{s}}$ is the smallest ideal containing $\set{f_{1}, \ldots, f_{s}}$.
\end{exercise}
\begin{pf}
  We already know that $\ideal{f_{1}, \ldots, f_{s}}$ is an ideal by \Cref{ex:UAG_1.1.2}.
  Now suppose that $J$ is an ideal containing $\set{f_{1}, \ldots, f_{s}}$.
  Then, since ideals are closed under addition and scaling, we have that for all $p_{1}, \ldots, p_{s} \in k[x_{1}, \ldots, x_{n}]$ that $\sum_{i = 1}^{s} p_{i} \cdot f_{i} \in J$.
  Thus, $\ideal{f_{1}, \ldots, f_{s}} \subseteq J$.
\end{pf}

\begin{exercise}[\tcite{book:UAG} 1.1.4]\label{ex:UAG_1.1.4}
  Using \Cref{ex:UAG_1.1.3}, formulate and prove a general criterion for the equality of $I = \ideal{f_{1}, \ldots, f_{s}}$ and $J = \ideal{g_{1}, \ldots, g_{t}}$.
\end{exercise}
\begin{pf}
  We claim that $\ideal{f_{1}, \ldots, f_{s}} = \ideal{g_{1}, \ldots, g_{t}}$ if and only if $\set{g_{1}, \ldots, g_{t}} \subseteq I$ and $\set{f_{1}, \ldots, f_{s}} \subseteq J$.
  The forward implication is immediate.
  Then by \Cref{ex:UAG_1.1.3}, if $\set{g_{1}, \ldots, g_{t}} \subseteq I$ then $J \subseteq I$.
  Similarly, $\set{f_{1}, \ldots, f_{s}} \subseteq J \implies I \subseteq J$ and overall $I = J$.
  This fact was used in \Cref{ex:UAG_1.1.1} (b).
\end{pf}

\begin{exercise}[\tcite{book:UAG} 1.1.5]\label{ex:UAG_1.1.5}
  Show that $\ideal{y - x^{2}, z - x^{3}} = \ideal{y - x^{2}, z - xy}$ in $\Q[x, y, z]$.
\end{exercise}
\begin{pf}
  It suffices to show that $z - x^{3} \in \ideal{y - x^{2}, z - xy}$ and and $z - xy \in \ideal{x - y^{2}, z - x^{3}}$.
  Indeed we have that $(z - xy) + x(y - x^{2}) = z - x^{3}$ which also yields that $z - xy = z - x^{3} - x(y - x^{2})$.
\end{pf}

\clearpage

\begin{exercise}[\tcite{book:UAG} 1.1.6]\label{ex:UAG_1.1.6}
  Show that every ideal $I \subseteq k[x]$ is generated by a single polynomial.
\end{exercise}
\begin{pf}
  If $I = \set{0}$ then $I = \ideal{0}$.
  So suppose $I \neq 0$.
  Let $d \in I$ be of minimal degree.
  \quest{$d = \gcd(I)$ but I need infinite Bezout.}
  Then we claim that $\ideal{d} = I$.
  Since $d \in I$, we have that $\ideal{d} \subseteq I$.
  Now let $f \in I$.
  By Euclidean division, there exists $q, r \in k[x]$ such that $f = qd + r$ where either $r = 0$ or $0 \leq \deg(r) \leq \deg(d) - 1$.
  If $r = 0$ then $f \in \ideal{d}$ and we are done.
  So suppose $r \neq 0$.
  Then $f, qd \in I \implies r = f - qd \in I$.
  Thus, $r \in I$ is of degree strictly less than $d$, contradicting the minimality of the degree of $d$.
  So we must have that $r = 0$ and overall $\ideal{d} = I$.
\end{pf}

\begin{exercise}[\tcite{book:UAG} 1.1.7]\label{ex:UAG_1.1.7}
  \begin{enumerate}[label= (\alph*)]
    \item Show that $\sqrt{\ideal{x^{n}}} = \ideal{x}$ in $k[x]$.
    \item If $p(x) = (x - a_{1})^{e_{1}} \cdots (x - a_{m})^{e_{m}}$, find $\sqrt{\ideal{p(x)}}$.
    \item Let $k = \C$.
          What are the radical ideals in $\sqrt{\C[x]}$?
  \end{enumerate}
\end{exercise}
\begin{pf}
  \begin{enumerate}[label= (\alph*)]
    \item Suppose $f(x) \in \ideal{x}$.
          Then $f(x)^{m} \in \ideal{x^{n}}$ so $f(x) \in \sqrt{\ideal{x^{n}}}$
          Now suppose that $f(x) \in \sqrt{\ideal{x^{n}}}$.
          Then $\exists k$ such that $f(x)^{k} \in \ideal{x^{n}}$.
          Thus $f(x)^{k}$ is a multiple of $x^{n}$.
          This implies that $f(x)^{k}$ is a multiple of $x$.
          Then notice that the unique factorization of $f(x)^{k}$ into irreducibles is the $k$th power of the factorization of $f(x)$ into irreducibles.
          Thus $x$ must be a factor of $f(x)$ and so $f(x) \in \ideal{x}$.
          Note, this heavily uses the fact that $k[x]$ is a unique factorization domain for all fields $k$.
    \item We claim that $\sqrt{\ideal{p(x)}} = \ideal{(x - a_{1}) \cdots (x - a_{m})} = I$.
          Suppose $f(x) \in I$.
          Let $k = \max{e_{1}, \ldots, e_{n}}$.
          Then $p(x) \mid f(x)^{k}$ so $f(x) \in \sqrt{\ideal{p(x)}}$.
          Now suppose that $f(x) \in \sqrt{\ideal{p(x)}}$.
          Then $\exists k$ such that $f(x)^{k} \in \ideal{p(x)}$.
          Thus $f(x)^{k}$ is a multiple of each $(x - a_{i})$.
          Then notice that the unique factorization of $f(x)^{k}$ into irreducibles is the $k$th power of the factorization of $f(x)$ into irreducibles.
          Thus $f(x)$ is a multiple of each $(x - a_{i})$ and so $f(x) \in I$.
    \item Radical ideals are the ideals $I$ such that $\sqrt{I} = I$.
          Notice that $\C[x]$ is a principal ideal domain and so any such $I$ must be generated by a single polynomial.
          Since every polynomial in $\C[x]$ splits into linear factors, (b) immediately implies that the only radical ideals of $\C[x]$ are the ones which are of the form $\ideal{(x - a_{1}) \cdots (x - a_{m})}$ for $a_{1}, \ldots, a_{m} \in \C$.
  \end{enumerate}
\end{pf}

\printbibliography
\end{document}
