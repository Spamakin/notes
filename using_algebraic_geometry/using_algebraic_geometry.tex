\documentclass[letterpaper, 11pt, oneside]{book}

\usepackage{style}  % If you feel like procrastinating, mess with this file
\usepackage{algo}   % Thank you Jeff, very cool!

\addbibresource{refs.bib}

% Required reading
% https://jmlr.csail.mit.edu/reviewing-papers/knuth_mathematical_writing.pdf
%   Along with required viewing:
%   https://www.youtube.com/watch?v=N6QEgbPWUrg&list=PLOdeqCXq1tXihn5KmyB2YTOqgxaUkcNYG
% https://faculty.math.illinois.edu/~west/grammar.html

% % % % % % % % % %
%     Cursor      %
%     Parking     %
%     Lot         %
% % % % % % % % % %

% Disable check for mismatched parens/brackets/braces
%   chktex-file 9
% Disable check for different counts of parents/brackets/braces
%   chktex-file 17
% Exclude these environments from syntax checking
%   VerbEnvir { tikzcd }

\regtotcounter{figure}

\title{\vspace{-100pt} {\Huge Using Algebraic Geometry} \\ {\small With $\total{figure}$ Figures}}
\author{\Large Anakin Dey}
\DTMsavenow{now}
\date{\small Last Edited on \today\ at \DTMfetchhour{now}:\DTMfetchminute{now}}

% Cover page number chicanery
\newcommand{\CoverName}{Cover}

\begin{document}
\frontmatter
\renewcommand{\thepage}{\CoverName}
\maketitle

\pagenumbering{roman}

\tableofcontents
\clearpage


% \listoftheorems[ignoreall, show={defn}, title={List of Definitions}]
%
% \listoftheorems[ignoreall, show={ex}, title={List of Examples and Counterexamples}]

\chapter*{Preface}

At the time of writing this, I am starting my PhD at The Ohio State University.
Currently a large part of my interests in algebra are about algorithms as they relate to polynomials and algebraic geometry.
I've been doing a bunch of problems from \emph{Ideals, Varieties, and Algorithms}~\cite{book:IVA}.
However,  it seems that \emph{Using Algebraic Geometry}~\cite{book:UAG} moves through the material faster as it assumes you know more algebra.
So I've moved onto working through this book as well as trying to comprehend Sturmfel's \emph{Algorithms in Invariant Theory}~\cite{book:AlgosInInvTheory}.

\mainmatter

\chapter{Introduction}

\section{Polynomials and Ideals}

\begin{exercise}[\tcite{book:UAG} 1.1.1]\label{ex:UAG_1.1.1}
  \begin{enumerate}[label= (\alph*)]
    \item Show that $x^{2} \in \ideal{x - y^{2}, xy}$ in $k[x, y]$.
    \item Show that $\ideal{x - y^{2}, xy, y^{2}} = \ideal{x, y^{2}}$.
    \item Is $\ideal{x - y^{2}, xy} = \ideal{x^{2}, xy}$? Why or why not?
  \end{enumerate}
\end{exercise}

\begin{pf}
  \begin{enumerate}[label= (\alph*)]
    \item We have that $x(x - y^{2}) + y(xy) = x^{2} - xy^{2} + xy^{2} = x^{2}$.
    \item It suffices to check for generators.
          We have that $x + (-1)(y^{2}) = x - y^{2}, y(x) = xy$, and $y^{2} = y^{2}$ showing that $\ideal{x - y^{2}, xy, y^{2}} \subseteq \ideal{x, y^{2}}$.
          Then $x - y^{2} + y^{2} = x$ and $y^{2} = y^{2}$ shows the reverse containment and overall the ideals are equal.
    \item We already know from 1.\ that $x^{2}$ lives in $\ideal{x - y^{2}, xy}$.
          Since $xy = xy$, we overall have that $\ideal{x^{2}, xy} \subseteq \ideal{x - y^{2}, xy}$.
          It remains to check if $x - y^{2} \in \ideal{x^{2}, xy}$.
          However, notice that every element of $\ideal{x^{2}, xy}$ is divisible by $x$ while $x - y^{2}$ is clearly not divisible by $x$.
          Thus $x - y^{2} \notin \ideal{x^{2}, xy}$ and the two ideals are not equal.
  \end{enumerate}
\end{pf}

\clearpage

\begin{exercise}[\tcite{book:UAG} 1.1.2]\label{ex:UAG_1.1.2}
  Show that $\ideal{f_{1}, \ldots, f_{s}}$ is closed under sums in $k[x_{1}, \ldots, x_{n}]$.
  Also show that if $f \in \ideal{f_{1}, \ldots, f_{s}}$ and $p \in k[x_{1}, \ldots, x_{n}]$ then $p \cdot f \in \ideal{f_{1}, \ldots, f_{s}}$.
\end{exercise}
\begin{pf}
  Let $f, g \in \ideal{f_{1}, \ldots, f_{s}}$.
  Then $\exists p_{1}, \ldots, p_{s}, q_{1}, \ldots, q_{s}$ such that $f = \sum_{i = 1}^{s} p_{i} \cdot f_{i}$ and $g = \sum_{i = 1}^{s} q_{i} \cdot f_{i}$.
  Thus $f + g = \sum_{i = 1}^{s} (p_{i} + q_{i}) \cdot f_{i}$ which shows that $f + g \in \ideal{f_{1}, \ldots, f_{s}}$.
  Then let $p \in k[x_{1}, \ldots, x_{n}]$.
  We have that $p \cdot f = p\sum_{i = 1}^{s} p_{i} f_{i} = \sum_{i = 1}^{s} (p \cdot p_{i}) \cdot f_{i}$ which shows that $\ideal{f_{1}, \ldots, f_{s}}$ is an ideal.
\end{pf}

\begin{exercise}[\tcite{book:UAG} 1.1.3]\label{ex:UAG_1.1.3}
  Show that $\ideal{f_{1}, \ldots, f_{s}}$ is the smallest ideal containing $\set{f_{1}, \ldots, f_{s}}$.
\end{exercise}
\begin{pf}
  We already know that $\ideal{f_{1}, \ldots, f_{s}}$ is an ideal by \Cref{ex:UAG_1.1.2}.
  Now suppose that $J$ is an ideal containing $\set{f_{1}, \ldots, f_{s}}$.
  Then, since ideals are closed under addition and scaling, we have that for all $p_{1}, \ldots, p_{s} \in k[x_{1}, \ldots, x_{n}]$ that $\sum_{i = 1}^{s} p_{i} \cdot f_{i} \in J$.
  Thus, $\ideal{f_{1}, \ldots, f_{s}} \subseteq J$.
\end{pf}

\begin{exercise}[\tcite{book:UAG} 1.1.4]\label{ex:UAG_1.1.4}
  Using \Cref{ex:UAG_1.1.3}, formulate and prove a general criterion for the equality of $I = \ideal{f_{1}, \ldots, f_{s}}$ and $J = \ideal{g_{1}, \ldots, g_{t}}$.
\end{exercise}
\begin{pf}
  We claim that $\ideal{f_{1}, \ldots, f_{s}} = \ideal{g_{1}, \ldots, g_{t}}$ if and only if $\set{g_{1}, \ldots, g_{t}} \subseteq I$ and $\set{f_{1}, \ldots, f_{s}} \subseteq J$.
  The forward implication is immediate.
  Then by \Cref{ex:UAG_1.1.3}, if $\set{g_{1}, \ldots, g_{t}} \subseteq I$ then $J \subseteq I$.
  Similarly, $\set{f_{1}, \ldots, f_{s}} \subseteq J \implies I \subseteq J$ and overall $I = J$.
  This fact was used in \Cref{ex:UAG_1.1.1} (b).
\end{pf}

\begin{exercise}[\tcite{book:UAG} 1.1.5]\label{ex:UAG_1.1.5}
  Show that $\ideal{y - x^{2}, z - x^{3}} = \ideal{y - x^{2}, z - xy}$ in $\Q[x, y, z]$.
\end{exercise}
\begin{pf}
  It suffices to show that $z - x^{3} \in \ideal{y - x^{2}, z - xy}$ and and $z - xy \in \ideal{x - y^{2}, z - x^{3}}$.
  Indeed we have that $(z - xy) + x(y - x^{2}) = z - x^{3}$ which also yields that $z - xy = z - x^{3} - x(y - x^{2})$.
\end{pf}

\clearpage

\begin{exercise}[\tcite{book:UAG} 1.1.6]\label{ex:UAG_1.1.6}
  Show that every ideal $I \subseteq k[x]$ is generated by a single polynomial.
\end{exercise}
\begin{pf}
  If $I = \set{0}$ then $I = \ideal{0}$.
  So suppose $I \neq 0$.
  Let $d \in I$ be of minimal degree.
  \quest{$d = \gcd(I)$ but I need infinite Bezout.}
  Then we claim that $\ideal{d} = I$.
  Since $d \in I$, we have that $\ideal{d} \subseteq I$.
  Now let $f \in I$.
  By Euclidean division, there exists $q, r \in k[x]$ such that $f = qd + r$ where either $r = 0$ or $0 \leq \deg(r) \leq \deg(d) - 1$.
  If $r = 0$ then $f \in \ideal{d}$ and we are done.
  So suppose $r \neq 0$.
  Then $f, qd \in I \implies r = f - qd \in I$.
  Thus, $r \in I$ is of degree strictly less than $d$, contradicting the minimality of the degree of $d$.
  So we must have that $r = 0$ and overall $\ideal{d} = I$.
\end{pf}

\begin{exercise}[\tcite{book:UAG} 1.1.7]\label{ex:UAG_1.1.7}
  \begin{enumerate}[label= (\alph*)]
    \item Show that $\sqrt{\ideal{x^{n}}} = \ideal{x}$ in $k[x]$.
    \item If $p(x) = (x - a_{1})^{e_{1}} \cdots (x - a_{m})^{e_{m}}$, find $\sqrt{\ideal{p(x)}}$.
    \item Let $k = \C$.
          What are the radical ideals in $\sqrt{\C[x]}$?
  \end{enumerate}
\end{exercise}
\begin{pf}
  \begin{enumerate}[label= (\alph*)]
    \item Suppose $f(x) \in \ideal{x}$.
          Then $f(x)^{m} \in \ideal{x^{n}}$ so $f(x) \in \sqrt{\ideal{x^{n}}}$
          Now suppose that $f(x) \in \sqrt{\ideal{x^{n}}}$.
          Then $\exists k$ such that $f(x)^{k} \in \ideal{x^{n}}$.
          Thus $f(x)^{k}$ is a multiple of $x^{n}$.
          This implies that $f(x)^{k}$ is a multiple of $x$.
          Then notice that the unique factorization of $f(x)^{k}$ into irreducibles is the $k$th power of the factorization of $f(x)$ into irreducibles.
          Thus $x$ must be a factor of $f(x)$ and so $f(x) \in \ideal{x}$.
          Note, this heavily uses the fact that $k[x]$ is a unique factorization domain for all fields $k$.
    \item We claim that $\sqrt{\ideal{p(x)}} = \ideal{(x - a_{1}) \cdots (x - a_{m})} = I$.
          Suppose $f(x) \in I$.
          Let $k = \max{e_{1}, \ldots, e_{n}}$.
          Then $p(x) \mid f(x)^{k}$ so $f(x) \in \sqrt{\ideal{p(x)}}$.
          Now suppose that $f(x) \in \sqrt{\ideal{p(x)}}$.
          Then $\exists k$ such that $f(x)^{k} \in \ideal{p(x)}$.
          Thus $f(x)^{k}$ is a multiple of each $(x - a_{i})$.
          Then notice that the unique factorization of $f(x)^{k}$ into irreducibles is the $k$th power of the factorization of $f(x)$ into irreducibles.
          Thus $f(x)$ is a multiple of each $(x - a_{i})$ and so $f(x) \in I$.
    \item Radical ideals are the ideals $I$ such that $\sqrt{I} = I$.
          Notice that $\C[x]$ is a principal ideal domain and so any such $I$ must be generated by a single polynomial.
          Since every polynomial in $\C[x]$ splits into linear factors, (b) immediately implies that the only radical ideals of $\C[x]$ are the ones which are of the form $\ideal{(x - a_{1}) \cdots (x - a_{m})}$ for $a_{1}, \ldots, a_{m} \in \C[x]$.
  \end{enumerate}
\end{pf}

\begin{exercise}[\tcite{book:UAG} 1.1.8]\label{ex:UAG_1.1.8}
  \begin{enumerate}[label= (\alph*)]
    \item Show that a prime ideal is radical.
    \item What are the prime ideals in $\C[x]$?
          What about the prime ideals in $\R[x]$ or $\Q[x]$?
  \end{enumerate}
\end{exercise}
\begin{pf}
  \begin{enumerate}[label= (\alph*)]
    \item Let $\mathfrak{p}$ be a prime ideal in $k[\overline{x}]$.
          Clearly $\mathfrak{p} \subseteq \sqrt{\mathfrak{p}}$ always.
          Let $f(\overline{x}) \in \sqrt{\mathfrak{p}}$.
          Then $f(\overline{x})^{m} \in \mathfrak{p}$ for some $m \in \Z_{\geq 1}$.
          We prove the reverse inclusion by induction on $m$.
          If $m = 1$ then $f(\overline{x}) =f(\overline{x})^{1} \in \mathfrak{p}$.
          Now let $m > 1$ and suppose the claim holds for all $k \leq m$.
          Then suppose $f(\overline{x})^{m + 1} \in \mathfrak{p}$.
          Then $f(\overline{x}) \cdot f(\overline{x})^{m} \in \mathfrak{p}$/
          Either $f(\overline{x}) \in \mathfrak{p}$ or $f(\overline{x})^{m} \in \mathfrak{p}$ which by induction implies that $f(\overline{x}) \in \mathfrak{p}$.
          Thus, $f(\overline{x})^{m} \in \mathfrak{p} \implies f(\overline{x}) \in \mathfrak{p}$ for all $m \in \Z_{\geq 1}$ and so $\sqrt{\mathfrak{p}} \subseteq \mathfrak{p}$.
          Thus, all prime ideals are radical.
    \item Notice that for all fields $k$ that $k[x]$ is a principal ideal domain.
          Thus, all the prime ideals are the ones generated by a single irreducible polynomial.
          Also, in $k[x]$ we have that $(0)$ is a prime ideal as well as $k[x]$ is an integral domain.
          In $\C[x]$, these are the ideals generated by $x - z$ for some $z \in \C$.
          In $\R[x]$, the primes are the ideals generated by $x - r$ for some $r \in \R$ or $x^{2} + r$ for some positive $r \in R$.
          \quest{What would be a general condition for $\Q[x]$?}
  \end{enumerate}
\end{pf}

\clearpage

\begin{exercise}[\tcite{book:UAG} 1.1.9]\label{ex:UAG_1.1.9}
  \begin{enumerate}[label= (\alph*)]
    \item Show that $\ideal{x_{1}, \ldots, x_{n}}$ is maximal in $k[x_{1}, \ldots, x_{n}]$.
    \item Show that for any point $(a_{1}, \ldots, a_{n}) \in k^{n}$ that $\ideal{x_{1} - a_{1}, \ldots, x_{n} - a_{n}}$ is maximal in $k[x_{1}, \ldots, x_{n}]$.
    \item Show that $\ideal{x^{2} + 1}$ is maximal in $\R[x]$.
          Is $\ideal{x^{2} + 1}$ maximal in $\C[x]$?
  \end{enumerate}
\end{exercise}
\begin{pf}
  \begin{enumerate}[label= (\alph*)]
    \item First, observe that $\ideal{x_{1}, \ldots, x_{n}}$ is the ideal consisting exactly of polynomials which have no constant term.
          Let $I$ be an ideal in $k[x_{1}, \ldots, x_{n}]$ such that $\ideal{x_{1}, \ldots, x_{n}} \subsetneq I$.
          Thus there exists $f(x_{1}, \ldots, x_{n}) \in I \setminus \ideal{x_{1}, \ldots, x_{n}}$.
          We have by our observation that $f$ has a nonzero constant term $z$.
          Then note that the non-constant terms of $f$ form a polynomial $g(x_{1}, \ldots, x_{n})$ in $\ideal{x_{1}, \ldots, x_{n}}$.
          Thus, we have that $z = f(x) - g(x) \in I$.
          Since $I$ contains a nonzero constant term, we must have that $I = k[x_{1}, \ldots, x_{n}]$.
    \item Recall that an ideal $I$ is maximal if and only if $R/I$ is a field.
          Let $I = \ideal{x_{1} - a_{1}, \ldots, x_{n} - a_{n}}$.
          Consider the evaluation map $\text{ev}_{\overline{a}}\colon k[x_{1}, \ldots, x_{n}] \to k$ sending $f(x_{1}, \ldots, x_{n}) \mapsto f(a_{1}, \ldots, a_{n})$.
          Clearly this map is surjective.
          Then since for all $i$ we have that $x_{i} \equiv a_{i} \pmod{I}$, we have that $f(x_{1}, \ldots, x_{n}) \equiv f(a_{1}, \ldots, a_{n}) \pmod{I}$ for all $f(x_{1}, \ldots, x_{n}) \in k[x_{1}, \ldots, x_{n}]$.
          Thus, $\text{ev}_{\overline{a}}(f) = f(a_{1}, \ldots, a_{n}) = 0$ if and only if $f(x_{1}, \ldots, x_{n}) \in I$.
          Thus, $\ker(\text{ev}_{\overline{a}}) = I$ and $k[x_{1}, \ldots, x_{n}] / I$ is a field, meaning $\ideal{x_{1} - a_{1}, \ldots, x_{n} - a_{n}}$ is maximal.
    \item Since $\R[x]$ is a principal ideal domain, any ideal $I$ strictly containing $\ideal{x^{2} + 1}$ is of the form $\ideal{g(x)}$ for some $g(x) \mid x^{2} + 1$.
          However, since $x^{2} + 1$ is irreducible in $\R[x]$, we have that $g(x)$ is either $z(x^{2} + 1)$ for some nonzero $z \in \C$ or $g(x) = z$ for some nonzero $z \in \C$, meaning $\ideal{g(x)} = \ideal{x^{2} + 1}$ or or $\ideal{g(x)} = \R[x]$.
          Thus, $\ideal{x^{2} + 1}$ is maximal.
          However, in $\C[x]$, we have that $x^{2} + 1 = (x + i)(x - i)$ and so $\ideal{x^{2} + 1} \subsetneq \ideal{x - i} \subsetneq \C[x]$.
  \end{enumerate}
\end{pf}

\clearpage

\begin{exercise}[\tcite{book:UAG} 1.1.10]\label{ex:UAG_1.1.10}
  \begin{enumerate}[label= (\alph*)]
    \item Let $I = \ideal{x^{2} + y^{2}, x^{2} - z^{3}} \subseteq k[x, y, z]$.
          Show that $y^{2} + z^{3}$ is in the first elimination ideal with respect to the ordering $x > y > z$.
    \item Show that if $I$ is an ideal in $k[x_{1}, \ldots, x_{n}]$ then for all $\ell \geq 1$, $I_{\ell}$ is an ideal in $k[x_{\ell + 1}, \ldots, x_{n}]$.
  \end{enumerate}
\end{exercise}
\begin{pf}
  \begin{enumerate}[label= (\alph*)]
    \item Since $x^{2} + y^{2} - (x^{2} - z^{3}) = y^{2} + z^{3}$ is an element of $I$ which does not depend on $x$, $y^{2} + z^{3}$ is in $I_{1}$.
    \item For all $\ell \geq 1$, we have that $0 \in I_{\ell}$.
          Then, if $f(x_{\ell + 1}, \ldots, x_{n}), g(x_{\ell + 1}, \ldots, x_{n})$ are two polynomials in $I$ who do not depend on the first $\ell$ variables, then so is $f + g$.
          Finally, let $r(x_{\ell} + 1, \ldots, x_{n}) \in k[x_{\ell + 1}, \ldots, x_{n}]$.
          Then $r \cdot f \in I_{\ell}$ since $r \cdot f \in I$ and still does not depend on any of the first $\ell$ variables.
  \end{enumerate}
\end{pf}

\begin{exercise}[\tcite{book:UAG} 1.1.11]\label{ex:UAG_1.1.11}
  Let $I, J$ be ideals in $k[\overline{x}]$.
  \begin{enumerate}[label= (\alph*)]
    \item Show that $I + J$ is an ideal.
    \item Show that $I + J$ is the smallest ideal containing $I \cup J$.
    \item If $I = \ideal{f_{1}, \ldots, f_{s}}$ and $J = \ideal{g_{1}, \ldots, g_{t}}$, what is a finite generating set of $I + J$?
  \end{enumerate}
\end{exercise}
\begin{pf}
  \begin{enumerate}[label= (\alph*)]
    \item \quest{meh}
    \item \quest{meh}
    \item We claim that $I + J = \ideal{f_{1}, \ldots, f_{s}, g_{1}, \ldots, g_{t}}$.
          Clearly $I, J \subseteq \ideal{f_{1}, \ldots, f_{s}, g_{1}, \ldots, g_{t}}$ and thus so is $I \cup J$.
          By (b), this shows that $I + J \subseteq \ideal{f_{1}, \ldots, f_{s}, g_{1}, \ldots, g_{t}}$.
          Then, since $f_{i} = f_{i} + 0$ and $g_{j} = 0 + g_{j}$ for all $i, j$, we have the reverse inclusion and thus the two ideals are equal.
  \end{enumerate}
\end{pf}

\clearpage

\begin{exercise}[\tcite{book:UAG} 1.1.12]\label{ex:UAG_1.1.12}
  Let $I, J$ be ideals in $k[\overline{x}]$.
  \begin{enumerate}[label= (\alph*)]
    \item Show that $I \cap J$ is an ideal.
    \item Show that $IJ \subseteq I \cap J$.
          Give an example where $IJ \subsetneq I \cap J$.
  \end{enumerate}
\end{exercise}
\begin{pf}
  \begin{enumerate}[label= (\alph*)]
    \item \quest{meh}
    \item Suppose that $h(\overline{x}) \in IJ$.
          Note that $IJ$ is generated by the products $f(\overline{x}) \cdot g(\overline{x})$ for $f(\overline{x}) \in I$, and $g(\overline{x}) \in J$.
          Then $h(\overline{x})$ consists of sums of terms of the form $r(\overline{x}) \cdot f(\overline{x}) \cdot g(\overline{x})$ for $r(\overline{x}) \in k[\overline{x}]$, $f(\overline{x}) \in I$, and $g(\overline{x}) \in J$.
          Thus, each term is in both $I$ and $J$ and overall so is $h(\overline{x})$.

          To see an example where $IJ \subsetneq I \cap J$, consider $I = \ideal{x^{2}y}$ and $J = \ideal{xy^{2}}$ in $k[x, y]$.
          Then $I \cap J = \gen{x^{2}y^{2}}$ and $IJ = \gen{x^{3}y^{3}}$.
          Thus $IJ \subsetneq I \cap J$ as $I \cap J$ contains $x^{2}y^{2}$ and $IJ$ does not contain $x^{2}y^{2}$.
  \end{enumerate}
\end{pf}

\chapter{Solving Polynomial Equations}

\section{Solving Polynomial Systems by Elimination}

\begin{exercise}[\quest{\tcite{book:UAG} 2.1.1}]\label{ex:UAG_2.1.1}

\end{exercise}

\begin{exercise}[\quest{\tcite{book:UAG} 2.1.2}]\label{ex:UAG_2.1.2}

\end{exercise}

\begin{exercise}[\tcite{book:UAG} 2.1.3]\label{ex:UAG_2.1.3}
  Suppose $p(z) = z^{n} + a_{n - 1}z^{n - 1} + \cdots + a_{0}$ is a monic polynomial in $\C[z]$.
  Then all roots $\overline{z}$ of $p(z)$ satisfy
  \[
    \overline{z} \leq B \defeq \max\set{1, \abs{a_{n - 1}} + \cdots + \abs{a_{0}}}.
  \]
\end{exercise}
\begin{pf}
  We may freely rewrite the polynomial as $p(z) = z^{n} - a_{n - 1}z^{n - 1} - \cdots - a_{0}$
  We have that $0 = \overline{z}^{n} - a_{n - 1}\overline{z}^{n - 1} - \cdots - a_{0}$ and so $\overline{z}^{n} = a_{n - 1}\overline{z}^{n - 1} + \cdots + a_{0}$.
  Suppose now that $\abs{\overline{z}} \geq 1$.
  Then
  \[
    \abs{\overline{z}}^{n} = \abs{a_{n - 1}\overline{z}^{n - 1} + \cdots + a_{0}} \leq \abs{a_{n - 1}}\abs{z}^{n - 1} + \cdots + a_{0} \leq \abs{a_{n - 1}}\overline{z}^{n - 1} + \cdots \abs{a_{0}}\overline{z}^{n - 1}.
  \]
  Thus, $\abs{\overline{z}} \leq \abs{a_{n - 1}} + \cdots + \abs{a_{0}}$.
  However, we assumed that $\abs{\overline{z}} \geq 1$.
  This may not be the case.
  Thus, $\abs{\overline{z}} \leq B \defeq \max\set{1, \abs{a_{n - 1}} + \cdots + \abs{a_{0}}}$.
\end{pf}

\begin{exercise}[\quest{\tcite{book:UAG} 2.1.4}]\label{ex:UAG_2.1.4}
  Numerically find all roots of $2z^{6} + 2z^{5} - z^{4} - z^{3} - 2z^{2} - 2z - 2$.
\end{exercise}

\clearpage

\begin{exercise}[\tcite{book:UAG} 2.1.5]\label{ex:UAG_2.1.5}
  Verify that if $x > y$ then $G = \bqty{x^{2} + 2x + 3 + y^{5} - y, y^{6} - y^{2} + 2y}$ is a lex \Grobner\ basis for the ideal that $G$ generates in $\mathbb{R}[x, y]$
\end{exercise}
\begin{pf}
  We apply Buchberger's Criterion.
  Let $f(x, y) = x^{2} + 2x + 3 + y^{5} - y$ and $g(x, y) = y^{6} - y^{2} + 2y$.
  Then we have that
  \[
    S(f, g) = \frac{x^{2}y^{6}}{x^{2}} \cdot (x^{2} + 2x + 3 + y^{5} - y) - \frac{x^{2}y^{6}}{y^{6}} \cdot (y^{6} - y^{2} + 2y) = y^{6} \cdot (x^{2} + 2x + 3 + y^{5} - y) - x^{2} \cdot (y^{6} - y^{2} + 2y).
  \]
  This shows that $\overline{S(f, g)}^{G} = 0$ which yields that $G$ is a \Grobner\ basis.
\end{pf}

\begin{exercise}[\quest{\tcite{book:UAG} 2.1.6}]\label{ex:UAG_2.1.6}
\end{exercise}

\begin{exercise}[\quest{\tcite{book:UAG} 2.1.7}]\label{ex:UAG_2.1.7}
\end{exercise}

\begin{exercise}[\tcite{book:UAG} 2.1.8]\label{ex:UAG_2.1.8}
  Newton's method for an equation $p(z) = 0$ is the sequence of points $\set{z_{k}}_{k \geq 0}$ starting from a chosen $z_{0}$ and defining $z_{k + 1} = N_{p}(z_{k})$ for $N_{p}(z) = z - \frac{p(z)}{p'(z)}$.
  \begin{enumerate}[label= (\alph*)]
    \item Prove that a simple root of a polynomial $p(z)$ is a fixed point of $N_{p}(z)$.
    \item Show that multiple roots of $p(z)$ are removable singularities of $N_{p}(z)$.
          That is, show that $\abs{N_{p}(z)}$ is bounded in a neighborhood of each multiple root.
          How should $N_{p}(z)$ be defined at a multiple root of $p(z)$ to make $N_{p}(z)$ continuous.
    \item Show that $N_{p}'(\overline{z}) = 0$ if $\overline{z}$ is a simple root, meaning that $p(\overline{z}) = 0$ and $p'(\overline{z}) \neq 0$.
    \item Show that if $\overline{z}$ is a root of multiplicity $k$ of $p(z)$, meaning $p(overline{z}) = p'(\overline{z}) = \cdots = p^{(k - 1)}(\overline{z}) = 0$ and $p^{(k)}(\overline{z}) \neq 0$, then
          \[
            \lim_{z \to \overline{z}} N_{p}'(z) = 1 - \frac{1}{k}.
          \]
    \item Show that by replacing $p(z)$ with
          \[
            p_{\emph{red}}(z) = \frac{p(z)}{\gcd{p(z), p'(z)}}
          \]
          that the difficulty in (d) is eliminated as all roots of $p_{\red}(z)$ are simple.
  \end{enumerate}
\end{exercise}
\clearpage
\begin{pf}
  \begin{enumerate}[label= (\alph*)]
    \item Let $\overline{z}$ be a simple root of $p(z)$, so $p(z) = 0$ but $p'(z) \neq 0$.
          Then $N_{p}(\overline{z}) = \overline{z} - \frac{p(\overline{z})}{p'(\overline{z})} = \overline{z}$ meaning $\overline{z}$ is a fixed point of $N_{p}(z)$.
    \item Suppose that $\overline{z}$ is a multiple root of $p(z)$ with multiplicity $m \geq 2$.
          Then we may express $p(z) = \tilde(p)(z)(z - \overline{z})^{m}$ such that $\tilde{p}(\overline{z}) \neq 0$.
          Thus, we have that
          \begin{align*}
            N_{p}(z) &\defeq z - \frac{p(z)}{p'(z)} \\
            &= z - \frac{\tilde{p}(z)(z - \overline{z})^{m}}{\tilde{p}'(z)(z - \overline{z})^{m} + m\tilde{p}(z)(z - \overline{z})^{m - 1}} = z - \frac{\tilde{p}(z)(z - \overline{z})}{\tilde{p}'(z)(z - \overline{z}) + m\tilde{p}(z)}
          \end{align*}
          Note that $m\tilde{p}(\overline{z}) \neq 0$.
          Thus, we have that
          \[
            \abs{N_{p}(\overline{z})} = \abs{\overline{z} - \frac{\tilde{p}(\overline{z})(\overline{z} - \overline{z})}{\tilde{p}'(\overline{z})(\overline{z} - \overline{z}) + m\tilde{p}(\overline{z})}} = \abs{\overline{z}} \leq \lc(p) \cdot B
          \]
          where $B$ is the value from~\Cref{ex:UAG_2.1.3} and $\lc(p)$ is the leading coefficient of $p(z)$.
    \item Suppose now that $\overline{z}$ is a simple root of $p(\overline{z})$.
          Then we may express $p(z) = \tilde{p}(z)(z - \overline{z})$ such that $\tilde{p}(\overline{z}) \neq 0$.
          We have that
          \[
            p'(z) = \tilde{p}'(z)(z - \overline{z}) + \tilde{p}(z)
          \]
          and evaluation of $p'(z)$ at $\overline{z}$ is nonzero.
    \item Let $\overline{z}$ be a root of multiplicity $m$.
          Following (b), we write $p(z) = \tilde{p}(z)(z - \overline{z})^{m}$ such that $\tilde{p}(\overline{z}) \neq 0$.
          Then we have, by differentiating the expression for $N_{p}(z)$ from (b), that
          \[
            N'_{p}(z) = 1 - \frac{(\tilde{p}'(z)(z - \overline{z}) + \tilde{p}(z))(\tilde{p}'(z)(z - \overline{z}) + m\tilde{p}(z)) - (\tilde{p}(z)(z - \tilde{z}))(\tilde{p}''(z)(z - \overline{z}) + \tilde{p}'(z) + m\tilde{p}'(z))}{(\tilde{p}'(z)(z - \overline{z}) + m\tilde{p}(z))^{2}}.
          \]
          Evaluation at $z = \overline{z}$ yields that $\lim_{z \to \overline{z}} N_{p}'(z) = 1 - \frac{1}{m}$.
    \item Let $\overline{z}$ be a root of multiplicity $m$.
          Following (b), we write $p(z) = \tilde{p}(z)(z - \overline{z})^{m}$ such that $\tilde{p}(\overline{z}) \neq 0$.
          Then
          \[
            p'(z) = \tilde{p}'(z - \overline{z})^{m} + m\tilde{p}(z)(z - \overline{z})^{m - 1} = (z - \overline{z})^{m - 1}(\tilde{p}'(z)(z - \overline{z}) + m\tilde{p}(z)).
          \]
          Notice that $\tilde{p}'(\overline{z})(\overline{z} - \overline{z}) + m\tilde{p}(\overline{z}) = m\tilde{p}(\overline{z}) \neq 0$.
          Thus, a root of multiplicity $m \geq 1$ of $p(z)$ is a root of multiplicity $m - 1$ of $p'(z)$.
          This implies that if we have roots $\overline{z}_{1}, \ldots, \overline{z}_{k}$ with multiplicities $m_{1}, \ldots, m_{k} \geq 1$, then $\gcd(p(z), p'(z)) = (z - \overline{z}_{1})^{m_{1}} \cdots (z - \overline{z}_{k})^{m_{k}}$.
          Thus, the polynomial $p_{\mathrm{red}}(z) = \frac{p(z)}{\gcd(p(z), p'(z))}$ has the same roots of $p(z)$ but all with multiplicity $1$ which is the best case for Newton's method.
  \end{enumerate}
\end{pf}

\begin{exercise}[\tcite{book:UAG} 2.1.9]\label{ex:UAG_2.1.9}
  \begin{enumerate}[label= (\alph*)]
    \item What happens if you do Newton's method to solve $z^{2} + 1 = 0$ starting from a real $z_{0}$ versus a complex $z_{0}$?
    \item Let $p(z) = z^{4} - z^{2} - \frac{11}{36}$.
          Let $N_{p}(z) = z - \frac{p(z)}{p'(z)}$.
          Show that $N_{p}\pqty{\pm\frac{1}{\sqrt{6}}} = \mp \frac{1}{\sqrt{6}}$ and $N_{p}'\pqty{\frac{1}{\sqrt{6}}} = 0$.
  \end{enumerate}
\end{exercise}
\begin{pf}
  \begin{enumerate}[label= (\alph*)]
    \item Let $p(z) = z^{2} + 1$.
          We have that
          \[
            N_{p}(z) = z - \frac{z^{2} + 1}{2z} = \frac{2z^{2} - z^{2} + 1}{2z} = \frac{z^{2} + 1}{2z} = \frac{x^{2} + 2ixy - y^{2} + 1}{2x + 2iy}.
          \]
          If $z$ is real then $y = 0$ and so $N_{p}(x) = \frac{x^{2} + 1}{2x}$ which is always real.
          Thus, Newton's method will never reach the imaginary roots of $z^{2} + 1$.
          However, if we begin with a guess with nonzero imaginary part, then the guess does converge as expected.
    \item \quest{Just basic arithmetic not worth doing.}
  \end{enumerate}
\end{pf}

\begin{exercise}[\tcite{book:UAG} 2.1.10]\label{ex:UAG_2.1.10}
  Let $p(z) = z^{n} + a_{n - 1}z^{n - 1} + \cdots + a_{0}$ be a monic polynomial in $\C[z]$.
  Then all roots $\overline{z}$ of $p(z)$ satisfy
  \[
    \overline{z} \leq B \defeq 1 + \max\set{\abs{a_{n - 1}}, \ldots, \abs{a_{0}}}.
  \]
\end{exercise}
\begin{pf}
  Let $\overline{z}$ be a root of $p(z)$.
  Then $-\overline{z}^{n} = a_{n - 1}\overline{z}^{n - 1} + \cdots + a_{0}$ and so
  \begin{align*}
    \abs{\overline{z}}^{n} &= \abs{a_{n - 1}\overline{z}^{n - 1} + \cdots + a_{0}} \\
                           &\leq \max_{i}\set{\abs{a_{i}}} \cdot \abs{\overline{z}^{n - 1} + \cdots + 1} \\
                           &\leq \max_{i}\set{\abs{a_{i}}} \cdot \pqty{\abs{\overline{z}}^{n - 1} + \cdots + 1} \\
                           &= \max_{i}\set{\abs{a_{i}}} \cdot \frac{\abs{\overline{z}}^{n} + 1}{\abs{\overline{z}} - 1} \leq \max_{i}\set{\abs{a_{i}}} \cdot \frac{\abs{\overline{z}}^{n}}{\abs{\overline{z}} - 1}.
  \end{align*}
  Thus, $\abs{\overline{z}}^{n} \leq \max_{i}\set{\abs{a_{i}}} \cdot \frac{\abs{z}^{n}}{\abs{z} - 1}$ which implies that $\abs{z} - 1 \leq \max_{i}\set{\abs{a_{i}}}$.
  Thus, $\abs{z} \leq 1 + \max_{i}\set{\abs{a_{i}}}$.
\end{pf}

\printbibliography
\end{document}
