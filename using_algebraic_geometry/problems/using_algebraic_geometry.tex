\documentclass[letterpaper, 11pt, oneside]{book}

\usepackage{style}  % If you feel like procrastinating, mess with this file
\usepackage{algo}   % Thank you Jeff, very cool!

\addbibresource{refs.bib}

% Required reading
% https://jmlr.csail.mit.edu/reviewing-papers/knuth_mathematical_writing.pdf
%   Along with required viewing:
%   https://www.youtube.com/watch?v=N6QEgbPWUrg&list=PLOdeqCXq1tXihn5KmyB2YTOqgxaUkcNYG
% https://faculty.math.illinois.edu/~west/grammar.html

% % % % % % % % % %
%     Cursor      %
%     Parking     %
%     Lot         %
% % % % % % % % % %

% Disable check for mismatched parens/brackets/braces
%   chktex-file 9
% Disable check for different counts of parents/brackets/braces
%   chktex-file 17
% Exclude these environments from syntax checking
%   VerbEnvir { tikzcd }

\regtotcounter{figure}

\title{\vspace{-100pt} {\Huge Using Algebraic Geometry} \\ {\small With $\total{figure}$ Figures}}
\author{\Large Anakin Dey}
\DTMsavenow{now}
\date{\small Last Edited on \today\ at \DTMfetchhour{now}:\DTMfetchminute{now}}

% Cover page number chicanery
\newcommand{\CoverName}{Cover}

\begin{document}
\frontmatter
\renewcommand{\thepage}{\CoverName}
\maketitle

\pagenumbering{roman}

\tableofcontents
\clearpage


% \listoftheorems[ignoreall, show={defn}, title={List of Definitions}]
%
% \listoftheorems[ignoreall, show={ex}, title={List of Examples and Counterexamples}]

\chapter*{Preface}

At the time of writing this, I am starting my PhD at The Ohio State University.
Currently a large part of my interests in algebra are about algorithms as they relate to polynomials and algebraic geometry.
I've been doing a bunch of problems from \emph{Ideals, Varieties, and Algorithms}~\cite{book:IVA}.
However,  it seems that \emph{Using Algebraic Geometry}~\cite{book:UAG} Ex. moves through the material faster as it assumes you know more algebra.
So I've moved onto working through this book as well as trying to comprehend Sturmfel's \emph{Algorithms in Invariant Theory}~\cite{book:AlgosInInvTheory}.

\mainmatter

\chapter{Introduction}

\section{Polynomials and Ideals}

\begin{sol}[\cite{book:UAG} Ex. 1.1.1]\label{ex:UAG_1.1.1}
  \begin{enumerate}
    \item We have that $x(x - y^{2}) + y(xy) = x^{2} - xy^{2} + xy^{2} = x^{2}$.
    \item It suffices to check for generators.
          We have that $x + (-1)(y^{2}) = x - y^{2}, y(x) = xy$, and $y^{2} = y^{2}$ showing that $\ideal{x - y^{2}, xy, y^{2}} \subseteq \ideal{x, y^{2}}$.
          Then $x - y^{2} + y^{2} = x$ and $y^{2} = y^{2}$ shows the reverse containment and overall the ideals are equal.
    \item We already know from 1.\ that $x^{2}$ lives in $\ideal{x - y^{2}, xy}$.
          Since $xy = xy$, we overall have that $\ideal{x^{2}, xy} \subseteq \ideal{x - y^{2}, xy}$.
          It remains to check if $x - y^{2} \in \ideal{x^{2}, xy}$.
          However, notice that every element of $\ideal{x^{2}, xy}$ is divisible by $x$ while $x - y^{2}$ is clearly not divisible by $x$.
          Thus $x - y^{2} \notin \ideal{x^{2}, xy}$ and the two ideals are not equal.
  \end{enumerate}
\end{sol}

\begin{sol}[\cite{book:UAG} Ex. 1.1.2]\label{ex:UAG_1.1.2}
  Let $f, g \in \ideal{f_{1}, \ldots, f_{s}}$.
  Then $\exists p_{1}, \ldots, p_{s}, q_{1}, \ldots, q_{s}$ such that $f = \sum_{i = 1}^{s} p_{i} \cdot f_{i}$ and $g = \sum_{i = 1}^{s} q_{i} \cdot f_{i}$.
  Thus $f + g = \sum_{i = 1}^{s} (p_{i} + q_{i}) \cdot f_{i}$ which shows that $f + g \in \ideal{f_{1}, \ldots, f_{s}}$.
  Then let $p \in k[x_{1}, \ldots, x_{n}]$.
  We have that $p \cdot f = p\sum_{i = 1}^{s} p_{i} f_{i} = \sum_{i = 1}^{s} (p \cdot p_{i}) \cdot f_{i}$ which shows that $\ideal{f_{1}, \ldots, f_{s}}$ is an ideal.
\end{sol}

\begin{sol}[\cite{book:UAG} Ex. 1.1.3]\label{ex:UAG_1.1.3}
  We already know that $\ideal{f_{1}, \ldots, f_{s}}$ is an ideal by \nameref{ex:UAG_1.1.2}.
  Now suppose that $J$ is an ideal containing $\set{f_{1}, \ldots, f_{s}}$.
  Then, since ideals are closed under addition and scaling, we have that for all $p_{1}, \ldots, p_{s} \in k[x_{1}, \ldots, x_{n}]$ that $\sum_{i = 1}^{s} p_{i} \cdot f_{i} \in J$.
  Thus, $\ideal{f_{1}, \ldots, f_{s}} \subseteq J$.
\end{sol}

\begin{sol}[\cite{book:UAG} Ex. 1.1.4]\label{ex:UAG_1.1.4}
  We claim that $\ideal{f_{1}, \ldots, f_{s}} = \ideal{g_{1}, \ldots, g_{t}}$ if and only if $\set{g_{1}, \ldots, g_{t}} \subseteq I$ and $\set{f_{1}, \ldots, f_{s}} \subseteq J$.
  The forward implication is immediate.
  Then by \nameref{ex:UAG_1.1.3}, if $\set{g_{1}, \ldots, g_{t}} \subseteq I$ then $J \subseteq I$.
  Similarly, $\set{f_{1}, \ldots, f_{s}} \subseteq J \implies I \subseteq J$ and overall $I = J$.
  This fact was used in \nameref{ex:UAG_1.1.1} (b).
\end{sol}

\begin{sol}[\cite{book:UAG} Ex. 1.1.5]\label{ex:UAG_1.1.5}
  It suffices to show that $z - x^{3} \in \ideal{y - x^{2}, z - xy}$ and and $z - xy \in \ideal{x - y^{2}, z - x^{3}}$.
  Indeed we have that $(z - xy) + x(y - x^{2}) = z - x^{3}$ which also yields that $z - xy = z - x^{3} - x(y - x^{2})$.
\end{sol}

\begin{sol}[\cite{book:UAG} Ex. 1.1.6]\label{ex:UAG_1.1.6}
  If $I = \set{0}$ then $I = \ideal{0}$.
  So suppose $I \neq 0$.
  Let $d \in I$ be of minimal degree.
  \quest{$d = \gcd(I)$ but I need infinite Bezout.}
  Then we claim that $\ideal{d} = I$.
  Since $d \in I$, we have that $\ideal{d} \subseteq I$.
  Now let $f \in I$.
  By Euclidean division, there exists $q, r \in k[x]$ such that $f = qd + r$ where either $r = 0$ or $0 \leq \deg(r) \leq \deg(d) - 1$.
  If $r = 0$ then $f \in \ideal{d}$ and we are done.
  So suppose $r \neq 0$.
  Then $f, qd \in I \implies r = f - qd \in I$.
  Thus, $r \in I$ is of degree strictly less than $d$, contradicting the minimality of the degree of $d$.
  So we must have that $r = 0$ and overall $\ideal{d} = I$.
\end{sol}

\begin{sol}[\cite{book:UAG} Ex. 1.1.7]\label{ex:UAG_1.1.7}
  \begin{enumerate}
    \item Suppose $f(x) \in \ideal{x}$.
          Then $f(x)^{m} \in \ideal{x^{n}}$ so $f(x) \in \sqrt{\ideal{x^{n}}}$
          Now suppose that $f(x) \in \sqrt{\ideal{x^{n}}}$.
          Then $\exists k$ such that $f(x)^{k} \in \ideal{x^{n}}$.
          Thus $f(x)^{k}$ is a multiple of $x^{n}$.
          This implies that $f(x)^{k}$ is a multiple of $x$.
          Then notice that the unique factorization of $f(x)^{k}$ into irreducibles is the $k$th power of the factorization of $f(x)$ into irreducibles.
          Thus $x$ must be a factor of $f(x)$ and so $f(x) \in \ideal{x}$.
          Note, this heavily uses the fact that $k[x]$ is a unique factorization domain for all fields $k$.
    \item We claim that $\sqrt{\ideal{p(x)}} = \ideal{(x - a_{1}) \cdots (x - a_{m})} = I$.
          Suppose $f(x) \in I$.
          Let $k = \max{e_{1}, \ldots, e_{n}}$.
          Then $p(x) \mid f(x)^{k}$ so $f(x) \in \sqrt{\ideal{p(x)}}$.
          Now suppose that $f(x) \in \sqrt{\ideal{p(x)}}$.
          Then $\exists k$ such that $f(x)^{k} \in \ideal{p(x)}$.
          Thus $f(x)^{k}$ is a multiple of each $(x - a_{i})$.
          Then notice that the unique factorization of $f(x)^{k}$ into irreducibles is the $k$th power of the factorization of $f(x)$ into irreducibles.
          Thus $f(x)$ is a multiple of each $(x - a_{i})$ and so $f(x) \in I$.
    \item Radical ideals are the ideals $I$ such that $\sqrt{I} = I$.
          Notice that $\C[x]$ is a principal ideal domain and so any such $I$ must be generated by a single polynomial.
          Since every polynomial in $\C[x]$ splits into linear factors, (b) immediately implies that the only radical ideals of $\C[x]$ are the ones which are of the form $\ideal{(x - a_{1}) \cdots (x - a_{m})}$ for $a_{1}, \ldots, a_{m} \in \C[x]$.
  \end{enumerate}
\end{sol}

\clearpage

\begin{sol}[\cite{book:UAG} Ex. 1.1.8]\label{ex:UAG_1.1.8}
  \begin{enumerate}
    \item Let $\mathfrak{p}$ be a prime ideal in $k[\overline{x}]$.
          Clearly $\mathfrak{p} \subseteq \sqrt{\mathfrak{p}}$ always.
          Let $f(\overline{x}) \in \sqrt{\mathfrak{p}}$.
          Then $f(\overline{x})^{m} \in \mathfrak{p}$ for some $m \in \Z_{\geq 1}$.
          We prove the reverse inclusion by induction on $m$.
          If $m = 1$ then $f(\overline{x}) =f(\overline{x})^{1} \in \mathfrak{p}$.
          Now let $m > 1$ and suppose the claim holds for all $k \leq m$.
          Then suppose $f(\overline{x})^{m + 1} \in \mathfrak{p}$.
          Then $f(\overline{x}) \cdot f(\overline{x})^{m} \in \mathfrak{p}$/
          Either $f(\overline{x}) \in \mathfrak{p}$ or $f(\overline{x})^{m} \in \mathfrak{p}$ which by induction implies that $f(\overline{x}) \in \mathfrak{p}$.
          Thus, $f(\overline{x})^{m} \in \mathfrak{p} \implies f(\overline{x}) \in \mathfrak{p}$ for all $m \in \Z_{\geq 1}$ and so $\sqrt{\mathfrak{p}} \subseteq \mathfrak{p}$.
          Thus, all prime ideals are radical.
    \item Notice that for all fields $k$ that $k[x]$ is a principal ideal domain.
          Thus, all the prime ideals are the ones generated by a single irreducible polynomial.
          Also, in $k[x]$ we have that $(0)$ is a prime ideal as well as $k[x]$ is an integral domain.
          In $\C[x]$, these are the ideals generated by $x - z$ for some $z \in \C$.
          In $\R[x]$, the primes are the ideals generated by $x - r$ for some $r \in \R$ or $x^{2} + r$ for some positive $r \in R$.
          \quest{What would be a general condition for $\Q[x]$?}
  \end{enumerate}
\end{sol}

\begin{sol}[\cite{book:UAG} Ex. 1.1.9]\label{ex:UAG_1.1.9}
  \begin{enumerate}
    \item First, observe that $\ideal{x_{1}, \ldots, x_{n}}$ is the ideal consisting exactly of polynomials which have no constant term.
          Let $I$ be an ideal in $k[x_{1}, \ldots, x_{n}]$ such that $\ideal{x_{1}, \ldots, x_{n}} \subsetneq I$.
          Thus there exists $f(x_{1}, \ldots, x_{n}) \in I \setminus \ideal{x_{1}, \ldots, x_{n}}$.
          We have by our observation that $f$ has a nonzero constant term $z$.
          Then note that the non-constant terms of $f$ form a polynomial $g(x_{1}, \ldots, x_{n})$ in $\ideal{x_{1}, \ldots, x_{n}}$.
          Thus, we have that $z = f(x) - g(x) \in I$.
          Since $I$ contains a nonzero constant term, we must have that $I = k[x_{1}, \ldots, x_{n}]$.
    \item Recall that an ideal $I$ is maximal if and only if $R/I$ is a field.
          Let $I = \ideal{x_{1} - a_{1}, \ldots, x_{n} - a_{n}}$.
          Consider the evaluation map $\text{ev}_{\overline{a}}\colon k[x_{1}, \ldots, x_{n}] \to k$ sending $f(x_{1}, \ldots, x_{n}) \mapsto f(a_{1}, \ldots, a_{n})$.
          Clearly this map is surjective.
          Then since for all $i$ we have that $x_{i} \equiv a_{i} \pmod{I}$, we have that $f(x_{1}, \ldots, x_{n}) \equiv f(a_{1}, \ldots, a_{n}) \pmod{I}$ for all $f(x_{1}, \ldots, x_{n}) \in k[x_{1}, \ldots, x_{n}]$.
          Thus, $\text{ev}_{\overline{a}}(f) = f(a_{1}, \ldots, a_{n}) = 0$ if and only if $f(x_{1}, \ldots, x_{n}) \in I$.
          Thus, $\ker(\text{ev}_{\overline{a}}) = I$ and $k[x_{1}, \ldots, x_{n}] / I$ is a field, meaning $\ideal{x_{1} - a_{1}, \ldots, x_{n} - a_{n}}$ is maximal.
    \item Since $\R[x]$ is a principal ideal domain, any ideal $I$ strictly containing $\ideal{x^{2} + 1}$ is of the form $\ideal{g(x)}$ for some $g(x) \mid x^{2} + 1$.
          However, since $x^{2} + 1$ is irreducible in $\R[x]$, we have that $g(x)$ is either $z(x^{2} + 1)$ for some nonzero $z \in \C$ or $g(x) = z$ for some nonzero $z \in \C$, meaning $\ideal{g(x)} = \ideal{x^{2} + 1}$ or or $\ideal{g(x)} = \R[x]$.
          Thus, $\ideal{x^{2} + 1}$ is maximal.
          However, in $\C[x]$, we have that $x^{2} + 1 = (x + i)(x - i)$ and so $\ideal{x^{2} + 1} \subsetneq \ideal{x - i} \subsetneq \C[x]$.
  \end{enumerate}
\end{sol}

\clearpage

\begin{sol}[\cite{book:UAG} Ex. 1.1.10]\label{ex:UAG_1.1.10}
  \begin{enumerate}
    \item Since $x^{2} + y^{2} - (x^{2} - z^{3}) = y^{2} + z^{3}$ is an element of $I$ which does not depend on $x$, $y^{2} + z^{3}$ is in $I_{1}$.
    \item For all $\ell \geq 1$, we have that $0 \in I_{\ell}$.
          Then, if $f(x_{\ell + 1}, \ldots, x_{n}), g(x_{\ell + 1}, \ldots, x_{n})$ are two polynomials in $I$ who do not depend on the first $\ell$ variables, then so is $f + g$.
          Finally, let $r(x_{\ell} + 1, \ldots, x_{n}) \in k[x_{\ell + 1}, \ldots, x_{n}]$.
          Then $r \cdot f \in I_{\ell}$ since $r \cdot f \in I$ and still does not depend on any of the first $\ell$ variables.
  \end{enumerate}
\end{sol}

\begin{sol}[\cite{book:UAG} Ex. 1.1.11]\label{ex:UAG_1.1.11}
  \begin{enumerate}
    \item \quest{meh}
    \item \quest{meh}
    \item We claim that $I + J = \ideal{f_{1}, \ldots, f_{s}, g_{1}, \ldots, g_{t}}$.
          Clearly $I, J \subseteq \ideal{f_{1}, \ldots, f_{s}, g_{1}, \ldots, g_{t}}$ and thus so is $I \cup J$.
          By (b), this shows that $I + J \subseteq \ideal{f_{1}, \ldots, f_{s}, g_{1}, \ldots, g_{t}}$.
          Then, since $f_{i} = f_{i} + 0$ and $g_{j} = 0 + g_{j}$ for all $i, j$, we have the reverse inclusion and thus the two ideals are equal.
  \end{enumerate}
\end{sol}

\begin{sol}[\cite{book:UAG} Ex. 1.1.12]\label{ex:UAG_1.1.12}
  \begin{enumerate}
    \item \quest{meh}
    \item Suppose that $h(\overline{x}) \in IJ$.
          Note that $IJ$ is generated by the products $f(\overline{x}) \cdot g(\overline{x})$ for $f(\overline{x}) \in I$, and $g(\overline{x}) \in J$.
          Then $h(\overline{x})$ consists of sums of terms of the form $r(\overline{x}) \cdot f(\overline{x}) \cdot g(\overline{x})$ for $r(\overline{x}) \in k[\overline{x}]$, $f(\overline{x}) \in I$, and $g(\overline{x}) \in J$.
          Thus, each term is in both $I$ and $J$ and overall so is $h(\overline{x})$.

          To see an example where $IJ \subsetneq I \cap J$, consider $I = \ideal{x^{2}y}$ and $J = \ideal{xy^{2}}$ in $k[x, y]$.
          Then $I \cap J = \gen{x^{2}y^{2}}$ and $IJ = \gen{x^{3}y^{3}}$.
          Thus $IJ \subsetneq I \cap J$ as $I \cap J$ contains $x^{2}y^{2}$ and $IJ$ does not contain $x^{2}y^{2}$.
  \end{enumerate}
\end{sol}

\clearpage

\section{\Grobner\ Bases}

\begin{sol}[\quest{\cite{book:UAG} Ex. 1.3.11}]\label{ex:UAG_1.3.11}

\end{sol}

\clearpage

\section{Affine Varieties}

\begin{sol}[\quest{\cite{book:UAG} Ex. 1.4.9}]\label{ex:UAG_1.4.9}

\end{sol}

\chapter{Solving Polynomial Equations}

\section{Solving Polynomial Systems by Elimination}

\begin{sol}[\quest{\cite{book:UAG} Ex. 2.1.1}]\label{ex:UAG_2.1.1}

\end{sol}

\begin{sol}[\quest{\cite{book:UAG} Ex. 2.1.2}]\label{ex:UAG_2.1.2}

\end{sol}

\begin{sol}[\cite{book:UAG} Ex. 2.1.3]\label{ex:UAG_2.1.3}
  We may freely rewrite the polynomial as $p(z) = z^{n} - a_{n - 1}z^{n - 1} - \cdots - a_{0}$
  We have that $0 = \overline{z}^{n} - a_{n - 1}\overline{z}^{n - 1} - \cdots - a_{0}$ and so $\overline{z}^{n} = a_{n - 1}\overline{z}^{n - 1} + \cdots + a_{0}$.
  Suppose now that $\abs{\overline{z}} \geq 1$.
  Then
  \[
    \abs{\overline{z}}^{n} = \abs{a_{n - 1}\overline{z}^{n - 1} + \cdots + a_{0}} \leq \abs{a_{n - 1}}\abs{z}^{n - 1} + \cdots + a_{0} \leq \abs{a_{n - 1}}\overline{z}^{n - 1} + \cdots \abs{a_{0}}\overline{z}^{n - 1}.
  \]
  Thus, $\abs{\overline{z}} \leq \abs{a_{n - 1}} + \cdots + \abs{a_{0}}$.
  However, we assumed that $\abs{\overline{z}} \geq 1$.
  This may not be the case.
  Thus, $\abs{\overline{z}} \leq B \defeq \max\set{1, \abs{a_{n - 1}} + \cdots + \abs{a_{0}}}$.
\end{sol}

\begin{sol}[\quest{\cite{book:UAG} Ex. 2.1.4}]\label{ex:UAG_2.1.4}
  Numerically find all roots of $2z^{6} + 2z^{5} - z^{4} - z^{3} - 2z^{2} - 2z - 2$.
\end{sol}

\clearpage

\begin{sol}[\cite{book:UAG} Ex. 2.1.5]\label{ex:UAG_2.1.5}
  We apply Buchberger's Criterion.
  Let $f(x, y) = x^{2} + 2x + 3 + y^{5} - y$ and $g(x, y) = y^{6} - y^{2} + 2y$.
  Then we have that
  \[
    S(f, g) = \frac{x^{2}y^{6}}{x^{2}} \cdot (x^{2} + 2x + 3 + y^{5} - y) - \frac{x^{2}y^{6}}{y^{6}} \cdot (y^{6} - y^{2} + 2y) = y^{6} \cdot (x^{2} + 2x + 3 + y^{5} - y) - x^{2} \cdot (y^{6} - y^{2} + 2y).
  \]
  This shows that $\overline{S(f, g)}^{G} = 0$ which yields that $G$ is a \Grobner\ basis.
\end{sol}

\begin{sol}[\quest{\cite{book:UAG} Ex. 2.1.6}]\label{ex:UAG_2.1.6}
\end{sol}

\begin{sol}[\quest{\cite{book:UAG} Ex. 2.1.7}]\label{ex:UAG_2.1.7}
\end{sol}

\clearpage

\begin{sol}[\cite{book:UAG} Ex. 2.1.8]\label{ex:UAG_2.1.8}
  \begin{enumerate}
    \item Let $\overline{z}$ be a simple root of $p(z)$, so $p(z) = 0$ but $p'(z) \neq 0$.
          Then $N_{p}(\overline{z}) = \overline{z} - \frac{p(\overline{z})}{p'(\overline{z})} = \overline{z}$ meaning $\overline{z}$ is a fixed point of $N_{p}(z)$.
    \item Suppose that $\overline{z}$ is a multiple root of $p(z)$ with multiplicity $m \geq 2$.
          Then we may express $p(z) = \tilde(p)(z)(z - \overline{z})^{m}$ such that $\tilde{p}(\overline{z}) \neq 0$.
          Thus, we have that
          \begin{align*}
            N_{p}(z) &\defeq z - \frac{p(z)}{p'(z)} \\
            &= z - \frac{\tilde{p}(z)(z - \overline{z})^{m}}{\tilde{p}'(z)(z - \overline{z})^{m} + m\tilde{p}(z)(z - \overline{z})^{m - 1}} = z - \frac{\tilde{p}(z)(z - \overline{z})}{\tilde{p}'(z)(z - \overline{z}) + m\tilde{p}(z)}
          \end{align*}
          Note that $m\tilde{p}(\overline{z}) \neq 0$.
          Thus, we have that
          \[
            \abs{N_{p}(\overline{z})} = \abs{\overline{z} - \frac{\tilde{p}(\overline{z})(\overline{z} - \overline{z})}{\tilde{p}'(\overline{z})(\overline{z} - \overline{z}) + m\tilde{p}(\overline{z})}} = \abs{\overline{z}} \leq \lc(p) \cdot B
          \]
          where $B$ is the value from~\nameref{ex:UAG_2.1.3} and $\lc(p)$ is the leading coefficient of $p(z)$.
    \item Suppose now that $\overline{z}$ is a simple root of $p(\overline{z})$.
          Then we may express $p(z) = \tilde{p}(z)(z - \overline{z})$ such that $\tilde{p}(\overline{z}) \neq 0$.
          We have that
          \[
            p'(z) = \tilde{p}'(z)(z - \overline{z}) + \tilde{p}(z)
          \]
          and evaluation of $p'(z)$ at $\overline{z}$ is nonzero.
    \item Let $\overline{z}$ be a root of multiplicity $m$.
          Following (b), we write $p(z) = \tilde{p}(z)(z - \overline{z})^{m}$ such that $\tilde{p}(\overline{z}) \neq 0$.
          Then we have, by differentiating the expression for $N_{p}(z)$ from (b), that
          \[
            N'_{p}(z) = 1 - \frac{(\tilde{p}'(z)(z - \overline{z}) + \tilde{p}(z))(\tilde{p}'(z)(z - \overline{z}) + m\tilde{p}(z)) - (\tilde{p}(z)(z - \tilde{z}))(\tilde{p}''(z)(z - \overline{z}) + \tilde{p}'(z) + m\tilde{p}'(z))}{(\tilde{p}'(z)(z - \overline{z}) + m\tilde{p}(z))^{2}}.
          \]
          Evaluation at $z = \overline{z}$ yields that $\lim_{z \to \overline{z}} N_{p}'(z) = 1 - \frac{1}{m}$.
    \item Let $\overline{z}$ be a root of multiplicity $m$.
          Following (b), we write $p(z) = \tilde{p}(z)(z - \overline{z})^{m}$ such that $\tilde{p}(\overline{z}) \neq 0$.
          Then
          \[
            p'(z) = \tilde{p}'(z - \overline{z})^{m} + m\tilde{p}(z)(z - \overline{z})^{m - 1} = (z - \overline{z})^{m - 1}(\tilde{p}'(z)(z - \overline{z}) + m\tilde{p}(z)).
          \]
          Notice that $\tilde{p}'(\overline{z})(\overline{z} - \overline{z}) + m\tilde{p}(\overline{z}) = m\tilde{p}(\overline{z}) \neq 0$.
          Thus, a root of multiplicity $m \geq 1$ of $p(z)$ is a root of multiplicity $m - 1$ of $p'(z)$.
          This implies that if we have roots $\overline{z}_{1}, \ldots, \overline{z}_{k}$ with multiplicities $m_{1}, \ldots, m_{k} \geq 1$, then $\gcd(p(z), p'(z)) = (z - \overline{z}_{1})^{m_{1}} \cdots (z - \overline{z}_{k})^{m_{k}}$.
          Thus, the polynomial $p_{\mathrm{red}}(z) = \frac{p(z)}{\gcd(p(z), p'(z))}$ has the same roots of $p(z)$ but all with multiplicity $1$ which is the best case for Newton's method.
  \end{enumerate}
\end{sol}

\begin{sol}[\cite{book:UAG} Ex. 2.1.9]\label{ex:UAG_2.1.9}
  \begin{enumerate}
    \item Let $p(z) = z^{2} + 1$.
          We have that
          \[
            N_{p}(z) = z - \frac{z^{2} + 1}{2z} = \frac{2z^{2} - z^{2} + 1}{2z} = \frac{z^{2} + 1}{2z} = \frac{x^{2} + 2ixy - y^{2} + 1}{2x + 2iy}.
          \]
          If $z$ is real then $y = 0$ and so $N_{p}(x) = \frac{x^{2} + 1}{2x}$ which is always real.
          Thus, Newton's method will never reach the imaginary roots of $z^{2} + 1$.
          However, if we begin with a guess with nonzero imaginary part, then the guess does converge as expected.
    \item \quest{Just basic arithmetic not worth doing.}
  \end{enumerate}
\end{sol}

\begin{sol}[\cite{book:UAG} Ex. 2.1.10]\label{ex:UAG_2.1.10}
  Let $\overline{z}$ be a root of $p(z)$.
  Then $-\overline{z}^{n} = a_{n - 1}\overline{z}^{n - 1} + \cdots + a_{0}$ and so
  \begin{align*}
    \abs{\overline{z}}^{n} &= \abs{a_{n - 1}\overline{z}^{n - 1} + \cdots + a_{0}} \\
                           &\leq \max_{i}\set{\abs{a_{i}}} \cdot \abs{\overline{z}^{n - 1} + \cdots + 1} \\
                           &\leq \max_{i}\set{\abs{a_{i}}} \cdot \pqty{\abs{\overline{z}}^{n - 1} + \cdots + 1} \\
                           &= \max_{i}\set{\abs{a_{i}}} \cdot \frac{\abs{\overline{z}}^{n} + 1}{\abs{\overline{z}} - 1} \leq \max_{i}\set{\abs{a_{i}}} \cdot \frac{\abs{\overline{z}}^{n}}{\abs{\overline{z}} - 1}.
  \end{align*}
  Thus, $\abs{\overline{z}}^{n} \leq \max_{i}\set{\abs{a_{i}}} \cdot \frac{\abs{z}^{n}}{\abs{z} - 1}$ which implies that $\abs{z} - 1 \leq \max_{i}\set{\abs{a_{i}}}$.
  Thus, $\abs{z} \leq 1 + \max_{i}\set{\abs{a_{i}}}$.
\end{sol}

\clearpage

\section{Finite Dimensional Algebras}

\begin{sol}[\quest{\cite{book:UAG} Ex. 2.2.1}]\label{ex:UAG_2.2.1}
\end{sol}

\begin{sol}[\cite{book:UAG} Ex. 2.2.2]\label{ex:UAG_2.2.2}
  It is clear that $\ideal{p_{i}(x_{i})} \subseteq I \cap k[x_{i}]$.
  Now suppose that $f(x_{i}) \in I \cap k[x_{i}]$.
  Then $\deg(f(x_{i}))$ must be $\geq m_{i}$.
  If not, then by the minimality of $m_{i}$ we would arrive at a contradiction.
  Now by the division algorithm, write $f(x_{i}) = q(x_{i})p_{i}(x_{i}) + r(x_{i})$ where $\deg(r_{x_{i}}) < m_{i}$.
  Then $r(x_{i}) = f(x_{i}) - q(x_{i})p(x_{i}) \in I$ and so $r(x_{i})$ must be $0$ since if not, we would arrive at a contradiction of the minimality of $m_{i}$.

  This gives us an algorithm to compute $p_{i}(x_{i})$.
  Let $I$ be a zero dimensional ideal and $G$ a \Grobner\ basis for $I$.
  Then we know there exists $m_{i}$ such that $\set{1, [x_{i}], \ldots, [x^{m_{i}}]}$ is linearly dependent in $k[\overline{x}] / I$.
  In fact, we may use the Finiteness Theorem to set $m_{i}$ to the smallest integer such that $x_{i}^{m_{i}} = \lt(g)$ for some $g \in G$.
  Since $k[x_{1}, \ldots, x_{n}] / I$ is a vector space, we can check linear independence in the usual way.
  See \texttt{code/ch2/2\_2\_2.sage} for a SageMath implementation of this.
\end{sol}

\begin{sol}[\cite{book:UAG} Ex. 2.2.3]\label{ex:UAG_2.2.3}
  Let $0 \neq f(x) \in \sqrt{\gen{p(x)}}$.
  Then there exists $m \geq 1$ such that $f^{m} \in \gen{p(x)}$ and so $p(x) \mid f(x)^{m}$.
  In particular, each linear factor $(x - \overline{z})$ of $p(x)$ divides $f(x)^{n}$ and so divides $f(x)$ as $(x - \overline{z})$ is irreducible.
  Thus, $p_{\red}(x) \mid f(x)$ and so $f(x) \in \gen{p_{\red}(x)}$.
  Conversely, suppose $f(x) \in \gen{p_{\red}(x)}$ so that $\gen{p_{\red}} \mid f(x)$.
  Label the roots of $p(x)$ as $\overline{z}_{1}, \ldots, \overline{z}_{r}$, each $\overline{z}_{i} \in \overline{k}$.
  Then for each $i$, $(x - \overline{z}_{i}) \mid f(x)$
  Let $m_{i}$ be the multiplicity of $z_{i}$ in $p(x)$ and $m = \max\set{m_{1}, \ldots, m_{r}}$.
  Then $p(x) \mid f(x)^{m}$ and so $f(x) \in \sqrt{\gen{p(x)}}$
\end{sol}

\begin{sol}[\cite{book:UAG} Ex. 2.2.4]\label{ex:UAG_2.2.4}
  We use the algorithm from \nameref{ex:UAG_2.2.2} implemented in \texttt{code/ch2/2\_2\_2sage}.
  See \texttt{code/ch2/2\_2\_2sage} for the code in action.
\end{sol}

\clearpage

\begin{sol}[\quest{\cite{book:UAG} Ex. 2.2.5}]\label{ex:UAG_2.2.5}
  Then $\sqrt{I} = I + \gen{x(x - 1), y(y - 2)}$.
  Since $I \subseteq \sqrt{I}$, we see that $\dim \C[x, y] / I \geq \dim \C[x, y] / \sqrt{I}$.
  A quick SageMath computation confirms this: $\dim \C[x, y] / I = 9$ and $\dim \C[x, y] / \sqrt{I} = 2$.
  See \texttt{code/ch2/2\_2\_5.sage} for the code in action.
  Then, since $I \subseteq \sqrt{I}$ we have that $V(\sqrt{I}) \subseteq V(I)$.
  Notice that
  \begin{align*}
    y^{4}x + 3x^{3} - y^{4} - 3x^{2} &= y^{4}(x - 1) + 3x^{2}(x - 1) = (y^{4} + 3x^{2})(x - 1) \\
    x^{2}y - 2x^{2} &= x^{2}(y - 2) \\
    2y^{4}x - x^{3} - 2y^{4} + x^{2} &= 2y^{4}(x - 1) - x^{2}(x - 1) = (2y^{4} - x^{2})(x - 1).
  \end{align*}
  Thus, $(1, 2)$ and $(0, 0)$ are the only two points in $V(I)$.
  Since it is evident that $V(\sqrt{I})$ contains these two points, we see in this case that $V(\sqrt{I}) = V(I)$.
\end{sol}

\begin{sol}[\cite{book:UAG} Ex. 2.2.6]\label{ex:UAG_2.2.6}
  A grevlex \Grobner\ basis for $I$ is $\set{y^{4} - 16y^{2}, x^{3} - x^{2}, -2x^{2}}$.
  Thus, by the Finiteness Theorem, we know that the for monomials $x^{a}y^{b}$ in $\C[x, y] / I$ we must have that $0 \leq a \leq 1$ and $0 \leq b \leq 3$.
  See \texttt{code/ch2/2\_2\_6.sage} for the code in action to compute the table.
\end{sol}

\begin{sol}[\cite{book:UAG} Ex. 2.2.7]\label{ex:UAG_2.2.7}
  We implement the algorithm described in \nameref{ex:UAG_1.3.11}.
  See \texttt{/code/ch2/2\_2\_7.sage} for the code in action.
\end{sol}

\begin{sol}[\cite{book:UAG} Ex. 2.2.8]\label{ex:UAG_2.2.8}
  \begin{enumerate}
    \item See \texttt{code/ch2/2\_2\_8.sage} for the code in action.
    \item Since each of the $I_{j}$ are maximal ideals and $I_{j} \subseteq \sqrt{I_{j}}$, we must have that $I = \sqrt{I_{j}}$.
          Thus $I(V(I_{j})) = I_{j}$ and we must have that $I_{j} = \sqrt{I_{j}}$.
          Since each $I_{j}$ is radical and $I = \bigcap_{j = 1}^{5} I_{j}$, we have by~\nameref{ex:UAG_2.2.7} that $I$ is radical.
  \end{enumerate}
\end{sol}

\clearpage

\begin{sol}[\cite{book:UAG} Ex. 2.2.9]\label{ex:UAG_2.2.9}
  \begin{enumerate}
    \item Let $f(\overline{x}) \in I + \ideal{p}$ and let $1 \leq j \leq d$.
          Then $f(\overline{x}) = g(\overline{x}) + h(\overline{x})p(x_{1})$ for some $g(\overline{x}) \in I$ and $h(\overline{x}) \in k[\overline{x}]$.
          We have that $(x_{1} - a_{j}) \mid p(x_{1})$ and so $h(\overline{x})p(x_{1}) \in \ideal{x_{1} - a_{j}}$.
          Thus, $f(\overline{x}) = g(\overline{x}) + h(\overline{x})p(x_{1}) \in I + \ideal{x_{1} - a_{j}}$.
          As $j$ was arbitrary, we have that $f(\overline{x}) \in \bigcap_{j} (I + \ideal{x_{1} - a_{j}})$.
    \item Let $f(\overline{x}) \in p_{j} \cdot (I + \ideal{x_{1} - a_{j}})$.
          Then $f(\overline{x}) = p_{j}(x_{1})\cdot(g(\overline{x}) + h(\overline{x})(x_{1} - a_{j}))$ for some $g(\overline{x}) \in I$ and $h(\overline{x}) \in k[\overline{x}]$.
          We have that $p_{j}(x_{1})g(\overline{x}) \in I$ and $p_{j}(x_{1})h(\overline{x})(x_{1} - a_{j}) = h(\overline{x})p(x_{1}) \in \ideal{p}$.
          Thus, $f(\overline(x)) = p_{j}(x_{1})g(\overline{x}) + h(\overline{x})p(x_{1}) \in I + \ideal{p}$.
    \item Let $d = \gcd(p_{1}, \ldots, p_{d})$.
          Then as $d \mid p_{1}$ and $d \mid p_{2}$, we have that $d \mid \prod_{j \neq 1, 2} (x_{1} - a_{j})$.
          Continuing on inductively, we have that for all $c \leq d$ that $d \mid \prod_{j \notin [c]} (x_{1} - a_{j})$.
          In particular, this means that $d \mid \prod_{j \notin [d]} (x_{1} - a_{j}) = 1$.
          Thus, $d$ itself is a unit in $k[\overline{x}]$ and $p_{i}$ and $p_{j}$ are coprime.
          By Bezout's Lemma, there exists polynomials $h_{1}, \ldots, h_{d} \in k[\overline{x}]$ such that $1 = \sum_{j = 1}^{d} h_{j}(\overline{x})p_{j}(x_{1})$.
    \item Now let $h(\overline{x}) \in \bigcap_{j = 1}^{d} (I + \ideal{x_{1} - a_{j}})$.
          As all the $p_{j}$ are coprime, we have that there exist polynomials $h_{1}, \ldots, h_{d} \in k[\overline{x}]$ such that $1 = \sum_{j = 1}^{d} h_{j}(\overline{x})p_{j}(x_{1})$.
          Thus, $h = \sum_{j = 1}^{d} h_{j}(\overline{x})p_{j}(x_{1})h(\overline{x})$.
          Then for all $1 \leq j \leq d$, we have that as $p_{j}(x_{1})h(\overline{x}) \in p_{j} \cdot (I + \ideal{x_{1} - a_{j}}) \subseteq I + \ideal{p}$.
          Thus, each summand of $\sum_{j = 1}^{d} h_{j}(\overline{x})p_{j}(x_{1})h(\overline{x})$ is in $I + \ideal{p}$ and so overall $h \in I + \ideal{p}$.
  \end{enumerate}
\end{sol}

\begin{sol}[\cite{book:UAG} Ex. 2.2.10]\label{ex:UAG_2.2.10}
  \begin{enumerate}
    \item Let $\overline{f}^{G} = \sum_{j = 1}^{d} c_{j}(f) x^{\alpha(j)}$ and $\overline{g}^{G} = \sum_{j = 1}^{d} c_{j}(g) x^{\alpha(j)}$.
          Then by combining like terms, we have that $\overline{f}^{G} + \overline{g}^{G} = \sum_{j = 1}^{d} (c_{j}(f) + c_{j}(g)) x^{\alpha(j)}$.
          On the other hand, we have that $\overline{f + g}^{G} = \sum_{j = 1}^{d} c_{j}(f + g) x^{\alpha(j)}$.
          Since $\overline{f}^{G} + \overline{g}^{G} = \overline{f + g}^{G}$ and each of the $x^{\alpha(j)}$ are linearly independent, we may equate coefficients and conclude that $c_{j}(f) + c_{j}(g) = c_{j}(f + g)$.
          For $\lambda \in k$, $\overline{\lambda f}^{G} = \sum_{j = 1}^{d} c_{j}(\lambda f) x^{\alpha(j)}$.
          Now notice that $\overline{\lambda f}^{G} = \lambda \overline{f}^{G}$ as we are working over a field.
          Thus, we have by equating coefficients that $c_{j}(\lambda f) = \lambda c_{j}(f)$.
          Thus, $c_{j}$ is a linear function $A \to k$.
    \item Let $\alpha_{j} \in A^{*}$ be the linear map $\alpha_{j}(f) = c_{j}(f)$.
          Notice that for all $1 \leq i, j \leq d$ we have that $\alpha_{j}(x^{\alpha(i)}) = c_{j}(x^{\alpha(i)}) = \delta_{i,j}$.
          Suppose there exists $\lambda_{1}, \ldots, \lambda_{d} \in k$ such that $\lambda_{1}\alpha_{1} + \cdots + \lambda_{d}\alpha_{d} = 0$.
          Then for all $1 \leq i \leq d$ we have that
          \[
            0 = \pqty{\sum_{j = 1}^{d} \lambda_{j} \alpha_{j}}(x^{\alpha(i)}) = \sum_{j = 1}^{d} \lambda_{j} \alpha_{j}(x^{\alpha(i)}) = \lambda_{i}
          \]
          and so for all $1 \leq i \leq d$, $\lambda_{i} = 0$ meaning that $\set{\alpha_{1}, \ldots, \alpha_{d}}$ is linearly independent.
          Since we know that $d = \dim A = \dim A^{*}$, we have that $\set{\alpha_{1}, \ldots, \alpha_{d}}$ is a basis for $A^{*}$.
    \item This was proven in \textbf{(b)}.
  \end{enumerate}
\end{sol}

\begin{sol}[\cite{book:UAG} Ex. 2.2.11]\label{ex:UAG_2.2.11}
  \begin{enumerate}
    \item We want a linear polynomial $\ell(\overline{x}) = \ell_{1}x_{1} + \cdots + \ell_{n}x_{n}$ takes distinct values at each of the $p_{i} \in \C^{n}$.
          Consider the space of all such $(\ell_{1}, \ldots, \ell_{n})$.
          This itself is a $\C$ vector space, call it $L$.
          Let $L_{i, j}$ be the subspace of $L$ corresponding to polynomials $\ell(\overline{x})$ such that $\ell(p_{i}) = \ell(p_{j})$.
          There are finitely many such $L_{i, j}$.
          We know that vector spaces over an infinite field cannot be expressed as the finite union of proper subspaces.
          Thus, $L \neq \bigcup_{1 \leq i \neq j \leq m} L_{i, j}$.
          This means there exists $(\ell_{1}, \ldots, \ell_{n}) \in L \setminus \bigcup_{1 \leq i \neq j \leq m} L_{i, j}$ such that $\ell(\overline{x}) = \ell_{1}x_{1} + \cdots + \ell_{n}x_{n}$ takes distinct values at each of the $p_{i}$.

          \quest{Can we do this constructively?}
    \item Let $\ell(\overline{x})$ be our constructed polynomial from \textbf{(a)}.
          For $1 \leq i \leq m$, we define $g_{i} \in \C[x_{1}, \ldots, x_{n}]$ as
          \[
            g_{i}(\overline{x}) = \frac{\Prod_{1 \leq i \neq j \leq m} \ell(\overline{x}) - \ell(\overline{p_{j}})}{\Prod_{1 \leq i \neq j \leq m} \ell(p_{i}) - \ell(\overline{p_{j}})}.
          \]
          Then clearly $g_{i}(p_{j}) = \delta_{ij}$ as desired.
  \end{enumerate}
\end{sol}

\begin{sol}[\cite{book:UAG} Ex. 2.2.12]\label{ex:UAG_2.2.12}
  \begin{enumerate}
    \item Clearly the map is linear.
          We now show it is well defined.
          Let $[f] = [g] \in \C[\overline{x}] / I$ meaning that $f - g \in I$.
          Thus, we have that
          \[
            \varphi([f]) - \varphi([g]) = \varphi([f - g]) = 0 \implies \varphi([f]) = \varphi([g])
          \]
          as desired.
    \item Let $[f], [g] \in \C[\overline{x}] / I$. Then we have
          \begin{align*}
            \varphi([f] \cdot [g]) &= \varphi([f \cdot g]) \\
                                   &= ((f \cdot g)(p_{1}), \ldots, (f \cdot g)(p_{m})) \\
                                   &= (f(p_{1}) \cdot g(p_{1}), \ldots, f(p_{m}) \cdot g(p_{m})) \\
                                   &= (f(p_{1}), \ldots, f(p_{m})) \cdot (g(p_{1}), \ldots, g(p_{m})) = \varphi([f]) \cdot \varphi([g]).
          \end{align*}
          Thus, $\varphi$ is a homomorphism of rings.
          In fact, it is a homomorphism of $\C$-algebras as $(\lambda \cdot f)(x) \defeq \lambda \cdot f(\overline{x})$ for all $\lambda \in \C$ and $f \in \C[\overline{x}]$ and this descends to $\C[\overline{x}] / I$.
    \item We have that $\varphi$ is surjective and that $I \subseteq \ker(\varphi)$ as in the proof of~\cite[Theorem 2.10]{book:UAG}.
          So we want to show that $\ker(\varphi) \subseteq I$ if and only if $I = \sqrt{I}$ which holds exactly as in the proof of~\cite[Theorem 2.10]{book:UAG}.
  \end{enumerate}
\end{sol}

\clearpage

\section{\Grobner\ Basis Conversion}

\begin{sol}[\cite{book:UAG} Ex. 2.3.2]\label{ex:UAG_2.3.2}
  Recall that \emph{lex} ordering compares the exponents of $x_{1}$, and then in the case of equality compares the exponents of $x_{2}$, and continues on in this manner.
  As such $\overline{x}^{\alpha} = x_{1}^{\alpha_{1}} \cdots x_{n}^{\alpha_{n}} \geq x_{1}^{a}$ if and only if $alpha_{1} \geq a_{1}$ which is equivalent to saying that $x_{1}^{a_{1}} \mid x_1^{\alpha_{1}} \mid \overline{x}^{\alpha}$
\end{sol}

\begin{sol}[\cite{book:UAG} Ex. 2.3.3]\label{ex:UAG_2.3.3}
  \begin{enumerate}
    \item Suppose for some $1 \leq i \leq k$ we have that $\lt(g_{i}) \mid x_{1}^{a_{1} + 1}$.
          This would imply that $\lt(g_{i})$ is a power of $x_{1}$.
          However, we assumed that we were in the situation of the Next Monomial procedure, meaning that the algorithm has not terminated which in turn implies that we have not added any polynomials to $G_{\emph{lex}}$ such that their leading term is a power of $x_{1}$.
          Thus, no such $\lt(g_{i})$ divides $x_{1}^{a_{1} + 1}$.
    \item Clearly we have that
          \[
            \overline{x}^{\beta} = x_{1}^{a_{1}} \cdots x_{k - 1}^{a_{k - 1}} x_{k}^{a_{k} + 1} > x_{1}^{a_{1}} \cdots x_{k - 1}^{a_{k - 1}} x_{k}^{a_{k}} = \overline{x}^{\alpha}.
          \]
          To show that $\overline{x}^{\beta}$ is the smallest monomial greater than $\overline{x}^{\alpha}$ where no $\lt(g_{i})$ divides $\overline{x}^{\beta}$, we want to show that $k + 1 \leq j, \ell \leq n$ and monomial $\overline{x}^{\gamma}$ of the form
          \begin{equation}\label{eq:2.3.2(b)-conditions}
            \overline{x}^{\gamma} = x_{1}^{a_{1}} \cdots x_{k}^{a_{k}} x_{k + 1}^{c_{k + 1}} \cdots x_{j}^{c_{j}},\ c_{k + 1} \geq a_{k + 1}, \ldots, c_{\ell - 1} \geq a_{\ell - 1}, c_{\ell} > a_{\ell}, c_{\ell + 1} \geq 0, \ldots, c_{j} \geq 0.
          \end{equation}
          sharing the same properties as $\overline{x}^{\beta}$.
          First, we indeed see that $\overline{x}^{\beta} > \overline{x}^{\gamma}$ as $a_{k} + 1 > a_{k}$ and $x^{\gamma} > x^{\alpha}$ by the assumption on $\ell$.
          We also see that any $x^{\gamma}$ such that $x^{\beta} > x^{\gamma} > x^{\alpha}$ must satisfy \Cref{eq:2.3.2(b)-conditions}.
          Suppose towards contradiction that none of the $\lt(g_{i})$ divide $\overline{x}^{\gamma}$.
          Then we have that
          \[
            \overline{x}^{\beta} > x_{1}^{a_{1}} \cdots x_{k}^{a_{k}} x_{k + 1}^{c_{k + 1}} \cdots x_{j}^{c_{j}} \geq x_{1}^{a_{1}} \cdots x_{k}^{a_{k}} x_{k + 1}^{c_{k + 1}} \cdots x_{\ell}^{c_{\ell}} \geq x_{1}^{a_{1}} \cdots x_{k}^{a_{k}} x_{k + 1}^{c_{k + 1}} \cdots x_{\ell}^{a_{\ell} + 1} > \overline{x}^{\alpha}.
          \]
          Since No $\lt(g_{i})$ divides $\overline{x}^{\gamma}$, we have that no $\lt(g_{i})$ divides $x_{1}^{a_{1}} \cdots x_{k}^{a_{k}} x_{k + 1}^{c_{k + 1}} \cdots x_{\ell}^{a_{\ell} + 1}$.
          As $\ell > k$, this contradicts the maximality of $k$.
          Thus, we cannot have that none of the $\lt(g_{i})$ divide $\overline{x}^{\gamma}$.
  \end{enumerate}
\end{sol}


\printbibliography
\end{document}
