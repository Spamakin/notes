\documentclass[letterpaper, 11pt, oneside]{book}

\usepackage{style}  % If you feel like procrastinating, mess with this file
\usepackage{algo}   % Thank you Jeff, very cool!

\addbibresource{refs.bib}

% Required reading
% https://jmlr.csail.mit.edu/reviewing-papers/knuth_mathematical_writing.pdf
%   Along with required viewing:
%   https://www.youtube.com/watch?v=N6QEgbPWUrg&list=PLOdeqCXq1tXihn5KmyB2YTOqgxaUkcNYG
% https://faculty.math.illinois.edu/~west/grammar.html

% % % % % % % % % %
%     Cursor      %
%     Parking     %
%     Lot         %
% % % % % % % % % %

% Disable check for mismatched parens/brackets/braces
%   chktex-file 9
% Disable check for different counts of parents/brackets/braces
%   chktex-file 17
% Exclude these environments from syntax checking
%   VerbEnvir { tikzcd }

\regtotcounter{figure}

\title{\vspace{-100pt} \includegraphics[width=0.75\textwidth]{figs/water.png} \\ {\Huge Representation Theory Notes and Exercises} \\ {\small With $\total{figure}$ Figures}}
\nocite{title_pic}
\author{\Large Anakin Dey}
\DTMsavenow{now}
\date{\small Last Edited on \today\ at \DTMfetchhour{now}:\DTMfetchminute{now}}

% Cover page number chicanery
\newcommand{\CoverName}{Cover}

\begin{document}
\frontmatter
\renewcommand{\thepage}{\CoverName}
\maketitle

\pagenumbering{roman}

\section*{TODOs}

\quest{Change the style of enumerates from ``1.'' to ``(1)''}

\quest{Proper Exercise Header}

\quest{Proper Chapter Header}

\quest{Remove indent at end of thmtool environments}

\tableofcontents
\clearpage


\listoftheorems[ignoreall, show={defn}, title={List of Definitions}]

\listoftheorems[ignoreall, show={ex}, title={List of Examples and Counterexamples}]

\chapter*{Preface}

This is a set of notes on group representation theory mainly based on J.P. Serre's text \emph{Linear Representations of Finite Groups}~\cite{book:SerreLinReps}.
Occasionally, other sources may be used, such as the set of notes by Charles Rezk created for Math 427~\cite{note:rezk_reps}.
The goal of these notes is to eventually work towards algebraic combinatorics such as Fulton's text \emph{Young Tableaux}~\cite{book:FultonTableaux}, as much of algebraic combinatorics is motivated by questions stemming from representation theory.
At the time of writing this, another goal is the applications of representation theory to computational complexity: see~\cite{misc:PanovaComputational} for a recent survey on this connection.

\mainmatter

\chapter{Generalities on Linear Representations}

Unless otherwise specified, $V$ will denote a vector space, usually over the field $\C$.
We will restrict ourselves to finite dimensional vector spaces.
Similarly, we will restrict ourselves to finite groups.

\begin{defn}[Linear Representation, Representation Space]
  Let $G$ be a group with identity $e$.
  A \emph{linear representation} of $G$ in $V$ is a homomorphism $\rho\colon G \to \GL(V)$.
  We will frequently, and often interchangeably, write $\rho_{s} \defeq \rho(s)$.
  Given $\rho$, we will say that $V$ is a \emph{representation space} or \emph{representation} of $G$.
\end{defn}

\begin{defn}[Degree]
  Let $\rho\colon G \to V$ be a representation of $G$ in a vector space $V$.
  Then the \emph{degree} of $\rho$ is $\dim(V)$.
\end{defn}

Let $\rho\colon G \to V$ be a representation of $G$ in a vector space $V$ with $n \defeq \dim(V)$.
Fix a basis $(e_{j})$ of $V$.
Then since each $\rho_{s}$ is an invertible linear transformation of $V$, we may define an $n \times n$ matrix $R_{s} \equiv (r_{ij}(s))$ where each $r_{ij}(s)$ is defined by the identity
\[
  \rho_{s}(e_{j}) = \sum_{i = 1}^{n} r_{ij}(s) e_{i}.
\]

\begin{defn}[Matrix of a Representation]
  We call $R_{s} = (r_{ij}(s))$ above the \emph{matrix of $\rho_{s}$} with respect to the basis $(e_{j})$.
\end{defn}
Note that $R_{s}$ satisfies the following:
\begin{align*}
  \det(R_{s}) \neq 0, && R_{st} = R_{s} \cdot R_{t} \equiv r_{ij}(st) = \sum_{k = 1}^{n} r_{ik}(s) \cdot r_{kj}(s) \quad \forall s, t \in G.
\end{align*}

\clearpage

Recall that two $n \times n$ matrices $A, A'$ are \emph{similar} if there exists an invertible matrix $T$ such that $T A = A' T$.
We may extend this notion to representations.
\begin{defn}[Similar/Isomorphic Representations]
  Let $\rho$ and $\rho'$ be two representations of the same group $G$ in vector spaces $V$ and $V'$ respectively.
  We say $\rho$ and $\rho'$ are \emph{similar} or \emph{isomorphic} if there exists an isomorphism $\tau \colon V \to V'$ such that for all $s \in G$, $\tau$ satisfies $\tau \circ \rho(s) = \rho'(s) \circ \tau$.
  If $R_{s}, R'_{s}$ are the corresponding matrices then this is equivalent to saying there exists an invertible matrix $T$ such that $T R_{s} = R'_{s} T$ for all $s \in G$.
\end{defn}
Note that if $\rho$ and $\rho'$ are isomorphic, then they must have the same degree.

We now give some examples of these things.
\begin{ex}[Unit/Trivial Representation]\label{ex:unit_trivial_representation}
  Let $G$ be a finite group.
  Representations of degree $1$ must be of the form $\rho\colon G \to \C^{\times}$.
  Since elements $s$ of $G$ are of finite order, $\rho(s)$ must also be of finite order.
  Thus, for all $s \in G$, $\rho(s)$ is a root of unity.
  If we take $\rho(s) = 1$ for all $s \in G$, we obtain the \emph{unit} or \emph{trivial} representation of $G$.
  This also means that $R_{s} = 1$ for all $s$.
\end{ex}

\begin{ex}[Regular Representation]\label{ex:regular_representation_Z3}
  Let $g$ be the order of $G$, and let $V$ be a vector space of dimension $g$ with a basis $(e_{t})_{t \in G}$.
  For each $s \in G$, define $\rho_{s}$ as the linear map $\rho_{s}\colon V \to V$ such that $\rho_{s}(e_{t}) = e_{st}$.
  This is a linear representation of $G$ called the \emph{regular} representation of $G$.
  Since for each $s \in G$, $e_{s} = \rho_{s}(e_{1})$ and thus the images of $e_{1}$ form a basis of $V$.
  Conversely, let $W$ be a representation of $G$ with a vector $w$ satisfying the collection of all $\rho_{s}(w)$, $s \in G$, forms a basis of $W$.
  Then $W$ is isomorphic to the regular representation of $G$ by the isomorphism $\tau(e_{s}) = \rho_{s}(w)$.

  For example, let $G = \Z_{3}$ and $V = \C^{3}$ with $e_{0} = (1, 0, 0)$, $e_{1} = (0, 1, 0)$, and $e_{2} = (0, 0, 1)$.
  Then for example, $\rho_{0}, \rho_{1}, \rho_{2}\colon \C^{3} \to \C^{3}$ are the linear maps such that
  \begin{align*}
    \rho_{0}(e_{0}) = e_{0 + 0} = e_{0} && \rho_{0}(e_{1}) = e_{0 + 1} = e_{1} && \rho_{0}(e_{2}) = e_{0 + 2} = e_{2} \\
    \rho_{1}(e_{0}) = e_{1 + 0} = e_{1} && \rho_{1}(e_{1}) = e_{1 + 1} = e_{2} && \rho_{1}(e_{2}) = e_{1 + 2} = e_{0} \\
    \rho_{2}(e_{0}) = e_{2 + 0} = e_{2} && \rho_{2}(e_{1}) = e_{2 + 1} = e_{0} && \rho_{2}(e_{2}) = e_{2 + 2} = e_{1} \\
  \end{align*}
  With this, the matrix representations of $\rho_{0}, \rho_{1}$ and $\rho_{2}$ is similarly straightforward:
  \begin{align*}
    R_{0} = \begin{pmatrix} 1 & 0 & 0 \\ 0 & 1 & 0 \\ 0 & 0 & 1 \end{pmatrix} && R_{1} = \begin{pmatrix} 0 & 0 & 1 \\ 1 & 0 & 0 \\ 0 & 1 & 0 \end{pmatrix} && R_{2} = \begin{pmatrix} 0 & 1 & 0 \\ 0 & 0 & 1 \\ 1 & 0 & 0 \end{pmatrix}
  \end{align*}
\end{ex}

\clearpage

\begin{ex}[Permutation Representation]
  We may generalize the regular representation to any group action $G \acts X$, $X$ a finite set.
  Recall that for such an action, the map $x \mapsto sx$ for each $s \in G$ is a permutation $X \tofrom X$.
  Let $V$ be a vector space with dimension the size of $X$, and so a basis $(e_{x})_{x \in X}$.
  Define a representation $\rho$ of $G$ by defining $\rho_{s}$ as the linear map sending $e_{x} \mapsto e_{sx}$.
  This representation is known as the \emph{Permutation} representation of $G$ associated with $X$.
  If we consider $X = [n]$ and $G = S_{n}$, then take $V = \C^{n}$ as our vector space and $e_{i}$ as the standard basis vector.
  Then $\rho_{\sigma}(e_{j}) = e_{\sigma_{j}}$.
  Thus for each $\sigma \in S_{n}$, we have that $R_{\sigma} = (r_{ij}(\sigma))$ where entry $r_{ij}(\sigma) = 1$ if $i = \sigma_{j}$ and $0$ otherwise.
\end{ex}

\begin{defn}[Stable/Invariant Subspaces, Subrepresentation]
  Let $\rho\colon G \to \GL(V)$ be a linear representation and $W \subseteq V$ a subspace of $V$.
  We say that $W$ is \emph{stable} under the action of $G$ if $x \in W$ implies that $\rho_{s}(x) \in W$ for all $s \in G$.,
  Thus, the restriction $\rho_{s}^{W} \defeq \rho_{s}\mid_{W}$ is an isomorphism of $W$ onto itself.
  Restrictions satisfy the property that $\rho_{s}^{W} \circ \rho_{t}^{W} = \rho_{st}^{W}$.
  Thus, $\rho^{W}\colon G \to \GL(W)$ is a linear representation of $G$ in $W$ and we say that $W$ is a \emph{subrepresentation} of $V$.
\end{defn}

\begin{ex}[Subrepresentations of the Regular Representation]
  Let $G$ be a group.
  Recall the regular representation $V$ given in \Cref{ex:regular_representation_Z3}.
  Let $W$ be the 1 dimensional subspace of $V$ generated by the element $x = \sum_{s \in G} e_{s}$.
  Then note that $\rho_{s}(x) = x$ for all $s \in G$ and thus $W$ is a subrepresentation of $V$.
  Furthermore, this is isomorphic to the unit representation \Cref{ex:unit_trivial_representation} with $\tau\colon C^{\times} \to W$ such that $\tau(1) = x$.
  For example, let $G = Z_{3}$ and $\rho\colon Z_{3} \to \C^{3}$ the representation given in \Cref{ex:regular_representation_Z3}.
  Then $x = (1, 1, 1)$ and for example we have that
  \[
    \rho_{1}(x) = \rho(1)(e_{0}) + \rho_{1}(e_{1}) + \rho_{1}(e_{2}) = e_{1} + e_{2} + e_{0} = x.
  \]
\end{ex}

\begin{thrm}\label{thrm:stable_complements_of_subspace}
  Let $\rho\colon G \to \GL(V)$ be a linear representation of $G$ in $V$ and let $W$ be a subspace of $V$ stable under $G$.
  Then there exists a complement $W^{0}$ of $W$ in $V$ which is stable under $G$.
\end{thrm}
\begin{pf}
  Let $W'$ be an arbitrary complement of $W$ in $V$, and let $p\colon V \to W$ be the projection.
  Then we form the average $p^{0}$ of conjugates of $p$ by elements in $G$:
  \[
    p^{0} \defeq \frac{1}{\abs{G}} \sum_{t \in G} \rho_{t} \circ p \circ \rho_{t}^{-1}.
  \]
  Since $p\colon V \to W$ and $\rho_{t}$ preserves $W$, we have that $p^{0}$ maps $V$ onto $W$.
  Furthermore, note that $\rho_{t}^{-1}$ also preserves $W$.
  \clearpage
  Thus we have that
  \begin{align*}
    (p \circ \rho_{t}^{-1})(x) = \rho_{t}^{-1}(x), && (\rho_{t} \circ p \circ \rho_{t}^{-1})(x) = x, && p^{0}(x) = x.
  \end{align*}
  Thus, $p^{0}$ is a projection of $V$ onto $W$, corresponding to some complement $W^{0}$ of $W$.
  Moreover, we have that $\rho_{s} \circ p^{0} = p^{0} \circ \rho_{s}$ for all $s \in G$ because
  \[
    \rho_{s} \circ p^{0} \circ \rho_{s}^{-1} = \frac{1}{\abs{G}} \sum_{t \in G} \rho_{s} \circ \rho_{t} \circ p \circ \rho_{t}^{-1} \circ \rho_{s}^{-1} = \frac{1}{\abs{G}} \sum_{t \in G} \rho_{st} \circ p \rho_{st}^{-1} = p^{0}.
  \]
  Now suppose that $x \in W^{0}$ and $s \in G$, we have that $p^{0}(x) = 0$ and hence $(p^{0} \circ \rho_{s})(x) = (\rho_{s} \circ p^{0})(x) = 0$, meaning that $\rho_{s}(x) \in W^{0}$.
  This, $W^{0}$ is stable under $G$.
\end{pf}

Suppose that $V$ had an innerproduct $\inner{x, y}$, and furthermore suppose this inner product was invariant under $G$ meaning that for all $s \in G$, $\inner{\rho_{s}(x), \rho_{s}(y)} = \inner{x, y}$.
We may also reduce to this case by replacing $\inner{x, y}$ with $\sum_{t \in G}\inner{\rho_{t}(x), \rho_{t}(y)}$.
With this, the orthogonal complement $W^{\bot}$ of $W$ in $V$ is a complement of $W$ stable under $G$.
Note that the invariance of $\inner{x, y}$ means that if $(e_{i})$ is an orthonormal basis of $V$, then $R_{s}$ is a unitary matrix \quest{proof?}.

Using the notation of \Cref{thrm:stable_complements_of_subspace}, let $x \in V$ and $w, w^{0}$ be the projections of $x$ on $W$ and $W^{0}$ respectively.
Thus for all $s \in G$, $\rho_{s}(x) = \rho_{s}(w) + \rho_{s}(w^{0})$.
Since $W$ and $W^{0}$ are stable under $G$, we have that $\rho_{s}(w) \in W$ and $\rho_{s}(w^{0}) \in W^{0}$.
This means that $\rho_{s}(w)$ and $\rho_{s}(w^{0})$ are the projections of $\rho_{s}(x)$ and in turn the representations of $W$ and $W^{0}$ determine the representations of $V$.
\begin{defn}[Direct Sum of Representations]
  Given the above, we write $V = W \oplus W^{0}$ as the \emph{direct sum} of $W$ and $W^{0}$.
  We identify elements $v \in V$ as pairs $(w, w^{0})$ given by their projections.
\end{defn}
If the representations $W$ and $W^{0}$ are given in matrices $R_{s}$ and $R_{s}^{0}$, then the matrix form of the representation $V$ is given by
\[
  \begin{pmatrix} R_{s} & 0 \\ 0 & R_{s}^{0} \end{pmatrix}.
\]
Similar results hold for arbitrarily many, but finite, direct sums of representations.

\printbibliography
\end{document}
