\documentclass[letterpaper, 11pt, oneside]{book}

\usepackage{style}  % If you feel like procrastinating, mess with this file
\usepackage{algo}   % Thank you Jeff, very cool!

\addbibresource{refs.bib}

% Required reading
% https://jmlr.csail.mit.edu/reviewing-papers/knuth_mathematical_writing.pdf
%   Along with required viewing:
%   https://www.youtube.com/watch?v=N6QEgbPWUrg&list=PLOdeqCXq1tXihn5KmyB2YTOqgxaUkcNYG
% https://faculty.math.illinois.edu/~west/grammar.html

% % % % % % % % % %
%     Cursor      %
%     Parking     %
%     Lot         %
% % % % % % % % % %

% Disable check for mismatched parens/brackets/braces
%   chktex-file 9
% Disable check for different counts of parents/brackets/braces
%   chktex-file 17
% Exclude these environments from syntax checking
%   VerbEnvir { tikzcd }

\regtotcounter{figure}

\title{\vspace{-100pt} \includegraphics[width=0.75\textwidth]{figs/water.png} \\ {\Huge Representation Theory Notes and Exercises} \\ {\small With $\total{figure}$ Figures}}
\nocite{title_pic}
\author{\Large Anakin Dey}
\DTMsavenow{now}
\date{\small Last Edited on \today\ at \DTMfetchhour{now}:\DTMfetchminute{now}}

% Cover page number chicanery
\newcommand{\CoverName}{Cover}

\begin{document}
\frontmatter
\renewcommand{\thepage}{\CoverName}
\maketitle

\pagenumbering{roman}

\section*{TODOs}

\quest{Change the style of enumerates from ``1.'' to ``(1)''}

\quest{Proper Exercise Header}

\quest{Proper Chapter Header}

\tableofcontents
\clearpage


\listoftheorems[ignoreall, show={defn}, title={List of Definitions}]

\listoftheorems[ignoreall, show={ex}, title={List of Examples and Counterexamples}]

\chapter*{Preface}

This is a set of notes on group representation theory mainly based on J.P. Serre's text \emph{Linear Representations of Finite Groups}~\cite{book:SerreLinReps}.
Occasionally, other sources may be used.
The goal of these notes is to eventually work towards algebraic combinatorics such as Fulton's text \emph{Young Tableaux}~\cite{book:FultonTableaux}, as much of algebraic combinatorics is motivated by questions stemming from representation theory.
At the time of writing this, another goal is the applications of representation theory to computational complexity: see~\cite{misc:PanovaComputational} for a recent survey on this connection.

\mainmatter

\chapter{Generalities on Linear Representations}

Unless otherwise specified, $V$ will denote a vector space, usually over the field $\C$.
We will restrict ourselves to finite dimensional vector spaces.
Similarly, we will restrict ourselves to finite groups.

\begin{defn}[Linear Representation, Representation Space]
  Let $G$ be a group with identity $e$.
  A \emph{linear representation} of $G$ in $V$ is a homomorphism $\rho\colon G \to \GL(V)$.
  We will frequently, and often interchangably, write $\rho_{s} \defeq \rho(s)$.
  Given $\rho$, we will say that $V$ is a \emph{representation space} or \emph{representation} of $G$.
\end{defn}

\begin{defn}[Degree]
  Let $\rho\colon G \to V$ be a representation of $G$ in a vector space $V$.
  Then the \emph{degree} of $\rho$ is $\dim(V)$.
\end{defn}

Let $\rho\colon G \to V$ be a representation of $G$ in a vector space $V$ with $n \defeq \dim(V)$.
Fix a basis $(e_{j})$ of $V$.
Then since each $\rho_{s}$ is an invertible linear transformation of $V$, we may define an $n \times n$ matrix $R_{s} \equiv (r_{ij}(s))$ where each $r_{ij}(s)$ is defined by the identity
\[
  \rho_{s}(e_{j}) = \sum_{i = 1}^{n} r_{ij}(s) e_{i}.
\]

\begin{defn}[Matrix of a Representation]
  We call $R_{s} = (r_{ij}(s))$ above the \emph{matrix of $\rho_{s}$} with respect to the basis $(e_{j})$.
\end{defn}
Note that $R_{s}$ satisfies the following:
\begin{align*}
  \det(R_{s}) \neq 0, && R_{st} = R_{s} \cdot R_{t} \equiv r_{ij}(st) = \sum_{k = 1}^{n} r_{ik}(s) \cdot r_{kj}(s) \quad \forall s, t \in G.
\end{align*}

\clearpage

Recall that two $n \times n$ matrices $A, A'$ are \emph{similar} if there exists an invertible matrix $T$ such that $T A = A' T$.
We may extend this notion to representations.
\begin{defn}[Similar/Isomorphic Representations]
  Let $\rho$ and $\rho'$ be two representations of the same group $G$ in vector spaces $V$ and $V'$ respectively.
  We say $\rho$ and $\rho'$ are \emph{similar} or \emph{isomorphic} if there exists an isomorphism $\tau \colon V \to V'$ such that for all $s \in G$, $\tau$ satisfies $\tau \circ \rho(s) = \rho'(s) \circ \tau$.
  If $R_{s}, R'_{s}$ are the corresponding matrices then this is equivalent to saying there exists an invertible matrix $T$ such that $T R_{s} = R'_{s} T$ for all $s \in G$.
\end{defn}
Note that if $\rho$ and $\rho'$ are isomorphic, then they must have the same degree.

We now give some examples of these things.
\begin{ex}[Unit/Trivial Representation]
  Let $G$ be a finite group.
  Representations of degree $1$ must be of the form $\rho\colon G \to \C^{\times}$.
  Since elements $s$ of $G$ are of finite order, $\rho(s)$ must also be of finite order.
  Thus, for all $s \in G$, $\rho(s)$ is a root of unity.
  If we take $\rho(s) = 1$ for all $s \in G$, we obtain the \emph{unit} or \emph{trivial} representation of $G$.
\end{ex}

\begin{ex}[Regular Representation]
  Let $g$ be the order of $G$, and let $V$ be a vector space of dimension $g$ with a basis $(e_{t})_{t \in G}$.
  For each $s \in G$, define $\rho_{s}$ as the linear map $\rho_{s}\colon V \to V$ sending $e_{t} \mapsto e_{st}$.
  This is a linear representation of $G$ called the \emph{regular} representation of $G$.
  Since for each $s \in G$, $e_{s} = \rho_{s}(e_{1})$ and thus the images of $e_{1}$ form a basis of $V$.
  Conversely, let $W$ be a representation of $G$ with a vector $w$ satisfying the collection of all $\rho_{s}(w)$, $s \in G$, forms a basis of $W$.
  Then $W$ is isomorphic to the regular representation of $G$ by the isomorphism $\tau(e_{s}) = \rho_{s}(w)$.
\end{ex}

\begin{ex}[Permutation Representation]
  We may generalize the regular representation to any group action $G \acts X$, $X$ a finite set.
  Recall that for such an action, the map $x \mapsto sx$ for each $s \in G$ is a permutation $X \tofrom X$.
  Let $V$ be a vector space with dimension the size of $X$, and so a basis $(e_{x})_{x \in X}$.
  Define a representation $\rho$ of $G$ by defining $\rho_{s}$ as the linear map sending $e_{x} \mapsto e_{sx}$.
  This representation is known as the \emph{Permutation} representation of $G$ associated with $X$.
\end{ex}

\printbibliography
\end{document}
