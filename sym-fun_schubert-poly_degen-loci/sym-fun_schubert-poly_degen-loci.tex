\documentclass[letterpaper, 11pt, oneside]{book}

\usepackage{style}  % If you feel like procrastinating, mess with this file
\usepackage{algo}   % Thank you Jeff, very cool!

\addbibresource{refs.bib}

% Required reading
% https://jmlr.csail.mit.edu/reviewing-papers/knuth_mathematical_writing.pdf
%   Along with required viewing:
%   https://www.youtube.com/watch?v=N6QEgbPWUrg&list=PLOdeqCXq1tXihn5KmyB2YTOqgxaUkcNYG
% https://faculty.math.illinois.edu/~west/grammar.html

% % % % % % % % % %
%     Cursor      %
%     Parking     %
%     Lot         %
% % % % % % % % % %

% Disable check for mismatched parens/brackets/braces
%   chktex-file 9
% Disable check for different counts of parents/brackets/braces
%   chktex-file 17
% Exclude these environments from syntax checking
%   VerbEnvir { tikzcd }

\regtotcounter{figure}

\title{\vspace{-100pt} {\Huge Young Tableau, Symmetric Functions, \protect\\ Schubert Polynomials, and Degeneracy Loci} \\ {\small With $\total{figure}$ Figures}}
\author{\Large Anakin Dey}
\DTMsavenow{now}
\date{\small Last Edited on \today\ at \DTMfetchhour{now}:\DTMfetchminute{now}}

% Cover page number chicanery
\newcommand{\CoverName}{Cover}

\begin{document}
\frontmatter
\renewcommand{\thepage}{\CoverName}
\maketitle

\pagenumbering{roman}

\tableofcontents
\clearpage


% \listoftheorems[ignoreall, show={defn}, title={List of Definitions}]
%
% \listoftheorems[ignoreall, show={ex}, title={List of Examples and Counterexamples}]

\chapter*{Preface}

These are notes for a reading course under Professor \href{https://people.math.osu.edu/anderson.2804/index.html}{Dave Anderson}.
They begin with a review of some material from Fulton's \emph{Young Tableaux}\footnote{which throughout these notes will be spelled as ``tableaux'' or ``tableau'' with no real consistency.}~\cite{book:YT}.
However, the primary focus is Manivel's \emph{Symmetric Functions, Schubert Polynomials, and Degeneracy Loci}~\cite{book:ManivelSFSPDL} which one could see as a quasi-sequel.

\mainmatter

\chapter{\cite{book:YT} Geometry}

\begin{sol}[\cite{book:YT} \S 9.1 Ex. 1]\label{ex:YT_9.1.1}
  Choose a basis $\set{e_{1}, \ldots, e_{m}}$ so that $E$ can be identified with $\C^{m}$.
  Let $i_{1} < \cdots < i_{d - 1}$ and $j_{1} < \cdots j_{d + 1}$ be sequences in $[m]$.
  Apply \S 9.1 Equation (1) with $k = 1$ to the sequences $j_{2} < \cdots < j_{d + 1}$ and $i_{1} < \cdots < i_{d - 1}, j_{1}$ by fixing $j_{1}$ to be the vector swapped successively with the $j_{2} < \cdots < j_{d + 1}$.
  Reordering the indices and applying the appropriate sign change yields the desired alternating summation.
\end{sol}

\begin{sol}[\cite{book:YT} \S 9.1 Ex. 2]\label{ex:YT_9.1.2}
  We have that $V \subseteq E = \C^{4}$ is given as the kernel of multiplication of a matrix $A = (a_{i, j})_{\substack{1 \leq i \leq 4 \\ 1 \leq j \leq 2}}$.
  To find this matrix, the given conditions of the $x_{i, j}$ describe the following determinantal conditions on the entries of $A$:
  \begin{align*}
    x_{1, 2} = 1 &\iff \Delta_{1, 2}(A) = 1, \\
    x_{1, 3} = 2 &\iff \Delta_{1, 3}(A) = 2, \\
    x_{1, 4} = 1 &\iff \Delta_{1, 4}(A) = 1, \\
    x_{2, 3} = 1 &\iff \Delta_{2, 3}(A) = 1, \\
    x_{2, 4} = 2 &\iff \Delta_{2, 4}(A) = 2, \\
    x_{3, 4} = 3 &\iff \Delta_{3, 4}(A) = 3.
  \end{align*}
  From here, we must make an assumption based on which affine portion of $\mathbb{P}^{5}$ our matrix lives in.
  This amounts to picking some $i_{1}, i_{2}$ so that the minor given by those columns is the identity matrix.
  For the given conditions, we could pick $(i_{1}, i_{2}) = (1, 2)$, $(1, 4)$, or $(2, 3)$.
  We give $A$ for each of these choices respectively:
  \begin{align*}
    A =
    \begin{pmatrix}
      1 & 0 & -1 & -2 \\
      0 & 1 & 2 & 1
    \end{pmatrix},
    &&
    A =
    \begin{pmatrix}
      1 & 2 & 3 & 0 \\
      0 & 1 & 2 & 1
    \end{pmatrix},
    &&
    A =
    \begin{pmatrix}
      2 & 1 & 0 & -3 \\
      -1 & 0 & 1 & 2
    \end{pmatrix}.
  \end{align*}
  It is clear that these are the same matrix up to row operations.
\end{sol}

\printbibliography
\end{document}
